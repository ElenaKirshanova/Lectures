% !TeX spellcheck = ru_RU-Russian
% !TeX program = xelatex
% !BIB TS-program = biber

\documentclass{beamer}

\usepackage[utf8]{inputenc}
\usepackage[russian]{babel}

%---tikz----
\usepackage{tikz}
\usetikzlibrary{arrows, chains, matrix, positioning, scopes, patterns, shapes}
\usepackage{pgfplots, subfigure}
\usepackage{extarrows}

\usepackage[backend=biber,firstinits=true,hyperref=true,style=numeric-comp]{biblatex}

\usepackage{../beamerthemeec2020}
{\footnotesize\bibliography{../biblio}}

\title{Эллиптические кривые}
\subtitle{Лекция 6. Алгоритмы подсчета точек кривой II}
\author{Семён Новосёлов}
\institute{БФУ им. И. Канта}
\date{2025}

\begin{document}

\frame{\titlepage}

%\begin{frame}{Алгоритм Сазерленда}
%    
%\end{frame}

\begin{frame}{Алгоритм Схоофа\footnote{\textit(гол.) Schoof = Схооф, в рус. лит. больше известен как Шуф.}}
\[
E/\mathbb{F}_q: y^2 = x^3 + a x + b
\]
Имеем:
\begin{itemize}
    \item $|E(\mathbb{F}_q)| = q + 1 - t$, где $t$ -- след эндоморфизма Фробениуса, $|t| \leq 2 \sqrt{q}$.
\end{itemize}

\vspace{1em}
\structure{Идея:} найти $t~(\bmod~\ell_i)$ для малых простых чисел $\ell_1, \ldots, \ell_n$ и восстановить $t$ по КТО и неравенству для следа~$t$.

\begin{itemize}
    \item $|t| \leq 2 \sqrt{q}$ \structure{$\implies$} $\prod_{i=1}^{n} \ell_i > 4 \sqrt{q}$ \structure{$\implies$} $\ell_n = O(\log{q})$
\end{itemize}
\end{frame}

\begin{frame}{Число точек по модулю $\ell = 2$}
    \begin{itemize}
        \item $\#E(\mathbb{F}_q)$ -- чётно \structure{$\iff$} $E(\mathbb{F}_q)$ содержит точку ($\neq \mathcal{O}$) порядка~$2$
        \vspace{0.5em}
        \item точка $P$ порядка $2$ имеет $y_P=0$ \structure{$\iff$} $x_P^3 + a x_P + b = 0$ в~$\mathbb{F}_q$
        \vspace{0.5em}
        \item проверка наличия точек порядка $2$: $\gcd(x^q - x, x^3 + a x + b) \neq 1$ в $\mathbb{F}_q[x]$
        \begin{itemize}
            \item[$\implies$] $O(\log^3{q})$, быстрое возведение в степень в $\mathbb{F}_q[x]/(x^3 + a x + b)$
         \end{itemize}
    \end{itemize}
\end{frame}

\begin{frame}{Число точек по модулю $\ell > 2$}
\[
E[\ell] = \{P \in E(\overline{\mathbb{F}}_q)~|~[\ell] P = \mathcal{O}\} \simeq \mathbb{Z}/\ell\mathbb{Z} \times \mathbb{Z}/\ell \mathbb{Z}
\]
\begin{itemize}
    \item $\varphi_q: (x, y) \mapsto (x^q, y^q)$ -- эндоморфизм Фробениуса,
    \[
    \varphi_q^2 - [t] \varphi_q + [q] = 0
    \]
    или
    \[
    (x^{q^2}, y^{q^2}) - [t] (x^{q}, y^{q}) + [q](x,y) = P_\infty.
    \]
    \item для ограничения $\varphi_q$ на $E[\ell]$ имеем:
    \[
    (x^{q^2}, y^{q^2}) - [t'] (x^{q}, y^{q}) + [q'](x,y) = P_\infty,
    \]
    где $t', q' \in \{0, ..., \ell-1\}$ и $t = t' \pmod{\ell}$, $q = q' \pmod{\ell}$.
\end{itemize}
\end{frame}

\begin{frame}
\begin{equation}
    \label{eq:char_poly_mod_l}
(x^{q^2}, y^{q^2}) - [t'] (x^{q}, y^{q}) + [q'](x,y) = P_\infty
\end{equation}
\begin{itemize}
    \item $\psi_\ell(x) \in \mathbb{F}_q[x]$, $\ell$-многочлен деления (может быть эффективно вычислен по рек. формуле)
    \item $P = (x_P, y_P) \in E[\ell]$ \structure{$\iff$} $\psi_\ell(x_P) = 0$
    \item из \eqref{eq:char_poly_mod_l} получаем
    \[(x^{q^2}, y^{q^2}) + [q'](x,y) = [t'] (x^{q}, y^{q})\]
    по модулю $\psi_\ell(x)$ и $E(x,y) = y^2 - x^3 - a x - b$
\end{itemize}
\end{frame}

\begin{frame}
    \begin{equation}
    \label{eq:chi_mod_l_2}
    (x^{q^2}, y^{q^2}) + [q'](x,y) = [t'] (x^{q}, y^{q}) 
    ~\bmod (\psi_\ell(x), E(x,y))
    \end{equation}
    \begin{itemize}
        \item $x^q, y^q, x^{q^2}, y^{q^2} \pmod{\psi_\ell} $ \structure{$\implies$} быстрое возведение в степень
        \item $[q'](x,y)$ и $[t'](x^q, y^q)$ $\pmod{\psi_\ell(x)}$ \structure{$\implies$} многочлены $q'$ и $t'$-деления
    \end{itemize}
    \vspace{1em}
    Значения $t' = t\bmod{\ell}$ (соотв. $\#E(\mathbb{F}_q)\bmod{\ell}$) находим перебором возможных вариантов для $t'$ пока не выполнится \eqref{eq:chi_mod_l_2}.
\end{frame}

\begin{frame}{Алгоритм Схоофа}
\structure{Вход}: $E/\mathbb{F}_q$\\
\structure{Выход:} $\#E(\mathbb{F}_q)$\\
\begin{enumerate}
    \item $M = 2, \ell = 3, S = \{(t \bmod{2}, 2)\}$
    \item \structure{while} $M < 4 \sqrt{q}$ \structure{do:}
    \item \quad \structure{for} $t' = 0, \ldots, \ell-1$ \structure{do:}
    \item \quad \quad \structure{if} $\varphi_q^2(P) + [q'] P = [t']\varphi_q(P) \pmod{(\psi_\ell, E)}$  \structure{do:}\\
    \quad \quad \quad \structure{break}
    \item \quad $S = S \cup \{ (t',\ell) \}$
    \item \quad $M = M \cdot \ell$
    \item \quad $\ell = next\_prime(\ell)$
    \item найти $t$ по КТО, используя $S$
    \item \structure{return} $q + 1 - t$
\end{enumerate}
\end{frame}

\begin{frame}{Анализ сложности}
\structure{Оценка размера $\ell$.}
\[\ell = O(\log{q})\]
\ProofBegin
\begin{enumerate}
    \item[1.] для однозначного восстановления $t$ по $M = \prod\limits_{i=1}^{n} \ell_i$, необх. $M > 4 \sqrt{q}$.
    \item[2.] $M = p_n\#$ -- праймориал \MyImplies $M = n^{(1+o(1)) n}$.
\end{enumerate}

\vspace{0.5em}
Объединяя пункты 1 и 2 (и взяв логарифм) получаем:

\[O(n \log{n}) = O(\log{q}) \text{\MyImplies} n = O\left(\frac{\log{q}}{\log{\log{q}}}\right).\]

\vspace{1em}
При этом~$\ell = \ell_n = O(n \log{n})$ (теорема о распределении простых чисел) \MyImplies $\ell = O(\log{q})$.

\ProofEnd
\end{frame}

\begin{frame}
\structure{Оценка сложности операций.}

Базовые операции:\footnote{Hoeven J.v.d., Larrieu R. - Fast reduction of bivariate polynomials with respect to sufficiently regular Gröbner bases. 2018.}
\begin{itemize}
    \item Редукция многочлена степени $d$ по модулю $\psi_\ell$ и $E$:
    
    $\widetilde{O}(d^2 + \deg{\psi_\ell}^2)$ операций в $\mathbb{F}_q$.
    
    $\deg{\psi} = \frac{\ell^2 - 1}{2}$ \MyImplies $\widetilde{O}(d^2 + \ell^4)$.
    
    \item Умножение в кольце $\mathbb{F}_q[x, y] / (\psi_\ell, E)$:
    
    $\widetilde{O}(\ell^4)$ операций в $\mathbb{F}_q$.
\end{itemize}
\end{frame}

\begin{frame}
Проверка условия $\varphi_q^2(P) + [q'] P = [t']\varphi_q(P) \pmod{(\psi_\ell, E)}$:
\begin{itemize}
    \item $(x^q, y^q), (x^{q^2}, y^{q^2}) \pmod{\psi_\ell, E} $ \MyImplies
    быстрое возведение в степени $q$ и $q^2$
    \MyImplies
    $O(\log{q})$ умножений в~$\mathbb{F}_q[x, y] / (\psi_\ell, E)$
    \MyImplies
    $\widetilde{O}(\ell^4 \log{q})$ операций в $\mathbb{F}_q$.
    
    %$\deg{\psi_\ell} = \frac{\ell^2 - 1}{2} = O(\ell^2)$ \structure{$\implies$} $O((\log{q}) (\ell^2 \log{q})^2)$
    
    \vspace{0.5em}
    \item $[q'] P$ \MyImplies рекур. формулы для многочленов деления%, $O(\ell)$ 
    
    \vspace{0.5em}
    \item $[t'](x^q, y^q)$ $\pmod{\psi_\ell, E}$:
    
    $x^q$ и $y^q$ $\pmod{\psi_\ell, E}$ -- многочлены степени $< \ell^2$, уже известны, $t' < \ell$
    \MyImplies
    используя рек. формулы для мн. деления имеем макс. $\ell$ операций умножения + редукции многочленов степени $< 2 \ell^2$ в~$\mathbb{F}_q[x, y] / (\psi_\ell, E)$
    
    \MyImplies
    $\widetilde{O}(\ell^5)$ операций в $\mathbb{F}_q$.
    %\structure{$\implies$} макс. $\ell$ раз, $O(\ell (\ell^2 \log{q})^2)$
\end{itemize}
\end{frame}

\begin{frame}
Перебирая $t'$ нужно проверять условие макс. $\ell$ раз
\MyImplies
$\widetilde{O}(\ell^5 \log{q} + \ell^6) = \widetilde{O}(\ell^5 \log{q})$ операций в $\mathbb{F}_q$.

\vspace{1em}
Всего в алгоритме делается $n = O(\frac{\log{q}}{\log{\log{q}}})$ итераций.

\vspace{1em}
\structure{Итого:}
$O(\frac{\log{q}}{\log{\log{q}}}) \cdot \widetilde{O}(\ell^5 \log{q}) = \widetilde{O}(\log^7{q})$ операций в $\mathbb{F}_q$ или~$\widetilde{O}(\log^8{q})$ битовых операций.

%$O(\log{q} \cdot (\log{q} (\ell^2 \log{q})^2 + \ell (\ell^2 \log{q})^2)) = O(\log^8{q})$
\end{frame}

\begin{frame}{Алгоритм Схоофа: дальнейшие улучшения}
    \structure{Schoof-Elkies-Atkin (SEA)}:
    \begin{itemize}
        \item замена многочленов деления на многочлены $g_\ell$, задающие изогении (степени: $O(\ell^2)$ \structure{$\implies$} $O(\ell)$)
        \item факторизация модулярных многочленов для нахождения ядер изогений (нулей $g_\ell$)
        \item эвристическая сложность: $O(\log^4{q})$
    \end{itemize}
\end{frame}

\begin{frame}{Литература}
    %\nocite{Menezes1993}
    \nocite{Blake1999}
    \nocite{Schoof1985}
    \nocite{CohenFrey+2005}
    \nocite{Washington2008}
    \printbibliography

    \begin{center}
        \begin{tcolorbox}[enhanced,hbox,colback=block-green-color-bg,colframe=subsection-color!120,title=Контакты,center title]
            \begin{varwidth}{\textwidth}
                \begin{center}
                    \href{mailto:snovoselov@kantiana.ru}{snovoselov@kantiana.ru}
                \end{center}
            \end{varwidth}
        \end{tcolorbox}	
    \end{center}
    
    \structure{Страница курса:}\\
    {\footnotesize
    	\href{https://crypto-kantiana.com/semyon.novoselov/teaching/elliptic_curves_2025}{crypto-kantiana.com/semyon.novoselov/teaching/elliptic\_curves\_2025}
    }
\end{frame}

\end{document}