% !TeX program = xelatex

\documentclass[11pt]{exam}

%---enable russian----

\usepackage[utf8]{inputenc}
%\usepackage[russian]{babel}
\usepackage[english,main=russian]{babel}


\usepackage{cmap}
\usepackage{fontspec}
\usepackage{ulem}

\setmainfont{Roboto}
\setsansfont{Roboto}
\setmonofont{Roboto Mono}


\usepackage[margin=0.73in]{geometry}
%\usepackage[top=1in, bottom=1in, left=1in, right=1in]{geometry}

\usepackage{graphicx}
\usepackage{url}
\usepackage{latexsym}
\usepackage{amscd,amsmath,amsthm}
\usepackage{mathtools}
\usepackage{amsfonts}
\usepackage{amssymb}
\usepackage[dvipsnames]{xcolor}
\usepackage{hyperref}

\usepackage{algorithmicx, enumitem, algpseudocode, algorithm, caption}
\usepackage{tikz}
\usetikzlibrary{arrows,decorations.pathmorphing,backgrounds,calc}
\usetikzlibrary{chains, matrix, positioning, scopes, patterns, shapes}
\usepackage{pgfplots, subfigure}
\usetikzlibrary{automata}

\usepackage{fontawesome5}


\usepackage[most]{tcolorbox}
\usepackage{varwidth}

\usepackage[backend=biber,firstinits=true,hyperref=true,style=numeric-comp]{biblatex}

\definecolor{text-color}{RGB}{0,0,0} %%% text color
\definecolor{subsection-color}{RGB}{40, 166, 103} %%% subsection color
\definecolor{title-and-section-color}{RGB}{61, 147, 221} %%% title and section color
\definecolor{bg-color}{RGB}{254, 255, 244} %%% background color
\definecolor{struct-color}{RGB}{224, 62, 73} %%% structure

\definecolor{color-cont-bg}{RGB}{204, 239, 171}
\definecolor{block-green-color-bg}{RGB}{204, 239, 171}

\definecolor{plot-red-color}{RGB}{224, 62, 73}
\definecolor{plot-green-color}{RGB}{77, 158, 99}
\definecolor{plot-blue-color}{RGB}{61, 147, 221}

\definecolor{box-red-color}{RGB}{224, 62, 73}
\definecolor{box-green-color}{RGB}{77, 158, 99}
\definecolor{box-blue-color}{RGB}{61, 147, 221}

\bibliography{../biblio}

%%%%%%%%%%%%%%%%%%%%%
% Handling comments and versions %%%
%%%%%%%%%%%%%%%%%%%%%

%\renewcommand{\comment}[1]{\texttt{[#1]}}


%%%%%%%%%%%%%%%%%%%%%%%%%%%
%% THEOREMS
%%%%%%%%%%%%%%%%%%%%%%%%%%%

\newtheorem{theorem}{Theorem}[section]
\newtheorem{axiom}[theorem]{Axiom}
\newtheorem{conclusion}[theorem]{Conclusion}
\newtheorem{condition}[theorem]{Condition}
\newtheorem{conjecture}[theorem]{Conjecture}
\newtheorem{corollary}[theorem]{Corollary}
\newtheorem{criterion}[theorem]{Criterion}
\newtheorem{definition}[theorem]{Definition}
\newtheorem{lemma}[theorem]{Lemma}
\newtheorem{notation}[theorem]{Notation}
\newtheorem{proposition}[theorem]{Proposition}


\theoremstyle{definition}
\newtheorem{problem}{Problem}


\newcommand{\nc}{\newcommand}
\nc{\eps}{\varepsilon}
\nc{\RR}{{{\mathbb R}}}
\nc{\CC}{{{\mathbb C}}}
\nc{\FF}{{{\mathbb F}}}
\nc{\NN}{{{\mathbb N}}}
\nc{\ZZ}{{{\mathbb Z}}}
\nc{\PP}{{{\mathbb P}}}
\nc{\QQ}{{{\mathbb Q}}}
\nc{\UU}{{{\mathbb U}}}
\nc{\OO}{{{\mathbb O}}}
\nc{\EE}{{{\mathbb E}}}

\newcommand{\val}{\operatorname{val}}
\newcommand{\wt}{\ensuremath{\mathit{wt}}}
\newcommand{\Id}{\ensuremath{I}}
\newcommand{\transpose}{\mkern0.7mu^{\mathsf{ t}}}
\newcommand*{\ScProd}[2]{\ensuremath{\langle#1\mathbin{,}#2\rangle}} %Scalar Product
\renewcommand{\char}{\ensuremath{\mathsf{char}}}

\DeclareMathOperator{\Vol}{Vol}

\newcommand*{\structure}[1]{\textcolor{struct-color}{#1}}

%\pretolerance=1000

%%%%%%%%%%%%%%%%%%%%%%%%%%%%%%%%
%%%%%%%%%%%%%%%%%%%%%%%%%%%%%%%%
%% DOCUMENT STARTS
%%%%%%%%%%%%%%%%%%%%%%%%%%%%%%%%
%%%%%%%%%%%%%%%%%%%%%%%%%%%%%%%%


\begin{document}
{\noindent
\textsc{БФУ им. И. Канта -- Компьютерный практикум по криптографии на эллиптических кривых }\\[5pt]
Преподаватель {С. Новоселов}   \hfill{Осень 2025\\}
\hrule
\begin{center}
	{\LARGE\textbf{
			Лекция 8. Тест на простоту на эллиптических кривых \\[5pt]
		}}

\end{center}
\hrule \vspace{5mm}

\thispagestyle{empty}

Тесты на простоту используются в криптографии для генерации простых чисел как, например, в криптосистеме RSA. Кроме того, генерация групп с простым порядком для криптосистем основанных на сложности решения дискретного логарифма также требует проверки порядка группы на простоту. Тест на простоту на эллиптических кривых был предложен  Шафи Гольдвассер и Джо Килианом~\cite{GoldwasserKilian1999} в 1986 году и является на сегодняшний день самым быстрым алгоритмов доказательства простоты числа.

\section{Тест на простоту Миллера-Рабина}
Тест Миллера-Рабина основан на следующей теореме.

\begin{center}
	\begin{tcolorbox}[enhanced,hbox,colback=title-and-section-color!5,colframe=title-and-section-color!120,title=Малая теорема Ферма,center title]
		\begin{varwidth}{\textwidth}
			\begin{center}
				$p$ -- простое, $p \nmid a$ \structure{$\implies$} $a^{p-1} - 1 \equiv 0 \pmod{p}$.
			\end{center}
		\end{varwidth}
	\end{tcolorbox}
\end{center}

Число~$p-1$ можно записать как $p-1 = 2^k \cdot q$, где~$q$ -- нечётное. Значит $a^{p-1} - 1$ можно представить в виде:
\begin{equation*}
	\begin{split}
		a^{p-1} - 1 & = (a^{2^{k-1} \cdot q} - 1) \cdot (a^{2^{k-1} \cdot q} + 1) = \\&= (a^q - 1) (a^{2^{k-1} \cdot q} + 1) \cdot \ldots \cdot (a^{2 \cdot q} + 1) \cdot (a^q + 1)
	\end{split}
\end{equation*}
Следовательно, если~$p$ -- простое, то по малой теореме Ферма $p$ делит один из множителей~$a^{p-1} - 1$, т.е. выполняется одно из условий:
\begin{equation}
	\label{Miller-Rabin_conditions}
	a^q \equiv 1, a^{2^{k-1} \cdot q} \equiv -1, \ldots, a^{q} \equiv -1.
\end{equation}
С другой стороны, если~$p$ -- не простое, то существует~$a$ такое, что все сравнения~\eqref{Miller-Rabin_conditions} не выполняются. Число $a$ т.ч. $a^q \not\equiv 1, a^{2^{k-1} \cdot q} \not\equiv -1, \ldots, a^{q} \not\equiv -1$ называется \structure{свидетелем}, что $n$ -- составное.

\structure{Идея} алгоритма Миллера-Рабина для проверки $n$ на простоту заключается в поиске таких свидетелей среди случайных чисел. Если среди достаточно большого случайных чисел~$a$ такого свидетеля не нашлось, то с большой вероятностью число простое.

\begin{algorithm}
	\caption{Алгоритм Миллера-Рабина}\label{alg:miller-rabin}
	\begin{algorithmic}
		\Require $n, a\in \mathbb{Z}$
		\Ensure «составное» или «возможно простое»
		\State$n-1 = 2^kq,\ q$ -- нечётное
		\State $a = a^q\bmod n$
		\If{$a \equiv 1\bmod n$}
		\Return <<возможно простое>>
		\EndIf
		\For{$i = 0, \ldots, k-1$}
		\If{$a \equiv -1 \bmod{n}$}
		\Return <<возможно простое>>
		\EndIf
		\State $a = a^2 \bmod{n}$
		\EndFor
		\Return <<составное>>
	\end{algorithmic}
\end{algorithm}

Более точно, алгоритм выполняется $K$ раз для случайных~$a \in [2, n-2]$. Вероятность ошибки, при которой число~$p$ -- не простое, равна~$2^{-2 K}$. Время работы алгоритма: $\mathcal{O}(K \log^3{n})$.

Алгоритм можно сделать \structure{детерминированным} (предполагая GRH) перебрав все $a ≤ 2 \ln^2 n$. Сложность в этом случае будет равна~$\mathcal{O}(\log n)$. Для маленьких $n$ достаточно перебрать только небольшое количество чисел~$a$. Например, для $n \leq 2^{64}$ достаточно проверить $a = 2$, $3$, $5$, $7$, $11$, $13$, $17$, $19$, $23$, $29$, $31$ и $37$.

\section{Тест на простоту на эллиптических кривых}
\structure{Задача:} по данному (большому) числу $p$ определить, является ли $p$ простым числом и, если да, вывести доказательство (\structure{сертификат}) простоты $p$.

Идея алгоритма Гольдвассер-Килиана состоит в заменене группы $\mathbb{Z}_n^\times$ как в алгоритме Миллера-Рабина или тесте
Поклингтона-Леммера на $E(\mathbb{Z}_n)$.

Число точек на эллиптической кривой при такой замене ведёт себя как случайное число из интервала Хассе-Вейля. Более точно, имеет место следующая теорема, доказанная Ленстрой в 1987 году.
\begin{center}
	\begin{tcolorbox}[enhanced,hbox,colback=title-and-section-color!5,colframe=title-and-section-color!120,title= Теорема о распределении порядков случайных ЭК,center title]
		\begin{varwidth}{\textwidth}
			\begin{center}
				Пусть $p>5$ -- простое, $S \subseteq [ p+1-\lfloor\sqrt{p}\rfloor, p+1+\lfloor\sqrt{p}\rfloor ]$ и $A, B \leftarrow \mathbb{F}_p$. Тогда $\exists\ c$ -- константа, т.ч.
				\[
				\Pr\left[\#E_{A,B}(\mathbb{F}_p) \in S \right] > \frac{c}{\log p} \cdot \frac{|S|-2}{2\lfloor\sqrt{p}\rfloor + 1},
				\]
				где $\#E_{A,B}(\mathbb{F}_p)$ -- число точек на $E_{A,B}: y = x^3 + Ax+B$.
			\end{center}
		\end{varwidth}
	\end{tcolorbox}	
\end{center}

Имея точку $L \in E(\mathbb{Z}_n)$, можно с помощью перейти к $L_p = L \bmod{p} \in E(\mathbb{Z}_p)$ с сохранением групповой операции. Более точно имеет место следующая лемма. 
\begin{lemma}
\label{lemma:points_En_to_Ep}
Пусть $n \in \mathbb{Z},\ 2,3\nmid \ n; p>3$ -- простой делитель $n$ и $4A^3 + 27B^2 \not\equiv 0\bmod p$.
\begin{itemize}
	\item Для любого $x \in \mathbb{Z}/n\mathbb{Z}$ пусть $x_p := x\bmod p$.
	\item Для любой точки $L = (x,y) \in E_{A,B}( \mathbb{Z}/n\mathbb{Z})$ пусть $L_p = (x_p, y_p) \in E_{A,B}(\mathbb{F}_p)$.  %$\infty_p = \infty_x \in E_{A, B}(\mathbb{Z}/n\mathbb{Z})$.
\end{itemize}  

Тогда $\forall L, M \in E_{A,B}(\mathbb{Z}/n\mathbb{Z})$, если $L+M$ определено, то $(L+M)_p = L_p+M_p$.
\end{lemma}

Для доказательства простоты числа мы будем рекурсивно применять следующий критерий простоты.

   \begin{center}
	\begin{tcolorbox}[enhanced,hbox,colback=title-and-section-color!5,colframe=title-and-section-color!120,title= Теорема (Критерий простоты),center title]
		\begin{varwidth}{\textwidth}
			\label{t3}
			Положим:
			\begin{itemize}
				\item $n \in \mathbb{Z}$, $A, B \in \mathbb{Z}/n\mathbb{Z}$
				\item $2,3 \nmid n$, $\gcd(4A^3 + 27B^2, n)=1$
				\item $L \in E_{A,B}(\mathbb{Z}/n\mathbb{Z})$, $L\neq \infty$
			\end{itemize}
			\vspace{0.5em}
			Тогда:
			
			\vspace{0.1em}
			%\begin{center}
			$\exists$ простое $q > (n^{1/4} + 1)^2$, т.ч. $qL = \infty$ \structure{$\implies$} $n$ -- простое.
			%\end{center}
		\end{varwidth}
	\end{tcolorbox}
\end{center}
\structure{$\triangleleft$}
От противного: пусть $n$ -- составное \structure{$\Rightarrow$} $\exists p > 3$, т.ч. $p \mid n$, $p$ -- простое. Т.к. $p$ -- простое, то~$p \leq \sqrt{n}$. Заметим что $\gcd(4A^3 + 27B^2, p) \neq 0 \bmod p$, иначе получается противоречие с условием $\gcd(4A^3 + 27B^2, n) = 1$. По Лемме~\ref{lemma:points_En_to_Ep} имеем $L_p \in E_{A,B}(\mathbb{F}_p)$ и $q\cdot L_p = (qL)_p = \infty_p = \infty$ \structure{$\Rightarrow$} $\operatorname{ord}(L_p) \mid q$ \structure{$\Rightarrow$} $\operatorname{ord}(L_p) = q$, т.к. $q$ -- простое. По условию теоремы~$(n^{1/4} + 1)^2 < q = \operatorname{ord}(L_p) \leq \#E_{A,B}(\mathbb{F}_p) < (\sqrt{p} + 1)^2$. Следовательно, $p > \sqrt{n}$. Это противоречие, значит, $n$ -- простое.
\structure{$\triangleright$}

$\ldots \ldots \ldots$

\nocite{Blake1999}
\nocite{Washington2008}
\nocite{Menezes1993}
\nocite{HankersonMenezesVanstone2006}
\nocite{Silverman2009}
\printbibliography

\end{document}
