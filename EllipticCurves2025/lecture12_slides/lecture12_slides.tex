% !TeX program = xelatex
% !BIB TS-program = biber

\documentclass{beamer}

\usepackage[utf8]{inputenc}
\usepackage[russian]{babel}

%---tikz----
\usepackage{tikz}
\usetikzlibrary{arrows, chains, matrix, positioning, scopes, patterns, shapes}
\usepackage{pgfplots, subfigure}
\usepackage{extarrows}
\usepackage{tikz-cd}
\usepackage{makecell}
\usepackage{quiver}
\usetikzlibrary{babel}

\usepackage[backend=biber,firstinits=true,hyperref=true,style=numeric-comp]{biblatex}

\usepackage{../beamerthemeec2020}
\usepackage{mathtools}

{\footnotesize\bibliography{../biblio}}

\title{Эллиптические кривые}
\subtitle{Лекция 12. Схема подписи SQISign}
\author{Семён Новосёлов}
\institute{БФУ им. И. Канта}
\date{2025}

\begin{document}

\frame{\titlepage}

\begin{frame}{Введение}
	\structure{SQISign} (Short Quaternion and Isogeny Signature) -- схема подписи, построенная применением преобразования Фиата-Шамира к протоколу доказательства с нулевым разглашением ($\Sigma$-протоколу).
	\vspace{1em}
	
	\begin{scriptsize}
	\begin{itemize}
		\item[\structure{\faScroll}] De Feo L., Kohel D., Leroux A., Petit C., Wesolowski B.\\``SQISign: compact post-quantum signatures from quaternions and isogenies''. AsiaCrypt 2020. \url{https://eprint.iacr.org/2020/1240}
	\end{itemize}
	\end{scriptsize}
	
	\vspace{1em}
	
	Последняя версия схемы:
	\begin{scriptsize}
	\begin{itemize}
		\item[\structure{\faGlobe}] \url{https://sqisign.org}
	\end{itemize}
	\end{scriptsize}
\end{frame}

\begin{frame}[fragile]{$\Sigma$-протокол}
%Состоит из двух участников: \\доказывающий (\UserA) и проверяющий (\UserB).
%\vspace{1em}
%Позволяет \UserA~~доказать \UserB~~знание секрета без раскрытия самого секрета.
Позволяет доказать знание секрета без раскрытия самого секрета.
\vspace{1em}

\begin{table}[]
	\label{tab:my-table}
	\begin{tabular}{ccc}
		\UserA($sk,pk$) &  & \UserB($pk$) \\
		 &  & \\
		сген. обязательство $com$ & $\xrightarrow{com}$ & \ldots \\
		%& $\xrightarrow{com}$   & \\
		 &  & \\
		\ldots & $\xleftarrow{chl}$ & сген. вызов $chl$ \\
		%&  $\xleftarrow{chl}$ & \\
		 &  & \\
		сген. ответ $rsp$ & $\xlongrightarrow{rsp}$ & \ldots \\
		%& $\xlongrightarrow{rsp}$   & \\
		& & вернуть ``accept'' или ``reject''
	\end{tabular}
\end{table}
\end{frame}

\begin{frame}{Преобразование Фиата-Шамира}
Позволяет построить цифровую подпись из любого $\Sigma$-протокола.
\vspace{1em}

\begin{itemize}
	\item $m$ -- сообщение для подписи
	\item при генерации $chl$ заменяем \UserB~на хэш-функцию:
	\[
	chl = hash(m || com)
	\]
	\item \structure{подпись} $s := (com, rsp)$
	\item \structure{проверка}: вычисляем $chl = hash(m || com)$ и проверяем условие $\Sigma$-протокола
\end{itemize}

\end{frame}

%\begin{frame}{Пример. Подпись EdDSA}
%	содержимое...
%\end{frame}

\begin{frame}{Подпись SQISign. Параметры}
	\begin{itemize}
		\item $E_0$ -- (открытая) кривая с известным $End(E_0)$
		\item $\phi_{sk}: E_0 \rightarrow E_{pk}$ -- случайная секретная изогения
		\item $E_{pk}$ -- открытый ключ
		\item $End(E_{pk})$ -- секретный ключ
	\end{itemize}
	
	\vspace{1em}
	\structure{Замечание:}
	\begin{itemize}
		\item зная $\phi: E_0 \rightarrow E$ можно вычислить $End(E)$ за полиномиальное время
		\item зная $End(E_1)$ и $End(E_2)$ можно вычислить изогению $\phi: E_1 \rightarrow E_2$ за полиномиальное время
	\end{itemize}
\end{frame}

\begin{frame}{Подпись SQISign. $\Sigma$-протокол}
\begin{columns}
	\begin{column}{0.7\textwidth}
		\begin{enumerate}
			\item \UserA~генерирует секретную $\phi_{com}$,\\
			отсылает $E_{com}$
			\item \UserB~генерирует случ. $\phi_{chl}$,\\
			отправляет $\phi_{chl}$
			\item \UserA~вычисляет и отправляет~$\phi_{rsp}$
			\item \UserB~проверяет, что $\phi_{rsp}$  -- изогения $E_{chl} \rightarrow E_{com}$
		\end{enumerate}
	\end{column}
	\begin{column}{0.3\textwidth}
		

\tikzset{every picture/.style={line width=0.75pt}} %set default line width to 0.75pt        

\begin{tikzpicture}[x=0.75pt,y=0.75pt,yscale=-1,xscale=1]
%uncomment if require: \path (0,300); %set diagram left start at 0, and has height of 300


% Text Node
\draw (42.5,26.4) node [anchor=north west][inner sep=0.75pt]    {$E_{0}$};
% Text Node
\draw (120.5,26.4) node [anchor=north west][inner sep=0.75pt]    {$E_{pk}$};
% Text Node
\draw (79,15.4) node [anchor=north west][inner sep=0.75pt]  [font=\scriptsize]  {$\phi _{sk}$};
% Text Node
\draw (119,111.4) node [anchor=north west][inner sep=0.75pt]    {$E_{chl}$};
% Text Node
\draw (142,68.4) node [anchor=north west][inner sep=0.75pt]  [font=\scriptsize]  {$\phi _{chl}$};
% Text Node
\draw (36,111.4) node [anchor=north west][inner sep=0.75pt]    {$E_{com}$};
% Text Node
\draw (23,67.4) node [anchor=north west][inner sep=0.75pt]  [font=\scriptsize]  {$\phi _{com}$};
% Text Node
\draw (88,127.4) node [anchor=north west][inner sep=0.75pt]  [font=\scriptsize]  {$\phi _{rsp}$};
% Connection
\draw [color={rgb, 255:red, 224; green, 62; blue, 73 }  ,draw opacity=1 ] [dash pattern={on 4.5pt off 4.5pt}]  (65.5,35) -- (114.5,35) ;
\draw [shift={(117.5,35)}, rotate = 180] [fill={rgb, 255:red, 224; green, 62; blue, 73 }  ,fill opacity=1 ][line width=0.08]  [draw opacity=0] (8.93,-4.29) -- (0,0) -- (8.93,4.29) -- cycle    ;
% Connection
\draw [color={rgb, 255:red, 74; green, 144; blue, 226 }  ,draw opacity=1 ]   (133,48) -- (133,104) ;
\draw [shift={(133,107)}, rotate = 270] [fill={rgb, 255:red, 74; green, 144; blue, 226 }  ,fill opacity=1 ][line width=0.08]  [draw opacity=0] (8.93,-4.29) -- (0,0) -- (8.93,4.29) -- cycle    ;
% Connection
\draw [color={rgb, 255:red, 224; green, 62; blue, 73 }  ,draw opacity=1 ] [dash pattern={on 4.5pt off 4.5pt}]  (52.5,48) -- (52.5,104) ;
\draw [shift={(52.5,107)}, rotate = 270] [fill={rgb, 255:red, 224; green, 62; blue, 73 }  ,fill opacity=1 ][line width=0.08]  [draw opacity=0] (8.93,-4.29) -- (0,0) -- (8.93,4.29) -- cycle    ;
% Connection
\draw [color={rgb, 255:red, 74; green, 144; blue, 226 }  ,draw opacity=1 ]   (116,120) -- (75,120) ;
\draw [shift={(72,120)}, rotate = 360] [fill={rgb, 255:red, 74; green, 144; blue, 226 }  ,fill opacity=1 ][line width=0.08]  [draw opacity=0] (8.93,-4.29) -- (0,0) -- (8.93,4.29) -- cycle    ;

\end{tikzpicture}

	\end{column}
\end{columns}
\pause
\vspace{1em}
\structure{Корректность:}
\begin{itemize}
	\item \UserA~знает $End(E_0)$ и $\phi_{com}$ \structure{$\implies$} вычисляет $End(E_{com})$
	\item \UserA~знает $End(E_{pk})$ и $\phi_{chl}$ \structure{$\implies$} находит $End(E_{chl})$
	\item \UserA~знает $End(E_{chl}), End(E_{com})$ \structure{$\implies$} вычисляет $\phi_{rsp}$
\end{itemize}
\end{frame}


\begin{frame}{Подпись SQISign. Преобразование Фиата-Шамира}
	\structure{Генерация $chl$}:
		\[
		chl = hash(j(E_{pk}) || j(E_{com}) || m),
		\]$
		\phi_{chl}$ строится детерминировано по $chl$.
		\vspace{1em}
		
		\structure{Подпись:} $s := (E_{com}, \phi_{rsp})$
		\vspace{1em}
		
		\structure{Проверка:} находим $\phi_{chl}$ и проверяем, что\\ $\phi_{rsp}$ -- изогения $E_{chl} \rightarrow E_{com}$.
\end{frame}

%\begin{frame}{Эндоморфизмы и изогении}
%\end{frame}

\begin{frame}{Размеры ключей/скорость}
	\begin{table}[h]
		\begin{center}
			\begin{tabular}{|c|c|c|c|c|}
				\hline
				Схема & \thead{Уровень\\ стойкости} &
				\thead{Открытый\\ключ}
				& \thead{Закрытый\\ключ} & \thead{Подпись} \\
				\hline
				SQISign%https://eprint.iacr.org/2020/1240.pdf
				& & & & \\
				\hline
				SQISignHD %https://ia.cr/2023/436
				& & & & \\
				\hline
				SQISign2D-West %https://eprint.iacr.org/2024/760.pdf
				& & & & \\
				\hline
				SQISign2D-East %https://ia.cr/2024/771
				& & & & \\
				\hline
				SQIPrime %https://ia.cr/2024/773
				& & & & \\
				\hline
			\end{tabular}
		\end{center}
		\caption{Размеры ключей (в байтах) для схем подписи на изогениях.}
		\label{Table1}
	\end{table}
\end{frame}

\begin{frame}{Литература}
%\begin{frame}
	%\nocite{Menezes1993}
	%\nocite{Lenstra1987}
	%\nocite{Blake1999}
	%\nocite{CohenFrey+2005}
	%\nocite{Washington2008}
	%\nocite{GoldwasserKilian1999}
	%\nocite{CohenLenstra1984}
	%\nocite{JaoDeFeo2011}
	%\nocite{Galbraith2012}
	%\nocite{DeFeo2018}
	%\nocite{Costello2019}
	%\nocite{SIKE}
	%\nocite{SafeCurves}
    %\nocite{CastryckDecru2022}
	%\printbibliography
	\begin{scriptsize}
	\begin{itemize}
		%\item[\structure{{\faScroll}}] Castryck W., Decru T. An efficient key recovery attack on SIDH. 2022.
		%\vspace{0.5em}
		\item[\structure{{\faBook}}] Boneh D., Shoup V. ``Graduate Course in Applied Cryptography''
		\url{https://toc.cryptobook.us}
		\vspace{0.5em}

		\item[\structure{{\faGlobe}}] SQIsign: Algorithm specifications and supporting documentation. \url{https://sqisign.org/spec/sqisign-20250707.pdf}
		\vspace{0.5em}
%
%		\item[\structure{{\faYoutube}}]
%		Выступление Castryck на ANTS XV:
%		\url{https://www.youtube.com/watch?v=_eNv7An3Qj0}
	\end{itemize}
	\end{scriptsize}
	

    \begin{center}
        \begin{tcolorbox}[enhanced,hbox,colback=block-green-color-bg,colframe=subsection-color!120,title=Контакты,center title]
            \begin{varwidth}{\textwidth}
                \begin{center}
                    \href{mailto:snovoselov@kantiana.ru}{snovoselov@kantiana.ru}
                \end{center}
            \end{varwidth}
        \end{tcolorbox}
    \end{center}
\end{frame}

%\begin{frame}{TODO}
%	\begin{itemize}
%		\item добавить определения всех обозначений
%	\end{itemize}
%\end{frame}

\end{document}
