% !TeX program = xelatex
% !BIB TS-program = biber

\documentclass{beamer}

\usepackage[utf8]{inputenc}
\usepackage[russian]{babel}

%---tikz----
\usepackage{tikz}
\usetikzlibrary{arrows, chains, matrix, positioning, scopes, patterns, shapes}
\usepackage{pgfplots, subfigure}
\usepackage{extarrows}
\usepackage{tikz-cd}
\usepackage{makecell}
\usepackage{quiver}
\usetikzlibrary{babel}

\usepackage[backend=biber,firstinits=true,hyperref=true,style=numeric-comp]{biblatex}

\usepackage{../beamerthemeec2020}
{\footnotesize\bibliography{../biblio}}

\title{Эллиптические кривые}
\subtitle{Лекция 12. Схема подписи SQISign}
\author{Семён Новосёлов}
\institute{БФУ им. И. Канта}
\date{2025}

\begin{document}

\frame{\titlepage}

\newcommand{\UserA}{{\structure{{\Large\faUserSecret}}}}
\newcommand{\UserB}{{\structure{{\Large\faCat}}}}

\begin{frame}{Введение}
	\structure{SQISign} (Short Quaternion and Isogeny Signature) -- схема подписи, построенная применением преобразования Фиата-Шамира к протоколу доказательства с нулевым разглашением ($\Sigma$-протоколу).
	\vspace{1em}
	
	\begin{scriptsize}
	\begin{itemize}
		\item[\structure{\faScroll}] De Feo L., Kohel D., Leroux A., Petit C., Wesolowski B.\\``SQISign: compact post-quantum signatures from quaternions and isogenies''. AsiaCrypt 2020. \url{https://eprint.iacr.org/2020/1240}
	\end{itemize}
	\end{scriptsize}
	
	\vspace{1em}
	
	Последняя версия схемы:
	\begin{scriptsize}
	\begin{itemize}
		\item[\structure{\faGlobe}] \url{https://sqisign.org}
	\end{itemize}
	\end{scriptsize}
\end{frame}

\begin{frame}{$\Sigma$-протокол}
	содержимое...
\end{frame}

\begin{frame}{Преобразование Фиата-Шамира}
	содержимое...
\end{frame}

\begin{frame}{Пример. Подпись EdDSA}
	содержимое...
\end{frame}

\begin{frame}{Подпись SQISign}
	содержимое...
\end{frame}

\begin{frame}{Размеры ключей/скорость}
	\begin{table}[h]
		\begin{center}
			\begin{tabular}{|c|c|c|c|c|}
				\hline
				Схема & \thead{Уровень\\ стойкости} &
				\thead{Открытый\\ключ}
				& \thead{Закрытый\\ключ} & \thead{Подпись} \\
				\hline
				SQISign%https://eprint.iacr.org/2020/1240.pdf
				& & & & \\
				\hline
				SQISignHD %https://ia.cr/2023/436
				& & & & \\
				\hline
				SQISignHD-West %https://eprint.iacr.org/2024/760.pdf
				& & & & \\
				\hline
				SQISignHD-East %https://ia.cr/2024/771
				& & & & \\
				\hline
				SQIPrime %https://ia.cr/2024/773
				& & & & \\
				\hline
			\end{tabular}
		\end{center}
		\caption{Размеры ключей (в байтах) для схем подписи на изогениях.}
		\label{Table1}
	\end{table}
\end{frame}

\begin{frame}{Литература}
%\begin{frame}
	%\nocite{Menezes1993}
	%\nocite{Lenstra1987}
	%\nocite{Blake1999}
	%\nocite{CohenFrey+2005}
	%\nocite{Washington2008}
	%\nocite{GoldwasserKilian1999}
	%\nocite{CohenLenstra1984}
	%\nocite{JaoDeFeo2011}
	%\nocite{Galbraith2012}
	%\nocite{DeFeo2018}
	%\nocite{Costello2019}
	%\nocite{SIKE}
	%\nocite{SafeCurves}
    %\nocite{CastryckDecru2022}
	%\printbibliography
	\begin{scriptsize}
	\begin{itemize}
		%\item[\structure{{\faScroll}}] Castryck W., Decru T. An efficient key recovery attack on SIDH. 2022.
		%\vspace{0.5em}
		\item[\structure{{\faBook}}] Boneh D., Shoup V. ``Graduate Course in Applied Cryptography''
		\url{https://toc.cryptobook.us}
		\vspace{0.5em}

		\item[\structure{{\faGlobe}}] SQIsign: Algorithm specifications and supporting documentation. \url{https://sqisign.org/spec/sqisign-20250707.pdf}
		\vspace{0.5em}
%
%		\item[\structure{{\faYoutube}}]
%		Выступление Castryck на ANTS XV:
%		\url{https://www.youtube.com/watch?v=_eNv7An3Qj0}
	\end{itemize}
	\end{scriptsize}
	

    \begin{center}
        \begin{tcolorbox}[enhanced,hbox,colback=block-green-color-bg,colframe=subsection-color!120,title=Контакты,center title]
            \begin{varwidth}{\textwidth}
                \begin{center}
                    \href{mailto:snovoselov@kantiana.ru}{snovoselov@kantiana.ru}
                \end{center}
            \end{varwidth}
        \end{tcolorbox}
    \end{center}\end{frame}

\end{document}
