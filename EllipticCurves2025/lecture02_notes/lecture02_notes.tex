\documentclass[11pt]{exam}

%---enable russian----

\usepackage[utf8]{inputenc}
%\usepackage[russian]{babel}
\usepackage[english,main=russian]{babel}



\usepackage[margin=0.73in]{geometry}
%\usepackage[top=1in, bottom=1in, left=1in, right=1in]{geometry}

\usepackage{graphicx}
\usepackage{url}
\usepackage{latexsym}
\usepackage{amscd,amsmath,amsthm}
\usepackage{mathtools}
\usepackage{amsfonts}
\usepackage{amssymb}
\usepackage[dvipsnames]{xcolor}
\usepackage{hyperref}

\usepackage{algorithmicx, enumitem, algpseudocode, algorithm, caption}
\usepackage{tikz}
\usetikzlibrary{arrows,decorations.pathmorphing,backgrounds,calc}
\usetikzlibrary{chains, matrix, positioning, scopes, patterns, shapes}
\usepackage{pgfplots, subfigure}
\usetikzlibrary{automata}


\usepackage[most]{tcolorbox}

\usepackage[backend=biber,firstinits=true,hyperref=true,style=numeric-comp]{biblatex}

\definecolor{text-color}{RGB}{0,0,0} %%% text color
\definecolor{subsection-color}{RGB}{40, 166, 103} %%% subsection color
\definecolor{title-and-section-color}{RGB}{61, 147, 221} %%% title and section color
\definecolor{bg-color}{RGB}{254, 255, 244} %%% background color
\definecolor{struct-color}{RGB}{224, 62, 73} %%% structure

\definecolor{color-cont-bg}{RGB}{204, 239, 171}
\definecolor{block-green-color-bg}{RGB}{204, 239, 171}

\definecolor{plot-red-color}{RGB}{224, 62, 73}
\definecolor{plot-green-color}{RGB}{77, 158, 99}
\definecolor{plot-blue-color}{RGB}{61, 147, 221}

\definecolor{box-red-color}{RGB}{224, 62, 73}
\definecolor{box-green-color}{RGB}{77, 158, 99}
\definecolor{box-blue-color}{RGB}{61, 147, 221}

\bibliography{../biblio}

%%%%%%%%%%%%%%%%%%%%%
% Handling comments and versions %%%
%%%%%%%%%%%%%%%%%%%%%

%\renewcommand{\comment}[1]{\texttt{[#1]}}


%%%%%%%%%%%%%%%%%%%%%%%%%%%
%% THEOREMS
%%%%%%%%%%%%%%%%%%%%%%%%%%%

\newtheorem{theorem}{Theorem}[section]
\newtheorem{axiom}[theorem]{Axiom}
\newtheorem{conclusion}[theorem]{Conclusion}
\newtheorem{condition}[theorem]{Condition}
\newtheorem{conjecture}[theorem]{Conjecture}
\newtheorem{corollary}[theorem]{Corollary}
\newtheorem{criterion}[theorem]{Criterion}
\newtheorem{definition}[theorem]{Definition}
\newtheorem{lemma}[theorem]{Lemma}
\newtheorem{notation}[theorem]{Notation}
\newtheorem{proposition}[theorem]{Proposition}


\theoremstyle{definition}
\newtheorem{problem}{Problem}


\newcommand{\nc}{\newcommand}
\nc{\eps}{\varepsilon}
\nc{\RR}{{{\mathbb R}}}
\nc{\CC}{{{\mathbb C}}}
\nc{\FF}{{{\mathbb F}}}
\nc{\NN}{{{\mathbb N}}}
\nc{\ZZ}{{{\mathbb Z}}}
\nc{\PP}{{{\mathbb P}}}
\nc{\QQ}{{{\mathbb Q}}}
\nc{\UU}{{{\mathbb U}}}
\nc{\OO}{{{\mathbb O}}}
\nc{\EE}{{{\mathbb E}}}

\newcommand{\val}{\operatorname{val}}
\newcommand{\wt}{\ensuremath{\mathit{wt}}}
\newcommand{\Id}{\ensuremath{I}}
\newcommand{\transpose}{\mkern0.7mu^{\mathsf{ t}}}
\newcommand*{\ScProd}[2]{\ensuremath{\langle#1\mathbin{,}#2\rangle}} %Scalar Product
\renewcommand{\char}{\ensuremath{\mathsf{char}}}

\DeclareMathOperator{\Vol}{Vol}

%\pretolerance=1000

%%%%%%%%%%%%%%%%%%%%%%%%%%%%%%%%
%%%%%%%%%%%%%%%%%%%%%%%%%%%%%%%%
%% DOCUMENT STARTS
%%%%%%%%%%%%%%%%%%%%%%%%%%%%%%%%
%%%%%%%%%%%%%%%%%%%%%%%%%%%%%%%%


\begin{document}	
	{\noindent
		\textsc{БФУ им. И. Канта -- Компьютерный практикум по криптографии на эллиптических кривых }\\[5pt]
		Преподаватель {С. Новоселов}   \hfill{Осень 2025\\}
	\hrule
	\begin{center}
		{\LARGE\textbf{
				Лекция 2. Групповой закон \\[5pt]
		}}
		
	\end{center}
	\hrule \vspace{5mm}
	
	\thispagestyle{empty}
	
	\section{Определение}
	
	Пусть~$E/K : y^2 = x^3 + Ax + B$ -- эллиптическая кривая над полем~$K$. На множестве точек эллиптической кривой (плюс бесконечно удалённая точка) можно задать групповую операцию. Для кривых над полем действительных чисел~$\mathbb{R}$ это можно сделать следующим образом (см. рисунок~\ref{fig:group:law}). Пусть~$P_1 = (x_1, y_1)$ и~$P_2 = (x_2, y_2)$ -- точки эллиптической кривой.
	\begin{figure}[h!]
		\caption{Групповой закон над~$\mathbb{R}$}
		\label{fig:group:law}
		\centering
		\begin{tikzpicture}[scale=0.85]
			\begin{axis}[
				xmin=-4,
				xmax=5,
				xtick=\empty,
				ytick=\empty,
				ymin=-5,
				ymax=5,
				xlabel={$x$},
				ylabel={$y$},
				scale only axis,
				axis lines=middle,
				style={thick},
				domain=-2.279018:3,      
				samples=201,
				smooth,   
				clip=false,
				axis equal image=true,
				]
				\addplot[color=plot-blue-color] {sqrt(x^3-3*x+5)} node[right] {$E$};
				\addplot[color=plot-blue-color] {-sqrt(x^3-3*x+5)};
				\addplot[color=plot-red-color] coordinates {(-3, -0.0890722)(3, 3.06557)} node[right] {$\ell$};
				\addplot[color=plot-green-color] coordinates {(1.9, 3.5)(1.9, -3.5)};
				\draw [fill=black] (axis cs: 0.65, 1.83) circle (2pt);
				\draw[color=black] (axis cs: 1.2, 1.3) node [left]{$P_2$};
				\draw [fill=black] (axis cs: -2.26, 0.3) circle (2pt);
				\draw[color=black] (axis cs: -2.3, 0.3) node [left]{$P_1$};
				\draw [fill=black] (axis cs: 1.9, 2.5) circle (2pt);
				\draw[color=black] (axis cs: 1.9, 2.8) node [left]{$P_3'$};
				\draw [fill=black] (axis cs: 1.9, -2.5) circle (2pt);
				\draw[color=black] (axis cs: 1.9, -2.8) node [left]{$P_3$};
			\end{axis}
		\end{tikzpicture}
	\end{figure}
	Проведём прямую через данные точки, возьмём точку пересечения прямой с эллиптической кривой и отразим её относительно оси $x$, получив точку~$P_3 = \left(x_3, y_3\right)$. Решая систему уравнений, состоящую из уравнения кривой и уравнения прямой через две точки, получаем
	\begin{equation*}
		\begin{split}
			x_3 &= m^2 - x_1 - x_2, \\
			y_3 &= m\left( x_1 - x_3 \right) - y_1, \\
			m &= \frac{y_2 - y_1}{x_2 - x_1}.
		\end{split}
	\end{equation*}
	Может получиться так, что точки пересечения нет, как на рисунке~\ref{fig:group:law:inf}. В этом случае полагаем~$P_3 = \mathcal{O}$.
	\begin{figure}[h!]
		\centering
		\caption{Групповой закон над $\mathbb{R}$. Случай~$x_1 = x_2$.}
		\label{fig:group:law:inf}
		\begin{tikzpicture}[scale=0.85]
			\begin{axis}[
				xmin=-4,
				xmax=5,
				xtick=\empty,
				ytick=\empty,
				ymin=-5,
				ymax=5,
				xlabel={$x$},
				ylabel={$y$},
				scale only axis,
				axis lines=middle,
				style={thick},
				domain=-2.279018:3,      
				samples=201,
				smooth,   
				clip=false,
				axis equal image=true,
				]
				\addplot[color=plot-blue-color] {sqrt(x^3-3*x+5)} node[right] {$E$};
				\addplot[color=plot-blue-color] {-sqrt(x^3-3*x+5)};
				\addplot[color=plot-red-color] coordinates {(-1.0, 3.65)(-1.0, -3.65)};
				\draw [fill=black] (axis cs: -1.0, 2.65) circle (2pt);
				\draw[color=black] (axis cs: -1.0, 3.0) node [left]{$P_1$};
				\draw [fill=black] (axis cs: -1.0, -2.65) circle (2pt);
				\draw[color=black] (axis cs: -1.0, -3.0) node [left]{$P_2$};
			\end{axis}
		\end{tikzpicture}
	\end{figure}

	В случае, если $P_1 = P_2$, то вместо прямой, проходящей через две точки, берётся касательная к кривой в точке~$P_1$. В этом случае координаты точки $P_3$ получаются следующими:
	\begin{equation*}
		\begin{split}
			x_3 &= m^2 - 2 x_1, \\
			y_3 &= m\left( x_1 - x_3 \right) - y_1,\\
			m   &= \frac{3x_1^2 + A}{2 y_1}. \\ 
		\end{split}
	\end{equation*}
	\begin{figure}[h!]
			\caption{Групповой закон над $\mathbb{R}$. Удвоение точек.}
			\label{fig:group:law:dbl}
		\centering
		\begin{tikzpicture}[scale=0.85]
			\begin{axis}[
				xmin=-4,
				xmax=5,
				xtick=\empty,
				ytick=\empty,
				ymin=-5,
				ymax=5,
				xlabel={$x$},
				ylabel={$y$},
				scale only axis,
				axis lines=middle,
				style={thick},
				domain=-2.279018:3,      
				samples=201,
				smooth,   
				clip=false,
				axis equal image=true,
				]
				\addplot[color=plot-blue-color] {sqrt(x^3-3*x+5)} node[right] {$E$};
				\addplot[color=plot-blue-color] {-sqrt(x^3-3*x+5)};
				\addplot[color=plot-red-color] {2.65 + ((-3.0) + 3*(-1.2)^2)/(2*2.65) * (x - (-1.2))}  node[right] {$\ell$};
				\addplot[color=plot-green-color] coordinates {(2.5, 4.5)(2.5, -4.5)};
				\draw [fill=black] (axis cs: -1.2, 2.65) circle (2pt);
				\draw[color=black] (axis cs: -1.0, 3.0) node [left]{$P_1$};
				\draw [fill=black] (axis cs: 2.5, 3.6) circle (2pt);
				\draw[color=black] (axis cs: 2.5, 3.9) node [left]{$P_3'$};
				\draw [fill=black] (axis cs: 2.5, -3.6) circle (2pt);
				\draw[color=black] (axis cs: 2.5, -3.9) node [left]{$P_3$};
			\end{axis}
		\end{tikzpicture}
	\end{figure}
	
	Можно показать, что заданный таким способом групповой закон удовлетворяет всем аксиомам аддитивной абелевой группы.
	\begin{tcolorbox}[colframe=title-and-section-color!120, colback=title-and-section-color!5, title=Теорема, center title]
		%Операция сложения на $E$ обладает свойствами:
		\begin{enumerate}
			\item $P_1 + P_2 = P_2 + P_1$ \hfill \textit{(коммутативность)}
			
			\item $P + \mathcal{O} = P$ $\forall P \in E$ \hfill \textit{($\exists$ нейтральный элемент)}
			
			\item $\forall P \in E$ $\exists P' \in E: P + P' = \mathcal{O}$ \hfill \textit{($\exists$ обратный элемент)}
			
			\item $\left( P_1 + P_2 \right) + P_3 = P_1 + \left( P_2 + P_3 \right)$ \hfill \textit{(ассоциативность)}
		\end{enumerate}
	\end{tcolorbox}
	Более того, поскольку в формулах используются только операции сложения, умножения и деления, то они работают и для любого поля. А не только для~$\mathbb{R}$. Для криптографических приложений наиболее интересен случай~$K = \mathbb{F}_q$.
	
	%\section{Быстрое скалярное умножение}
	
	%\section{Протокол Диффи-Хэллмана}
	
	%\section{Цифровая подпись}

\nocite{Blake1999}
\nocite{Washington2008}
\nocite{Menezes1993}
\nocite{HankersonMenezesVanstone2006}
\nocite{Silverman2009}
\printbibliography

\end{document}
