% !TeX program = xelatex
% !BIB TS-program = biber

\documentclass{beamer}

\usepackage[utf8]{inputenc}
\usepackage[russian]{babel}

%---tikz----
\usepackage{tikz}
\usetikzlibrary{arrows, chains, matrix, positioning, scopes, patterns, shapes}
\usepackage{pgfplots, subfigure}
\usepackage{extarrows}
\usepackage{tikz-cd}
\usepackage{makecell}
\usepackage{quiver}
\usetikzlibrary{babel}

\usepackage[backend=biber,firstinits=true,hyperref=true,style=numeric-comp]{biblatex}

\usepackage{../beamerthemeec2020}
\usepackage{mathtools}

{\footnotesize\bibliography{../biblio}}

\title{Эллиптические кривые}
\subtitle{Лекция 13. Криптоанализ схем на изогениях}
\author{Семён Новосёлов}
\institute{БФУ им. И. Канта}
\date{2025}

\begin{document}

\frame{\titlepage}

\begin{frame}{Граф изогений}
	Граф с вершинами -- $j$-инвариантами эллиптических кривых и рёбрами -- изогениями.
	\vspace{1em}
	
	\structure{Обозначение:} $X(K, \ell)$, где $K$ --  поле, $\ell$ -- степень изогении.
\end{frame}

\begin{frame}{Задача поиска изогении}
	\begin{center}
	\begin{tcolorbox}[enhanced,hbox,colback=title-and-section-color!5,colframe=title-and-section-color!120,title=Общая задача нахождения изогении,center title]
		\begin{varwidth}{\textwidth}
			\begin{center}
				Даны две изогенные кривые~$E_1$ и $E_2$.
				
				Вычислить изогению между ними.
			\end{center}
		\end{varwidth}
	\end{tcolorbox}	
\end{center}
	
	\vspace{1em}
	Теорема Тейта: $E_1 \sim E_2$ \structure{$\iff$} $\#E_1 = \#E_2$
	
	\structure{$\implies$} легко проверить существование изогении.
\end{frame}


\begin{frame}{Методы поиска изогений}
	\begin{itemize}
		\item на основе парадокса дней рождений
		\item сведение задачи к вычислению $End(E)$
	\end{itemize}
\end{frame}

\begin{frame}{Методы на основе парадокса дней рождений}
\begin{itemize}
	\item модификация алгоритмов BSGS/Полларда/vOW
	\item исп. случайные блуждания в графе изогений
\end{itemize}
\vspace{1em}

\structure{Сложность} для суперсингулярных кривых:
\begin{itemize}
	\item $\widetilde{\mathcal{O}}(p^{1/2})$ по времени/памяти в худшем случае или
	\item $\widetilde{\mathcal{O}}(p^{3/4})$ по времени в среднем для алгоритма с малой памятью (vOW)
\end{itemize}
\vspace{1em}
Число изогенных суперсингулярных кривых~$\approx \frac{p}{12}$.
\end{frame}

\begin{frame}{Как задавать случайные блуждания?}
	\begin{itemize}
		\item через ядра изогений -- подгруппы $E[\ell]$
		\item через модулярные многочлены
	\end{itemize}
	\vspace{1em}

	\structure{Модулярный многочлен} -- это многочлен $\Phi_\ell \in \mathbb{Z}[X,Y]$ т.ч. $\Phi_\ell(j(E_1), j(E_2)) = 0$ \structure{$\iff$} $\exists$ изогения степени $\ell$ между $E_1$ и $E_2$.
	\vspace{0.5em}
	
	\structure{$\implies$} находясь в вершине~$j(E)$, находим корни $\Phi_\ell(j(E), Y)$ и берём случайный.
\end{frame}

\begin{frame}{Суперсингулярная задача поиска изогении}
	Эллиптическая кривая $E/\mathbb{F}_{q}$, $q=p^n$ -- \structure{суперсингулярная}, если $p \mid t = \#E(\mathbb{F}_q) - q- 1$.
	
	\vspace{1em}

	\structure{Факт:} $j(E) \in \mathbb{F}_{p^2}$ \hspace{8em} \textcolor{gray}{[Silverman, Th. 3.1.(a).iii]}
	
	\vspace{1em}
	\structure{$\implies$} суперсингулярный граф изогений определён над $\mathbb{F}_{p^2}$
	
	\vspace{1em}
	Также над $\mathbb{F}_{p^2}$ он связен для любой степени изогении $\ell$.
\end{frame}

\begin{frame}{Алгоритм Delfs-Galbraith}
	Пусть $E_1$, $E_2$ определены над $\mathbb{F}_{p^2}$.
	\vspace{1em}
	
	\structure{Идея:}
	\begin{itemize}
		\item найти изогении $\phi_1: E_1 \rightarrow E_1'/\mathbb{F}_p$ и $\phi_2: E_2 \rightarrow E_2'/\mathbb{F}_p$,
		\item найти изогению $\phi': E_1' \rightarrow E_2'$ в графе изогений над $\mathbb{F}_p$
		\item вернуть $\phi: E_1 \rightarrow E_2$ как $\phi = \widehat{\phi}_2 \circ \phi' \circ \phi_1$
	\end{itemize}

	%\vspace{1em}
	%todo: добавить график
\end{frame}

\begin{frame}
	Граф изогений $X(\mathbb{F}_p, \ell)$ меньше, чем $X(\mathbb{F}_{p^2}, \ell)$ -- состоит из $\widetilde{\mathcal{O}}(\sqrt{p})$ вершин, но он не связный и надо брать несколько $\ell$.
	\\
	\structure{$\implies$} сложность поиска $\phi'$ равна $\widetilde{\mathcal{O}}(p^\frac{1}{4})$
	
	\vspace{1em}
	Общая \structure{сложность} алгоритма Delfs-Galbraith: $\widetilde{\mathcal{O}}(p^{1/2})$, доминирующие шаги -- нахождение $\phi_1, \phi_2$.
\end{frame}

%\begin{frame}{Модификации алгоритма Delfs-Galbraith}
%	https://eprint.iacr.org/2021/1488
%	посмотреть ещё цитирования, улучшения в константе только
%\end{frame}


%\begin{frame}{Предварительные сведения: Дуальная изогения}
%	содержимое...
%\end{frame}

\begin{frame}{Сведение задачи к вычислению $End(E)$}
	\begin{itemize}
		\item Задачи поиска изогении и вычисления $End(E)$ эквивалентны \textcolor{gray}{[Wesolowski'21]} (одна за полиномиальное время сводится к другой).
		\vspace{0.5em}
		
		\item Один эндоморфизм \structure{$\implies$} всё кольцо за полиномиальное время~\textcolor{gray}{[Page-Wesolowski'23]}
		\vspace{0.5em}
		
		\item Детектирование эндоморфизмов малых степеней возможно с помощью нахождения классового многочлена Гильберта\\
		\structure{$\implies$} алгоритм нахождения изогений~\textcolor{gray}{[Love-Boneh'19]}
	\end{itemize}
\end{frame}

\begin{frame}{Квантовые атаки -- сложность}
	\begin{itemize}
		\item суперсингулярные кривые: $\widetilde{\mathcal{O}}(p^{1/2})$ \structure{$\implies$} $\widetilde{\mathcal{O}}(p^{1/4})$
		\vspace{0.5em}
		\item обычных кривые: $L(1/2)$
	\end{itemize}
	%todo: описать принцип
\end{frame}

%\begin{frame}{Выводы}
%	\begin{itemize}
%		\item обычные кривые: сложность классическая/квантовая
%		\item суперсингулярные кривые: сложность классическая/квантовая 
%	\end{itemize}
%\end{frame}

\begin{frame}{Литература}
%\begin{frame}
	%\nocite{Menezes1993}
	%\nocite{Lenstra1987}
	%\nocite{Blake1999}
	%\nocite{CohenFrey+2005}
	%\nocite{Washington2008}
	%\nocite{GoldwasserKilian1999}
	%\nocite{CohenLenstra1984}
	%\nocite{JaoDeFeo2011}
	%\nocite{Galbraith2012}
	%\nocite{DeFeo2018}
	%\nocite{Costello2019}
	%\nocite{SIKE}
	%\nocite{SafeCurves}
    %\nocite{CastryckDecru2022}
	%\printbibliography
	\begin{scriptsize}
	\begin{itemize}
		%\item[\structure{{\faScroll}}] Castryck W., Decru T. An efficient key recovery attack on SIDH. 2022.
		%\vspace{0.5em}
		\item[\structure{{\faBook}}] Silverman J.H. ``The Arithmetic of Elliptic Curves'', 2ed (2009)
		\vspace{0.5em}

		\item[\structure{{\faScroll}}] Delfs C., Galbraith S.D. ``Computing isogenies between supersingular elliptic curves over~$\mathbb{F}_p$''. 2016. DCC. \url{https://arxiv.org/pdf/1310.7789}
		\vspace{0.5em}
		
		\item[\structure{{\faScroll}}] Wesolowski B. ``The supersingular isogeny path and endomorphism ring problems are equivalent'' (2021)
		\url{https://ieeexplore.ieee.org/document/9719728}
		\vspace{0.5em}
		
		\item[\structure{{\faScroll}}] Page A., Wesolowski B. ``The supersingular Endomorphism Ring and One Endomorphism problems are equivalent'' (2023)
		\url{https://eprint.iacr.org/2023/1399}
		
		\item[\structure{\faScroll}] Love J., Boneh D. - Supersingular Curves With Small Non-integer Endomorphisms (2019)
		\url{https://arxiv.org/pdf/1910.03180}

		%\item[\structure{{\faGlobe}}] SQIsign: Algorithm specifications and supporting documentation. \url{https://sqisign.org/spec/sqisign-20250707.pdf}
		%\vspace{0.5em}
%
%		\item[\structure{{\faYoutube}}]
%		Выступление Castryck на ANTS XV:
%		\url{https://www.youtube.com/watch?v=_eNv7An3Qj0}
	\end{itemize}
	\end{scriptsize}
	

    \begin{center}
        \begin{tcolorbox}[enhanced,hbox,colback=block-green-color-bg,colframe=subsection-color!120,title=Контакты,center title]
            \begin{varwidth}{\textwidth}
                \begin{center}
                    \href{mailto:snovoselov@kantiana.ru}{snovoselov@kantiana.ru}
                \end{center}
            \end{varwidth}
        \end{tcolorbox}
    \end{center}
\end{frame}

\end{document}
