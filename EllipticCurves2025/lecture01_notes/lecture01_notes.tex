\documentclass[11pt]{exam}

%---enable russian----

\usepackage[utf8]{inputenc}
%\usepackage[russian]{babel}
\usepackage[english,main=russian]{babel}



\usepackage[margin=0.73in]{geometry}
%\usepackage[top=1in, bottom=1in, left=1in, right=1in]{geometry}

\usepackage{graphicx}
\usepackage{url}
\usepackage{latexsym}
\usepackage{amscd,amsmath,amsthm}
\usepackage{mathtools}
\usepackage{amsfonts}
\usepackage{amssymb}
\usepackage[dvipsnames]{xcolor}
\usepackage{hyperref}

\usepackage{algorithmicx, enumitem, algpseudocode, algorithm, caption}
\usepackage{tikz}
\usetikzlibrary{arrows,decorations.pathmorphing,backgrounds,calc}
\usetikzlibrary{chains, matrix, positioning, scopes, patterns, shapes}
\usepackage{pgfplots, subfigure}
\usetikzlibrary{automata}

\usepackage[backend=biber,firstinits=true,hyperref=true,style=numeric-comp]{biblatex}

\bibliography{../biblio}

%%%%%%%%%%%%%%%%%%%%%
% Handling comments and versions %%%
%%%%%%%%%%%%%%%%%%%%%

%\renewcommand{\comment}[1]{\texttt{[#1]}}


%%%%%%%%%%%%%%%%%%%%%%%%%%%
%% THEOREMS
%%%%%%%%%%%%%%%%%%%%%%%%%%%

\newtheorem{theorem}{Theorem}[section]
\newtheorem{axiom}[theorem]{Axiom}
\newtheorem{conclusion}[theorem]{Conclusion}
\newtheorem{condition}[theorem]{Condition}
\newtheorem{conjecture}[theorem]{Conjecture}
\newtheorem{corollary}[theorem]{Corollary}
\newtheorem{criterion}[theorem]{Criterion}
\newtheorem{definition}[theorem]{Definition}
\newtheorem{lemma}[theorem]{Lemma}
\newtheorem{notation}[theorem]{Notation}
\newtheorem{proposition}[theorem]{Proposition}


\theoremstyle{definition}
\newtheorem{problem}{Problem}


\newcommand{\nc}{\newcommand}
\nc{\eps}{\varepsilon}
\nc{\RR}{{{\mathbb R}}}
\nc{\CC}{{{\mathbb C}}}
\nc{\FF}{{{\mathbb F}}}
\nc{\NN}{{{\mathbb N}}}
\nc{\ZZ}{{{\mathbb Z}}}
\nc{\PP}{{{\mathbb P}}}
\nc{\QQ}{{{\mathbb Q}}}
\nc{\UU}{{{\mathbb U}}}
\nc{\OO}{{{\mathbb O}}}
\nc{\EE}{{{\mathbb E}}}

\newcommand{\val}{\operatorname{val}}
\newcommand{\wt}{\ensuremath{\mathit{wt}}}
\newcommand{\Id}{\ensuremath{I}}
\newcommand{\transpose}{\mkern0.7mu^{\mathsf{ t}}}
\newcommand*{\ScProd}[2]{\ensuremath{\langle#1\mathbin{,}#2\rangle}} %Scalar Product
\renewcommand{\char}{\ensuremath{\mathsf{char}}}

\DeclareMathOperator{\Vol}{Vol}

%\pretolerance=1000

%%%%%%%%%%%%%%%%%%%%%%%%%%%%%%%%
%%%%%%%%%%%%%%%%%%%%%%%%%%%%%%%%
%% DOCUMENT STARTS
%%%%%%%%%%%%%%%%%%%%%%%%%%%%%%%%
%%%%%%%%%%%%%%%%%%%%%%%%%%%%%%%%


\begin{document}	
	{\noindent
		\textsc{БФУ им. И. Канта -- Компьютерный практикум по криптографии на эллиптических кривых }\\[5pt]
		Преподаватель {С. Новоселов}   \hfill{Осень 2025\\}
	\hrule
	\begin{center}
		{\LARGE\textbf{
				Лекция 1. Введение \\[5pt]
		}}
		
	\end{center}
	\hrule \vspace{5mm}
	
	\thispagestyle{empty}
	
	Эллиптические кривые используются в криптографии как основа для построения множества криптографических примитивов, схем и протоколов, таких как цифровых подписи и схемы обмена ключами.
	
	В настоящее время в криптографии выделяется два направления: классическая криптография и посквантовая. К классической криптографии относятся криптографические схемы, построенные на основе сложности задач дискретного логарифмирования, факторизации целых чисел и других. Однако, данные задачи могут быть взломаны на квантовом компьютере за полиномиальное время. В связи с этим появилось новое направление криптографических исследований -- постквантовая криптография требует от схем стойкость к атакам на квантовом компьютере.
	
	Среди распространённых классических схем на эллиптических кривых имеются схема подписи (ECDSA) и обмен ключами по протоколу Диффи-Хэлмана на эллиптических кривых (ECDH).
	
	Из перспективных постквантновых схем на эллиптических кривых можно отметить схемы, построенные на сложности задачи вычисления изогении между эллиптическими кривыми --  цифровую подпись SQISign и схему обмена ключами CSIDH.
	
	Классическая криптография в настоящее время широко используется на практике в большом числе приложений.
	
	\begin{enumerate}
		\item Обмен ключами и цифровые подписи в протоколе HTTPS (TLS), который используется для шифрования данных при доступе к ресурсам в интернете (см. Рис.~\ref{fig:tls:example}).
		\item В составе протоколов для построения VPN: обмен ключами в протоколе WireGuard, встроенном в ядро Linux, осуществляется на базе эллиптической кривой Curve25519.
		\item Кривые Эдвардса Ed25519 можно использовать для цифровой подписи в протоколе SSH.
		\item Bitcoin/Ethereum используют кривую Secp256k1 для цифровой
		подписи транзакций.
	\end{enumerate}

	\begin{figure}
		\caption{Пример. Wikipedia использует сертификат на основе ECDSA}
		\label{fig:tls:example}
		\includegraphics[scale=0.5]{../images/example_tls}
	\end{figure}
	
	Хотя на практике современные квантовые компьютеры в настоящее время не могут взломать классические криптосистемы на эллиптических кривых, предполагается, что такие компьютеры могут быть созданы в течении 10 лет.
	
	Полный отказ от классической криптографии (ECDSA,
	ECDH, FFDH, RSA) рекомендован NIST начиная с 2035 года. А с 2030 года классические схемы объявлены устаревшими. Это значит, что начиная с этих дат будет прекращена поддержка классической криптографии в популярных протоколах сети Интернет (TLS, https, ssh).
	
	В качестве замены предлагаются различные схемы постквантовой криптографии. В эллиптической криптографии, это SQISign вместо ECDSA, CSIDH вместо ECDH. Однако, в настоящее время схемы на изогениях не утверждены в качестве стандартов.
	
\section{План курса}
\begin{enumerate}
	\item Введение
	\item Групповой закон на эллиптической кривой
	\item Точки $n$-кручения
	\item Алгоритмы подсчета $\mathbb{F}_q$-рациональных точек кривой
	\item Алгоритм факторизации на эллиптических кривых
	\item Тест на простоту Goldwasser-Kilian
	\item Выбор эллиптической кривой для криптографии
	\item Криптографические схемы на изогениях
\end{enumerate}

Заметим, что и классическая криптография и постквантовая криптография имеют одну и ту же теоретическую базу. И всё, что используется в классической криптографии на эллиптических кривых, используется и в посткватновой. Поэтому мы начинаем с классической криптографии и потом переходим к постквантовой.

\section{Порядок работы}
Формат: лекции + сдача лабораторных работ.
\vspace{0.5em}
\begin{itemize}
	\item после лекций выкладываются лабораторные работы
	\item срок сдачи: {2} недели
	\item за пределами срока сдачи работы принимаются с низким приоритетом и в количестве не более {2} за пару
\end{itemize}
\vspace{0.5em}
{Экзамен}: сдать {85\%} лабораторных и {курсовую}.
\begin{itemize}
	\item оценка -- среднее по оценкам за сдачу лабораторных
\end{itemize}

\section{Основные определения}
	Уравнение Вейерштрасса \textbf{в аффинных координатах}:
\begin{equation}\label{eq:weierstrassequation}
	f: y^2+a_1xy + a_3y = x^3 + a_2x^2 + a_4x + a_6
\end{equation}

Уравнение над полем~$K$ -- \textbf{гладкое}, если во множестве его решений над~$\overline{K}$ нет сингулярных точек.
\vspace{1em}

\textbf{Эллиптическая кривая} задаётся как \[E(K) = \{ (x,y) \in K \times K: f(x,y)=0 \} \cup \{\mathcal{O}\}
\]
для гладкого~$f$.
\begin{itemize}
	\item $\mathcal{O}$ -- точка в бесконечности.
	\item $K$ -- некоторое поле, чаще всего $K = \mathbb{F}_q$.
\end{itemize}

\textbf{Уравнение Вейерштрасса} в проективных координатах: 
\begin{equation}
	\label{eq:weierstrass}
	F: Y^2Z + a_1 X Y Z + a_3 Y Z^2 = X^3 + a_2 X^2 Z + a_4 X Z^2 + a_6 Z^3,
\end{equation}
где $a_i \in K$.
\begin{itemize}
	%\item \structure{гладкое} (или несингулярное): $\forall P \in \mathbb{P}^2(K)$, $\frac{dF}{dX}(P) \neq 0$, $\frac{dF}{dY}(P) \neq 0$ или $\frac{dF}{dZ}(P) \neq 0$.
	\item \textbf{сингулярное}: $\exists P \in \mathbb{P}^2(K): \frac{dF}{dX}(P) = \frac{dF}{dY}(P) = \frac{dF}{dZ}(P) = 0$.
	\item  \textbf{гладкое} (или несингулярное) в противном случае.
\end{itemize}
%\footnotetext[1]{проективная плоскость над $K$ — множество классов эквивалентности на $K^3\setminus \{0,0,0\}$, т.е. $\overrightarrow{X} \sim \overrightarrow{Y}$, если $x_1=u*y_1,x_2=u*y_2,x_3=u*y_3$ }
%\structure{Эллиптическая кривая}:

\vspace{1em}

\textbf{Эллиптическая кривая $E$} -- множество точек $\mathbb{P}^2(K)$, удовлетворяющих гладкой кривой \eqref{eq:weierstrass}.

\begin{itemize}
	\item Точка в бесконечности: $\exists! \mathcal{O} = (0: 1: 0)$.
\end{itemize}

\begin{itemize}
	\item Проективные координаты позволяют избежать деления в арифметике за счёт доп. умножений.
\end{itemize}

\subsection{Дискриминант}
Как проверить, что уравнение Вейерштрасса задаёт эллиптическую кривую? Для решения задачи используется дискриминант кривой. Обозначим
\begin{equation}
	\begin{split}
		d_2 &= a_1^2 + 4a_2 \\
		d_4 &= 2a_4 + a_1a_3 \\
		d_6 &= a_3^2 + 4a_6 \\
		d_8 &= a_1^2a_6 + 4a_2a_6 - a_1a_3a_4 + a_2a_3^2 - a_4^2 \\
		c_4 &= d_2^2 - 24d_4 \\
		%\text{Для проверки: } 4d_8 &= d_2d_6 - d_4^2
	\end{split}
\end{equation}
Тогда \textbf{дискриминант} уравнения \eqref{eq:weierstrassequation} определяется как 
\[
\Delta = -d_2^2d_8 - 8d_4^3-27d_6^2+9d_2d_4d_6.
\]

Данные формулы можно получить применив общую теорию дискриминантов~\cite[Ch.~13]{GelfandKapranovZelevinsky1994} к уравнению Вейерштрасса. 

\begin{theorem}[\cite{Silverman2009}, Prop.~1.4]
~
\begin{enumerate}
	\item $\Delta \neq 0 \iff$ кривая гладкая ($\implies$ задаёт эллиптическую кривую) 
	\item $\Delta = 0, c_4 \neq 0 \iff$ кривая обладает узлом (node) 
	\item $\Delta = c_4 = 0 \iff$ кривая обладает точкой перегиба (cusp)
\end{enumerate}
\end{theorem}

	\begin{figure}[h!]
		\caption{Случай $\Delta < 0$}
		\begin{tikzpicture}
			\begin{axis}[
				xmin=-5,
				xmax=5,
				ymin=-7,
				ymax=7,
				xlabel={$x$},
				ylabel={$y$},
				scale only axis,
				axis lines=middle,
				domain=-2.1038:3,
				samples=200,
				smooth,
				clip=false,
				axis equal image=true,
				]
				\addplot [red] {sqrt(x^3-3*x+3)}
				node[right] {$y^2=x^3-3x+3$};
				\addplot [red] {-sqrt(x^3-3*x+3)};
			\end{axis}
		\end{tikzpicture} 
	\end{figure}

	\begin{figure}[h!]
			\caption{Случай $\Delta > 0$}
			\centering
			\begin{tikzpicture}
				\begin{axis}[
					xmin=-1.5,
					xmax=1.5,
					ymin=-2,
					ymax=2,
					xlabel={$x$},
					ylabel={$y$},
					scale only axis,
					axis lines=middle,
					domain=-1:0, 
					samples=200,
					smooth,
					clip=false,
					axis equal image=true,
					]
					\addplot [red] {sqrt(x^3-x)};
					\addplot [red] {-sqrt(x^3-x)};
				\end{axis}
				\begin{axis}[
					xmin=-1.5,
					xmax=1.5,
					ymin=-2,
					ymax=2,
					xlabel={$x$},
					ylabel={$y$},
					scale only axis,
					axis lines=middle,
					domain=0.1:1.5, 
					samples=200,
					smooth,
					clip=false,
					axis equal image=true,
					]
					\addplot [red] {sqrt(x^3-x)}
					node[right] {$y^2=x^3-x$};
					\addplot [red] {-sqrt(x^3-x)};
				\end{axis}
			\end{tikzpicture}
		\end{figure}

\subsection{Изоморфизмы эллиптических кривых}
Пусть $E_1/K, E_2/K$ -- эллиптические кривые с уравнениями:
\begin{equation}
	\begin{split}
		E_1&: y^2+a_1xy + a_3y = x^3 + a_2x^2 + a_4x + a_6 \\
		E_2&: y^2+a_1'xy + a_3'y = x^3 + a_2'x^2 + a_4'x + a_6'
	\end{split}
\end{equation}

$E_1/K, E_2/K$ \textbf{изоморфны}, если они изоморфны как проективные многообразия, т.е. $\exists$ морфизмы $\phi: E_1/K \to E_2/K, \psi: E_2/K \to E_1/K$ (определённые над $K$), такие что $\psi \circ \phi = id_{E_1}, \phi \circ \psi = id_{E_2}$.

%\begin{tcolorbox}[colframe=title-and-section-color!120, colback=title-and-section-color!5, title=Теорема, center title]
\begin{theorem}
	$E_1 \simeq E_2 \iff \exists u,r,s,t \in K, u \neq 0$ такие, что замена
\begin{equation}
	\label{eq:isom}
	(x,y) \mapsto (u^2x+r, u^3y+ u^2sx+t)
\end{equation}
преобразует кривую $E_1$ в $E_2$.
\end{theorem}
%\end{tcolorbox}

Изоморфизм кривых задаёт отношение эквивалентности.

Зачем нужны изоморфизмы на практике? Они позволяют подбирать форму кривой под нужные свойства в арифметике. Например:
\begin{itemize}
	\item с меньшим количеством коэффициентов: ускорение вычислений;
	\item с константным временем выполнения группового закона: противодействие атакам по побочным каналам.
\end{itemize}

\subsection{Краткие формы}
    \begin{equation*}
	E/K: y^2 + a_1 xy + a_3 y = x^3 + a_2 x^2 + a_4 x + a_6 \tag{\ref{eq:weierstrassequation}}
\end{equation*}
\textbf{$\operatorname{char} K \neq 2$:}
Изоморфизм \[(x, y)\mapsto \left(x, \frac{1}{2}(y-a_1x-a_3)\right)\] преобразует $E/K$ к виду:
\begin{equation}
	\label{eq:char_neq2}
	E/K: y^2 = 4x^3 + d_2x^2 + 2d_4 + d_6.
\end{equation}

\textbf{$\operatorname{char} K \neq 2, 3$:}
Изоморфизм
\[
(x, y) \mapsto \left(\frac{x-3d_2}{36}, \frac{y}{216}\right)
\]
Преобразует~\eqref{eq:char_neq2} к виду:
\begin{align}
	E/K: y^2 = x^3 + ax + b
\end{align}
\[
a = -27 c_4
\]
\[
b = -56(d_2^3 + 36 d_2 d_4 - 216 d_6) 
\]

В последнем случае, 
\begin{align*}
	\Delta &= -16(4a^3 + 27b^2) \\ \nonumber
\end{align*}
\textbf{$char K = 2$:} 
\[
a_1 \neq 0 \implies (x, y) \mapsto \left(a_1^2x+\frac{a_3}{a_1}; \, a_1^3y + \frac{a_1^2a_4+a_3^2}{a_1^3}\right)
\]
\begin{equation}
	E/K: y^2+xy=x^3+a_2'x^2+a_6'
\end{equation}

\[
a_1 \neq 0 \implies (x, y) \mapsto (x+a_2, y)
\]
\begin{equation}
	E/K: y^2+a_3y = x^3+a_4x+a_6
\end{equation}

\subsection{Определение изоморфности кривых}
\textbf{$j$-инвариант} эллиптической кривой $E$:
\[
j(E) = \frac{c_4^3}{\Delta}
\] или для краткой формы  $j(E) = -1728 \frac{4a^3}{\Delta}$.
\begin{theorem}
	$E_1 \simeq E_2$ над $\overline{K} \iff j(E_1)=j(E_2)$.
\end{theorem}
Теорема даёт необходимые условия изоморфности, но не достаточности, так как кривые могут быть изоморфны над алгебраическим замыканием поля, но неизоморфными над базовым полем.

Таким образом определение изоморфности кривых над полем $K$ состоит в проверке условий теоремы, а затем, если условия теоремы выполняются, то составлении и решении системы уравнений, используя \eqref{eq:isom}.

\nocite{Blake1999}
\nocite{Washington2008}
\nocite{Menezes1993}
\nocite{HankersonMenezesVanstone2006}
\nocite{Silverman2009}
\printbibliography

\end{document}
