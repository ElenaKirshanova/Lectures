\documentclass[11pt]{exam}
%%%%%%%%%%%%%%%%%%%%%%%%%%%%%%%%
%\noprintanswers % pour enlever les réponses
%\printanswers

\unframedsolutions
\SolutionEmphasis{\itshape\small}
\renewcommand{\solutiontitle}{\noindent\textbf{A: }}
%%%%%%%%%%%%%%%%%%%%%%%%%%%%%%%%

\usepackage[T2A]{fontenc}
\usepackage[utf8]{inputenc}
\usepackage[english, russian]{babel}


\usepackage[margin=0.73in]{geometry}
%\usepackage[top=1in, bottom=1in, left=1in, right=1in]{geometry}

%\usepackage{fullpage}


\usepackage{hyperref}
\usepackage{appendix}
\usepackage{enumerate}


\usepackage{times,graphicx,epsfig,amsmath,latexsym,amssymb,verbatim}%,revsymb}
\usepackage{algorithmicx, enumitem, algpseudocode, algorithm, caption}


%%%%%%%%%%%%%%%%%%%%%
% Handling comments and versions %%%
%%%%%%%%%%%%%%%%%%%%%
\newcommand{\extra}[1]{}

\renewcommand{\comment}[1]{\texttt{[#1]}}


%%%%%%%%%%%%%%%%%%%%%%%%%%%
%% THEOREMS
%%%%%%%%%%%%%%%%%%%%%%%%%%%

\usepackage{amsmath,amssymb,amsfonts}
\usepackage{amsthm}

\newtheorem{theorem}{Theorem}[section]
\newtheorem{axiom}[theorem]{Axiom}
\newtheorem{conclusion}[theorem]{Conclusion}
\newtheorem{condition}[theorem]{Condition}
\newtheorem{conjecture}[theorem]{Conjecture}
\newtheorem{corollary}[theorem]{Corollary}
\newtheorem{criterion}[theorem]{Criterion}
\newtheorem{definition}[theorem]{Definition}
\newtheorem{lemma}[theorem]{Lemma}
\newtheorem{notation}[theorem]{Notation}
\newtheorem{proposition}[theorem]{Proposition}


\theoremstyle{definition}
\newtheorem{problem}{Problem}


\newcommand{\nc}{\newcommand}
\nc{\eps}{\varepsilon}
\nc{\RR}{{{\mathbb R}}}
\nc{\CC}{{{\mathbb C}}}
\nc{\FF}{{{\mathbb F}}}
\nc{\NN}{{{\mathbb N}}}
\nc{\ZZ}{{{\mathbb Z}}}
\nc{\PP}{{{\mathbb P}}}
\nc{\QQ}{{{\mathbb Q}}}
\nc{\UU}{{{\mathbb U}}}
\nc{\OO}{{{\mathbb O}}}
\nc{\EE}{{{\mathbb E}}}

\newcommand{\val}{\operatorname{val}}

\newcommand{\wt}{\ensuremath{\mathit{wt}}}
\newcommand{\Id}{\ensuremath{I}}
\newcommand{\transpose}{\mkern0.7mu^{\mathsf{ t}}}
\newcommand*{\ScProd}[2]{\ensuremath{\langle#1\mathbin{,}#2\rangle}} %Scalar Product
\newcommand*\norm[1]{\left\lVert#1\right\rVert}

\pretolerance=1000

%%%%%%%%%%%%%%%%%%%%%%%%%%%%%%%%
%%%%%%%%%%%%%%%%%%%%%%%%%%%%%%%%
%% DOCUMENT STARTS
%%%%%%%%%%%%%%%%%%%%%%%%%%%%%%%%
%%%%%%%%%%%%%%%%%%%%%%%%%%%%%%%%
\usepackage{tikz}
\usetikzlibrary{automata}
\DeclareMathOperator{\Vol}{Vol}

\begin{document}
	{\noindent
		\textsc{БФУ им. И. Канта -- Криптография на решётках}
		\hfill {Е. Киршанова // 2023\\}
	\hrule
	\begin{center}
		{\Large\textbf{
				\textsc{Практика № 6} \\[5pt] {24.04.23}
		} }
	\end{center}
	\hrule \vspace{5mm}
	
	\thispagestyle{empty}
	
	\vspace{0.2cm}
	\section{Лемма из лекции}
		Используя обозначения Леммы, докажите, что
		\[
		S_A=
		\left[
		\begin{array}{c|c}
		I & W \\
		\hline
		0 & S
		\end{array}  \right] \cdot 
		\left[
		\begin{array}{c|c}
		I & 0 \\
		\hline
		R & W	\end{array}
		\right]
		-\text{базис } A^\perp.
		\]
		Для этого сперва покажите, что $S_A \cdot A = 0 \bmod p$, затем, что $\det S_A = q^n$ и $\det A^\perp = q^n$.
%	\section{Leftover Hash Lemma}
%	Докажите, что $\Delta[(A, r^t A), (A,u)] \leq 2^{-\Omega(n)}$ для $A \leftarrow \mathcal{U}(\ZZ_q^{m \times n})$, $u \leftarrow \mathcal{U}(\ZZ_q^n)$, $r \leftarrow D_{\ZZ^m, \sigma}$, $m \geq n \log q$, $q-$ простое.}
%
%	\begin{enumerate}
%		\item Постройте изоморфизм $\ZZ_q^n \cong \ZZ^m / A^\perp$.
%		
%		Вывод: $D_{\ZZ^m, \sigma} \cdot A$ следует случайному равномерному распределению $\iff$ $D_{\ZZ^m, \sigma}^t \bmod A^\perp $ случайно равномерно в $\ZZ^m / A^\perp$.
%		
%		\item Докажите, что $\Pr_{b \leftarrow D_{\ZZ^m, \sigma}}[b -\text{класс смежности в } A^\perp] \approx \frac{\rho_\sigma(A^\perp)}{\rho_\sigma(\ZZ)}$ и, что эта величина независима от $b$. Для этого можете использовать зависимость  $\eta_{\varepsilon}(A^\perp)$ от $\lambda_1$ дуальной решетки, а для $\lambda_1$ неравенство Минковского-Хлавки (см. Лекцию №2).  
%	\end{enumerate}
%		
		
\end{document}