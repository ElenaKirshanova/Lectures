\documentclass[11pt]{exam}
%%%%%%%%%%%%%%%%%%%%%%%%%%%%%%%%
%\noprintanswers % pour enlever les réponses
%\printanswers

\unframedsolutions
\SolutionEmphasis{\itshape\small}
\renewcommand{\solutiontitle}{\noindent\textbf{A: }}
%%%%%%%%%%%%%%%%%%%%%%%%%%%%%%%%

\usepackage[T2A]{fontenc}
\usepackage[utf8]{inputenc}
\usepackage[english, russian]{babel}


\usepackage[margin=0.73in]{geometry}
%\usepackage[top=1in, bottom=1in, left=1in, right=1in]{geometry}

%\usepackage{fullpage}


\usepackage{hyperref}
\usepackage{appendix}
\usepackage{enumerate}


\usepackage{times,graphicx,epsfig,amsmath,latexsym,amssymb,verbatim}%,revsymb}
\usepackage{algorithmicx, enumitem, algpseudocode, algorithm, caption}


%%%%%%%%%%%%%%%%%%%%%
% Handling comments and versions %%%
%%%%%%%%%%%%%%%%%%%%%
\newcommand{\extra}[1]{}

\renewcommand{\comment}[1]{\texttt{[#1]}}


%%%%%%%%%%%%%%%%%%%%%%%%%%%
%% THEOREMS
%%%%%%%%%%%%%%%%%%%%%%%%%%%

\usepackage{amsmath,amssymb,amsfonts}
\usepackage{amsthm}

\newtheorem{theorem}{Theorem}[section]
\newtheorem{axiom}[theorem]{Axiom}
\newtheorem{conclusion}[theorem]{Conclusion}
\newtheorem{condition}[theorem]{Condition}
\newtheorem{conjecture}[theorem]{Conjecture}
\newtheorem{corollary}[theorem]{Corollary}
\newtheorem{criterion}[theorem]{Criterion}
\newtheorem{definition}[theorem]{Definition}
\newtheorem{lemma}[theorem]{Lemma}
\newtheorem{notation}[theorem]{Notation}
\newtheorem{proposition}[theorem]{Proposition}


\theoremstyle{definition}
\newtheorem{problem}{Problem}


\newcommand{\nc}{\newcommand}
\nc{\eps}{\varepsilon}
\nc{\RR}{{{\mathbb R}}}
\nc{\CC}{{{\mathbb C}}}
\nc{\FF}{{{\mathbb F}}}
\nc{\NN}{{{\mathbb N}}}
\nc{\ZZ}{{{\mathbb Z}}}
\nc{\PP}{{{\mathbb P}}}
\nc{\QQ}{{{\mathbb Q}}}
\nc{\UU}{{{\mathbb U}}}
\nc{\OO}{{{\mathbb O}}}
\nc{\EE}{{{\mathbb E}}}

\newcommand{\val}{\operatorname{val}}

\newcommand{\wt}{\ensuremath{\mathit{wt}}}
\newcommand{\Id}{\ensuremath{I}}
\newcommand{\transpose}{\mkern0.7mu^{\mathsf{ t}}}
\newcommand*{\ScProd}[2]{\ensuremath{\langle#1\mathbin{,}#2\rangle}} %Scalar Product

\newcommand*\abs[1]{\left\lvert#1\right\rvert}
\newcommand*\norm[1]{\left\lVert#1\right\rVert}

\pretolerance=1000

%%%%%%%%%%%%%%%%%%%%%%%%%%%%%%%%
%%%%%%%%%%%%%%%%%%%%%%%%%%%%%%%%
%% DOCUMENT STARTS
%%%%%%%%%%%%%%%%%%%%%%%%%%%%%%%%
%%%%%%%%%%%%%%%%%%%%%%%%%%%%%%%%
\usepackage{tikz}
\usetikzlibrary{automata}
\DeclareMathOperator{\Vol}{Vol}

\begin{document}
	{\noindent
		\textsc{БФУ им. И. Канта -- Криптография на решётках}
		\hfill {Е. Киршанова // 2023\\}
	\hrule
	\begin{center}
		{\Large\textbf{
				\textsc{Практика № 3} \\[5pt] {20.02.23}
		} }
	\end{center}
	\hrule \vspace{5mm}
	
	\thispagestyle{empty}
	
	\vspace{0.2cm}
%	\section{Теорема Минковского-Хлавки}
%	Докажите, что с вероятностью $\geq 1 - 2^{-m}$ для $G \in \ZZ_q^{m \times n}$ выполняется
%	\[
%		\lambda_1^{\infty} (L(G)) \geq \frac{1}{4} q^{1 - n/m}.
%	\]
%	Для этого
%		\begin{enumerate}
%			\item Зафиксируйте $B = \frac{1}{4} q^{1 - n/m}$ и рассмотрите $\Pr_{G}[ \lambda_1^{\infty} (L(G))< B ]$,
%			\item Покажите, что 
%			\[\Pr_{G}[ \lambda_1^{\infty} (L(G))< B ] \leq \sum_{s \in \ZZ_q^n} \sum_{\substack{y \in \ZZ^m \\ \abs{y}_\infty < B}} \Pr[y  = Gs \bmod q]\],
%			\item Покажите, что 
%			\[
%				\sum_{s \in \ZZ_q^n} \sum_{\substack{y \in \ZZ^m \\ \abs{y}_\infty < B}} \Pr[y  = Gs \bmod q] 
%				\begin{cases}
%				 = 0, \hspace{25pt} s = 0, \\
%				 < 2^{-m}, \quad s \neq 0.
%				\end{cases}
%			\]
%		\end{enumerate}
\begin{section}{NP полнота задачи CVP}
	Цель этого упражнения: доказать NP-полноту задачи нахождения ближайшего вектора (CVP). Ддя этого сделаем редукцию от задачи от рюкзаке (известная NP-полная задача) к CVP.
	
	\begin{definition}[Задача о рюкзаке (Subset sum/knapsack)] Пусть на вход даны $a_1, \ldots, a_n, s \in \ZZ$. На выходе задача имеет решение ``ДА'', если существуют $x_i \in \{0,1\}$, такие что $\sum_i x_i a_i = s$;  и решение ``НЕТ'', иначе.
	\end{definition}

	Покажите, что решение задачи CVP для определенной решетки даёт решение задачи о рюкзаке. Рассмотрим решетку, порожденную столбцами матрицы из входных данных задачи о рюкзаке:
	\[
		B = \begin{pmatrix}
			a_1 & a_2 & \ldots & a_n \\
			2 & 0 & \ldots & 0 \\
			\vdots & \vdots & \ddots & \vdots \\
			0 & 0 & \ldots & 2 \\
		\end{pmatrix} \in \ZZ^{(n+1)\times n},
	\]
	и вектор 
	\[
		t = \begin{pmatrix}
			s \\
			1 \\
			\vdots \\
			1
	\end{pmatrix} \in \ZZ^{n+1}.
	\]
	
	\begin{enumerate}
		\item Покажите, что если существует решение задачи о рюкзаке, то $\mathrm{dist}(\mathcal{L}(B), t)  = \sqrt{n}$. \\
		
		Отсюда делаем вывод: если $\mathrm{CVP}_1(\mathcal{L}(B), t, r=\sqrt{n})$, то выводим ``ДА'' для задачи о рюкзаке. Иначе ``НЕТ''.
		
		
		\item Покажите, что  $\mathrm{CVP}_1(\mathcal{L}(B), t, r=\sqrt{n})$ выводит ``ДА''  \emph{только} для инстанций ``ДА'' задачи о рюкзаке. То есть, в  $\mathcal{L}(B)$ нет других (кроме $Bx$) ближайших к $t$ векторов. Например, можно показать, что $\mathrm{dist}(Bx', t)> \sqrt{n}$ для $x' \notin \{0,1\}$.
	\end{enumerate}
	
\end{section}
		
\end{document}