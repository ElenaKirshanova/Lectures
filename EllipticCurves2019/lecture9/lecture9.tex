\documentclass[12pt, final]{article}
\usepackage{amsmath,amssymb,amsthm}
\usepackage{mathtools}
\usepackage{algorithm}
\usepackage[noend]{algpseudocode} 
\usepackage{enumitem}

\usepackage{graphicx}

%---enable russian----

\usepackage[utf8]{inputenc} 
\usepackage[russian]{babel}


%---tikz----
\usepackage{tikz}
\usetikzlibrary{arrows, chains, matrix, positioning, scopes, patterns, shapes}
\usepackage{pgfplots, subfigure}
\usepackage{tikz-cd} 

\usepackage{biblatex}

% MACROS 
% PROBABILITY SYMBOLS
\newcommand*\PROB\Pr 
\DeclareMathOperator*{\EXPECT}{\mathbb{E}}


% GROUPS/DISTRIBUTIONS/SETS/LISTS
\newcommand{\N}{{{\mathbb N}}}
\newcommand{\Z}{{{\mathbb Z}}}
\newcommand{\Q}{{{\mathbb Q}}}
\newcommand{\R}{{{\mathbb R}}}
\newcommand{\F}{{{\mathbb F}}}
\newcommand{\Fqd}{{{\F_{q^{d_i}}}}}
\newcommand{\PP}{{{\mathbb P}}}
\newcommand*{\IZ}{\ensuremath{\mathbb{Z}}}
\newcommand*{\IN}{\ensuremath{\N}}
\newcommand*{\IR}{{{\mathbb R}}}
\newcommand{\Zp}{\ints_p} % Integers modulo p
\newcommand{\Zq}{\ints_q} % Integers modulo q
\newcommand{\Zn}{\ints_N} % Integers modulo N
\newcommand{\Zr}{\ensuremath{\mathbb{Z}/r\mathbb{Z}}} % Integers modulo N
\newcommand*{\dDR}{\mathrm{d}} %de-Rham-Differential (the d in dx, dy, dz and so on)
\newcommand{\transpose}{\mkern0.1mu^{\mathsf{t}}}
\newcommand*{\union}{\mathbin{\cup}}

% Landau 
\newcommand{\bigO}{\mathcal{O}}
\newcommand*{\OLandau}{\bigO}
\newcommand*{\WLandau}{\Omega}
\newcommand*{\xOLandau}{\widetilde{\OLandau}}
\newcommand*{\xWLandau}{\widetilde{\WLandau}}
\newcommand*{\TLandau}{\Theta}
\newcommand*{\xTLandau}{\widetilde{\TLandau}}
\newcommand{\smallo}{o} %technically, an omicron
\newcommand{\softO}{\widetilde{\bigO}}
\newcommand{\wLandau}{\omega}
\newcommand{\negl}{\mathrm{negl}} 

% Misc
\newcommand{\eps}{\varepsilon}
\newcommand{\inprod}[1]{\left\langle #1 \right\rangle}
\DeclarePairedDelimiter{\abs}{\lvert}{\rvert}

%proper overline reduced by some mu's, re-adjust if needed
\newcommand{\overbar}[1]{\overline{\mkern-0.0mu#1\mkern-4.0mu}}

 
\newcommand{\handout}[5]{
  \noindent
  \begin{center}
  \framebox{
    \vbox{
      \hbox to 5.78in { {\bf  } \hfill #2 }
      \vspace{4mm}
      \hbox to 5.78in { {\Large \hfill #5  \hfill} }
      \vspace{2mm}
      \hbox to 5.78in { {\em #3 \hfill #4} }
    }
  }
  \end{center}
  \vspace*{4mm}
}

\newcommand{\lecture}[4]{\handout{#1}{#2}{#3}{Оформил #4}{Лекция #1}}

\newtheorem{theorem}{Теорема}
\newtheorem{corollary}[theorem]{Следствие}
\newtheorem{lemma}[theorem]{Лемма}
\newtheorem{observation}[theorem]{Observation}
\newtheorem{proposition}[theorem]{Предложение}

\theoremstyle{definition}
\newtheorem{definition}[theorem]{Определение}

\newtheorem{claim}[theorem]{Утверждение}
\newtheorem{fact}[theorem]{Факт}
\newtheorem{assumption}[theorem]{Предположение}

\theoremstyle{definition}
\newtheorem{examples}[theorem]{Примеры}

\theoremstyle{definition}
\newtheorem{example}[theorem]{Пример}
\newtheorem{remark}[theorem]{Замечание}

% 1-inch margins
\topmargin 0pt
\advance \topmargin by -\headheight
\advance \topmargin by -\headsep
\textheight 8.9in
\oddsidemargin 0pt
\evensidemargin \oddsidemargin
\marginparwidth 0.5in
\textwidth 7in

\parindent 0in
\parskip 1.5ex


\begin{document}
\lecture{№9}{Осень 2019}{Лектор: Елена Киршанова}{Филипп Максимов}

\section{Изогении. Протокол обмена ключами SIKE}

\subsection{Изогении}
\definition{}
Пусть $E_1, E_2$ — эллиптические кривые. \\
\textbf{Изогения} $\phi : E_1 \rightarrow E_2$ — морфизм, т.ч. $\phi(\bigO_{E_1}) = \bigO_{E_2}$

«Морфизм» означает, что $\phi$ задаётся рациональными многочленами, т.е.
\[
    \phi(x,y) = \left(\frac{f_1(x,y)}{f_2(x,y)}, \frac{g_1(x,y)}{g_2(x,y)} \right) = \left(\frac{p(x)}{q(x)}, y\cdot r(x)\right)
\]
Изогения — гомоморфизм групп

\textbf{Степень изогении} — степень $\phi$, т.е. $\deg\phi = \max\{\deg p(x), \deg q(x)\}$.\\
Если $E_1 = E_2, \phi$ — эндоморфизм. 

\subsubsection{Примеры}
1. «Умножение на $m$» — описывается многочленами деления
\begin{align*}
     [m] : E &\rightarrow E\\
     p &\mapsto m\cdot P
\end{align*}
$E: y^2 = x^3 + x$ над $\Q$
\[
    [2]\cdot P = \left(\frac{(x^2-1)^2}{4(x^3+x)},\; \frac{y\cdot (y^6 + 5x^4 + 5x^2 - 1}{8(x^3+x)^2} \right)
\]
\[  
    \ker[2] = \{\bigO, (x_P, 0)\ :\ x_P^3+x=0\}
\]
\[
    \#\ker[2] = 4 = \deg [2] 
\]
Это совпадение не случайно (для сепарабельных изогений $\ker\phi = \deg\phi$).

2. Frobenius:
\begin{align*}
    E_1: y^2 &= x^3+Ax+B\\
    E_2: y^2 &= x^3+A^Px+B^P
\end{align*}
\[
    \phi(x_1, y_1) = (x_1^P, y_1^P) \text{ — изогения между } E_1, E_2
\]
\[
    \phi = (x^P, (x^3 + Ax + B)^{\frac{p-1}{2}})
\]
\[
    \ker\phi = \bigO_E, \deg\phi = p \text{ (несепарабельая изогения) }
\]

\subsubsection{Факт 1: Теорема Тэйта об изогениях}
$E_1, E_2$ изогенны над $\F_q \Leftrightarrow \#E_1(\F_q) = \#E(\F_q)$, т.е. рациональные многочлены определены над $\F_q$

Т.е. существует эффективный метод определения, изогенны ли кривые

\subsubsection{Факт 2: Vélu}
\textit{Доказательство: Galbraith «Mathematics of Public Key Cryptography», Chapter 25}

Пусть $E$ — эллиптическая кривая над $\F_q, G$ — конечная группа $E(\overbar{\F_q})$.\\
Тогда существует элл. кривая $E'$ над $\F_q$ и сепарабельная изогения $\phi: E \rightarrow E' $, определённая над $\F_q$, степени $\#G$, т.ч. $\ker\phi = G$.

Кроме того, если $\psi: E \rightarrow E''$ — другая сепарабельная изогения степени $\#G$ с $\ker\psi = G$, то $j(E')=j(E'')$, т.е. кривые изоморфны.\\
Значит, образ $E$ корректно определён до изоморфизма.

Обозначение: $E' = E/G$. Это не фактор группа, а кривая, отличная от (но изогенная) $E$.

Vélu описал явные формулы для $E', \phi$

\subsubsection{Факт 3}
\textit{Доказательство в L. de Feo «Algorithmes Rapides pour les Tours de Corps Finis et let Isogénies»}

Пусть $E: y^2 = x^3 + ax + b$ — эллиптическая кривая над полем $K$,\; $G \subset E(\overbar{K})$ — конечная подгруппа.

Сепарабельная изогения $\phi: E \rightarrow E'$ с ядром $G$ может быть записана 
\[
    \phi(P) = [x(P) + \sum_{Q\in G \backslash \{0\}} (x(P+Q) - x(Q)),\;\; y(P) + \sum_{Q\in G \backslash \{0\}}(y(P+Q) - y(Q))],
\]
а изогенная кривая $E/G$ задаётся уравнением $y^2 = x^3 + a'x+b'$, где 
\[
    a' = a - 5\cdot \sum_{Q\in G \backslash \{0\}} (3 x(Q)^2 + a)
\]
\[
    b' = b - 7\cdot \sum_{Q\in G \backslash \{0\}}(5x(Q)^3 + 3ax + 2b)
\]
Сложность вычисления $E/G: \bigO(|G|)$

Если $G$ -- подгруппа большого порядка вычисления $E/G$ является трудной задачей

\subsubsection{SIDH — Supersingular Isogeny Diffie--Hellman}
1. «Стандартный» протокол обмена ключами DH в абстрактной группе $G$: \\ 
$G$ — группа, $\langle g\rangle = G$\\
$\phi_A(x) = [a]\cdot x$ — групповой гомоморфизм

\[
\begin{tikzcd}
    Alice & & Bob \\
    & \arrow{ld}[swap]{\phi_a} g \arrow{rd}{\phi_b}  \\
    \phi_a(g) & \arrow[yshift=-0.7ex]{r}{\phi_b(g)} \arrow[yshift=0.7ex]{l}{\phi_a(g)}  & \phi_b(g)\\
\end{tikzcd}
\]
\[
\phi_a(\phi_b(g)) = \phi_b(\phi_a(g)) = [ab]\cdot g
\]

2. В качестве группового гомоморфизма можно взять изогении $\Rightarrow$ протокол SIKE (de Feo \& Jao 2011)
 
\[
\begin{tikzcd}
    &\arrow{ld}[swap]{\phi_{Alice}} E  \arrow{rd}{\phi_{Bob}} &\\
    E/\langle P\rangle\arrow{rd}[swap]{\phi_{Alice}}  &  & \arrow{ld}{\phi_{Bob}}  E/\langle Q\rangle\\
    & E/\langle P,Q\rangle &\\
\end{tikzcd}
\]
$E$ — суперсингулярная кривая, т.е. $\#E(\F_q)=1\bmod  (\operatorname{char} \F_q) \Leftrightarrow E[char\F_q] = \{\bigO_E\}$

$\ker(\phi_{Bob})=\langle Q\rangle$

$\ker(\phi^{'}_{Alice})=\langle\phi_{Bob}(P)\rangle$
$\ker(\phi^{'}_{Bob})=\langle\phi_{Alice}(P)\rangle$

Детальнее:

$E[\ell_A] = (\Z/e_a\Z) \oplus (\Z/e_a\Z)$ — группа точек $\ell_A$-кручения \\
$E[\ell_B] = (\Z/e_b\Z) \oplus (\Z/e_b\Z)$ — группа точек $\ell_B$-кручения

$E[\ell_A]=\langle R_A, S_A\rangle$ — образующие\\
$E[\ell_B]=\langle R_B, S_B\rangle$ — образующие

(!) В лабораторной №8 даны подгруппы $E[\ell_A]$, $E[\ell_B]$ с одной образующей

\begin{center}
\begin{tabular}{ c c c }
 \boxed{Alice} & $(E,R_A,S_A,R_B,S_B)$ & \boxed{Bob} \\ 
 1. $a\leftarrow (0, \ell_A]$ &  & 1. $b\leftarrow (0, \ell_B]$  \\  
 $T_a=R_A+aS_A \in E[\ell_A]$ &  & $T_b=R_B+bS_B \in E[\ell_B]$\\
 $\phi_A=R\rightarrow E/\langle T_a\rangle$ & $(\phi_A(R_B),\phi_A(S_B), E/\langle T_a\rangle)\rightarrow$ & $\phi_B=R\rightarrow E/\langle T_b\rangle$\\
 & $\leftarrow(\phi_B(R_A),\phi_B(S_A), E/\langle T_b\rangle)$ &\\
 2. $T^{'}_a=\phi_B(R_A) + a\phi_B(S_A)$ & & 2. $T^{'}_b=\phi_A(R_B) + b\phi_B(S_B)$\\
 $=\phi_B(R_a+aS_A)=\phi_B(T_a)$ & & $=\phi_A(R_B+bS_B)=\phi_A(T_b)$\\
 & & \\
 3.$T_a^{'}=\phi_B(R_A)+a\phi_B(S_A)$ & & 3.$T_b^{'}=\phi_A(R_B)+b\phi_A(S_B)$\\
 $=\phi_B(R_A+aS_a)=\phi_B(T_a)$ & & $=\phi_A(R_B+bS_B)=\phi_A(T_b)$\\
 & & \\
 $\phi_a^{'}:E/\langle T_b\rangle \rightarrow (E/\langle T_b\rangle)/\langle T_a'\rangle$ & & $\phi_b^{'}:E/\langle T_a\rangle \rightarrow (E/\langle T_a\rangle)/\langle T_b'\rangle$
\end{tabular}
\end{center}
\[
    E_{A,B}:=\phi_a^{'}\circ \phi_B(E)= \phi_b^{'}\circ \phi_a(E)=E/\langle T_a, T_b\rangle
\]

Уравнения кривых, полученные \textbf{A} и \textbf{B} могут быть не идентичны, но обязательно измофорными $\Leftrightarrow$ j-инвариант одинаков $\Rightarrow j(E_{A,B})$ — общий секретный ключ.
\end{document}