\documentclass[11pt]{exam}

%---enable russian----

\usepackage[utf8]{inputenc}
\usepackage[russian]{babel}



\usepackage[margin=0.73in]{geometry}
%\usepackage[top=1in, bottom=1in, left=1in, right=1in]{geometry}

\usepackage{graphicx}
\usepackage{url}
\usepackage{latexsym}
\usepackage{amscd,amsmath,amsthm}
\usepackage{mathtools}
\usepackage{amsfonts}
\usepackage{amssymb}
\usepackage[dvipsnames]{xcolor}
\usepackage{hyperref}

\usepackage{algorithmicx, enumitem, algpseudocode, algorithm, caption}
\usepackage{tikz}
\usetikzlibrary{automata}

%%%%%%%%%%%%%%%%%%%%%
% Handling comments and versions %%%
%%%%%%%%%%%%%%%%%%%%%

%\renewcommand{\comment}[1]{\texttt{[#1]}}


%%%%%%%%%%%%%%%%%%%%%%%%%%%
%% THEOREMS
%%%%%%%%%%%%%%%%%%%%%%%%%%%

\newtheorem{theorem}{Theorem}[section]
\newtheorem{axiom}[theorem]{Axiom}
\newtheorem{conclusion}[theorem]{Conclusion}
\newtheorem{condition}[theorem]{Condition}
\newtheorem{conjecture}[theorem]{Conjecture}
\newtheorem{corollary}[theorem]{Corollary}
\newtheorem{criterion}[theorem]{Criterion}
\newtheorem{definition}[theorem]{Definition}
\newtheorem{lemma}[theorem]{Lemma}
\newtheorem{notation}[theorem]{Notation}
\newtheorem{proposition}[theorem]{Proposition}


\theoremstyle{definition}
\newtheorem{problem}{Problem}


\newcommand{\nc}{\newcommand}
\nc{\eps}{\varepsilon}
\nc{\RR}{{{\mathbb R}}}
\nc{\CC}{{{\mathbb C}}}
\nc{\FF}{{{\mathbb F}}}
\nc{\NN}{{{\mathbb N}}}
\nc{\ZZ}{{{\mathbb Z}}}
\nc{\PP}{{{\mathbb P}}}
\nc{\QQ}{{{\mathbb Q}}}
\nc{\UU}{{{\mathbb U}}}
\nc{\OO}{{{\mathbb O}}}
\nc{\EE}{{{\mathbb E}}}

\newcommand{\val}{\operatorname{val}}
\newcommand{\wt}{\ensuremath{\mathit{wt}}}
\newcommand{\Id}{\ensuremath{I}}
\newcommand{\transpose}{\mkern0.7mu^{\mathsf{ t}}}
\newcommand*{\ScProd}[2]{\ensuremath{\langle#1\mathbin{,}#2\rangle}} %Scalar Product
\renewcommand{\char}{\ensuremath{\mathsf{char}}}

\DeclareMathOperator{\Vol}{Vol}

%\pretolerance=1000

%%%%%%%%%%%%%%%%%%%%%%%%%%%%%%%%
%%%%%%%%%%%%%%%%%%%%%%%%%%%%%%%%
%% DOCUMENT STARTS
%%%%%%%%%%%%%%%%%%%%%%%%%%%%%%%%
%%%%%%%%%%%%%%%%%%%%%%%%%%%%%%%%


\begin{document}	
	{\noindent
		\textsc{БФУ им. И. Канта -- Компьютерный практикум по криптографии на эллиптических кривых }\\[5pt]
		Преподаватель {Е. Киршанова}   \hfill{2019--2020\\}
	\hrule
	\begin{center}
		{\LARGE\textbf{
				Лабораторная работа № 6 \\[5pt]
		}} 
			Опубликована \textbf{15.11.2019} \\[5pt] 
			Дэдлайн \textbf{29.11.2019}
		
	\end{center}
	\hrule \vspace{5mm}
	
	\thispagestyle{empty}
	
	Разработать программу в системе компьютерной алгебры Maple или Sage (в одной на выбор), реализующую следующие функции:
	
	\begin{enumerate}
		\item \texttt{Prove\_prime($p$)}, где $p$ -- простое или составное число. Функция реализует алгоритм теста на простоту Goldwasser-Killain  возвращает (с большой вероятностью) либо $С$-сертификат простоты $p$, либо делитель $p$; с малой вероятностью возвращает ``fail''.
		
		\item \texttt{Check\_prime($p_0, C = [A_0, B_0, L_0, p_1], \ldots [A_i, B_i, L_i, p_{i+1}]$)}, где $C$ -- сертификат простоты числа $p_0$. Функция реализует алгоритм проверки сертификата на простоту и возвращает либо ``Accept'' (в случае принятия сертификата), либо ``Reject'' с пояснением, почему.
		
		
		
	\end{enumerate}
\normalsize
\section*{Требования к сдаче}
\begin{itemize}
	\item 
	 Для программ разработанных в системе Maple, следует сдавать подгружаемый модуль.
	 \item Исходный код должен содержать комментарии к каждой из функций с описанием входных и выходных параметров
\end{itemize}
\end{document}