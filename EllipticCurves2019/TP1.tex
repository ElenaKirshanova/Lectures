\documentclass[11pt]{exam}

%---enable russian----

\usepackage[utf8]{inputenc}
\usepackage[russian]{babel}



\usepackage[margin=0.73in]{geometry}
%\usepackage[top=1in, bottom=1in, left=1in, right=1in]{geometry}

\usepackage{graphicx}
\usepackage{url}
\usepackage{latexsym}
\usepackage{amscd,amsmath,amsthm}
\usepackage{mathtools}
\usepackage{amsfonts}
\usepackage{amssymb}
\usepackage[dvipsnames]{xcolor}
\usepackage{hyperref}

\usepackage{algorithmicx, enumitem, algpseudocode, algorithm, caption}
\usepackage{tikz}
\usetikzlibrary{automata}

%%%%%%%%%%%%%%%%%%%%%
% Handling comments and versions %%%
%%%%%%%%%%%%%%%%%%%%%

%\renewcommand{\comment}[1]{\texttt{[#1]}}


%%%%%%%%%%%%%%%%%%%%%%%%%%%
%% THEOREMS
%%%%%%%%%%%%%%%%%%%%%%%%%%%

\newtheorem{theorem}{Theorem}[section]
\newtheorem{axiom}[theorem]{Axiom}
\newtheorem{conclusion}[theorem]{Conclusion}
\newtheorem{condition}[theorem]{Condition}
\newtheorem{conjecture}[theorem]{Conjecture}
\newtheorem{corollary}[theorem]{Corollary}
\newtheorem{criterion}[theorem]{Criterion}
\newtheorem{definition}[theorem]{Definition}
\newtheorem{lemma}[theorem]{Lemma}
\newtheorem{notation}[theorem]{Notation}
\newtheorem{proposition}[theorem]{Proposition}


\theoremstyle{definition}
\newtheorem{problem}{Problem}


\newcommand{\nc}{\newcommand}
\nc{\eps}{\varepsilon}
\nc{\RR}{{{\mathbb R}}}
\nc{\CC}{{{\mathbb C}}}
\nc{\FF}{{{\mathbb F}}}
\nc{\NN}{{{\mathbb N}}}
\nc{\ZZ}{{{\mathbb Z}}}
\nc{\PP}{{{\mathbb P}}}
\nc{\QQ}{{{\mathbb Q}}}
\nc{\UU}{{{\mathbb U}}}
\nc{\OO}{{{\mathbb O}}}
\nc{\EE}{{{\mathbb E}}}

\newcommand{\val}{\operatorname{val}}
\newcommand{\wt}{\ensuremath{\mathit{wt}}}
\newcommand{\Id}{\ensuremath{I}}
\newcommand{\transpose}{\mkern0.7mu^{\mathsf{ t}}}
\newcommand*{\ScProd}[2]{\ensuremath{\langle#1\mathbin{,}#2\rangle}} %Scalar Product
\renewcommand{\char}{\ensuremath{\mathsf{char}}}

\DeclareMathOperator{\Vol}{Vol}

%\pretolerance=1000

%%%%%%%%%%%%%%%%%%%%%%%%%%%%%%%%
%%%%%%%%%%%%%%%%%%%%%%%%%%%%%%%%
%% DOCUMENT STARTS
%%%%%%%%%%%%%%%%%%%%%%%%%%%%%%%%
%%%%%%%%%%%%%%%%%%%%%%%%%%%%%%%%


\begin{document}	
	{\noindent
		\textsc{БФУ им. И. Канта -- Компьютерный практикум по криптографии на эллиптических кривых }\\[5pt]
		Преподаватель {Е. Киршанова}   \hfill{2019--2020\\}
	\hrule
	\begin{center}
		{\LARGE\textbf{
				Лабораторная работа № 1 \\[5pt]
		}} 
			Опубликована \textbf{06.09.2019} \\[5pt] 
			Дэдлайн \textbf{27.09.2019}
		
	\end{center}
	\hrule \vspace{5mm}
	
	\thispagestyle{empty}
	
	Разработать программу в системе компьютерной алгебры Maple или Sage (в одной на выбор), реализующую следующие функции:
	
	\begin{enumerate}
		\item \texttt{jInvariant($\mathtt{a_1, a_2, a_3, a_4, a_6}$)}, где $\mathtt{a_1, a_2, a_3, a_4, a_6}$ -- коэффициенты кривой, заданной уравнение Вейерштрасса. Если кривая является эллиптической, функция возвращает $j$-инвариант кривой, иначе сообщение о том, что кривая сингулярна.
		
		\item \texttt{randIsomorphic($\mathtt{a_1=0, a_2=0, a_3=0, a_4=0, a_6=0, a=0, b=0}$)}, где $\mathtt{a_1, a_2, a_3, a_4, a_6}$,$ \mathtt{a,b}$ -- коэффициенты эллиптической кривой $E_1$ в общем случае, или в случае $\char(K) \neq 2,3$. Функция возвращает коэффициенты кривой $E_2$, изоморфной  $E_1$ над $\QQ$ путём случайного выбора параметров $(u,r,s,t)$. Если коэффициенты $\mathtt{a_1, a_2, a_3, a_4, a_6}$ задают сингулярную кривую, функция терминирует с соответствующим сообщением.
		
		\item \texttt{isIsomorphic($\mathtt{a_1, a_2, a_3, a_4, a_6, \_a_1, \_a_2, \_a_3, \_a_4, \_a_6}, p$)}, где $\mathtt{a_1, a_2, a_3, a_4, a_6}$ -- коэффициенты эллиптической кривой $E_1$, $\mathtt{\_a_1, \_a_2, \_a_3, \_a_4, \_a_6}$ -- коэффициенты эллиптической кривой $E_2$, $p$ -- простое число (означает кривые заданы над $\FF_p$) или $0$ (кривые заданы над $\QQ$). Функция определяет, являются ли кривые изоморфными над $\FF_p$ (или $\QQ$), и возвращает одно из значений $\in \{ \mathtt{isomorphic}, \mathtt{non-isomorphic} \}$.
		Если коэффициенты $\mathtt{a_1, a_2, a_3, a_4, a_6}$ или $\mathtt{\_a_1, \_a_2, \_a_3, \_a_4, \_a_6}$ задают сингулярную кривую, функция терминирует с соответствующим сообщением.
		
		\item \texttt{findExtension($\mathtt{a_1, a_2, a_3, a_4, a_6, \_a_1, \_a_2, \_a_3, \_a_4, \_a_6}, p$)},  коэффициенты эллиптической кривой $E_1$, $\mathtt{\_a_1, \_a_2, \_a_3, \_a_4, \_a_6}$ -- коэффициенты эллиптической кривой $E_2$, заданные над $\FF_p$ ($p$ интерпретировать аналогично предыдущей функции).Функция определяет, над каким полем кривые  $E_1 \cong E_2$ и возвращает степень расширения этого поля над $\FF_p$.
		
		
		Если коэффициенты $\mathtt{a_1, a_2, a_3, a_4, a_6}$ или $\mathtt{\_a_1, \_a_2, \_a_3, \_a_4, \_a_6}$ задают сингулярную кривую, функция терминирует с соответствующим сообщением.
	\end{enumerate}

%	 \textnormal{(Default) Основной шрифт документа}\\
%	 \textrm{(Roman) С засечками}\\
%	 \textit{(Italic) Курсив — не наклонный!}\\
%	 \textsl{(Slanted) Наклонный — не курсив!}\\
%	 \textbf{(Bold) Жирный}\\
%	 \textbf{\textit{(Bold italic) Жирный курсив}}\\
%	 \textbf{\textsl{(Bold slanted) Жирный наклонный}}\\
%	 \texttt{(Monospace) Моноширинный}\\
%	 \textsc{(Small caps) «Малые заглавные»}\\
%	 \textbf{\textsc{(Bold Small caps) Жирный «Малые заглавные»}}
%	 \textsf{(Sans serif) без засечек}
\section*{Требования к сдаче}
\begin{itemize}
	\item 
	 Для программ разработанных в системе Maple, следует сдавать подгружаемый модуль.
	 \item Исходный код должен содержать комментарии к каждой из функций с описанием входных и выходных параметров
\end{itemize}
\end{document}