\documentclass[12pt]{article}
\usepackage{amsmath,amssymb,amsthm}
\usepackage{algorithm}
\usepackage[noend]{algpseudocode} 
\usepackage{enumitem}

%---enable russian----

\usepackage[utf8]{inputenc}
\usepackage[russian]{babel}


%---tikz----
\usepackage{tikz}
\usetikzlibrary{arrows, chains, matrix, positioning, scopes, patterns, shapes}
\usepackage{pgfplots, subfigure}

\usepackage{biblatex}

% MACROS 
% PROBABILITY SYMBOLS
\newcommand*\PROB\Pr 
\DeclareMathOperator*{\EXPECT}{\mathbb{E}}


% GROUPS/DISTRIBUTIONS/SETS/LISTS
\newcommand{\N}{{{\mathbb N}}}
\newcommand{\Z}{{{\mathbb Z}}}
\newcommand{\Q}{{{\mathbb Q}}}
\newcommand{\R}{{{\mathbb R}}}
\newcommand{\F}{{{\mathbb F}}}
\newcommand{\PP}{{{\mathbb P}}}
\newcommand*{\IZ}{\ensuremath{\mathbb{Z}}}
\newcommand*{\IN}{\ensuremath{\mathbb{N}}}
\newcommand*{\IR}{{{\mathbb R}}}
\newcommand{\Zp}{\ints_p} % Integers modulo p
\newcommand{\Zq}{\ints_q} % Integers modulo q
\newcommand{\Zn}{\ints_N} % Integers modulo N
\newcommand{\Zr}{\ensuremath{\mathbb{Z}/r\mathbb{Z}}} % Integers modulo N
\newcommand*{\dDR}{\mathrm{d}} %de-Rham-Differential (the d in dx, dy, dz and so on)
\newcommand{\transpose}{\mkern0.1mu^{\mathsf{t}}}
\newcommand*{\union}{\mathbin{\cup}}

% Landau 
\newcommand{\bigO}{\mathcal{O}}
\newcommand*{\OLandau}{\bigO}
\newcommand*{\WLandau}{\Omega}
\newcommand*{\xOLandau}{\widetilde{\OLandau}}
\newcommand*{\xWLandau}{\widetilde{\WLandau}}
\newcommand*{\TLandau}{\Theta}
\newcommand*{\xTLandau}{\widetilde{\TLandau}}
\newcommand{\smallo}{o} %technically, an omicron
\newcommand{\softO}{\widetilde{\bigO}}
\newcommand{\wLandau}{\omega}
\newcommand{\negl}{\mathrm{negl}} 

% Misc
\newcommand{\eps}{\varepsilon}
\newcommand{\inprod}[1]{\left\langle #1 \right\rangle}

 
\newcommand{\handout}[5]{
  \noindent
  \begin{center}
  \framebox{
    \vbox{
      \hbox to 5.78in { {\bf  } \hfill #2 }
      \vspace{4mm}
      \hbox to 5.78in { {\Large \hfill #5  \hfill} }
      \vspace{2mm}
      \hbox to 5.78in { {\em #3 \hfill #4} }
    }
  }
  \end{center}
  \vspace*{4mm}
}

\newcommand{\lecture}[4]{\handout{#1}{#2}{#3}{Оформил #4}{Лекция #1}}

\newtheorem{theorem}{Теорема}
\newtheorem{corollary}[theorem]{Следствие}
\newtheorem{lemma}[theorem]{Лемма}
\newtheorem{observation}[theorem]{Observation}
\newtheorem{proposition}[theorem]{Предложение}

\theoremstyle{definition}
\newtheorem{definition}[theorem]{Определение}

\newtheorem{claim}[theorem]{Утверждение}
\newtheorem{fact}[theorem]{Факт}
\newtheorem{assumption}[theorem]{Предположение}

\theoremstyle{definition}
\newtheorem{examples}[theorem]{Примеры}

\theoremstyle{definition}
\newtheorem{example}[theorem]{Пример}

% 1-inch margins
\topmargin 0pt
\advance \topmargin by -\headheight
\advance \topmargin by -\headsep
\textheight 8.9in
\oddsidemargin 0pt
\evensidemargin \oddsidemargin
\marginparwidth 0.5in
\textwidth 6.5in

\parindent 0in
\parskip 1.5ex

\addbibresource{books.bib}

\begin{document}
    
\lecture{2 --- 20.09.2019}{Осень 2019}{Лектор: Елена Киршанова}{Филипп Максимов}

\section{Групповой закон на эллиптической кривой}
    \begin{gather*}
        E: y^2 = x^3 + Ax+B \ \ (char K \neq 2, 3) \\ 
        P_1 = (x_1, y_1) \in E \\
        P_2 = (x_2, y_2) \in E
    \end{gather*}
    
    \begin{enumerate}
        \item Проведём прямую через $P_1, P_2$. Она пересечёт кривую  в 3-ей точке $P_3$ (кривая задана уравнением степени 3).
        \item Отразим $P_3$ относительно $Ox$
        \item Положим $P_3' = P_1 + P_2 \ \ \ (\boldsymbol{!} \ P_1 + P_2 \neq (x_1+x_2, y_1+y_2))$.
    \end{enumerate}
    
    \begin{figure}[h!]
    \centering
    \begin{tikzpicture}
    \begin{axis}[
            xmin=-4,
            xmax=5,
            ymin=-5,
            ymax=5,
            xlabel={$x$},
            ylabel={$y$},
            scale only axis,
            axis lines=middle,
            domain=-2.279018:3,      
            samples=201,
            smooth,   
            clip=false,
            % use same unit vectors on the axis
            axis equal image=true,
        ]
        
        \addplot[blue] {sqrt(x^3-3*x+5)} node[right] {$E$};
        \addplot[blue] {-sqrt(x^3-3*x+5)};
        \addplot[red] coordinates {(-3, -0.0890722)(3, 3.06557)} node[right] {$l$};
        \addplot[green] coordinates {(1.9, 3.5)(1.9, -3.5)};
    %\begin{scriptsize}
        %== P ==
        \draw [fill=black] (axis cs: 0.65, 1.83) circle (2pt);
        \draw[color=black] (axis cs: 1.2, 1.3) node [left]{$P_2$};
        \draw [fill=black] (axis cs: -2.26, 0.3) circle (2pt);
        \draw[color=black] (axis cs: -2.3, 0.3) node [left]{$P_1$};
        \draw [fill=black] (axis cs: 1.9, 2.5) circle (2pt);
        \draw[color=black] (axis cs: 1.9, 2.8) node [left]{$P_3$};
        \draw [fill=black] (axis cs: 1.9, -2.5) circle (2pt);
        \draw[color=black] (axis cs: 1.9, -2.8) node [left]{$P_3'$};
    %\end{scriptsize}
    \end{axis}
    \end{tikzpicture}
    \end{figure}
    
    Получим координаты $P_3'$: 
    \[
    m = \cfrac{y_2-y_1}{x_2-x_1}   \text{ -- наклон } l. 
    \]
    \begin{itemize}
    \item Положим $x_2 \neq x_1 \Rightarrow$ уравнение $l: y = m(x-x_1)+y_1$.
    
    Найдём пересечение $l$ c $E: (m(x-x_1)+y_1)^2 = x^3 + Ax + B$ — уравнение 3-ей степени с 3-мя корнями, два корня известны ($x_1, x_2$). Кроме того, для любого кубического уравнения с корнями $r, s, t:$
    \begin{gather*}
    x^3 + ax^2+bx+c = (x-r)(x-s)(x-t)=x^3 - (r+s+t)x^2 + ... \\
    \Rightarrow r+s+t=-a \\ 
    \Rightarrow t = -a-r-s
    \end{gather*}
    В нашем случае $x^3 - m^2x^2+...=0 \Rightarrow$ 3-ий корень: 
    \begin{gather*}
        x_3 = m^2 - x_1 - x^2 \text{ и } y = m(x-x_1)+y_1 \\
        P_3 = (m^2 - x_1 - x_2, \ m(x-x_1)+ y_1)
    \end{gather*}
    Отражение относительно $Ox:$
    \[
    P_3' = (m^3 - x_1  - x_2, \ m(x_1 - x)-y_1)
    \]

    \item Положим $x_2=x_1, \ y_1\neq y_2\Rightarrow l$ — вертикальная прямая $\Rightarrow$ пересекает $E$ в $\bigO. \\ \bigO \text{ относительно Ox } = \bigO \Rightarrow P_3' = \bigO$
    
    \item Положим $x_2=x_1, \ y_1 = y_2 \ \ (P_1=P_2) \Rightarrow l$ — касательная к $E \Rightarrow m=\frac{dy}{dx}, \text{ где } y=f(x).$
    \begin{gather*}
        2y\frac{dy}{dx} = 3x^2 + A \Rightarrow \\
        m = \frac{3x_1^2 + A}{2y_1} \text{ в } m: P_1=P_2
    \end{gather*}

    Если $y_1=0 \Rightarrow$ вертикальная прямая $\Rightarrow P_1 + P_2 = \infty$\\
    Важно: $3x_1^2 + A \neq \bigO$ при $y_1 = 0$.
    
    Если ${y_1} \ne 0 \Rightarrow l: y = m\left( {x - {x_1}} \right) + {y_1}$. Аналогично получаем кубическое уравнение 
    \[
    0 = {x^3} - {m^2}{x^2} +  \ldots ,
    \]
    однако теперь нам известен 1 корень, ${x_1}$ степени 2. Аналогичными рассуждениями, приходим: 
    \begin{gather*}
    {x_3} = {m^2} - 2{x_1},\quad {y_3} = m\left( {{x_1} - {x_3}} \right) - {y_1} \hfill \\
    P'_3 = \left( {{x_3},\;{y_3}} \right) \hfill \\
    \end{gather*}
    
    \item Положим ${P_2} = \mathcal{O} $, проходящая через ${P_1}$ и $\mathcal{O} $~---~вертикальная прямая $ \cap E = {P_1}$. Отражение относительно $OX$ даст
    \[
    {P_1} \Rightarrow {P_1} + \mathcal{O}  = {P_1}\quad \forall {P_1} \in E.
    \]
     
 \end{itemize}

    \begin{figure}[h!]
    \centering
    \begin{tikzpicture}
    \begin{axis}[
            xmin=-4,
            xmax=5,
            ymin=-5,
            ymax=5,
            xlabel={$x$},
            ylabel={$y$},
            scale only axis,
            axis lines=middle,
            domain=-2.279018:3,      
            samples=201,
            smooth,   
            clip=false,
            % use same unit vectors on the axis
            axis equal image=true,
        ]
        
        \addplot[blue] {sqrt(x^3-3*x+5)} node[right] {$E$};
        \addplot[blue] {-sqrt(x^3-3*x+5)};
        \addplot[green] coordinates {(-1.3, 3.5)(-1.3, -3.5)};
    %\begin{scriptsize}
        %== P ==
        \draw [fill=black] (axis cs: -1.3, 2.58) circle (2pt);
        \draw[color=black] (axis cs: -1.3, 2.95) node [left]{$P_1 = P_3'$};
        \draw [fill=black] (axis cs: -1.3, -2.58) circle (2pt);
        \draw[color=black] (axis cs: -1.25, -2.9) node [left]{$P_3$};
    %\end{scriptsize}
    \end{axis}
    \end{tikzpicture}
    \end{figure}
    
    \begin{corollary}[Закон сложения <<+>> на $E$]
    	\label{corol_01}
    	\[
    	E:{y^2} = {x^3} + Ax + B,\quad {P_1} = \left( {{x_1},{y_1}} \right), {P_2} = \left( {{x_2},{y_2}} \right) \in E, {P_1},{P_2} \ne \bigO
    	\]
    	определим ${P_3} = {P_1} + {P_2} = \left( {{x_3},{y_3}} \right):$
    	\begin{enumerate}
    		\item Если ${x_1} \ne {x_2}$;
    		
    		\begin{gather*}
            {x_3} = {m^2} - {x_1} - {x_2} \\
    		{y_3} = m\left( {{x_1} - {x_3}} \right) - {y_1}, \\
    		\text{где } m = \frac{{{y_2} - {y_1}}}{{{x_2} - {x_1}}}
    		\end{gather*}
    		
    		\[
    		\boxed{1\text{Inv}  + 3{\text{Mul}}\;{\text{в}}\;{\text{K}}}
    		\]
    		
    		\item Если ${x_1} = {x_2}$, но ${y_1} \ne {y_2}$: 
    		\begin{gather*}
    		{P_1} + {P_2} = \mathcal{O}
    		\end{gather*}

    		\item Если ${P_1} = {P_2}$, но ${y_1} \ne 0$: 
    		\begin{gather*}
    		{x_3} = {m^2} - 2{x_1} \hfill \\
    		{y_3} = m\left( {{x_1} - {x_3}} \right) - {y_1},{\text{ где }}m = \frac{{3x_1^2 + A}}{{2{y_1}}} \hfill \\ 
    		\end{gather*}
    		
    		\item  Если ${P_1} = {P_2}$, ${y_1} = 0$, то 
    		\begin{gather*}
    		{P_1} + {P_2} = \mathcal{O} \hfill \\
    		\forall P \in E \hfill \\ 
    		\end{gather*} 
    		\[
    		\boxed{{\text{1I + 4M}}}
    		\]
    		
    		\item Определим $P + \mathcal{O}  = P.$
    		
    		Если $P = \left( {{x_1},{y_1}} \right) \in \underbrace {K \times K}_{{\text{поле}}}$, ${P_2} = \left( {{x_2},{y_2}} \right) \in {K^2}$ и $A,B \in K \Rightarrow {P_3} \in {K^2}$. 
    	\end{enumerate}
    \end{corollary}
     
    \begin{theorem}
    	\label{theor_02}
    	Операция сложения на $E$, заданная в Cледствии~\ref{corol_01}, обладает следующими свойствами:
    	\begin{enumerate}
    		\item Коммутативность: ${P_1} + {P_2} = {P_2} + {P_1}$ $\forall {P_1},{P_2} \in E$
    		
    		\item $\exists $ нейтральный элемент: $P + \mathcal{O} = P$ $\forall P \in E$
    		
    		\item $\exists$ обратный элемент: $\forall P \in E$ $\exists P' \in E:\:P + P' = \mathcal{O}$ (обычно $P'$ обозначается $-P$)
    		
    		\item Ассоциативность: $\left( {{P_1} + {P_2}} \right) + {P_3} = {P_1} + \left( {{P_2} + {P_3}} \right)$ $\forall {P_1},{P_2},{P_3} \in E$.
    	\end{enumerate}
    \end{theorem}
    
    Вывод: закон сложения <<+>> точек на $E$ задает аддитивную абелеву группу.
    
    \begin{proof}
    1~--~5 тривиально; 4~--~см. Washington, Sec. 2.4.
    \end{proof}
    
    \underline{Замечания} 
    \begin{enumerate}
    \item Для сокращённого уравнения Вейерштрасса $\left(char K \ne 2,3 \right)$ для $P = \left( {x,y} \right)$ 
    \[
    - P = \left( {x, - y} \right).
    \] 
    Однако, для обобщенного уравнения $\left( {{y^2} + {a_1}xy + {a_3}y = {X^3} + {a_2}{X^2} + {a_4}X + {a_6}} \right)$ 
    \[ 
    - P = \left( {X,\; - {a_1}X - {a_3} - y} \right)
    \]
    
    \item Если $E$ задана над $\F_q \Rightarrow $ получаем \underline{конечную} абелеву группу ($ \Rightarrow $ приложения в криптографии).

    Если $E$ задана над $\Q$, $E\left( \Q \right)$~---~\underline{конечно-порожденная} абелева группа. 

    \item Если $char(K) = 2, \quad E:\:{y^2} + xy = {x^3} + {a_2}{x^2} + {a_6}$. \\
    Если ${P_1} \ne {P_2}$:
    \begin{gather*}
    m = \frac{y_1 + y_2}{x_1 + x_2} \\ 
    x_3 = m^2 + m + x_1 + x_2 + a_2 \\ 
    y_3 = ( x_1 + x_3)m + x_3 + y_1 \\
    \end{gather*}
    Если ${P_1} = {P_2}$:
    \begin{gather*}
    m = \frac{y_1}{x_1} + x_1 \\
    x_3 = m^2 + m + a_2 \\
    y_3 =( {x_1 + x_3})m + x_3 + y_1
    \end{gather*}
    \end{enumerate}
\section{Умножение точки на число}
    $P \to \left[ k \right] \cdot P = \underbrace {P + P +  \ldots  + P}_{k{\text{ раз}}}$ (наивно: $k-1$ Операций сложения $k$ раз).
    
    \subsection{Алгоритм à la быстрое возведение в степень (бинарный метод)}
    $$k = \sum\limits_{j = 0}^{l - 1} {{k_j}{2^j}} ,\quad {k_j} \in \left\{ {0,1} \right\}
    $$
    
    \begin{enumerate}
    	\item $Q \leftarrow \mathcal{O}$
    	
    	\item For j = $l - 1$ to 0 by - 1:
    	
    	$Q \leftarrow \left[ 2 \right]Q = Q + Q$
    	
    	If ${K_j} = 1:$
    	
    	$\quad Q \leftarrow Q + P$
    	
    	\item Return Q
    \end{enumerate}
    
    \underline{Пример}
    
    $k = 5, k_0 = 1, k_2 = 1$
    
    $Q \leftarrow \mathcal{O}$
    
    $j = 2: \quad Q \leftarrow P$
    
    $j = 1: \quad Q = 2P$
    
    $j = 0: \quad Q = 4P + P = 5P$.
    
    Сложность: 
    
    количество операций дублирования точки: $\bigO(\lg K)$
      
    количество операций сложения двух точек: $\omega t(K)\sim O (\lg K)$ ($\omega$~--~вес Хэмминга $K$) $ \Rightarrow \bigO \left( {\lg k} \right)$ операций сложения (дублирования).
\section{Ускорение операции «+» с проективным координатами}
\begin{gather*}
    % y_2z = x^3+Axz^2+Bz^3\\
    % Y^2Z = X^3+ AXZ^2 + BZ^3\\
    x = \frac{X}{Z}, y = \frac{Y}{Z}\\
    \Rightarrow m = \frac{\frac{Y_2}{Z_2} - \frac{Y_1}{Z_1}}{\frac{X_2}{Z_2} - \frac{X_1}{Z_1}} = \frac{Y_2 Z_1 - Y_1 Z_2}{X_2 Z_1 - X_1 Z_2}
    \begin{array}{*{20}{c}}
    {: = u} \\ 
    {: = v} 
    \end{array} \\
    % \Rightarrow \frac{X_3}{Z_3} = {\left( {\frac{Y_2 Z_1 - Y_1 Z_1}{{X_2}{Z_1} - {X_1}{Z_2}}} \right)^2} - \frac{X_1}{Z_1} - \frac{X_2}{Z_2} = \frac{u^2}{v^2} - \frac{x_1}{z_1} - \frac{x_2}{z_2} \\
    % \frac{Y_3}{Z_3} = \frac{{Y_2}{Z_1} - {Y_1}{Z_2}}{{X_2}{Z_1} - {X_1}{Z_2}} \cdot \left( {\frac{X_1}{Z_1} - \frac{X_3}{Z_3}} \right) - \frac{Y_1}{Z_1} = \frac{u}{v} \left(2\frac{X_1}{Z_1} + \frac{u^2}{v^2} + \frac{x_2}{z_2} \right) - \frac{Y_1}{Z_3}. \\ \\
    \Rightarrow \frac{X_3}{Z_3} = \frac{u^2}{v^2} - \frac{X_1}{Z_1} - \frac{X_2}{Z_2}, \\
    \frac{Y_3}{Z_3} = \frac{u}{v}\left(\frac{X_1}{Z_1} - \frac{X_3}{z_3}\right) - \frac{Y_1}{Z_1}.\\\\
    \text{Пусть } Z_3 = v^3 Z_1 Z_2. \\
    % \Rightarrow {X_3} = {u^2}v{z_1}{z_2} - {z_2}{x_1}{v^2} - {x_1}{v^3}{z_1} \\
    % \left[ {{Y_3} = u\left( {2{x_1}{v^2}{z_2} + \underbrace {{u^2}v{z_1}{z_2}}_{} + {v^2}{z_1}{x_2}} \right)} \right]\\
    % X_3 = v\left( {u^2 z_1 z_2 - z_2 x_1 v^2 - x_2 z_1 v^2} \right) = v({u^2}{z_1}{z_2} - {v^2}\left( { - {z_2}{x_1} + {z_1}{x_1}} \right)) - 2 v^2 z_2 x_1\\
    % = v(\underbrace {{u^2}{z_1}{z_2} - {v^3} - 2{v^2}{z_2}{x_1}}_w)\\
    % \frac{Y_3}{Z_3} = \frac{u}{v}\left( {\frac{x_1}{z_1} - \frac{x_3}{z_3}} \right) - \frac{Y_1}{Z_1}\\
    \Rightarrow {X_3} = v(\underbrace{u^2 Z_1 Z_2 - v^3 - 2 v^2 X_1 Z_2}_w), \\
    Y_3 = u(X_1 v^2 Z_2 - w) - v^3 Z_2 Y_1
\end{gather*}

	В результате получаем алгоритм вычисления суммы точек за 12 умножений в $K$ (вместо 3х умножений + 1 деление в $K$).

\end{document}
