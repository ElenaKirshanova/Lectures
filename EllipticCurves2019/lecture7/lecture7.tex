\documentclass[12pt]{article}
\usepackage{amsmath,amssymb,amsthm}
\usepackage{mathtools}
\usepackage{algorithm}
\usepackage[noend]{algpseudocode} 
\usepackage{enumitem}

\usepackage{graphicx}

%---enable russian----

\usepackage[utf8]{inputenc} 
\usepackage[russian]{babel}


%---tikz----
\usepackage{tikz}
\usetikzlibrary{arrows, chains, matrix, positioning, scopes, patterns, shapes}
\usepackage{pgfplots, subfigure}

\usepackage{biblatex}

% MACROS 
% PROBABILITY SYMBOLS
\newcommand*\PROB\Pr 
\DeclareMathOperator*{\EXPECT}{\mathbb{E}}


% GROUPS/DISTRIBUTIONS/SETS/LISTS
\newcommand{\N}{{{\mathbb N}}}
\newcommand{\Z}{{{\mathbb Z}}}
\newcommand{\Q}{{{\mathbb Q}}}
\newcommand{\R}{{{\mathbb R}}}
\newcommand{\F}{{{\mathbb F}}}
\newcommand{\Fqd}{{{\F_{q^{d_i}}}}}
\newcommand{\PP}{{{\mathbb P}}}
\newcommand*{\IZ}{\ensuremath{\mathbb{Z}}}
\newcommand*{\IN}{\ensuremath{\N}}
\newcommand*{\IR}{{{\mathbb R}}}
\newcommand{\Zp}{\ints_p} % Integers modulo p
\newcommand{\Zq}{\ints_q} % Integers modulo q
\newcommand{\Zn}{\ints_N} % Integers modulo N
\newcommand{\Zr}{\ensuremath{\mathbb{Z}/r\mathbb{Z}}} % Integers modulo N
\newcommand*{\dDR}{\mathrm{d}} %de-Rham-Differential (the d in dx, dy, dz and so on)
\newcommand{\transpose}{\mkern0.1mu^{\mathsf{t}}}
\newcommand*{\union}{\mathbin{\cup}}

% Landau 
\newcommand{\bigO}{\mathcal{O}}
\newcommand*{\OLandau}{\bigO}
\newcommand*{\WLandau}{\Omega}
\newcommand*{\xOLandau}{\widetilde{\OLandau}}
\newcommand*{\xWLandau}{\widetilde{\WLandau}}
\newcommand*{\TLandau}{\Theta}
\newcommand*{\xTLandau}{\widetilde{\TLandau}}
\newcommand{\smallo}{o} %technically, an omicron
\newcommand{\softO}{\widetilde{\bigO}}
\newcommand{\wLandau}{\omega}
\newcommand{\negl}{\mathrm{negl}} 

% Misc
\newcommand{\eps}{\varepsilon}
\newcommand{\inprod}[1]{\left\langle #1 \right\rangle}
\DeclarePairedDelimiter{\abs}{\lvert}{\rvert}

%proper overline reduced by some mu's, re-adjust if needed
\newcommand{\overbar}[1]{\overline{\mkern-0.0mu#1\mkern-4.0mu}}

 
\newcommand{\handout}[5]{
  \noindent
  \begin{center}
  \framebox{
    \vbox{
      \hbox to 5.78in { {\bf  } \hfill #2 }
      \vspace{4mm}
      \hbox to 5.78in { {\Large \hfill #5  \hfill} }
      \vspace{2mm}
      \hbox to 5.78in { {\em #3 \hfill #4} }
    }
  }
  \end{center}
  \vspace*{4mm}
}

\newcommand{\lecture}[4]{\handout{#1}{#2}{#3}{Оформил #4}{Лекция #1}}

\newtheorem{theorem}{Теорема}
\newtheorem{corollary}[theorem]{Следствие}
\newtheorem{lemma}[theorem]{Лемма}
\newtheorem{observation}[theorem]{Observation}
\newtheorem{proposition}[theorem]{Предложение}

\theoremstyle{definition}
\newtheorem{definition}[theorem]{Определение}

\newtheorem{claim}[theorem]{Утверждение}
\newtheorem{fact}[theorem]{Факт}
\newtheorem{assumption}[theorem]{Предположение}

\theoremstyle{definition}
\newtheorem{examples}[theorem]{Примеры}

\theoremstyle{definition}
\newtheorem{example}[theorem]{Пример}
\newtheorem{remark}[theorem]{Замечание}

% 1-inch margins
\topmargin 0pt
\advance \topmargin by -\headheight
\advance \topmargin by -\headsep
\textheight 8.9in
\oddsidemargin 0pt
\evensidemargin \oddsidemargin
\marginparwidth 0.5in
\textwidth 6.5in

\parindent 0in
\parskip 1.5ex

%\addbibresource{books.bib}

\begin{document}
\lecture{№7 — 1.11.19}{Осень 2019}{Лектор: Елена Киршанова}{Филипп Максимов}

\section{Алгоритм факторизации на эллиптических кривых}

\subsection{(p-1)-метод Полларда}

Не умаляя общности, $N = pq$ (легко обобщается на случай нескольких простых множителей)\\
$p-1$ факторизуется на «малые» простые\\
$q-1$ \underline{не} факторизуется на «малые» простые

Точнее, $p-1 = \prod p_i^{e_i},\ p_i \leq B_1, p_i^{e_i} \leq B_2$ (такие $p-1$ называются "$B_1$-гладкими")

\underline{Идея метода:}
\begin{itemize}
    \item $\forall a \in \Z_N^*$ и $\forall K$ - кратное $p-1$:
    \[
        a^k = (a^l)^{p-1} \equiv 1\ mod\ p
    \]
    \item (Теорема Ферма) Если $a^k \not\equiv 1\ mod\ q$, то $GCD(N, a^K -1) = p$
\end{itemize}

\subsubsection{Алгоритм}

Вход: $N = p \cdot q$

Выход: $p,q = \frac{N}{p}$, или «делители не найдены».

\begin{enumerate}
	\item Выбрать ${B_1},{B_2}$ — границы.
	
	\[
	    a\mathop  \leftarrow \limits^\$  \Z_N^*
	\]
	
	\item Для всех простых ${p_i} \leqslant {B_1}$: 
	
	\hspace{3em} $a \leftarrow a^{p_i^{e_i}}\bmod N$, где $e_i$ — макс., удовлетворяющее $p_i^{e_i} \leqslant {B_2}$.
	
	\item Если $\gcd ( a - 1,N )\not  \in \left\{ {1,N} \right\}$
	
	\hspace{3em} вернуть $\gcd ( a - 1,N ),\quad \dfrac{N}{\gcd ( a - 1,\;N )}$. 
	
	иначе 
	
	\hspace{3em} вернуть "делители не найдены".
\end{enumerate}

\underline{Корректность}

\begin{lemma}
	Пусть $N = p \cdot q$, $B_1,B_2 \in \N$, т.ч. $( p - 1 )$ — $B_i$-гладкое и 

	$p - 1 = \Pi p_i^{e_i}$, $p_i^{e_i} \leqslant B_2$. A $(q-1)$ — не $B_i$-гладкое. 

	Тогда алгоритм $(p-1)$ Полларда находит $p$ за время $\bigO( B_1 \lg^3 N)$ с вероятностью $1  - \frac{1}{B_1}$.
\end{lemma}

\begin{proof}
    Положим $K = \prod\limits_{\begin{subarray}{c} 
            p_i \leqslant B_1 \\ 
            p_i - \text{простые}
    \end{subarray}}  p_i^{e_i} $

    Так как $( q - 1 )$ — не $B_1$-гладкое $\exists $ $r$ — простое, $r > B_1 : r|q - 1$. 
    
    Если $r|\operatorname{ord}_{\Z_q^*}( a )$, то ${\operatorname{ord} _{\Z_q^*}}( a ) \not |\ K \Rightarrow {a^K} \ne 1\bmod q$.
    
    С другой стороны, $k$ — кратно $p - 1 \Rightarrow {a^K} = 1\bmod p$ и $\text{gcd}(a^k - 1,N) = p$. 
    
    Т.е.  необходимо показать, что $r|{\operatorname{ord} _{\Z_q^*}}( a )$ с большой вероятностью для $a\mathop  \leftarrow \limits^\$  \Z_N^*$. 
    
    $\Z_q^* = \{ \alpha ^1 \ldots \alpha ^{q - 1}\}$ — циклическая группа, т.е.  $a\bmod q = \alpha ^i$ для $i \in ( 1,q-1 )$. 
    
    Кроме того, ${\operatorname{ord} _{\Z_q^N}}( {\alpha ^i} ) = \dfrac{q - 1}{\gcd (i,\;q - 1)}$
\end{proof}

\begin{lemma}
	Покажем, что ${\operatorname{ord} _{\Z_q^*}}( {\alpha'} ) = \dfrac{q - 1}{\gcd ( {i,\;q - 1} )}$; Пусть $t = \operatorname{ord} ( \alpha' )$
	\[
    	\left. \begin{gathered}
    	( \alpha ^i )^t = 1 \hfill \\
    	\operatorname{ord} ( \alpha  )^0 = q - 1 \hfill \\ 
    	\end{gathered}  \right\} \Leftrightarrow ( q - 1 )|i \cdot t;
	\]
\end{lemma}

\begin{proof}
Положим $( q - 1)m = i \cdot t$ $(m \in \Z)$. 

\[
	\begin{gathered}
	\gcd ( q - 1,\;i ) |\ q - 1, \text{ положим }( q - 1 ) = q' \cdot \gcd ( q - 1,i ) \hfill\\
	\gcd ( q - 1,\;i ) |\ i, \text{ положим } i = i' \cdot \gcd ( q - 1,i ) \hfill
	\end{gathered}
	 \Rightarrow \gcd ( q',i' ) = 1. 
\]

Заметим, что $q' = \dfrac{q - 1}{\gcd ( q - 1,i )}$; покажем, что $t = q'$. 

$( q - 1 )m = i \cdot t$

$q' \cdot \gcd ( q - 1,i ) \cdot m = i' \cdot \gcd ( q - 1,i ) \cdot t$

$q' \cdot m = i' \cdot t\ \Rightarrow\ q'|i' \cdot t$, т.к.  $\gcd ( q',i' ) = 1$, $q'|t\ \Rightarrow\ $ $q' \leqslant t$. 

Покажем обратное неравенство: 
\[
    ( \alpha ^i )^{q_i} = \alpha ^i \cdot \frac{q - 1}{\gcd ( {i,q - 1} )} = \alpha ^{(q - 1) \cdot i'} = ( 1 )^{i'} = 1\bmod q
\]
\[ 
    \Rightarrow \text{Вывод: }\frac{{t \leqslant q'}}{{t = q'}}
\]
\[
    r \not |\  \operatorname{ord} ( {{\alpha ^i}} ) \Leftrightarrow r|i
\] 
т.к.  $i$ — случайное число $[ 1 \ldots q - 1 ]$, $i$ — кратно $r$ с вероятностью $\dfrac{q}{r} \cdot \dfrac{1}{q} = \dfrac{1}{r} \Rightarrow r|\operatorname{ord} ( \alpha ^i )$ с вероятностью $1 - \frac{1}{r} > 1   - \frac{1}{B_1}\ ( r > B_1 )$.

\end{proof}

\underline{Сложность} Существует не более $B_1$ простых $p_i$, таких что $p_i < B_1$ (точнее $\exists \sim \dfrac{B_1}{\lg ( B_1 )}$)

Шаг 2: $\bigO( \lg^3 N )$

Шаг 3: $\bigO( \lg^2 N )$

$ \Rightarrow \bigO( B_1 \cdot \lg^3 N )$.

\underline{Замечание}. Вероятность успеха и сложность алгоритма зависят от $|\Z_p^*| = p - 1:$ Если $\frac{p - 1}{2}$ — простое (т.е.  $p - 1 = 2 \cdot p'$) $ \Rightarrow {B_1} \approx p \Rightarrow $ сложность $\bigO( p \cdot \lg^3 N )$ — не лучше наивного брутфорса. 

Решение использовать эллиптические кривые, т.к.  $\# E( \Z_p ) \in [ p + 1 - 2\sqrt p ,\;p + 1 + 2\sqrt p ]$, и в этом интервале существует много гладких чисел. 

\subsection{Эллиптические кривые $\bmod\ N$}

\[
    \begin{gathered}
    E( \Z_N ) = \left\{ {( {x,y} ) \in {\Z_N} \times {\Z_N}:\:{y^2} = {x^3} + a x + b\bmod N} \right. \hfill \\
    \left. {\text{для }\gcd ( \N,\;4a^3 + 27 b^2 ) = 1} \right\} \cup \left\{ 0 \right\} \hfill 
    \end{gathered} 
\]

\underline{Важно!} Точки на $E( \Z_N )$ не образуют аддитивную группу!

(Пример: $E:\:y^2 = x^3 + 1\bmod 55$, $P = ( 10,11 ) \in E$, \\
для вычисления $2P$, необходимо найти $( 2y )^{-1} =  2 \cdot 11 )^{-1} \bmod 55$, но $\gcd (22,25) = 1$\\
$\Rightarrow $ обратного $\not\exists$).

\subsubsection{Закон "+"\ на $E( \Z_N )$:}

\underline{Вход} $P,Q \in E( \Z_N )$ $( {P,Q \ne \bigO} )$; 

\underline{Выход} либо $P + Q = ( x_3,y_3 )$,

\hspace{6ex} либо $d|N$. 

\begin{enumerate}
	\item Если $x_1 \equiv x_2\bmod N$ и $y_1 =  - y_2\bmod N$
	
	\hspace{5ex} Вернуть $\bigO$
	
	\item $d = \gcd (x_1 - x_2,N)$
	
	\hspace{5ex} Если $d\not  \in \left\{1,N\right\}$

	\hspace{10ex} Вернуть $d$
	
	\item Если $x_1 \equiv x_2\bmod N$
	
	\hspace{5ex} $d = \gcd (y_1 + y_2,\;N)$
	
	\hspace{5ex} Если $d > 1$
	
	\hspace{10ex} Вернуть $d$
	
	\item 
    \[
    	\alpha  = \left\{ \begin{gathered}
    	\frac{y_2 - y_1}{x_2 - x_1},\quad x_1 \ne x_2 \hfill \\
    	\frac{3x_1^2 + a}{y_1 + y_2},\quad x_1 = x_2 \hfill \\ 
    	\end{gathered}  \right.
	\]
	\[
    	\beta  = y_1 - \alpha x_1
	\]
	
	\item $x_3 = \alpha ^2 - x_1 - x_2\bmod N$
	
	$x_3 =  - ( \alpha x_3 + \beta  )\bmod N$
	
	Вернуть $( x_3, y_3 )$
	
\end{enumerate}

\begin{theorem}
	Пусть $P,Q \in E( \Z_N )$. 
	
	Тогда $P + Q$ на $E( \Z_N )$ либо идентично сложению на $E( \F_p ), E( \F_q )$,\\
	либо дает делитель $N$. 
\end{theorem}
\begin{proof}
$P = ( x_1 , y_1 ), Q( x_2, y_2 )$.

\underline{Случай 1.} $P + Q = \bigO$ на $E( \F_p )$ и на $E( \F_q )$
\[
    \begin{gathered}
    \Rightarrow \left\{ 
        \begin{gathered}
        \left| 
            \begin{gathered}
            x \equiv x_1 \hfill \\
            {y_1} \equiv y_2 \hfill \\ 
            \end{gathered} 
        \right.\bmod p \hfill \\
        \left| 
            \begin{gathered}
            x \equiv x_1 \hfill \\
            {y_1} \equiv y_2 \hfill \\ 
            \end{gathered} 
        \right.\bmod q \hfill \\ 
        \end{gathered}  
    \right\} \Rightarrow 
        \begin{array}{*{20}{c}}
        {x = x_1} \\ 
        {x = {y_1}} 
        \end{array}
    \bmod N \Rightarrow \hfill \\
    \Rightarrow P + Q = \bigO{\text{ на }}E( {{\F_N}} ) \hfill \\ 
    \end{gathered} 
\]

\underline{Случай 2.} $P + Q \ne \bigO$ на $E( \F_p )$, $E( \F_q )$. 

2.1. $x_1\not  \equiv x_2\bmod p$ и $x_1\not  \equiv x_2\bmod q$ \\
$ \Rightarrow $ формулы сложения $E( \F_p ), E( \F_q )$, $E( N )$ идентичны. 

2.2. $x_1 \not\equiv x_2\bmod p$, $x_1 \equiv x_2\bmod q \Rightarrow $ 

\hspace{5ex} Шаг 2:  $\gcd ( {x_1 - x_2,N} ) = q$

(Аналогично $x_1 = x_2\bmod p$, $x_1 \ne x_2\bmod q$)\\

2.3. $\left\{ \begin{gathered}
    x_1 = x_2\bmod N \hfill \\
    {y_1} \ne -y_2\bmod p \hfill \\ 
    \end{gathered}  \right. 
\Rightarrow $ уравнение $y^2 = x_1^3 + ax_1 + b$ (для $y$) имеет в точности 2 решения.

$y_{1,2} = \pm y\bmod p$, т.ч. ${y_1} \ne  - y_2\bmod p \Leftrightarrow y_1 = y_2\bmod p$.

В таком случае ${y_1} + y_2 = 2{y_1}\bmod p$, формулы сложения идентичны (то же самое при $q \leftrightarrow p$).
\end{proof}

\begin{corollary}
	Пусть $P + Q = \bigO$ на $E( \F_p )$ и $P + Q \ne \bigO$ на $E( \F_q )$. Тогда $P + Q$ на $E(\F_N)$ даст делитель $N$. 
\end{corollary}

\begin{proof}
    $P + Q = \bigO$ на $E( \F_p ) \Leftrightarrow \left\{ \begin{gathered}
    x_1 \equiv x_2\bmod p \hfill \\
    y_1 \equiv y_2\bmod p \hfill \\ 
    \end{gathered}  \right.$

    $P + Q \ne \bigO$ на $E\left( \F_q \right) \Leftrightarrow \left[ \begin{gathered}
    x_1\not  \equiv x_2\bmod q \Rightarrow \gcd \left( x_1 - x_2,N \right) = p\left( \text{Шаг 2} \right) \hfill \\
    y_1\not  \equiv  - y_2\bmod q \Rightarrow \gcd \left( y_1 + y_2,N \right) = q. \hfill \\ 
    \end{gathered}  \right.$
\end{proof}

\underline{Алгоритм факторизации ЕСМ}

Вход: $N = p \cdot q\quad ( p\sim q )$

Выход: $p, q$ или "делители не найдены"

\begin{enumerate}
	\item Выберем границы ${B_1},{B_2}$
	
	\item Выберем $( a,x,y ) \leftarrow \Z_N \times \Z_N \times \Z_N$
	
	\hspace{3ex} $b = y^2 - x^3 - ax\bmod N$ // Таким образом, мы выбираем точку с координатами $(x,y)$ на кривой $y^2 = x^3 + ax + b$.
	
	\item Если $\gcd ( 4a^3 + 27 b^2,\;N) = \left\{ \begin{gathered}
	1,\quad {\text{положим}}\quad P = (x,y) \hfill \\
	N,\quad {\text{идем на шаг 2}} \hfill \\
	{\text{иное}}{\text{, вернуть }}p,q \in \left\{p,q\right\} \hfill \\ 
	\end{gathered}  \right.$
	
	\item Для всех простых $p_1 < B_1$: 
	\[
	    P = p_i^{e_i} \cdot P{\text{ на }}E( \Z_N ){\text{ т.е. }}p_i^{e_i} < {B_i}
	\]
	
	Если какое-либо вычисление "+"\ на $E( \Z_N )$ позволяет делитель $N$, вернуть его.
	
	\item Либо повторить с Шаг 2, либо вернуть "делитель не найден". 
\end{enumerate}

\underline{Корректность}
\begin{lemma}
	Пусть $N = p \cdot q$, $E( \Z_N )$ — эллиптическая кривая, т. е. $| E( \F_p )|$ — $B_1$ — гладкое и $|E( \F_q )|$ — не $B_i$-гладкое. Тогда алгоритм ЕСМ возвращает $p,q$ за время $\bigO( B_1 \lg^3 N )$ с вероятностью $ \geqslant 1 - \frac{1}{B_1}$. 
\end{lemma}

\begin{proof}
 Пусть $K = \prod\limits_{\begin{subarray}{c} 
{p_i} - {\text{простое}} \\ 
{p_i} \leqslant {B_1} 
\end{subarray}}  p_i^{e_i} $

Так как $E( \F_q )$ — не $B_i$-гладкое, то $\exists r > B_1$, т. ч. 

Если $r|\operatorname{ord}_{E( \F_q )}( P )$, то $kP \ne \bigO$ на $E( \F_q )$.

С другой стороны, $K$ — кратно $\# E( \F_p ) \Rightarrow k \cdot P = \bigO$ на $E( \F_q )$. 

Т. е. когда мы вычисляем $kP$ на $E( \Z_N )$ мы получаем 
\[
    \begin{gathered}
    P' + Q' = \bigO\text{ на }E( \F_p ) \hfill \\
    P' + Q' = \bigO\text{ на }E( \F_q ) \hfill \\ 
    \end{gathered}  
\Rightarrow \text{по следствию 5 алгоритм вернет }( p,q ). 
\]
сложность и вероятность — аналогично $( p - 1 )$-методу. 
\end{proof}

\underline{Замечание} Баланс выбора $B_1$:

\hspace{5em} малое $B_1 \Rightarrow $ быстрый алгоритм, малая вероятность успеха 

\hspace{5em} большое $B \Rightarrow $ медленный алгоритм, большая вероятность успеха. 

Оптимально: ${B_1} \approx {L_p}[ \frac{1}{2},\;\frac{1}{\sqrt 2 }] = e^{\frac{1}{\sqrt 2 }(\log p)^\frac{1}{2}(\log\log p)^\frac{1}{2}} \Rightarrow $ время работы алгоритма: ${L_p}[ \frac{1}{2},\;\frac{1}{\sqrt 2 }]$ при предположении о гладкости чисел в интервале 
$[ {p + 1 - 2\sqrt p ,\;p + 1 + 2\sqrt p } ]$. 

ЕСМ — лучший на сегодня алгоритм для нахождения делителей $< 100$ бит. 
\end{document}