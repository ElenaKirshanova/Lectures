\documentclass[11pt]{article}
\usepackage{amsmath,amssymb,amsthm}
\usepackage{algorithm}
\usepackage[noend]{algpseudocode} 

%---enable russian----

\usepackage[utf8]{inputenc}
\usepackage[russian]{babel}


%---tikz----
\usepackage{tikz}
\usetikzlibrary{arrows, chains, matrix, positioning, scopes, patterns, shapes}

%----newcommands (math)----
\input{header} 


%-----newcommands specific to the scribe template 
\newcommand{\handout}[5]{
  \noindent
  \begin{center}
  \framebox{
    \vbox{
      \hbox to 5.78in { {\bf  } \hfill #2 }
      \vspace{4mm}
      \hbox to 5.78in { {\Large \hfill #5  \hfill} }
      \vspace{2mm}
      \hbox to 5.78in { {\em #3 \hfill #4} }
    }
  }
  \end{center}
  \vspace*{4mm}
}

\newcommand{\lecture}[4]{\handout{#1}{#2}{#3}{Выполнил(а): #4}{#1}}


\newtheorem{theorem}{Теорема}
\newtheorem{corollary}{Следствие}
\newtheorem{lemma}{Лемма}
\newtheorem{observation}{Observation}
\newtheorem{proposition}{Предложение}
\newtheorem{definition}{Определение}
\newtheorem{remark}{Замечание}
\newtheorem{claim}{Утверждение}
\newtheorem{fact}{Факт}
\newtheorem{assumption}{Предположение}

% 1-inch margins
\topmargin 0pt
\advance \topmargin by -\headheight
\advance \topmargin by -\headsep
\textheight 8.9in
\oddsidemargin 0pt
\evensidemargin \oddsidemargin
\marginparwidth 0.5in
\textwidth 6.5in

\parindent 0in
\parskip 1.5ex

\begin{document}

\lecture{Редукция базиса решетки}{25 января 2021 г.}{Лектор: Елена Киршанова}{ГЛАДКИЙ ДЕНИС}


\section{HNF (Hermite Normal Form): Эрмитова нормальная форма}

\[
	\forall B \in \mathbb{Z}^{n \times k} \; \exists \; U \in GL_k(\mathbb{Z}): \; B \cdot U =
	\begin{pmatrix}
		\begin{matrix} 
		0 & 0 & \dots & 0 \\
		x \\
		* & x \\ 
		\vdots & \hdots & \ddots & \\
		* & \hdots & * & x 
		\end{matrix}
		\quad
		\vline	
	    & \hbox{\Huge0}
	\end{pmatrix},
\]
и коэффициенты в строке с элементом $ x $ на главной диагонали лежат в интервале $ [0,x) $.

Полученная матрица уникальна для $ B $ и носит название HNF формы $ B $.

Находится HNF аналогично "Гауссовому преобразованию" (Gaussian elimination), где деление заменено на НОД.

\textit{Приложение}: $ B_1, B_2 $ - базисы $ L_1, L_2 \subseteq \mathbb{Z}^n $, HNF позволяет вычислить базис $ L_1 + L_2 = B_1 \cdot \mathbb{Z}^n + B_2 \cdot \mathbb{Z}^n $, а именно HNF($ B_1 \parallel B_2 $).

Сложность вычисления HNF:
$ \mathcal{O}(\max(n,k)^{\omega + 1} \cdot \lg{ \max{ \norm{b_i} } } ) $ - битовая сложность.


\section{QR - факторизация}

\begin{definition} \label{def:QRFactorization}
	Пусть $ B \in \mathbb{Z}^{n \times n}, \; \det{B} \neq 0. \; \exists Q $ - ортогональная ($ Q^T \cdot Q = Q \cdot Q^T = Id $) и R - треугольная матрица: $ B = Q \cdot R, \; r_{ii} > 0 \; \forall \; i $. \\
	Такая декомпозиция уникальна. \\
	QR-факторизация связана с ортогонализацией Грам-Шмидта:
	\begin{align*}
	&b_1^* = b_1 \\
	&b_1^* = b_1 - \sum_{j<i} \mu_{i,j}b_i^*, \; \mu_{i,j} = \frac{\ScProd{b_i}{b_j^*}}{\norm{b_j^*}^{2}} \\
	&B = Q \cdot R = \underbrace{Q \cdot diag(r_{ii})}_{B^*} \cdot \underbrace{diag(r_{ii})^{-1} \cdot R}_{\left( \mu_{i,j} \right)^T_{i,j} }
	\end{align*}
\end{definition}

\begin{remark} \label{rem1}
	QR-факторизация и ортогонализация Грам-Шмидта несут одну и ту же информацию о решётке. \\
	1) Q, R не обязаны быть рациональными; $ B^*, \mu $ - рациональны для $ B \in  \mathbb{Z}^{n \times n} $, и битовая длина числителя и знаменателя элементов  $ B^*, \mu $ полиномиальна от $ \lg (b_{i,j}) $.
\end{remark}

Сложность: $ \mathcal{O}(n^3) $ арифметическихз операций.

\begin{lemma} \label{lem:1}
	$ \forall x: \; \norm{ Bx } = \norm{ Rx }, \; B = QR $
\end{lemma}
\begin{proof}
	$  \norm{ Bx } = \norm{ QRx} = \norm{ Qy } = \sqrt{\ScProd{Qy}{Qy}} = \sqrt{\ScProd{y^TQ^T}{Qy}} = \\
	= \sqrt{y^T \cdot \underbrace{Q^T \cdot Q}_{Id} \cdot y} = \sqrt{y^T \cdot y} = \norm{ y } = \norm{ Rx } $.
\end{proof}

\begin{lemma} \label{lem:2}
	\[
	L = L(B), \; B = QR; \qquad \lambda_1(L) \geq \min_i (r_{ii});
	\]
\end{lemma}
\begin{proof}
	\begin{align*}
	&b = B \cdot x, \; x \in \mathbb{Z}^n, \; b = \lambda_1(L) \\
	&\hbox{Лемма 7} \Rightarrow \norm{ b } = \norm{ QRx } = \norm{ Rx } = \norm{ (..., \; r_{n-1,n-1} \cdot X_{n-1} + r_{n-1,n} \cdot X_n, \; r_{n,n} \cdot X_n) }
	\end{align*}
	\begin{itemize}
		\item $ X_n \neq 0, \hbox{ тогда} \norm{ (..., \; ..., \; \underbrace{r_{n,n} \cdot X_n}_{\neq 0}) } \; \geq r_{n,n}$;
		\item $ X_n = 0, \; X_{n-1} \neq 0, \hbox{ тогда} \norm{ (..., \; r_{n-1,n-1} \cdot X_{n-1}, \; 0) } \; \geq r_{n-1,n-1}$;
		\item $ X_{n-1} = X_n = 0, \; X_{n-2} \neq 0, \hbox{ тогда} \norm{ (..., \;  r_{n-2,n-2} \cdot X_{n-2}, \; 0, \; 0) } \; \geq r_{n-2,n-2}$;
	\end{itemize}
	Рассматриваем $ max \; i: x_i \neq 0 $.
\end{proof}
 

\end{document}