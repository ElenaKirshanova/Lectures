\documentclass[11pt]{article}
\usepackage{amsmath,amssymb,amsthm}
\usepackage{algorithm}
\usepackage[noend]{algpseudocode} 

%---enable russian----

\usepackage[utf8]{inputenc}
\usepackage[russian]{babel}


%---tikz----
\usepackage{tikz}
\usetikzlibrary{arrows, chains, matrix, positioning, scopes, patterns, shapes}

%----newcommands (math)----
% ==================================================================
% Definitions for this paper
% ==================================================================
\mathchardef\hyphen="2D

\usepackage{multirow}
\usepackage{multicol} % For multiple coloumn environments
%\usepackage{stmaryrd} % For set brackets
% \setlength{\columnsep}{15pt} % Defining the coloumn seperation
% \setlength{\columnseprule}{1pt} % Place a line between coloumns
% \newcommand{\tab}{\hspace*{2em}}

%subscripts

\newcommand*\SmallTextScript[2]{{\mathchoice{\displaystyle #2}
		{\textstyle #2}%dito
		{\scalebox{#1}{\ensuremath{\scriptstyle #2}}}%
		{\scalebox{#1}{\ensuremath{\scriptscriptstyle #2}}}%
}}


% ADVERSARIES AND SUCH
\newcommand*{\poly}{\ensuremath{\mathrm{poly}}}
\newcommand*{\eps}{\ensuremath{\varepsilon}}
\newcommand*{\alg}{\ensuremath{\mathcal{A}}}

% GROUPS/DISTRIBUTIONS/SETS/LISTS
\newcommand{\N}{{{\mathbb N}}}
\newcommand{\Z}{{{\mathbb Z}}}
\newcommand*{\IZ}{\ensuremath{\mathbb{Z}}}
\newcommand*{\IN}{\ensuremath{\mathbb{N}}}
\newcommand*{\IQ}{\ensuremath{\mathbb{Q}}}
\newcommand{\R}{{{\mathbb R}}}
\newcommand*{\IR}{{{\mathbb R}}}
\newcommand{\Zp}{\ints_p} % Integers modulo p
\newcommand{\Zq}{\ints_q} % Integers modulo q
\newcommand{\Zn}{\ints_N} % Integers modulo N
\newcommand{\F}{\ensuremath{\mathbb{F}}}
\newcommand{\CC}{\ensuremath{\mathbb{C}}}

\newcommand{\GF}{\ensuremath{\mathbb{F}_2}}
\newcommand{\GFn}{\ensuremath{\mathbb{F}^n_2}}

%%% ALGORITHMS/PROCEDURES %%%
\newcommand{\Dec}{\textsf{Dec}}
\newcommand{\Enc}{\textsf{Enc}}
\newcommand{\KeyGen}{\textsf{KeyGen}}
\newcommand{\Gen}{\textsf{Gen}}
\newcommand{\sk}{\textsf{sk}}
\newcommand{\pk}{\textsf{pk}}
\newcommand{\vk}{\textsf{vk}}
\newcommand{\mesS}{\ensuremath{\mathcal{M}}}
\newcommand{\keyS}{\ensuremath{\mathcal{K}}}
\newcommand{\cipS}{\ensuremath{\mathcal{C}}}
\newcommand{\tagS}{\ensuremath{\mathcal{T}}}
\newcommand{\mactag}{\textsf{tag}}
\newcommand{\Hash}{\ensuremath{\mathcal{H}}}
\newcommand{\EID}{\ensuremath{\mathtt{EphID}}}


\newcommand{\adv}{\ensuremath{\mathcal{A}}}

\newcommand{\LWE}{\mathsf{LWE}}
\newcommand{\DCP}{\mathsf{DCP}}
\newcommand{\EDCP}{\mathsf{EDCP}}
\newcommand{\UEDCP}{\mathsf{U \text{-} EDCP}}
\newcommand{\GEDCP}{\mathsf{G \text{-} EDCP}}



%% Landau and proba
\newcommand{\bigO}{\mathcal{O}}
\newcommand*{\OLandau}{\bigO}
\newcommand*{\WLandau}{\Omega}
\newcommand*{\xOLandau}{\widetilde{\OLandau}}
\newcommand*{\xWLandau}{\widetilde{\WLandau}}
\newcommand*{\TLandau}{\Theta}
\newcommand*{\xTLandau}{\widetilde{\TLandau}}
\newcommand{\smallo}{o} %technically, an omicron
\newcommand{\wLandau}{\omega}
\newcommand{\negl}{\mathrm{negl}}
\newcommand*\PROB\Pr 
\DeclareMathOperator*{\EXPECT}{\mathbb{E}}
\DeclareMathOperator*{\VARIANCE}{\mathbb{V}}
\DeclareMathOperator*{\LOGBIAS}{\mathbb{LB}}

\newcommand{\supp}{\ensuremath{\mathsf{sup}}}
\newcommand{\Distr}{\ensuremath{\mathcal{D}}}

% Lattices

% \newcommand{\coset}{\Lambda} % Lambda Lattice
% \newcommand{\cosetPerp}{\Lambda^{\bot}} % Lambda_Perp Lattice
% \newcommand{\gadget}{\textbf{G}} %Gaget matrix
% \newcommand{\mes}{\textbf{m}} %message vector
% \newcommand{\AMat}{\textbf{A}} %A matrices
% \newcommand{\BMat}{\textbf{B}} %B matrices
% \newcommand{\RMat}{\textbf{R}} %R matrices
% \newcommand{\HMat}{\textbf{H}} %H matrices
% \newcommand{\XMat}{\textbf{X}} %H matrices
% \newcommand{\mbar}{\bar{m}} %mBar dimension
% % \newcommand{\gauss}{\mathcal{D}} % gaussian distribution
% \newcommand{\Id}{\textbf{I}} % Identity matrix
% \newcommand{\er}{\textbf{e}} % gaussian distr. vectors
% % \newcommand{\cipher}{\textit{c}} % ciphertext
% \newcommand{\Olwe}{\mathcal{O}_{\textsf{LWE}}} %LWE oracle
% \newcommand{\OSample}{\mathcal{O}_{Sample}} %LWE oracle
% \newcommand{\SigmaB}{\boldsymbol{\Sigma}} %semi-deifinite matrix Sigma%
% % \newcommand{\mods}{\text{ mod}}


%Vectors and Matrices

\newcommand{\AMat}{\mathbf{A}} %A matrices
\newcommand{\BMat}{\mathbf{B}} %B matrices
\newcommand{\DMat}{\mathbf{D}} %Diagonal


\newcommand{\HMat}{\ensuremath{\mathbf{H}}}
\newcommand{\QMat}{\ensuremath{\mathbf{Q}}}
\newcommand{\Id}{\ensuremath{\mathbf{I}}}
\newcommand{\ZeroM}{\textbf{0}} % Zero matrix

\newcommand{\avec}{\ensuremath{\mathbf{a}}}
\newcommand{\bvec}{\ensuremath{\mathbf{b}}}
\newcommand{\cvec}{\ensuremath{\mathbf{c}}}
\newcommand{\evec}{\ensuremath{\mathbf{e}}}
\newcommand{\rvec}{\ensuremath{\mathbf{r}}}
\newcommand{\svec}{\ensuremath{\mathbf{s}}}
\newcommand{\tvec}{\ensuremath{\mathbf{t}}}
\newcommand{\vvec}{\ensuremath{\mathbf{v}}}
\newcommand{\zvec}{\ensuremath{\mathbf{z}}}
\newcommand{\xvec}{\ensuremath{\mathbf{x}}}
\newcommand{\yvec}{\ensuremath{\mathbf{y}}}
\newcommand{\uvec}{\ensuremath{\mathbf{u}}}
\newcommand{\zerovec}{\ensuremath{\mathbf{0}}}

\newcommand{\nth}{^{\mathrm{th}}}
\newcommand{\nd}{^{\mathrm{nd}}}

\newcommand{\RepMMT}{\ensuremath{\mathcal{R}_{\protect\SmallTextScript{0.70}{\texttt{MMT}}}}}
\newcommand{\RepBJMM}{\ensuremath{\mathcal{R}_{\protect\SmallTextScript{0.70}{\texttt{BJMM}}}}}
\newcommand{\XOR}{\ensuremath{\mathtt{3XOR}}}


% % % % % \newcommand{\mb}[1]{\mathbf{#1}} % does not compile otherwise
%%% Removed by Gotti; this is just asking to screw up with packages that (properly) define \mb (mathbold)

% \newcommand{\bL}{\|\bvec_1\|} % b1 length that appears way too often
% \newcommand{\dL}{\|\dvec_1\|} % b1 length that appears way too oftend

%Norms and Scalar products

\newcommand*\abs[1]{\left\lvert#1\right\rvert}
\newcommand*\norm[1]{\left\lVert#1\right\rVert}
\newcommand*\normalabs[1]{\lvert#1\rvert} 
\newcommand*\normalnorm[1]{\lVert#1\rVert}
\newcommand*\bignorm[1]{\bigl\lVert#1\bigr\rVert}
\newcommand*\bigabs[1]{\bigl\lvert#1\bigr\rvert}
\newcommand*\Bigabs[1]{\Bigl\lvert#1\Bigr\rvert}
\newcommand*{\ScProd}[2]{\ensuremath{\langle#1\mathbin{,}#2\rangle}} %Scalar Product
% \newcommand*{\ScProd}[2]{\ensuremath{\langle#1 \:{,}\:#2\rangle}} %Scalar Product
\newcommand*{\bigScProd}[2]{\ensuremath{\bigl\langle#1\mathbin{,}#2\bigr\rangle}} %Scalar Product
\newcommand*{\BigScProd}[2]{\ensuremath{\Bigl\langle#1\mathbin{,}#2\Bigr\rangle}} %Scalar Product
\newcommand{\dist}{\ensuremath{\text{dist}}}


%Some other math operators

\DeclareMathOperator{\Span}{Span} %span of vectors
\DeclareMathOperator{\vol}{\mathrm{vol}} %volume
\DeclareMathOperator{\LW}{LambertW} %Lambert W function
\DeclareMathOperator{\SD}{SD}
\DeclareMathOperator{\gradient}{grad}
\DeclareMathOperator{\TRACE}{Tr}
\newcommand*{\dDR}{\mathrm{d}} %de-Rham-Differential (the d in dx, dy, dz and so on)


%Lists
\renewcommand{\L}{\ensuremath{\mathcal{L}}}

\renewcommand{\P}{\ensuremath{\mathcal{P}}}

\newcommand*{\Lout}{\ensuremath{\L_{\mkern-0.5mu\protect\SmallTextScript{0.85}{\textup{out}}}}}
\newcommand*{\Sout}{\ensuremath{S_{\mkern-0.5mu\protect\SmallTextScript{0.85}{\textup{out}}}}}
\newcommand{\wt}{\ensuremath{\mathit{wt}}}


\newcommand*{\softO}{\widetilde{\bigO}}

\newcommand{\const}{\mathsf{c}} 


\newcommand{\transpose}{\mkern0.7mu^{\mathsf{ t}}}

%proper overline reduced by 1.5mu
\newcommand{\overbar}[1]{\mkern 1.5mu\overline{\mkern-1.5mu#1\mkern-1.5mu}\mkern 1.5mu}

\DeclareMathOperator{\erf}{erf} %error function
\DeclareMathOperator{\erfc}{erfc} %complementary error function
\newcommand{\Er}{\ensuremath{\mathrm{Er}}} %complementary error function


% LATTICES

\newcommand{\Lat}{\ensuremath{\mathcal{L}}}
\newcommand*{\Sphere}[1]{\ensuremath{\mathsf{S}^{#1}}}
%\DeclareMathOperator{\Conf}{Conf}
\newcommand{\Conf}{\mathcal{C}}

%Thick line for table
\setlength{\doublerulesep}{0pt}
\newcommand{\thickline}{\hline\hline\hline}


%circled text
\newcommand*\circled[1]{\tikz[baseline=(char.base)]{
    \node[shape=circle,draw,inner sep=0.3 pt] (char) {\scriptsize #1};}}


%Fix Algorithmicx package
\def\NoNumber#1{{\def\alglinenumber##1{}\State #1}\addtocounter{ALG@line}{-1}}

%For comments
\newcommand{\GColor}{ForestGreen}  %Damiens' color
\newcommand{\EColor}{MidnightBlue} %Elena's color

\newcommand*{\E}[1]{{\color{\EColor} #1} } 
\newcommand*{\G}[1]{{\color{\GColor} #1} } 

%Proper limit with the subscript underneath
% \newcommand{\Lim}[1]{\raisebox{0.5ex}{\scalebox{0.8}{$\displaystyle \lim_{#1}\;$}}}


%TIKZ dense dotted pattern

\pgfdeclarepatternformonly{my dots}{\pgfqpoint{-1pt}{-1pt}}{\pgfqpoint{2.0pt}{2.0pt}}{\pgfqpoint{2pt}{2pt}}%
{
	\pgfpathcircle{\pgfqpoint{0pt}{0pt}}{.35pt}
	\pgfpathcircle{\pgfqpoint{1pt}{1pt}}{.35pt}
	\pgfusepath{fill}
}


\tikzset{
	master/.style={
		execute at end picture={
			\coordinate (lower right) at (current bounding box.south east);
			\coordinate (upper left) at (current bounding box.north west);
		}
	},
	slave/.style={
		execute at end picture={
			\pgfresetboundingbox
			\path  (lower right)rectangle (upper left) ;
		}
	}
} 


%-----newcommands specific to the scribe template 
\newcommand{\handout}[5]{
  \noindent
  \begin{center}
  \framebox{
    \vbox{
      \hbox to 5.78in { {\bf  } \hfill #2 }
      \vspace{4mm}
      \hbox to 5.78in { {\Large \hfill #5  \hfill} }
      \vspace{2mm}
      \hbox to 5.78in { {\em #3 \hfill #4} }
    }
  }
  \end{center}
  \vspace*{4mm}
}

\newcommand{\lecture}[4]{\handout{#1}{#2}{#3}{Выполнил(а): #4}{#1}}


\newtheorem{theorem}{Теорема}
\newtheorem{corollary}{Следствие}
\newtheorem{lemma}{Лемма}
\newtheorem{observation}{Observation}
\newtheorem{proposition}{Предложение}
\newtheorem{definition}{Определение}
\newtheorem{remark}{Замечание}
\newtheorem{claim}{Утверждение}
\newtheorem{fact}{Факт}
\newtheorem{assumption}{Предположение}

% 1-inch margins
\topmargin 0pt
\advance \topmargin by -\headheight
\advance \topmargin by -\headsep
\textheight 8.9in
\oddsidemargin 0pt
\evensidemargin \oddsidemargin
\marginparwidth 0.5in
\textwidth 6.5in

\parindent 0in
\parskip 1.5ex

\begin{document}

\lecture{Теорема Минковского}{25 января 2021 г.}{Лектор: Елена Киршанова}{ГЛАДКИЙ ДЕНИС}

\section{Теорема Минковского}

\begin{theorem}[Минковского] \label{thm:theoremMinkowski}
	Для решётки $ L \subseteq \mathbb{R}^d $ ранга d справедливо:
	
	\begin{equation} \label{eq1}
	\lambda_1(L) \leq \sqrt{d}\cdot(\det{L})^{1/d} \qquad 
	\lambda_1(L) = \min_{\substack{b \in L \\ b \neq 0}} \norm{ b }
	\end{equation}
	\begin{equation} \label{eq2}
	\lambda_1^{\infty}(L) \leq (\det{L})^{1/d} \qquad 
	\lambda_1^{\infty}(L) = \min_{\substack{b \in L \\ b \neq 0}} \norm{ b }_{\infty} \qquad
	\norm{ \cdot }_{\infty} = \max_i \abs{ b_i }
	\end{equation}
\end{theorem}

Для доказательства Теоремы Минковского рассмотрим 2 другие теоремы.

\begin{theorem} \label{thm:theorem2}
	Пусть $ S \subseteq \mathbb{R}^d $ - симметрическое, выпуклое множество, т.ч. $ \vol{S} > 2^d \cdot \det{L} $.
	Тогда S содержит ненулевой вектор L.
	Если S компактно, достаточно условия \\
	$ \vol{S} \geq 2^d \cdot \det{L} $ 
\end{theorem}

Из Теоремы~\ref{thm:theorem2} следует Теорема Минковского:
\begin{align*}
S &= [\; \text{-}(\det{L})^{1/d}, (\det{L})^{1/d} \;] \\
\vol{S} &= 2^d \cdot ((\det{L})^{1/d})^d = 2^d \cdot \det{L} \\
\hbox{В } S \; \exists b \in L\backslash\{0\} \hbox{ и}
\norm{ b }_{\infty} &\leq (\det{L})^{1/d} \Rightarrow \hbox{выполняется } \eqref{eq2} \\
\norm{ b }_2 &\leq \sqrt{d} \cdot \norm{ b }_{\infty} \leq \sqrt{d} \cdot (\det{L})^{1/d} \Rightarrow \hbox{выполняется } \eqref{eq1}
\end{align*}

\begin{theorem}[Блихфельд] \label{thm:theoremBlichfeld}
	Пусть $ L \subseteq \mathbb{R}^d $ - решётка, $ E \subseteq \mathbb{R}^d $, т.ч. $ \vol{E} > \det{L} $.
	Тогда $ \exists \; z_1 \neq z_2 \in E $, т.ч. $ z_1 - z_2 \in L $.	
\end{theorem}

Из Теоремы~\ref{thm:theoremBlichfeld} Блихфельда следует Теорема~\ref{thm:theorem2}:
\begin{align*}
\hbox{В качестве } E = S/2. \hbox{ Тогда }  \vol{E} > \det{L} \Rightarrow \exists \; z_1 \neq z_2 \in E: \; z_1 - z_2 \in L; \\
z_1 - z_2 = 2 \cdot \dfrac{z_1 - z_2}{2} \Rightarrow \dfrac{z_1 - z_2}{2} \in \dfrac{S}{2} = E \Rightarrow z_1 - z_2 \in E \Rightarrow \exists \; z_1 - z_2 \neq 0 \in L.
\end{align*}

\begin{proof}
Рассмотрим $ \displaystyle \bigcup_{b \in L} \{\mathcal{P} + b\} $ - это разбиение $ \mathbb{R}^d $.
\[
E = \coprod_{b \in L} \{E \cap \mathcal{P} + b\}; \qquad \coprod - \hbox{ неперескающееся объединение}.
\]
\[
\vol{\mathcal{P}} = \vol{L} < \vol{E} = \sum_{b \in L}\vol{(E \cap (\mathcal{P} + b))} = \sum_{b \in L}\vol{((E - b) \cap \mathcal{P})}; \qquad (E - b) \cap \mathcal{P} \in \mathcal{P}
\]
\[
\exists \; b_1 \neq b_2 \in L: \; ( (E - b_1) \cap \mathcal{P}) \cap  ((E - b_2) \cap \mathcal{P}) \neq \emptyset
\]
\[
\hbox{Возьмём } z \in ( (E - b_1) \cap \mathcal{P}) \cap  ((E - b_2) \cap \mathcal{P}), \; z_1 = (z + b_1) \in E, \; z_2 = (z + b_2) \in E, \; z_1 - z_2 = b_1 - b_2 \in L.
\]
\end{proof}


\section{Построение решёток из кодов}

\begin{definition}["Конструкция А"] \label{def:constructionA}
	C - линейный $[m,n,q]$-код, где m - длина кода, n - размерность кода, q - простой модуль, т.е. $ C = G \cdot x, \; x \in \mathbb{Z}_q^n $ \\
	Определим $ L(C) = L(G) = C + q \cdot \mathbb{Z}^m = G \cdot \mathbb{Z}_q^n + q \cdot \mathbb{Z}^m $ \\
	\[
	C=\left[
	\begin{array}{cc}
	G_{top} \\ \hline
	G_{bot}
	\end{array}\right],
	G_{top} \in \mathbb{Z}_q^{n \times n},
	G_{bot} \in \mathbb{Z}_q^{m-n \times m-n}
	\]
	Положим $ G_{top} \in \mathbb{Z}_q^{n \times n} $ - обратима. Тогда
	\[
	G \cdot G_{top}^{-1} = 
	\left[\begin{array}{ c | c }
	Id_n & \mbox{\Large 0} \\
	\hline
	G_{bot} \cdot G_{top}^{-1} & q \cdot Id_{m-n}
	\end{array}\right] \in \mathbb{Z}_q^{m \times m},
	\]
	столбцы этой матрицы образуют базис $ L(C) = L(G) $.
	\begin{itemize}
		\item $ \dim{L} = m $;
		\item $ \det{L} = q^{m-n} $;
	\end{itemize}
	По теореме Минковского:
	$
	\lambda_1^{\infty}(L(C)) \leq (\det{L})^{1/d} = q^{\frac{m-n}{m}} = q^{1 - \frac{n}{m}}
	$.
\end{definition}

\begin{theorem}[Минковского-Хлавки] \label{thm:theoremMinkowskiKhlavka}
	С вероятностью $ \geq 1 - 2^{-m} $ над случайным выбором  \\
	$ G \in \mathbb{Z}_q^{m \times n} $ имеем 
	$
	\lambda_1^{\infty}(L(G)) \geq 1/4 \cdot q^{1 - \frac{n}{m}}
	$.
\end{theorem}

\begin{proof}
Зафиксируем $ B = 1/4 \cdot q^{1 - \frac{n}{m}} $
\begin{align*}
	&G \leftarrow \mathbb{Z}_q^{m \times n}: \Pr{ \left[ \lambda_1^{\infty}(L(G)) < B \right] }  = \Pr{ \left[ \exists s \in \mathbb{Z}_q^n, \; y \in \mathbb{Z}_q^m: 0 <  \norm{ y }_{\infty} < B, \; y = G \cdot s \; (mod \; q) \right] } \leq \\  &\leq \sum_{s \in \mathbb{Z}_q^n} \sum_{\substack{y \in \mathbb{Z}_q^m \\ 
	0 <  \norm{ y }_{\infty} < B}} \Pr{ \left[ y = G \cdot s \; (mod \; q) \right] }
	\leq \sum_{s \in \mathbb{Z}_q^n \backslash\{0\}} \sum_{\substack{y \in \mathbb{Z}_q^m \\ 
	0 <  \norm{ y }_{\infty} < B}} \left[ \prod_{i=1}^{m} \ScProd{g_i}{s} = y_i \; (mod \; q) \right] = \\
	&= q^n \cdot (2(B-1) + 1)^m \cdot q^{-m} = \frac{(2B-1)^m}{q^{m-n}} = \left( \frac{2B-1}{q^{1-\frac{n}{m}}}\right)^m < 2^{-m}.
\end{align*}
\end{proof}


\end{document}