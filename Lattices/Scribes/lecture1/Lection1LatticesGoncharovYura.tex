\documentclass[11pt]{article}
\usepackage{amsmath,amssymb,amsthm}
\usepackage{algorithm}
\usepackage[noend]{algpseudocode} 

%---enable russian----

\usepackage[utf8]{inputenc}
\usepackage[russian]{babel}


%---tikz----
\usepackage{tikz}
\usetikzlibrary{arrows, chains, matrix, positioning, scopes, patterns, shapes}

%----newcommands (math)----
% ==================================================================
% Definitions for this paper
% ==================================================================
\mathchardef\hyphen="2D

\usepackage{multirow}
\usepackage{multicol} % For multiple coloumn environments
%\usepackage{stmaryrd} % For set brackets
% \setlength{\columnsep}{15pt} % Defining the coloumn seperation
% \setlength{\columnseprule}{1pt} % Place a line between coloumns
% \newcommand{\tab}{\hspace*{2em}}

%subscripts

\newcommand*\SmallTextScript[2]{{\mathchoice{\displaystyle #2}
		{\textstyle #2}%dito
		{\scalebox{#1}{\ensuremath{\scriptstyle #2}}}%
		{\scalebox{#1}{\ensuremath{\scriptscriptstyle #2}}}%
}}


% ADVERSARIES AND SUCH
\newcommand*{\poly}{\ensuremath{\mathrm{poly}}}
\newcommand*{\eps}{\ensuremath{\varepsilon}}
\newcommand*{\alg}{\ensuremath{\mathcal{A}}}

% GROUPS/DISTRIBUTIONS/SETS/LISTS
\newcommand{\N}{{{\mathbb N}}}
\newcommand{\Z}{{{\mathbb Z}}}
\newcommand*{\IZ}{\ensuremath{\mathbb{Z}}}
\newcommand*{\IN}{\ensuremath{\mathbb{N}}}
\newcommand*{\IQ}{\ensuremath{\mathbb{Q}}}
\newcommand{\R}{{{\mathbb R}}}
\newcommand*{\IR}{{{\mathbb R}}}
\newcommand{\Zp}{\ints_p} % Integers modulo p
\newcommand{\Zq}{\ints_q} % Integers modulo q
\newcommand{\Zn}{\ints_N} % Integers modulo N
\newcommand{\F}{\ensuremath{\mathbb{F}}}
\newcommand{\CC}{\ensuremath{\mathbb{C}}}

\newcommand{\GF}{\ensuremath{\mathbb{F}_2}}
\newcommand{\GFn}{\ensuremath{\mathbb{F}^n_2}}

%%% ALGORITHMS/PROCEDURES %%%
\newcommand{\Dec}{\textsf{Dec}}
\newcommand{\Enc}{\textsf{Enc}}
\newcommand{\KeyGen}{\textsf{KeyGen}}
\newcommand{\Gen}{\textsf{Gen}}
\newcommand{\sk}{\textsf{sk}}
\newcommand{\pk}{\textsf{pk}}
\newcommand{\vk}{\textsf{vk}}
\newcommand{\mesS}{\ensuremath{\mathcal{M}}}
\newcommand{\keyS}{\ensuremath{\mathcal{K}}}
\newcommand{\cipS}{\ensuremath{\mathcal{C}}}
\newcommand{\tagS}{\ensuremath{\mathcal{T}}}
\newcommand{\mactag}{\textsf{tag}}
\newcommand{\Hash}{\ensuremath{\mathcal{H}}}
\newcommand{\EID}{\ensuremath{\mathtt{EphID}}}


\newcommand{\adv}{\ensuremath{\mathcal{A}}}

\newcommand{\LWE}{\mathsf{LWE}}
\newcommand{\DCP}{\mathsf{DCP}}
\newcommand{\EDCP}{\mathsf{EDCP}}
\newcommand{\UEDCP}{\mathsf{U \text{-} EDCP}}
\newcommand{\GEDCP}{\mathsf{G \text{-} EDCP}}



%% Landau and proba
\newcommand{\bigO}{\mathcal{O}}
\newcommand*{\OLandau}{\bigO}
\newcommand*{\WLandau}{\Omega}
\newcommand*{\xOLandau}{\widetilde{\OLandau}}
\newcommand*{\xWLandau}{\widetilde{\WLandau}}
\newcommand*{\TLandau}{\Theta}
\newcommand*{\xTLandau}{\widetilde{\TLandau}}
\newcommand{\smallo}{o} %technically, an omicron
\newcommand{\wLandau}{\omega}
\newcommand{\negl}{\mathrm{negl}}
\newcommand*\PROB\Pr 
\DeclareMathOperator*{\EXPECT}{\mathbb{E}}
\DeclareMathOperator*{\VARIANCE}{\mathbb{V}}
\DeclareMathOperator*{\LOGBIAS}{\mathbb{LB}}

\newcommand{\supp}{\ensuremath{\mathsf{sup}}}
\newcommand{\Distr}{\ensuremath{\mathcal{D}}}

% Lattices

% \newcommand{\coset}{\Lambda} % Lambda Lattice
% \newcommand{\cosetPerp}{\Lambda^{\bot}} % Lambda_Perp Lattice
% \newcommand{\gadget}{\textbf{G}} %Gaget matrix
% \newcommand{\mes}{\textbf{m}} %message vector
% \newcommand{\AMat}{\textbf{A}} %A matrices
% \newcommand{\BMat}{\textbf{B}} %B matrices
% \newcommand{\RMat}{\textbf{R}} %R matrices
% \newcommand{\HMat}{\textbf{H}} %H matrices
% \newcommand{\XMat}{\textbf{X}} %H matrices
% \newcommand{\mbar}{\bar{m}} %mBar dimension
% % \newcommand{\gauss}{\mathcal{D}} % gaussian distribution
% \newcommand{\Id}{\textbf{I}} % Identity matrix
% \newcommand{\er}{\textbf{e}} % gaussian distr. vectors
% % \newcommand{\cipher}{\textit{c}} % ciphertext
% \newcommand{\Olwe}{\mathcal{O}_{\textsf{LWE}}} %LWE oracle
% \newcommand{\OSample}{\mathcal{O}_{Sample}} %LWE oracle
% \newcommand{\SigmaB}{\boldsymbol{\Sigma}} %semi-deifinite matrix Sigma%
% % \newcommand{\mods}{\text{ mod}}


%Vectors and Matrices

\newcommand{\AMat}{\mathbf{A}} %A matrices
\newcommand{\BMat}{\mathbf{B}} %B matrices
\newcommand{\DMat}{\mathbf{D}} %Diagonal


\newcommand{\HMat}{\ensuremath{\mathbf{H}}}
\newcommand{\QMat}{\ensuremath{\mathbf{Q}}}
\newcommand{\Id}{\ensuremath{\mathbf{I}}}
\newcommand{\ZeroM}{\textbf{0}} % Zero matrix

\newcommand{\avec}{\ensuremath{\mathbf{a}}}
\newcommand{\bvec}{\ensuremath{\mathbf{b}}}
\newcommand{\cvec}{\ensuremath{\mathbf{c}}}
\newcommand{\evec}{\ensuremath{\mathbf{e}}}
\newcommand{\rvec}{\ensuremath{\mathbf{r}}}
\newcommand{\svec}{\ensuremath{\mathbf{s}}}
\newcommand{\tvec}{\ensuremath{\mathbf{t}}}
\newcommand{\vvec}{\ensuremath{\mathbf{v}}}
\newcommand{\zvec}{\ensuremath{\mathbf{z}}}
\newcommand{\xvec}{\ensuremath{\mathbf{x}}}
\newcommand{\yvec}{\ensuremath{\mathbf{y}}}
\newcommand{\uvec}{\ensuremath{\mathbf{u}}}
\newcommand{\zerovec}{\ensuremath{\mathbf{0}}}

\newcommand{\nth}{^{\mathrm{th}}}
\newcommand{\nd}{^{\mathrm{nd}}}

\newcommand{\RepMMT}{\ensuremath{\mathcal{R}_{\protect\SmallTextScript{0.70}{\texttt{MMT}}}}}
\newcommand{\RepBJMM}{\ensuremath{\mathcal{R}_{\protect\SmallTextScript{0.70}{\texttt{BJMM}}}}}
\newcommand{\XOR}{\ensuremath{\mathtt{3XOR}}}


% % % % % \newcommand{\mb}[1]{\mathbf{#1}} % does not compile otherwise
%%% Removed by Gotti; this is just asking to screw up with packages that (properly) define \mb (mathbold)

% \newcommand{\bL}{\|\bvec_1\|} % b1 length that appears way too often
% \newcommand{\dL}{\|\dvec_1\|} % b1 length that appears way too oftend

%Norms and Scalar products

\newcommand*\abs[1]{\left\lvert#1\right\rvert}
\newcommand*\norm[1]{\left\lVert#1\right\rVert}
\newcommand*\normalabs[1]{\lvert#1\rvert} 
\newcommand*\normalnorm[1]{\lVert#1\rVert}
\newcommand*\bignorm[1]{\bigl\lVert#1\bigr\rVert}
\newcommand*\bigabs[1]{\bigl\lvert#1\bigr\rvert}
\newcommand*\Bigabs[1]{\Bigl\lvert#1\Bigr\rvert}
\newcommand*{\ScProd}[2]{\ensuremath{\langle#1\mathbin{,}#2\rangle}} %Scalar Product
% \newcommand*{\ScProd}[2]{\ensuremath{\langle#1 \:{,}\:#2\rangle}} %Scalar Product
\newcommand*{\bigScProd}[2]{\ensuremath{\bigl\langle#1\mathbin{,}#2\bigr\rangle}} %Scalar Product
\newcommand*{\BigScProd}[2]{\ensuremath{\Bigl\langle#1\mathbin{,}#2\Bigr\rangle}} %Scalar Product
\newcommand{\dist}{\ensuremath{\text{dist}}}


%Some other math operators

\DeclareMathOperator{\Span}{Span} %span of vectors
\DeclareMathOperator{\vol}{\mathrm{vol}} %volume
\DeclareMathOperator{\LW}{LambertW} %Lambert W function
\DeclareMathOperator{\SD}{SD}
\DeclareMathOperator{\gradient}{grad}
\DeclareMathOperator{\TRACE}{Tr}
\newcommand*{\dDR}{\mathrm{d}} %de-Rham-Differential (the d in dx, dy, dz and so on)


%Lists
\renewcommand{\L}{\ensuremath{\mathcal{L}}}

\renewcommand{\P}{\ensuremath{\mathcal{P}}}

\newcommand*{\Lout}{\ensuremath{\L_{\mkern-0.5mu\protect\SmallTextScript{0.85}{\textup{out}}}}}
\newcommand*{\Sout}{\ensuremath{S_{\mkern-0.5mu\protect\SmallTextScript{0.85}{\textup{out}}}}}
\newcommand{\wt}{\ensuremath{\mathit{wt}}}


\newcommand*{\softO}{\widetilde{\bigO}}

\newcommand{\const}{\mathsf{c}} 


\newcommand{\transpose}{\mkern0.7mu^{\mathsf{ t}}}

%proper overline reduced by 1.5mu
\newcommand{\overbar}[1]{\mkern 1.5mu\overline{\mkern-1.5mu#1\mkern-1.5mu}\mkern 1.5mu}

\DeclareMathOperator{\erf}{erf} %error function
\DeclareMathOperator{\erfc}{erfc} %complementary error function
\newcommand{\Er}{\ensuremath{\mathrm{Er}}} %complementary error function


% LATTICES

\newcommand{\Lat}{\ensuremath{\mathcal{L}}}
\newcommand*{\Sphere}[1]{\ensuremath{\mathsf{S}^{#1}}}
%\DeclareMathOperator{\Conf}{Conf}
\newcommand{\Conf}{\mathcal{C}}

%Thick line for table
\setlength{\doublerulesep}{0pt}
\newcommand{\thickline}{\hline\hline\hline}


%circled text
\newcommand*\circled[1]{\tikz[baseline=(char.base)]{
    \node[shape=circle,draw,inner sep=0.3 pt] (char) {\scriptsize #1};}}


%Fix Algorithmicx package
\def\NoNumber#1{{\def\alglinenumber##1{}\State #1}\addtocounter{ALG@line}{-1}}

%For comments
\newcommand{\GColor}{ForestGreen}  %Damiens' color
\newcommand{\EColor}{MidnightBlue} %Elena's color

\newcommand*{\E}[1]{{\color{\EColor} #1} } 
\newcommand*{\G}[1]{{\color{\GColor} #1} } 

%Proper limit with the subscript underneath
% \newcommand{\Lim}[1]{\raisebox{0.5ex}{\scalebox{0.8}{$\displaystyle \lim_{#1}\;$}}}


%TIKZ dense dotted pattern

\pgfdeclarepatternformonly{my dots}{\pgfqpoint{-1pt}{-1pt}}{\pgfqpoint{2.0pt}{2.0pt}}{\pgfqpoint{2pt}{2pt}}%
{
	\pgfpathcircle{\pgfqpoint{0pt}{0pt}}{.35pt}
	\pgfpathcircle{\pgfqpoint{1pt}{1pt}}{.35pt}
	\pgfusepath{fill}
}


\tikzset{
	master/.style={
		execute at end picture={
			\coordinate (lower right) at (current bounding box.south east);
			\coordinate (upper left) at (current bounding box.north west);
		}
	},
	slave/.style={
		execute at end picture={
			\pgfresetboundingbox
			\path  (lower right)rectangle (upper left) ;
		}
	}
} 


%-----newcommands specific to the scribe template 
\newcommand{\handout}[5]{
  \noindent
  \begin{center}
  \framebox{
    \vbox{
      \hbox to 5.78in { {\bf  } \hfill #2 }
      \vspace{4mm}
      \hbox to 5.78in { {\Large \hfill #5  \hfill} }
      \vspace{2mm}
      \hbox to 5.78in { {\em #3 \hfill #4} }
    }
  }
  \end{center}
  \vspace*{4mm}
}

\newcommand{\lecture}[4]{\handout{#1}{#2}{#3}{Выполнил: #4}{#1}}


\newtheorem{theorem}{Теорерма}
\newtheorem{corollary}[theorem]{Следствие}
\newtheorem{lemma}{Лемма} % убрал [theorem] для нумерации как в лекции
\newtheorem{observation}[theorem]{Observation}
\newtheorem{proposition}[theorem]{Предложение}
\newtheorem{definition}[theorem]{Определение}
\newtheorem{claim}[theorem]{Утверждение}
\newtheorem{fact}[theorem]{Факт}
\newtheorem{assumption}[theorem]{Предположение}

% 1-inch margins
\topmargin 0pt
\advance \topmargin by -\headheight
\advance \topmargin by -\headsep
\textheight 8.9in
\oddsidemargin 0pt
\evensidemargin \oddsidemargin
\marginparwidth 0.5in
\textwidth 6.5in

\parindent 0in
\parskip 1.5ex

\begin{document}

\lecture{Лекция 1}{}{Лектор: Елена Киршанова}{Гончаров Юра}

\section{Евклидовы решётки}
\subsection{Определения}

\begin{definition} \label{thm:Lattice}
	Пусть $\{b_i\}_{i \leq d}$ - линейно независимые вектора в $\R^n$ ($d$ < $n$). \underline{Решётка}, порождённая $\{b_i\}$ - множество вида
\[
	L(\{b_i\}_{i \leq d}) = \sum_i \Z \cdot b_i = \Big\{\sum x_i b_i , x_i \in \Z \Big\}
\]
\end{definition}

\textbf{Альтернативное определение:} \textit{\underline{Решётка} - дискретная, конечно порождённая, аддитивная подгруппа в \big($\R^n, +$\big)}

\textbf{Примеры:}\\
1) $\Z^n$, $n \geq 1$; $\{b_i = e_i\}$; $e_i = (0, 0, ..., 1_i, ..., 0)$\\
2) $\forall$ погруппа $\Z^n$, например 2$\Z^n$\\
3) a$\Z$ + b$\Z$, a,b $\in \IQ$\\
\textcolor{red}{(!) $\Z$ + $\sqrt{2}\Z$ не является решёткой (см. упражнения)}\\

\begin{definition} \label{thm:Basis}
	Пусть $L$ = $L(\{b_i\})$ для линейно независимых $b_i \in \R^n$. Тогда $\{b_i\}$ образуют базис $L$.
\[
	B = \begin{bmatrix}
		1 & ... & 1 \\
		b_1 & ... & b_d \\
		1 & ... & 1
	\end{bmatrix} \in \R^{n \cdot d} \mbox{, } L\big(B\big) \mbox{ - решётка, порождённая } \{b_i\}_{i \leq d}
\]
\end{definition}

\begin{lemma} \label{lem:Lemma1}
	Пусть $\{b_i\}_{i \leq d}$ и $\{b'_i\}_{i \leq d'}$ - два множества линейно независимых векторов в $\R^n$. Тогда
	$L(\{b_i\}_{i \leq d}) = L(\{b'_i\}_{i \leq d'}) \Leftrightarrow \mbox{выполняются оба пункта:}$\\
	\begin{itemize}
	\item $d = d'$\\
	\item $\exists$ $U \in Gl_d(\Z)$ (называют унимодулярной матрицей с определителем = 1), такая что 
	\end{itemize}
\[B = B' \cdot U \mbox{ , где }
	B =
	\begin{bmatrix}
	1 & ... & 1 \\
	b_1 & ... & b_d \\
	1 & ... & 1
	\end{bmatrix}, 
	B' =
	\begin{bmatrix}
	1 & ... & 1 \\
	b'_1 & ... & b'_d \\
	1 & ... & 1
	\end{bmatrix}
\]
\end{lemma}

\begin{proof}
	"$\Leftarrow$" (см. упражнения)\\
	"$\Rightarrow$"\\
	1)  $d$ = dim\Big(Span$_\R(\{b_i\}_{i \leq d})$\Big) (это оболочка = $\sum_i \R \cdot b_i$) = dim\Big(Span$_\R(\{b'_i\}_{i \leq d'})$\Big) = $d'$\\
	2) Распишем каждую координату\\
	$b'_1 \in L(\{b_i\})$ $\Rightarrow$ $b'_1$ = $\sum_{j = 1}^d u_{j_1} \cdot b_i$\\
	$b'_2 \in L(\{b_i\})$ $\Rightarrow$ $b'_2$ = $\sum_{j = 2}^d u_{j_2} \cdot b_i$\\
	.\\
	.\\
	.\\
	$b'_d \in L(\{b_i\})$ $\Rightarrow$ $b'_d$ = $\sum_{j = d}^d u_{j_d} \cdot b_i$\\
	Отсюда получаем, что $B'$ = $B \cdot U$; аналогичными рассуждениями мы можем выразить $B$ = $B' \cdot U$, где $U, V \in \Z^{d \cdot d}$\\
	$B$ = $B' \cdot V$ = $B \cdot U \cdot V$ $\Leftrightarrow$ $B \Big( Id \cdot U \cdot V \Big)$ = 0\\
	$\Rightarrow$ так как матрица $B$ состоит их линейно независимых векторов, то $U \cdot V$ = $Id$\\
	$\det{U} \cdot \det{V}$ = 1
\end{proof}


\textbf{Замечание:} для $d \geq$ 2, $\forall$ фиксированная решётка имеет $\infty$ много различных базисов


\textbf{"Простые" задачи на решётках:}\\
1) Для $\nu \in \R^n$ и $L = L(B)$, определить $\nu \in L(B)$\\
Решить уравнение $\nu$ = $B \cdot X$\\

2) Определить, задают ли $B, B'$ одну и ту же решётку\\
Аналогично

\subsection{Инварианты решётки}

\begin{definition} \label{thm:Invariant}
	(Первый) минимум решётки $L$ (длина кратчайшего ненулевого вектора):
\[
	\lambda_1(L) = min \{ r : \exists b \in L \backslash \{0\} : ||b|| \leq r\}
\]
	Здесь $||\cdot||$ - Евклидова норма; $||x||$ = $\sqrt{\sum x_i^2}$;  $||x||_\infty$ = max\big($|x_i|$\big)
\end{definition}

\begin{lemma} \label{lem:Lemma2}
	$\lambda_1$ достигается не менее 2-х раз и не более 3$^d$ раз
\end{lemma}

\begin{proof}
	1) $||b_1||$ = $\lambda_1$ $\Rightarrow$ $-||b_1||$ = $\lambda_1$\\
	2) $\forall$ $b \in L$ такого, что $||b_1||$ = $\lambda_1$, нарисуем шар B\Bigg($b$ (центр), $\frac{\lambda_1}{2}$ (радиус) \Bigg); эти шары не пересекаются, иначе противоречие  $\lambda_1$\\
	С другой стороны, все эти шары лежат в другом шаре, который нарисован вокруг 0, но с радиусом $\frac{3\lambda_1}{2}$; B\Bigg(0, $\frac{3\lambda_1}{2}$ \Bigg) $\Rightarrow$ количество шаров $\leq$ $\frac{VolB\Big(0, \frac{3\lambda_1}{2}\Big)}{VolB\Big(0, \frac{\lambda_1}{2}\Big)}$ = $\frac{(\frac{3\lambda_1}{2})^d \cdot VolB\Big(0, 1 \Big)}{(\frac{\lambda_1}{2})^d \cdot VolB\Big(0, 1 \Big)}$ = $3^d$
\end{proof}

\begin{definition} \label{thm:Minimum}
	Последовательные минимумы решётки: для $i \leq d$ определим
\[
	\lambda_i(L) = min \{ r : dim \big( B(0,1) \cap L \big) \geq i \}
\]
\end{definition}

\begin{lemma} \label{lem:Lemma3}
	$\forall L$ $\exists c_1, ..., c_d \in L$ - линейно независимые, такие что $||c_i||$ = $\lambda_i(L)$ $\forall i \leq d$
	($\lambda_i$ достигаются $\forall L$)
\end{lemma}

\textcolor{red}{(!) $\exists$ решётки, ждя которых $\not\exists$ базиса, вектора которого достигают $\lambda_i$ одновременно}\\
Например, $	B = \begin{bmatrix}
	2 & 0 & 0 & 0 & 1 \\
	0 & 2 & 0 & 0 & 1 \\
	0 & 0 & 2 & 0 & 1 \\
	0 & 0 & 0 & 2 & 1 \\
	0 & 0 & 0 & 0 & 1 \\
	\end{bmatrix}; \begin{array}{rcl}
	\lambda_1 & = & 2 \\
	\lambda_2 & = & 2 \\
	\lambda_3 & = & 2 \\
	\lambda_4 & = & 2 \\
	\lambda_5 & = & 2
	\end{array};$\\
	(2,2,2,2,2) - (2,0,0,0,0) - (0,2,0,0,0) - (0,0,2,0,0) - (0,0,0,2,0) =  (0,0,0,0,2)

\begin{definition} \label{thm:Determinant}
	Пусть $B \in \R^{n \cdot d}$ - базисная матрица решётки $L$ (то есть столбцы образуют базис $L$)\\
	\underline{Определитель} $L$, $\det{(L)}$ - это $\sqrt{\det{(B^\intercal \cdot B)}}$ = $\sqrt{\det{(B^\intercal)} } \cdot \sqrt{\det{(B)}}$ = $\sqrt{\det{(B)^2}}$ = $|\det{(B)}$|\\
	Для $B \in \R^{n \cdot d}$, $\det{(L)}$ = |$\det{B}$|
\end{definition}

\begin{lemma} \label{lem:Lemma4}
	Если $B, B'$ - два базиса одной и той же решётки, то \\
	$\det{(B^\intercal \cdot B)}$ = $\det{(B'^\intercal \cdot B')}$\\
	($B$ = $B' \cdot U$ (\ref{lem:Lemma1}) $\Rightarrow$ $\det{(B^\intercal \cdot B)}$ = $\det{((B' \cdot U)}$ $^\intercal \cdot B' \cdot U$) = $\det{U^\intercal}$ $\cdot$ $\det{B'^\intercal}$ $\cdot$ $\det{B'}$ $\cdot$ $\det{U}$ = $\det{(B'^\intercal \cdot B'}$).)\\
	Определитель решётки задаёт "плотность": чем меньше определитель, тем "плотнее" решётка
\end{lemma}

\begin{definition} \label{thm:Par}
	$\mathcal{P}$\big($\{b_i\}_{i \leq d}$\big) = $\big\{ \sum y_i \cdot b_i ; y_i \in [0,1] \big\}$ - фундаментальный параллелепипед $L$.
	$\det{L}$ = Vol\big($\mathcal{P}$\big)
\end{definition}

\end{document}