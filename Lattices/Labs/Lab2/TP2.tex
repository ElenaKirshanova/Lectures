\documentclass[11pt]{exam}

%---enable russian----

\usepackage[utf8]{inputenc}
\usepackage[russian]{babel}



\usepackage[margin=0.73in]{geometry}
%\usepackage[top=1in, bottom=1in, left=1in, right=1in]{geometry}

\usepackage{graphicx}
\usepackage{url}
\usepackage{latexsym}
\usepackage{amscd,amsmath,amsthm}
\usepackage{mathtools}
\usepackage{amsfonts}
\usepackage{amssymb}
\usepackage[dvipsnames]{xcolor}
\usepackage{hyperref}

\usepackage{algorithmicx, enumitem, algpseudocode, algorithm, caption}
\usepackage{tikz}
\usetikzlibrary{automata}

\usepackage{textcomp}


%%%%%%%%%%%%%%%%%%%%%%%%%%%
%% THEOREMS
%%%%%%%%%%%%%%%%%%%%%%%%%%%

\usepackage{amsmath,amssymb,amsfonts}
\usepackage{amsthm}

\newtheorem{theorem}{Теорерма}
\newtheorem{corollary}[theorem]{Следствие}
\newtheorem{lemma}[theorem]{Лемма}
\newtheorem{observation}[theorem]{Observation}
\newtheorem{proposition}[theorem]{Предложение}
\newtheorem{definition}[theorem]{Определение}
\newtheorem{claim}[theorem]{Утверждение}
\newtheorem{fact}[theorem]{Факт}
\newtheorem{assumption}[theorem]{Предположение}

\theoremstyle{definition}
\newtheorem{problem}{Problem}


\newcommand{\nc}{\newcommand}
\nc{\eps}{\varepsilon}
\nc{\RR}{{{\mathbb R}}}
\nc{\CC}{{{\mathbb C}}}
\nc{\FF}{{{\mathbb F}}}
\nc{\NN}{{{\mathbb N}}}
\nc{\ZZ}{{{\mathbb Z}}}
\nc{\PP}{{{\mathbb P}}}
\nc{\QQ}{{{\mathbb Q}}}
\nc{\UU}{{{\mathbb U}}}
\nc{\OO}{{{\mathbb O}}}
\nc{\EE}{{{\mathbb E}}}

\newcommand{\bigO}{\mathcal{O}}

\newcommand{\val}{\operatorname{val}}

\newcommand{\wt}{\ensuremath{\mathit{wt}}}
\newcommand{\Id}{\ensuremath{I}}
\newcommand{\transpose}{\mkern0.7mu^{\mathsf{ t}}}
\newcommand*{\ScProd}[2]{\ensuremath{\langle#1\mathbin{,}#2\rangle}} %Scalar Product
\newcommand*\abs[1]{\left\lvert#1\right\rvert}
\newcommand*\norm[1]{\left\lVert#1\right\rVert}

\newcommand{\zvec}{\ensuremath{\mathbf{z}}}

\pretolerance=1000

%%%%%%%%%%%%%%%%%%%%%%%%%%%%%%%%
%%%%%%%%%%%%%%%%%%%%%%%%%%%%%%%%
%% DOCUMENT STARTS
%%%%%%%%%%%%%%%%%%%%%%%%%%%%%%%%
%%%%%%%%%%%%%%%%%%%%%%%%%%%%%%%%
\usepackage{tikz}
\usetikzlibrary{automata}
\DeclareMathOperator{\Vol}{Vol}

\begin{document}
	{\noindent
		\textsc{БФУ им. И. Канта -- Криптография на решётках}
		\hfill {Е. Киршанова // 2020--2021\\}
\hrule
\begin{center}
	{\LARGE
			Лабораторная работа № 2 \\[5pt]
			\textbf{Факторизация на решётках} \\[10pt]
	 	{09.03.2021} 
 	} 
\end{center}
\hrule \vspace{5mm}
	
	\thispagestyle{empty}
	
	\vspace{0.2cm}
	\section{Алгоритм факторизации Шнорра}
	
	Лабораторная работа вдохновлена недавнем пре-принтом К.П. Шнорра~\cite{Schnorr21}, вызвавший большой резонанс~\cite{twitter1,twitter2,stackexchange}. В частности, аннотация статьи утверждает, что ``This [атака] destroys the RSA cryptosystem.'' Суть этой лабораторной - попытаться факторизовать RSA модуль с помощью решёток. 
	
	Алгоритм имеет длинную историю~\cite{leogit}, мы будем придерживаться описания из~\cite{vera10}
	
	В основе алгоритма (как и для продвинутых алгоритмов Number Field Sieive), лежит идея поиска пар $(x,y)$, удовлетворяющих 
	\begin{align} \label{eq:congruence}
	x^2 \equiv y^2 \bmod N.
	\end{align}
	\noindent Если $x \neq \pm y \bmod N$, то вычисление $\gcd(N, x+y)$ даст нетривиальный делитель $N$. Метод Шнорра ищет такие пары с доп. условие гладкости.
	
	Число называется \emph{$B$-гладким}, если все его простые делители меньше $B$. Обозначим $p_i$--$i$-ое простое число и зафиксируем некоторое целое $d>1$. Основная вычислительная задача алгоритма Шнорра состоит в поиске четверок $(u,v,k,\gamma)$, таких что 1. $u,v, k$--$p_d$-гладкие; 2. $\gamma \geq 1$ -- целое; 3. выполняется Диофантово уравнение
	\[
		u = v + kN^\gamma.
	\]
	\noindent Эти четверки будут находится с помощью коротких векторов решетки специального вида. А именно, рассмотри решетку, порожденную столбцами матрицы $A$:
	\[
	A = \begin{pmatrix}
	\ln p_1 & 0 & \ldots & 0  & 0 \\
	0 & \ln p_2 & \ldots & 0  & 0\\
	\vdots & \vdots & \ddots & \vdots & \vdots\\
	0 & 0 & \ldots & \ln p_d & 0 \\
	C \ln p_1 & C \ln p_2 & \ldots & C \ln p_d & C \ln N
	\end{pmatrix} \in \ZZ^{d+1 \times d+1},
	\] 
	\noindent где $C$--некая (достаточно большая) константа. Для вектора $\zvec \in \ZZ^{d+1}$, справедливо
	\[
		A\zvec = \begin{pmatrix}
		z_1 \ln p_1 \\
		\vdots \\
		z_d \ln p_d \\
		C (\sum_i z_i \ln p_i + z_{d+1} \ln N).
		\end{pmatrix}
	\]
	Если мы будем ассоциировать с вектором $\zvec$, элементы $u, k, \gamma$ следующим образом
	\[
		 u = \prod_{z_i > 0} p_i^{z_i}, \quad k = \prod_{z_i < 0} p_i^{-z_i} \quad \text{ и } \quad \gamma = \abs{z_{d+1}},
	\] 	
	то 
	\[
		\norm{A\zvec}_1 = \sum_i^{d} \abs{z_i} \ln p_i  + C \abs{\sum_i^{d} z_i \ln p_1 - \abs{z_{d+1}} \ln N},
	\]
	и
	\[
	\norm{A\zvec}_1 = \ln u + \ln k  + C \abs{\ln u  - \ln(k N^\gamma)}.
	\]
	
	Отсюда, если $\norm{A\zvec}_1$ -- мала, то можно доказать следующее утверждение (см.~\cite{vera10} для доказательства):
	
	Если $\norm{A\zvec}_1 \leq 2C + 2 \sigma \ln p_d - \gamma \ln N$, то $\abs{u - kN^\gamma} < p_d^\sigma$.
	
	Делаем вывод: чтобы найти четверку $(u,v,k,\gamma)$, необходимо найти короткий вектор в решетке, порожденной  столбцами $A$. 
	Как найти из такой четверки $(x,y)$, удовлетворяющие~\ref{eq:congruence}?
	
	Найдем $d+1$ четверок  $(u,v,k,\gamma)$.  Для каждой такой четверки положим $a_{i,j} = z_j$ для $z_j >0$ (те $z_j$, что участвуют в записи $u_i$), а за $b_{i,j}$ обозначим степени в разложении $v_i = u_i - k_iN^\gamma_i = \prod_i p_i^{b_i,j}$ (для $i \leq d+1$). Обозначим далее вектора $\mathbf{a}_i = (a_{i,1}, \ldots, a_{i,d})$, $\mathbf{b}_i = (b_{i,1}, \ldots, b_{i,d})$. Всего имеем $(d+1)$ таких векторов, которые будем записывать в матрицу $M \in\ZZ^{d \times d+1}$ по столбцам.
	
	Для всякого вектора $\mathbf{c} \in \{0,1\}^{d+1}$, удовлетворяющего
	\[
		\mathbf{c}  \cdot M \equiv 0 \bmod 2, 
	\]
	положим
	\[
	 	x = \prod_{j=1}^d p_j^{\sum_{i=1}^{d+1} c_i (a_{i,j}+b_{i,j})/2} \bmod N \quad \quad	y = \prod_{j=1}^d p_j^{\sum_{i=1}^{d+1} c_i a_{i,j} } \bmod N.
	\]
 	Если $x \neq \pm y \bmod N$, вычислим нетривиальный делитель $N$ как $\gcd(N, x+y)$.
 	
	\subsection{Предложение Шнорра 2021}
	Алгоритм, описанный выше, требует поиска нескольких (минимум $d+1$) коротких векторов для решетки, порожденной матрицей $A$. Для этого, например, можно использовать алгоритм просеивания, возвращающий (почти) все короткие вектора решетки~\cite{g6k} (см \textsf{examples/all\_short\_vectors.py}).
	
	Шнорр в~\cite{Schnorr21} предлагает рандомизировать $A$ следующим образом. Выбираем случайную перестановку $f:[1,d] \rightarrow [1,d] $ и строим $A'$
	\[
	A' = 
	\begin{pmatrix}
		N f(1) & 0 &  \ldots & 0 & 0 \\
		0 & N f(2) & \ldots & 0 & 0 \\
		\vdots & \vdots & \ddots & \vdots & 0 \\
		0 & 0 & \vdots & N f(n)  & 0 \\
		C N \ln p_1 &C  N \ln p_2 & \ldots & C N \ln p_d & CN \ln NД,
	\end{pmatrix}
	\]
	для какого-то (большого) $C$. Найдя короткий вектора этой решетки $A\zvec$, ``уберем'' скаляр $C$, положив $\zvec' = \zvec / C$. Представим
	\[
		u = \prod_{i: \zvec'_i >0} p_i^{\zvec'_i} \quad \quad k = \prod_{i: \zvec'_i <0} p_i^{-\zvec'_i}.
	\] 
	Из найденных значений $(u,v)$ строим $(x,y)$ аналогично процедуре выше, \emph{если} $(u - kN)$ --- $p_d-$гладкое число.


	
	
	\subsection{Задание}
	
	Используя \textbf{любой} из подходов и их улучшений (смотрим ссылки), факторизовать 30-битный RSA-модуль $N$ (то есть $N = pq$, где $p \neq q$ -- простые числа по $\approx$15 бит каждое).
	
	\textbf{Бонус.} Команде\footnote{Команда состоит из 1-2 человек},  успешно факторизовавшей 80-битный RSA-модуль $N$ алгоритмом Шнорра, положен +1 балл на экзамене.
	
	\begin{thebibliography}{9}
		\bibitem{Schnorr21}
		Claus Peter Schnor.
		\textit{Fast Factoring Integers by SVP Algorithms}. \url{https://eprint.iacr.org/2021/232.pdf}		
		
		\bibitem{twitter1}
		\url{https://twitter.com/inf\_0\_/status/1367376526300172288}
		
		\bibitem{twitter2}
		\url{https://twitter.com/kennyog/status/1367132559117848583}
	
		\bibitem{stackexchange}
		\url{https://crypto.stackexchange.com/questions/88582/does-schnorrs-2021-factoring-method-show-that-the-rsa-cryptosystem-is-not-secur/88647#88647}
		
		\bibitem{leogit}
		\url{https://github.com/lducas/SchnorrGate}
		
		\bibitem{vera10}
		Antonio Vera.
		\textit{A note on integer factorization using lattices}
		\url{https://arxiv.org/pdf/1003.5461.pdf}
		
		\bibitem{g6k}
		\url{https://github.com/fplll/g6k}
	\end{thebibliography}	
		
\end{document}