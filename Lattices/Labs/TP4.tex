\documentclass[11pt]{exam}

%---enable russian----

\usepackage[utf8]{inputenc}
\usepackage[russian]{babel}



\usepackage[margin=0.73in]{geometry}
%\usepackage[top=1in, bottom=1in, left=1in, right=1in]{geometry}

\usepackage{graphicx}
\usepackage{url}
\usepackage{latexsym}
\usepackage{amscd,amsmath,amsthm}
\usepackage{mathtools}
\usepackage{amsfonts}
\usepackage{amssymb}
\usepackage[dvipsnames]{xcolor}
\usepackage{hyperref}

\usepackage{algorithmicx, enumitem, algpseudocode, algorithm, caption}
\usepackage{tikz}
\usetikzlibrary{automata}

\usepackage{textcomp}

\usepackage{kbordermatrix} % to label matrix rows and columns

%%%%%%%%%%%%%%%%%%%%%%%%%%%
%% THEOREMS
%%%%%%%%%%%%%%%%%%%%%%%%%%%

\usepackage{amsmath,amssymb,amsfonts}
\usepackage{amsthm}

\newtheorem{theorem}{Теорерма}
\newtheorem{corollary}[theorem]{Следствие}
\newtheorem{lemma}[theorem]{Лемма}
\newtheorem{observation}[theorem]{Observation}
\newtheorem{proposition}[theorem]{Предложение}
\newtheorem{definition}[theorem]{Определение}
\newtheorem{claim}[theorem]{Утверждение}
\newtheorem{fact}[theorem]{Факт}
\newtheorem{assumption}[theorem]{Предположение}

\theoremstyle{definition}
\newtheorem{problem}{Problem}


\newcommand{\nc}{\newcommand}
\nc{\eps}{\varepsilon}
\nc{\RR}{{{\mathbb R}}}
\nc{\CC}{{{\mathbb C}}}
\nc{\FF}{{{\mathbb F}}}
\nc{\NN}{{{\mathbb N}}}
\nc{\ZZ}{{{\mathbb Z}}}
\nc{\PP}{{{\mathbb P}}}
\nc{\QQ}{{{\mathbb Q}}}
\nc{\UU}{{{\mathbb U}}}
\nc{\OO}{{{\mathbb O}}}
\nc{\EE}{{{\mathbb E}}}

\newcommand{\bigO}{\mathcal{O}}

\newcommand{\val}{\operatorname{val}}

\newcommand{\wt}{\ensuremath{\mathit{wt}}}
\newcommand{\Id}{\ensuremath{I}}
\newcommand{\transpose}{\mkern0.7mu^{\mathsf{ t}}}
\newcommand*{\ScProd}[2]{\ensuremath{\langle#1\mathbin{,}#2\rangle}} %Scalar Product
\newcommand*\abs[1]{\left\lvert#1\right\rvert}
\newcommand*\norm[1]{\left\lVert#1\right\rVert}

\pretolerance=1000

%%%%%%%%%%%%%%%%%%%%%%%%%%%%%%%%
%%%%%%%%%%%%%%%%%%%%%%%%%%%%%%%%
%% DOCUMENT STARTS
%%%%%%%%%%%%%%%%%%%%%%%%%%%%%%%%
%%%%%%%%%%%%%%%%%%%%%%%%%%%%%%%%
\usepackage{tikz}
\usetikzlibrary{automata}
\DeclareMathOperator{\Vol}{Vol}

\begin{document}
	{\noindent
		\textsc{БФУ им. И. Канта -- Криптография на решётках}
		\hfill {Е. Киршанова // 2020--2021\\}
\hrule
\begin{center}
	{\LARGE
			Лабораторная работа № 4 \\[5pt]
			\textbf{Атака на подписи GGH/NTRU} \\[10pt]
	 	{05.05.2021} 
 	} 
\end{center}
\hrule \vspace{5mm}
	
	\thispagestyle{empty}
	
	\vspace{0.2cm}
	\section{Алгоритм подписи GGH}
	
	Цель этой лабораторной -- атаковать подпись GGH (названа в честь авторов Goldreich, Goldwasser, Halevi)~\cite{GGH}, на основе которой было предложено несколько популярных подписей, в том числе запатентованная\footnote{Патенты в крипто--зло!} подпись NTRU~\cite{NTRUSign}(сегодня мы знаем, как обойти эту атаку, и построить безопасную подпись, см.\ лекцию).
	
	Для начала рассмотрим алгоритм подписи. Пару (публичный, секретный) ключи формируют базисы $d$-размерной решетки (полного ранга). Секретный ключ $sk$ -- `хороший' базис решетки (например, HKZ-редуцированный), открытый ключ $pk$ -- `плохой' базис этой же решетки (например, HNF форма $sk$).
	
	Сообщения предполагаются выбранными из множества $\ZZ^d$ (любое отображение $\{0,1\}^\star \rightarrow \ZZ^d$ подойдет).
	
	Процедура генерации подписи берет на вход хороший базис решетки $sk$ и решает задачу нахождения ближайшего вектора (CVP) к $m$. Например, при достаточно редуцированном базисе для решения аппроксимации CVP запускается алгоритм Бабая. Заметьте, что вектор-ошибки $(m - \mathtt{CVP}(sk, m))$ лежит в фундаментальном параллелепипеде базиса $sk$, $\mathcal{P}(sk)$, и этот вектор ошибки легко вычислить, зная пару (сообщение, подпись).
	
	Процедура верификации, получив на вход тройку $(m, \sigma, pk)$, проверяет с помощью $pk$ -- базиса $L$, евклидову норму ошибки $\norm{m - \sigma}$. Если она мала, а именно меньше чем некая граница $\eta$ (полагаем $\eta$ известным), то подпись принимается, если нет, отклоняется.
	
	\begin{algorithm}
		\begin{algorithmic}[1]
			\Function{KeyGen}{$L = \mathcal{L}(B)$-- решетка, порожденная $B$}
			\State		\Return $pk = (\text{HNF}(B), \eta), sk = \text{HKZ}(B)$ \Comment{$\eta$ зависит от базиса Грам-Шмидта базиса $sk$.}
			\EndFunction
			
			\vspace{10pt}
			\Function{Sign}{$m, sk$}
			\State \Return $\sigma = \mathtt{CVP}(sk, m)$
			\EndFunction
			
			\vspace{10pt}
			\Function{Verify}{$m, \sigma, pk$}
			\State \Return $\norm{\sigma-m} \leq \eta$
			\EndFunction
		\end{algorithmic}
	\end{algorithm}
	
	В лабораторной работе, предлагается взять $sk = q \Id_d$ (с тривиальным алгоритмом Бабая) для некого фиксированного $q$, а в качестве $pk = U \cdot sk$, где $U$ -- случайная унимодулярная матрица.  Вообще, так генерировать публичный ключ нельзя, но это работает быстро и не влияет на суть лабораторной. С корректно сгенерированным $pk$ атака работает точно также.
	
	Пример реализации подписи можно найти по ссылке \url{https://crypto-kantiana.com/elena.kirshanova/teaching/lattices_2021/lab4\_GGHsign.sage}.
	
	
	
	\section{Атака}
	
	Проблема этой подписи заключается в том, что достаточное большое число подписей ``выдают'' форму фундаментального параллелепипеда $\mathcal{P}(sk)$, а вместе с ним и секретного ключа $sk$.  Атака была формализована в 2006 в работе Нгуена-Регева~\cite{NR06} и состоит из следующих шагов:
	\begin{enumerate}
		\item Получение большого числа подписей под одним ключом (в этой лабораторной мы планируем взломать один фиксированный ключ)
		\item Аппроксимация матрицы $sk^t sk$. Как это сделать описано в Разделе 4.1  статьи~\cite{NR06}. Из этой аппроксимации получить матрицу $L$, отображающую элементы $\mathcal{P}(sk)$ в элементы фундаментального параллелепипеда куба. В примере, выбранном в лабораторной, этот куб есть $(\Id_d)$ (в статье описывается общий случай, когда этот куб может быть ``перевернут'', то есть любой ортогональной матрицей).  Матрица $L$ получена на шаге 2 Алгоритма~1 в~\cite{NR06}.
		\item С помощью $L$ отобразить все элементы  $\mathcal{P}(sk)$  в элементы $\mathcal{P}(\Id_d)$, то есть в элементы куба со сторонами длины 1. Здесь допустима погрешность, так как на первом шаге мы получим не точное значение $sk^t sk$, а его аппроксимацию.
		\item Аппроксимацией моментов этого куба (см. раздел 4.3 в~\cite{NR06}), получить аппроксимацию $\Id_d$.
		Суть этого шага в следующем: имея выборку и некоего многомерного тела (в нашем случае $\mathcal{P}(\Id_d))$, мы можем аппроксимировать образующую этого параллелепипеда (в нашем случае $\Id_d$), имея достаточное число элементов в выборке. Это своего рода обобщение неравенства Чебышёва (где аппроксимируется первый момент -- мат. ожидание) на моменты более высоких порядков. 
		\item С помощью обратного отображения $L^{-1}$, отобразить $\Id_d$ в $sk$. Это сделано в шаге 5 Алгоритма~1 в~\cite{NR06}.
	\end{enumerate}
	

	\section{Задание к лабораторной}
	
	Задача: реализовать алгоритм, описанный в Разделе 4 статьи~\cite{NR06} и получить аппроксимацию $sk$.
	
	Пояснения к заданию:
	\begin{itemize}
	\item Для предложенной размерности $d=70$ понадобится не менее $100\,000$ подписей. Для такой размерности атака должна занимает около часа на 2,8 GHz Intel Core i5.  Демонстрировать работоспособность кода можно и на меньших размерностях, однако нужно показать результат работы на больших размерностях. 
	\item Параметры подписи, такие как форма $sk$, можно изменять (при этом убедитесь, что алгоритм Бабая работает верно). 
	\item Матрица $V$ раздела 4.1 в ~\cite{NR06} есть $sk$ в наших обозначениях. Однако обратите внимание, что в разделе 4.4  $V$ уже обозначает другую матрицу (в нашем примере, $\Id_d$).
	\item Вектор ошибки, возвращаемый алгоритмом Бабая в реализации, лежит в фундаментальном параллелепипеде $\mathcal{P}_{1/2}(sk) = \{ \sum_{i=1}^d x_i v_k \; x_i \in [-1/2, 1/2] \}$, в то время как атака в~\cite{NR06} описывает  $\mathcal{P}(sk) = \{ \sum_{i=1}^d x_i v_k \; x_i \in [-1, 1] \}$. Это, в частности, влияет на  фактор 3, описанный в Лемме 1 в~\cite{NR06}.
	\item  При реализации Алгоритма~2 из~\cite{NR06} обратите внимание, что только те $w$, 4-ый момент которых близок к 1/5, ведут к верной аппроксимации $\Id$, а следовательно, $sk$. Подтверждает этот факт Лемма 3 в~\cite{NR06}.
	Не все $w$ полезны. Параметр $\delta$ в этом алгоритме можно выбрать из диапазона $[0.65, 0.8]$.
	\item Аппроксимация $\text{mom}_{sk}(w)$, упомянутая в разделе 4.3 в~\cite{NR06} реализуется с помощью неравенства Чебышёва~\cite{Cheb}.
	
	\end{itemize}
	
	\begin{thebibliography}{9}
		\bibitem{NTRUSign} 
		J. Hoffstein, N. A. Howgrave Graham, J. Pipher, J. H. Silverman, and W. Whyte.
		\textit{NTRUSIGN: Digital signatures using the NTRU lattice.}
		
		\bibitem{GGH} 
		O. Goldreich, S. Goldwasser, and S. Halevi. 
		\textit{Public-key cryptosystems from lattice reduction problems.}
		
		\bibitem{NR06}
		Phong Q. Nguyen and Oded Regev.
		\textit{Learning a Parallelepiped: Cryptanalysis of GGH and NTRU Signatures}. EuroCrypt'06
		\url{https://cims.nyu.edu/~regev/papers/gghattack.pdf}
		
		\bibitem{Cheb}
		\url{https://en.wikipedia.org/wiki/Chebyshev\%27s_inequality#Probabilistic_statement}
	\end{thebibliography}	
		
\end{document}