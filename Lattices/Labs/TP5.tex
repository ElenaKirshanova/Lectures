\documentclass[11pt]{exam}

%---enable russian----

\usepackage[utf8]{inputenc}
\usepackage[russian]{babel}



\usepackage[margin=0.73in]{geometry}
%\usepackage[top=1in, bottom=1in, left=1in, right=1in]{geometry}

\usepackage{graphicx}
\usepackage{url}
\usepackage{latexsym}
\usepackage{amscd,amsmath,amsthm}
\usepackage{mathtools}
\usepackage{amsfonts}
\usepackage{amssymb}
\usepackage[dvipsnames]{xcolor}
\usepackage{hyperref}

\usepackage{algorithmicx, enumitem, algpseudocode, algorithm, caption}
\usepackage{tikz}
\usetikzlibrary{automata}

\usepackage{textcomp}

\usepackage{kbordermatrix} % to label matrix rows and columns

%%%%%%%%%%%%%%%%%%%%%%%%%%%
%% THEOREMS
%%%%%%%%%%%%%%%%%%%%%%%%%%%

\usepackage{amsmath,amssymb,amsfonts}
\usepackage{amsthm}

\newtheorem{theorem}{Теорерма}
\newtheorem{corollary}[theorem]{Следствие}
\newtheorem{lemma}[theorem]{Лемма}
\newtheorem{observation}[theorem]{Observation}
\newtheorem{proposition}[theorem]{Предложение}
\newtheorem{definition}[theorem]{Определение}
\newtheorem{claim}[theorem]{Утверждение}
\newtheorem{fact}[theorem]{Факт}
\newtheorem{assumption}[theorem]{Предположение}

\theoremstyle{definition}
\newtheorem{problem}{Problem}


\newcommand{\nc}{\newcommand}
\nc{\eps}{\varepsilon}
\nc{\RR}{{{\mathbb R}}}
\nc{\CC}{{{\mathbb C}}}
\nc{\FF}{{{\mathbb F}}}
\nc{\NN}{{{\mathbb N}}}
\nc{\ZZ}{{{\mathbb Z}}}
\nc{\PP}{{{\mathbb P}}}
\nc{\QQ}{{{\mathbb Q}}}
\nc{\UU}{{{\mathbb U}}}
\nc{\OO}{{{\mathbb O}}}
\nc{\EE}{{{\mathbb E}}}

\newcommand{\bigO}{\mathcal{O}}

\newcommand{\val}{\operatorname{val}}

\newcommand{\wt}{\ensuremath{\mathit{wt}}}
\newcommand{\Id}{\ensuremath{I}}
\newcommand{\transpose}{\mkern0.7mu^{\mathsf{ t}}}
\newcommand*{\ScProd}[2]{\ensuremath{\langle#1\mathbin{,}#2\rangle}} %Scalar Product
\newcommand*\abs[1]{\left\lvert#1\right\rvert}
\newcommand*\norm[1]{\left\lVert#1\right\rVert}

\pretolerance=1000

%%%%%%%%%%%%%%%%%%%%%%%%%%%%%%%%
%%%%%%%%%%%%%%%%%%%%%%%%%%%%%%%%
%% DOCUMENT STARTS
%%%%%%%%%%%%%%%%%%%%%%%%%%%%%%%%
%%%%%%%%%%%%%%%%%%%%%%%%%%%%%%%%
\usepackage{tikz}
\usetikzlibrary{automata}
\DeclareMathOperator{\Vol}{Vol}

\begin{document}
	{\noindent
		\textsc{БФУ им. И. Канта -- Криптография на решётках}
		\hfill {Е. Киршанова // 2020--2021\\}
\hrule
\begin{center}
	{\LARGE
			Дополнительная лабораторная работа № 5 \\[5pt]
			\textbf{Взлом хэш-функции} \\[10pt]
	 	{01.06.2021} 
 	} 
\end{center}
\hrule \vspace{5mm}
	
	\thispagestyle{empty}
	
	\vspace{0.2cm}
	\section{Хэш-функция на решетках}
	
	На задаче SIS можно построить хэш-функцию следующим образом:
	\begin{enumerate}
		\item Для фиксированных $m>n>1$ и $q>1$, выберем матрицу $A \leftarrow(\ZZ_q^{n\times m})$
		\item Хэш-функция $\mathcal{H}_A$ определяется отображением
		\begin{align*}
			\mathcal{H}_A: \{0,1\}^m &\rightarrow \ZZ_q^n \\
			x &\mapsto Ax \bmod q
		\end{align*}
	\end{enumerate}

	Цель работы: отыскать коллизию для $\mathcal{H}_A$.
	
	Заметим, что коллизией для $\mathcal{H}_A$ будет считаться любая пара бинарных векторов $(x, x')$, такая, что $x \neq x'$ и $Ax = Ax' \bmod q$. Из последнего равенства следует, что $A(x -x') = 0 \bmod q$, а значит, $x, x' \in L^\perp(A)$. Эту пару можно найти с помощью алгоритма редукции базиса $L^\perp(A)$.

	\section{Задание к лабораторной}
	
	Задача: для параметров $n = 12, m=113, q = 37$, найти коллизию для хэш-функции $\mathcal{H}_A$. Пример реализации хэш-функции можно найти по адресу 
	\url{https://crypto-kantiana.com/elena.kirshanova/teaching/lattices\_2021/lab5.sage}
	
	Код лабораторной должен вывести коллизию для $L^\perp(A)$. Алгоритм должен работать для случайной матрицы $A$.
	
\end{document}