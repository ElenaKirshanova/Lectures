% !BIB TS-program = biber
%% !TeX program = xelatex
% !TeX encoding = UTF-8

\documentclass[12pt]{article}
\usepackage{amsmath,amssymb,amsthm}
\usepackage{algorithm}
\usepackage[noend]{algpseudocode} 
\usepackage{enumitem}

%---enable russian----

\usepackage{cmap} % for russian text search support in pdf
\usepackage[utf8]{inputenc}
\usepackage[russian]{babel}
%\usepackage{euler}
%\usepackage{fontspec}
%\setmainfont{TeX Gyre Pagella}

%---tikz----
\usepackage{tikz}
\usetikzlibrary{arrows, chains, matrix, positioning, scopes, patterns, shapes}
\usepackage{pgfplots, subfigure}

\usepackage{hyperref}
\usepackage[backend=biber,hyperref=true,backref=true,style=alphabetic-verb]{biblatex}
\bibliography{../biblio}


% MACROS 
% PROBABILITY SYMBOLS
\newcommand*\PROB\Pr 
\DeclareMathOperator*{\EXPECT}{\mathbb{E}}


% GROUPS/DISTRIBUTIONS/SETS/LISTS
\newcommand{\N}{{{\mathbb N}}}
\newcommand{\Z}{{{\mathbb Z}}}
\newcommand{\Q}{{{\mathbb Q}}}
\newcommand{\R}{{{\mathbb R}}}
\newcommand{\F}{{{\mathbb F}}}
\newcommand{\PP}{{{\mathbb P}}}
\newcommand*{\IZ}{\ensuremath{\mathbb{Z}}}
\newcommand*{\IN}{\ensuremath{\mathbb{N}}}
\newcommand*{\IR}{{{\mathbb R}}}
\newcommand{\Zp}{\ints_p} % Integers modulo p
\newcommand{\Zq}{\ints_q} % Integers modulo q
\newcommand{\Zn}{\ints_N} % Integers modulo N
\newcommand{\Zr}{\ensuremath{\mathbb{Z}/r\mathbb{Z}}} % Integers modulo N
\newcommand*{\dDR}{\mathrm{d}} %de-Rham-Differential (the d in dx, dy, dz and so on)
\newcommand{\transpose}{\mkern0.1mu^{\mathsf{t}}}
\newcommand*{\union}{\mathbin{\cup}}

% Landau 
\newcommand{\bigO}{\mathcal{O}}
\newcommand*{\OLandau}{\bigO}
\newcommand*{\WLandau}{\Omega}
\newcommand*{\xOLandau}{\widetilde{\OLandau}}
\newcommand*{\xWLandau}{\widetilde{\WLandau}}
\newcommand*{\TLandau}{\Theta}
\newcommand*{\xTLandau}{\widetilde{\TLandau}}
\newcommand{\smallo}{o} %technically, an omicron
\newcommand{\softO}{\widetilde{\bigO}}
\newcommand{\wLandau}{\omega}
\newcommand{\negl}{\mathrm{negl}} 

% Misc
\newcommand{\eps}{\varepsilon}
\newcommand{\inprod}[1]{\left\langle #1 \right\rangle}

 
\newcommand{\handout}[5]{
  \noindent
  \begin{center}
  \framebox{
    \vbox{
      \hbox to 5.78in { {\bf  } \hfill #2 }
      \vspace{4mm}
      \hbox to 5.78in { {\Large \hfill #5  \hfill} }
      \vspace{2mm}
      \hbox to 5.78in { {\em #3 \hfill #4} }
    }
  }
  \end{center}
  \vspace*{4mm}
}

%\newcommand{\lecture}[4]{\handout{#1}{#2}{#3}{Оформил #4}{Лекция #1}}
\newcommand{\lecture}[4]{\handout{#1}{#2}{#3}{}{Лекция #1}}

\newtheorem{theorem}{Теорема}
\newtheorem{corollary}[theorem]{Следствие}
\newtheorem{lemma}[theorem]{Лемма}
\newtheorem{observation}[theorem]{Observation}
\newtheorem{proposition}[theorem]{Предложение}

\theoremstyle{definition}
\newtheorem{definition}[theorem]{Определение}

\newtheorem{claim}[theorem]{Утверждение}
\newtheorem{fact}[theorem]{Факт}
\newtheorem{assumption}[theorem]{Предположение}

\theoremstyle{definition}
\newtheorem{examples}[theorem]{Примеры}

\theoremstyle{definition}
\newtheorem{example}[theorem]{Пример}

% 1-inch margins
\topmargin 0pt
\advance \topmargin by -\headheight
\advance \topmargin by -\headsep
\textheight 8.9in
\oddsidemargin 0pt
\evensidemargin \oddsidemargin
\marginparwidth 0.5in
\textwidth 6.5in

\parindent 0in
\parskip 1.5ex

\begin{document}
    
\lecture{1 --- 01.09.2020}{Осень 2020}{Лектор: Семён Новосёлов}{}
    
\section{Введение}
\subsection{Мотивация}
Эллиптические кривые в широко используются в сети Интернет в составе таких протоколов как HTTPS(TLS), SSH и IPSec. Основное применение -- выработка общего ключа по протоколу Диффи-Хелмана (ECDH) и цифровые подписи (ECDSA). Используемые в настоящее время криптографические схемы основаны на задаче нахождения дискретного логарифма. В качестве альтернативы разрабатываются схемы (например, SIKE) основанные на сложности вычисления изогений -- нетривиальных гомоморфизмов эллиптических кривых, которые отличаются стойкостью к атакам на квантовом компьютере.

С точки зрения математики, эллиптические кривые естественным образом появляются при попытке применить теорию групп к исследованию множеств решений систем полиномиальных уравнений, т.е. многообразий и алгебраических множеств. По определению эллиптическая кривая представляет собой групповое (абелево) многообразие размерности $1$. Их изучению посвящено множество работ.

Данные лекции рассчитаны главным образом на изучение арифметических аспектов теории эллиптических кривых и их приложении к криптографии. Более основательное изложение можно найти в \cite{Silverman2009}.

\subsection{Обозначения}
$\F_q$ -- конечное поле, $|\F_q|=q=p^k$, $p$ -- простое, $K$ -- поле, $\overline{K}$ -- алгебраическое замыкание.
        
\subsection{Определения}
        
\begin{definition} \textit{Уравнение Вейерштрасса} в проективных координатах -- уравнение степени $3$ вида 
\begin{equation}
\label{eq:weierstrass_proj}
F: Y^2Z + a_1 X Y Z + a_3 Y Z^2 = X^3 + a_2 X^2 Z + a_4 X Z^2 + a_6 Z^3,
\end{equation}
где $a_i \in K$. Уравнение Вейерштрасса \textit{гладкое} (или несингулярное), если для любых проективных точек $P=(X:Y:Z) \in \PP^2(K)$ \footnotetext{проективная плоскость над $K$ — множество классов эквивалентности на $K^3\setminus \{0,0,0\}$, т.е. $\overrightarrow{X} \sim \overrightarrow{Y}$, если $x_1=u*y_1,x_2=u*y_2,x_3=u*y_3$}, удовлетворяющих уравнению~\eqref{eq:weierstrass_proj}, хотя бы одна из частных производных $\frac{dF}{dX},\frac{dF}{dY},\frac{dF}{dZ}$ не обращается в $0$ на $P$. Если все три частных производные обращаются в $0$ хотя бы на одной точке $P$ (точке сингулярности), то~\eqref{eq:weierstrass_proj} -- сингулярное уравнение.
\end{definition}
            
\begin{definition} 
\textit{Эллиптическая кривая} $E$ -- множество всех точек в $\PP^2(K)$, удовлетворяющих гладкому уравнению~\eqref{eq:weierstrass_proj}. 
                
Единственная точка $\bigO$ в $E$ с координатами $(0:1:0)$, называется точкой в бесконечности. 
\end{definition}
            
\begin{definition} 
Уравнение Вейерштрасса в аффинных координатах $(x=X/Z, y=Y/Z)$:
\begin{equation}\label{eq:weierstrass_affine}
f: y^2+a_1xy + a_3y = x^3 + a_2x^2 + a_4x + a_6
\end{equation}
Тогда $F(K) = \{ (x,y) \in K \times K: f(x,y)=0 \} \union \{\bigO\}$.
                
Если $\forall i: a_i \in K$, то будем говорить, что кривая $E$ определена над $K$.
\end{definition}

Заметим, что использование проективных координат позволяет при выполнении арифметических операций на кривой избежать деления в поле за счёт увеличения количества умножений. Поэтому выбор подходящих координат для применения на практике зависит от скорости выполнения умножения по сравнению со скоростью выполнения деления.

Для определения является ли уравнение Вейерштрассе сингулярным/несингулярным используется понятие дискриминанта. 

\begin{definition}
Обозначим
\begin{equation}
\begin{split}
d_2 &= a_1^2 + 4a_2, \\
d_4 &= 2a_4 + a_1a_3, \\
d_6 &= a_3^2 + 4a_6, \\
d_8 &= a_1^2a_6 + 4a_2a_6 - a_1a_3a_4 + a_2a_3^2 - a_4^2, \\
c_4 &= d_2^2 - 24d_4.
%\text{Для проверки: } 4d_8 &= d_2d_6 - d_4^2
\end{split}
\end{equation}

Тогда \textit{дискриминант} уравнения~\eqref{eq:weierstrass_affine} определяется как 
\[
\Delta = -d_2^2d_8 - 8d_4^3-27d_6^2+9d_2d_4d_6.
\]
\end{definition}
            
\begin{theorem}[\cite{Silverman2009}, Thm. 1.4]
	Кривая, заданная уравнением Вейерштрасса, может быть классифицирована следующим образом.
	\begin{enumerate}[itemsep=0pt, topsep=0pt, partopsep=0pt]
		\item Несингулярная $\iff \Delta \neq 0$ ($\implies$ задаёт эллиптическую кривую).
		\item Кривая, обладающая узлом (нодой) $\iff \Delta = 0, c_4 \neq 0$.
		\item Кривая, обладающая точкой возврата (каспом) $\iff \Delta = c_4 = 0$.
	\end{enumerate}
\end{theorem}
%
%
\begin{figure}[h!]
	\centering
	\subfigure[$\Delta < 0$]{
		\begin{tikzpicture}
		\begin{axis}[
		xmin=-5,
		xmax=5,
		ymin=-7,
		ymax=7,
		xlabel={$x$},
		ylabel={$y$},
		scale only axis,
		axis lines=middle,
		domain=-2.1038:3,
		samples=200,
		smooth,
		clip=false,
		axis equal image=true,
		]
		\addplot [red] {sqrt(x^3-3*x+3)}
		node[right] {$y^2=x^3-3x+3$};
		\addplot [red] {-sqrt(x^3-3*x+3)};
		\end{axis}
		\end{tikzpicture}
	}
	\subfigure[$\Delta > 0$]{
		\begin{tikzpicture}
		\begin{axis}[
		xmin=-1.5,
		xmax=1.5,
		ymin=-2,
		ymax=2,
		xlabel={$x$},
		ylabel={$y$},
		scale only axis,
		axis lines=middle,
		domain=-1:0, 
		samples=200,
		smooth,
		clip=false,
		axis equal image=true,
		]
		\addplot [red] {sqrt(x^3-x)};
		\addplot [red] {-sqrt(x^3-x)};
		\end{axis}
		\begin{axis}[
		xmin=-1.5,
		xmax=1.5,
		ymin=-2,
		ymax=2,
		xlabel={$x$},
		ylabel={$y$},
		scale only axis,
		axis lines=middle,
		domain=0.1:1.5, 
		samples=200,
		smooth,
		clip=false,
		axis equal image=true,
		]
		\addplot [red] {sqrt(x^3-x)}
		node[right] {$y^2=x^3-x$};
		\addplot [red] {-sqrt(x^3-x)};
		\end{axis}
		\end{tikzpicture}
	}
	\subfigure[$\Delta < 0$]{
		\begin{tikzpicture}
		\begin{axis}[
		xmin=-1.5,
		xmax=1.5,
		ymin=-1.5,
		ymax=1.5,
		xlabel={$x$},
		ylabel={$y$},
		scale only axis,
		axis lines=middle,
		domain=0:1, 
		samples=200,
		smooth,
		clip=false,
		axis equal image=true,
		]
		\addplot [red] {sqrt(x^3+x)}
		node[right] {$y^2=x^3+x$};
		\addplot [red] {-sqrt(x^3+x)};
		\end{axis}
		\end{tikzpicture}
	}
	\caption{Эллиптические кривые над $\R$}
\end{figure}

  \begin{figure}[h!]
	\centering
	\subfigure[Касп: одна касательная, $\Delta = 0$]{
		\begin{tikzpicture}
		\begin{axis}[
		xmin=-1.5,
		xmax=6,
		ymin=-6,
		ymax=6,
		xlabel={$x$},
		ylabel={$y$},
		scale only axis,
		axis lines=middle,
		domain=0:3, 
		samples=200,
		smooth,
		clip=false,
		axis equal image=true,
		]
		\addplot [red] {sqrt(x^3)}
		node[right] {$y^2=x^3$};
		\addplot [red] {-sqrt(x^3)};
		\end{axis}
		\end{tikzpicture}
	}
	\subfigure[Нода: две касательных, $\Delta = 0$]{
		\begin{tikzpicture}
		\begin{axis}[
		xmin=-1.5,
		xmax=1.5,
		ymin=-2,
		ymax=2,
		xlabel={$x$},
		ylabel={$y$},
		scale only axis,
		axis lines=middle,
		domain=-1:1, 
		samples=200,
		smooth,
		clip=false,
		axis equal image=true,
		]
		\addplot [red] {sqrt(x^3 + x^2)}
		node[right] {$y^2=x^3 + x^2$};
		\addplot [red] {-sqrt(x^3 + x^2)};
		\end{axis}
		\end{tikzpicture}
	}
	\caption{Сингулярные кубические кривые над $\R$}
\end{figure}

\subsection{Изоморфизмы эллиптических кривых}
Если существует взаимнооднозначное отображение между двумя эллиптическими кривыми, задаваемое полиномиальными функциями (морфизмами), то кривые называются изоморфными. Такие кривые эквивалентны и в принципе нету разницы с какой изоморфной кривой работать. Это свойство достаточно широко используется на практике. Например, для приведения кривой в более простую краткую форму (см. ниже), которая имеет меньше коэффициентов и соответственно арифметика на кривой более эффективная. Также существуют различные модели кривых с ещё более оптимизированной арифметикой (например, кривые Монтгомери) или имеющие другие интересные свойства (например, кривые в форме Шолтена\cite{JouxVitse2012} уязвимы к атакам на дискретный логарифм методом спуска Вейля). Однако, не всегда существуют изоморфизмы приводящие кривую в нужную форму.

В строгой и явной форме изоморфизм задаётся следующим образом.
\begin{definition} 
Две эллиптические кривые $E_1/K$ и $E_2/K$ изоморфны, если они изоморфны как проективные многообразия, т.е. $\exists$ морфизмы $\phi: E_1/K \to E_2/K$ и $\psi: E_2/K \to E_1/K$ (определённые над $K$) такие, что $\psi \circ \phi = id_{E_1}, \phi \circ \psi = id_{E_2}$.
\end{definition}

\begin{theorem}
\label{th:isomorphisms}
Пусть $E_1/K$, $E_2/K$ -- две эллиптические кривые, заданные уравнениями 
\begin{equation}
\label{eq:isom_E1_E2}
	\begin{split}
	E_1&: y^2+a_1xy + a_3y = x^3 + a_2x^2 + a_4x + a_6, \\
	E_2&: y^2+a_1'xy + a_3'y = x^3 + a_2'x^2 + a_4'x + a_6'.
	\end{split}
\end{equation}
Тогда
\begin{equation*}
E_1 \simeq E_2 \iff \exists u,r,s,t \in K, u\neq0, \text{такие что замена}
	\end{equation*}
	\begin{equation}
	\label{eq:isomorphism}
	(x,y) \mapsto (u^2x+r, u^3y+ u^2sx+t)
	\end{equation}
	преобразует уравнение кривой $E_1$ в уравнение кривой $E_2$. Изоморфизм кривых задаёт отношение эквивалентности.
	\begin{align*}
	\phi : (x,y)&\mapsto (u^{-2}(x-r), \,u^{-3}(y-sx-t+rs))\,\text{— точки $E_1$ в $E_2$}, \\
	\psi : (x,y)&\mapsto (u^2x+r, \,u^3y+u^2sx+t))\, \text{— точки $E_2$ в $E_1$}, \\
	\phi \circ \psi = i&d_{E_2}, \psi \circ \phi = id_{E_1}.
	\end{align*}
\end{theorem}

С помощью преобразования \eqref{eq:isomorphism}, можно вывести коэффициенты кривой $E_2$:
\begin{equation}
\label{eq:of_isomorphic_curve}
\begin{split}
a_1' &= \frac{1}{u}(a_1+2s), \\ 
a_2' &= \frac{1}{u^2}(a_2-sa_1+3r-s^2), \\
a_3' &= \frac{1}{u^3}(a_3+ra_1+2t), \\
a_4' &= \frac{1}{u^4}(a_4 - sa_3 + 2ra_2 - (t+rs)a_1 + 3r^2 - 2st), \\
a_6' &= \frac{1}{u^6}(a_6 + ra_4 + r^2a_2 + r^3 - ta_3 - t^2 - rta_1).
\end{split}
\end{equation}

Аналогично, можно получить уравнения для дискриминанта:
\begin{align*}
\Delta' &= \frac{1}{u^{12}}\Delta.
\end{align*}

Определить являются ли кривые изоморфными можно
%можно подстановкой \eqref{eq:isomorphism} в уравнение кривой и 
решением системы уравнений относительно $(u,r,s,t)$ составленной из \eqref{eq:of_isomorphic_curve}.

Вместо решения системы уравнений удобней было бы определить и использовать функцию от коэффициентов кривой, которая имеет одинаковые значения на изоморфных кривых, или другими словами была бы инвариантна относительно изоморфизмов. Такая функция существует для кривых с точками над замыканием поля, называется $j$-инвариантом. Определяется следующим образом.
%Над алгебраическим замыканием поля существует функция от коэффициентов кривой, называемая $j$-инвариантом, которая позволяет упростить задачу.

Пусть $c_4 = d_2^2 - 24 d_4$ тогда $j$-инвариант эллиптической кривой $E$, $j(E)$, определяется как 
\[
j(E) = \frac{c_4^3}{\Delta}.
\]

\begin{theorem}
	$E_1 \simeq E_2$ над $\Bar{K} \iff j(E_1)=j(E_2)$.
\end{theorem}
Доказательство теоремы можно найти в \cite[46]{Silverman2009}.
%\begin{proof}
%	Докажем для случая $char(K)\neq 2,3$ (см. Silverman для общего случая).
%	
%	$\implies$ следует из формул (10).
%	
%	$\Longleftarrow$ Рассмотрим
%	\begin{align*}
%	E_1&:y^2=x^3+ax+b \\
%	E_2&=(y')^2 = (x')^3+a'x'+b'
%	\end{align*}
%	Тогда из $j(E_1)=j(E_2) \Rightarrow$
%	\begin{align*}
%	\frac{(4a)^3}{4a^3+27b^2} &= \frac{(4a')^3}{4(a')^3+27(b')^2} \\
%	4^4a^3(a')^3 + 4^3a^327(b')^2 &= 4^4 a'^3a^3 + 4^3(a')^327b^2 \\
%	a^3b^{12} &= a^{13}b^2 \tag{*}
%	\end{align*}
%	Нас интересуют только изоморфизмы вида $(x,y)\mapsto(u^2x', u^3y')$ (следствие 3).
%	
%	Рассмотрим 3 случая: \newline
%	\textbf{Случай 1.} $a = 0 \,(\Rightarrow j=0)$. Тогда $b\neq0$ (т.к. $\Delta \neq0$), $a'=0$;
%	\begin{equation*}
%	y^2=x^3+b, \quad \quad y^{12} = x'^3 + b', \quad \quad u=(\frac{b}{b'})^{\frac{1}{6}}
%	\end{equation*}
%	
%	\textbf{Случай 2.} $b=0$ (Тогда $a \neq 0, b' = 0$).
%	\[
%	u = (\frac{a}{a'})^{\frac{1}{4}}
%	\]
%	
%	\textbf{Случай 3.} $b\neq0 \implies a'b' \neq0$ 
%	\[
%	u = (\frac{a}{a'})^{\frac{1}{4}} = (\frac{b}{b'})^{\frac{1}{6}}
%	\]
%\end{proof}

Кривые, изоморфные над замыканием называются \textit{кручениями}\footnote{\textit{анг.} twists}. Так как кривые изоморфные над базовым полем $K$, также должны быть изоморфными и над замыканием, то для определения изоморфности над $K$ следует сначала сравнить $j$-инварианты, а затем уже решать систему уравнений.

\subsection{Краткие формы уравнения Вейерштрасса}
С помощью изоморфных преобразований и дополнительных ограничений можно существенно упросить общее уравнение кривой.

Пусть эллиптическая кривая задана в полной форме:
\begin{equation*}
f: y^2+a_1xy + a_3y = x^3 + a_2x^2 + a_4x + a_6. \tag{\ref{eq:weierstrass_affine}}
\end{equation*}

\textbf{Случай $\operatorname{char}{K} \neq 2$}.
Дополним до полного квадрата:
\[
y^2 + 2y(a_1x+a_3)+(a_1x+a_3))^2- \frac{1}{4}(a_1x+a_3)^2
\]
\[
\implies 4(2y + \frac{a_1x+a_3}{2})^2 = 4 + \frac{1}{4}(a_1x+a_3)^2 + (a_2x^2 + a_4x + a_6) \quad| \cdot 4
\]
\[
\implies (2y + a_1x+a_3)^2 = 4x^3 + (a_1^2 + 4a_3)x^2 + (2a_1a_3 + 4a_2)x^2 + a_3^2 + 4a_6
\]
\[
\implies y = \frac{1}{2}(y' - a_1x - a_3)
\]
\[
\implies y^2 = 4x^3 + d_2x^2 + 2d_4 + d_6.
\]
Получаем, что отображение $(x, y)\mapsto(x, \frac{1}{2}(y-a_1x-a_3))$ для $E/K, \operatorname{char}{K} \neq 2$, преобразует кривую вида \eqref{eq:weierstrass_affine} к кривой
\begin{equation}
\label{eq:weierstrass_affine_char_neq_2}
E/K: y^2 = 4x^3 + d_2x^2 + 2d_4 + d_6.
\end{equation}

\textbf{Случай $\operatorname{char}{K} \neq 2, 3$}. 
Дополним правую часть \eqref{eq:weierstrass_affine_char_neq_2} до полного куба. Замена переменных
\[
(x, y) \mapsto \left(\frac{x-3d_2}{36}, \frac{y}{216}\right)
\]
преобразует \eqref{eq:weierstrass_affine_char_neq_2} в 
\begin{equation}
\label{eq:weierstrass_affine_char_neq_2_3}
E/K: y^2 = x^3 + ax + b,
\end{equation}
\[
a = -27c_4,
\]
\[
b = -56(d_2^3 + 36d_2d_4 - 216d_6). 
\]
В этом случае, 
\begin{align*}
\Delta &= -16(4a^3 + 27b^2) \\ \nonumber
j(E) &= -1728 \frac{4a^3}{\Delta}. \nonumber
\end{align*}

\textbf{Случай $\operatorname{char}{K} = 2$.} 
\begin{enumerate}
	\item $
	j(E)\neq0 \, (a_1\neq0) \implies (x, y) \mapsto (a_1^2x+\frac{a_3}{a_1}; \, a_1^3y + \frac{a_1^2a_4+a_3^2}{a_1^3}).
	$
	\begin{equation}
	\label{eq:weierstrass_affine_char_2_case_1}
	E/K: y^2+xy=x^3+a_2'x^2+a_6'
	\end{equation}
	\item $
	j(E)\neq0 \, (a_1\neq0) \implies (x, y) \mapsto (x+a_2, y).
	$
	\begin{equation}
	\label{eq:weierstrass_affine_char_2_case_2}
	E/K: y^2+a_3y = x^3+a_4x+a_6
	\end{equation}
\end{enumerate}

%\subsubsection{Проверка изоморфности.}
%
%\begin{examples} $ $\newline
%	Кривая $E$ из~\eqref{eq:isom_E1_E2} $\cong$ \eqref{eq:weierstrass_affine}, если $\operatorname{char} K \neq 2$. \newline
%	(5) $\cong$ \eqref{eq:isom_E1_E2} $\cong$ (2), если $char K \neq 2,3$.\newline
%	(6) $\cong$ \eqref{eq:weierstrass_affine}, если $char K = 2, j(E_1)\neq0$.\newline
%	(7) $\cong$ \eqref{eq:weierstrass_affine}, если $char K = 2, j(E_1)=0$
%\end{examples}


Для кривых в краткой форме изоморфизмы имеют более простой вид.

\begin{corollary}[из Теоремы~\ref{th:isomorphisms}]
	Если $E_1, E_2$ определены над $K$ и $\operatorname{char}{K} \neq 2,3$, то \eqref{eq:isomorphism} можно упростить до преобразования
	\[
	(x,y) \mapsto (u^2x, u^3y), u \neq 0.
	\]
\end{corollary}
%            
%            \textbf{Результаты}
%            \begin{enumerate}
%                \item Получить уравнение изоморфных для $E$ кривых над $K$, надо взять $(u,r,s,t)\in K$ и получить коэффициенты изоморфной кривой из (10). \newline
%                Сложность: $\bigO$ операций деления/умножения в $K$.
%                \item Проверить, являются ли две кривые $E_1/K, E_2/K$ изоморфными: решить (10) для неизвестных $(u,r,s,t)$. Если решение в $K$ существует, значит $E_1\cong E_2$. \newline
%                Сложность: полиномиальная -- решение системы уравнений в $K$.
%                \item Определить, над каким полем $L \subseteq K$ изоморфны кривые $E_1, E_2$: в каком расширении $K$ лежат решения системы (10).
%            \end{enumerate}

\nocite{Menezes1993,Blake1999,HankersonMenezesVanstone2006,Washington2008}
\printbibliography


\end{document}