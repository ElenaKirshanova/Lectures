% !TeX program = xelatex
% !BIB TS-program = biber

\documentclass{beamer}

\usepackage[utf8]{inputenc}
\usepackage[russian]{babel}

%---tikz----
\usepackage{tikz}
\usetikzlibrary{arrows, chains, matrix, positioning, scopes, patterns, shapes}
\usepackage{pgfplots, subfigure}
\usepackage{extarrows}

\usepackage[backend=biber,firstinits=true,hyperref=true,style=numeric-comp]{biblatex}

\usepackage{../beamerthemeec2020}
{\footnotesize\bibliography{../biblio}}

\title{Эллиптические кривые}
\subtitle{Лекция 2. Групповой закон}
\author{Семён Новосёлов\\
\footnotesize{на основе курса Елены Киршановой}}
\institute{БФУ им. И. Канта}
\date{2020}

\begin{document}

\frame{\titlepage}

\begin{frame}{Групповой закон}%{Аффинные координаты}
    \begin{center}
        $E/K: y^2 = x^3 + Ax + B$
    \end{center}
    \begin{columns}
        \begin{column}{0.5\textwidth}
            \begin{equation*}
                \begin{split}
                P_1 &= (x_1, y_1) \in E \\
                P_2 &= (x_2, y_2) \in E \\
                P_3 &= P_1 + P_2 = \left(x_3, y_3\right)
                \end{split}
            \end{equation*}
                \structure{Случай $x_1 \ne x_2$:}
                \begin{equation*}
                    \begin{split}
                        x_3 &= m^2 - x_1 - x_2 \\
                        y_3 &= m\left( x_1 - x_3 \right) - y_1 \\
                        m &= \frac{y_2 - y_1}{x_2 - x_1}
                    \end{split}
                \end{equation*}
            \begin{center}
                \begin{tcolorbox}[enhanced,hbox,colback=box-blue-color!15,colframe=box-blue-color,title=Сложность,center title]
                    \begin{varwidth}{\textwidth}
                        \begin{center}
                            I + $3$M в $K$
                        \end{center}
                    \end{varwidth}
                \end{tcolorbox}	
            \end{center}

        \end{column}
        \begin{column}{0.5\textwidth}
            \begin{figure}[h!]
                \centering
                \begin{tikzpicture}[scale=0.85]
                    \begin{axis}[
                        xmin=-4,
                        xmax=5,
                        xtick=\empty,
                        ytick=\empty,
                        ymin=-5,
                        ymax=5,
                        xlabel={$x$},
                        ylabel={$y$},
                        scale only axis,
                        axis lines=middle,
                        style={thick},
                        domain=-2.279018:3,      
                        samples=201,
                        smooth,   
                        clip=false,
                        axis equal image=true,
                        ]
                        \addplot[color=plot-blue-color] {sqrt(x^3-3*x+5)} node[right] {$E$};
                        \addplot[color=plot-blue-color] {-sqrt(x^3-3*x+5)};
                        \addplot[color=plot-red-color] coordinates {(-3, -0.0890722)(3, 3.06557)} node[right] {$\ell$};
                        \addplot[color=plot-green-color] coordinates {(1.9, 3.5)(1.9, -3.5)};
                        \draw [fill=black] (axis cs: 0.65, 1.83) circle (2pt);
                        \draw[color=black] (axis cs: 1.2, 1.3) node [left]{$P_2$};
                        \draw [fill=black] (axis cs: -2.26, 0.3) circle (2pt);
                        \draw[color=black] (axis cs: -2.3, 0.3) node [left]{$P_1$};
                        \draw [fill=black] (axis cs: 1.9, 2.5) circle (2pt);
                        \draw[color=black] (axis cs: 1.9, 2.8) node [left]{$P_3'$};
                        \draw [fill=black] (axis cs: 1.9, -2.5) circle (2pt);
                        \draw[color=black] (axis cs: 1.9, -2.8) node [left]{$P_3$};
                    \end{axis}
                \end{tikzpicture}
            \end{figure}
        \end{column}
    \end{columns}
\end{frame}


\begin{frame}{Групповой закон - 2}%{Аффинные координаты}
    \begin{center}
        $E/K: y^2 = x^3 + Ax + B$
    \end{center}
    \begin{columns}
        \begin{column}{0.5\textwidth}
            \begin{equation*}
                \begin{split}
                    P_1 &= (x_1, y_1) \in E \\
                    P_2 &= (x_2, y_2) \in E \\
                    P_3 &= P_1 + P_2 = \left(x_3, y_3\right)
                \end{split}
            \end{equation*}
            \structure{Случай $x_1 = x_2$, $y_1 \neq y_2$ или $P_1 = P_2, y_1 = 0$:}
            \begin{equation*}
                {P_1} + {P_2} = \mathcal{O}
            \end{equation*}            
        \end{column}
        \begin{column}{0.5\textwidth}
            \begin{figure}[h!]
                \centering
                \begin{tikzpicture}[scale=0.85]
                    \begin{axis}[
                        xmin=-4,
                        xmax=5,
                        xtick=\empty,
                        ytick=\empty,
                        ymin=-5,
                        ymax=5,
                        xlabel={$x$},
                        ylabel={$y$},
                        scale only axis,
                        axis lines=middle,
                        style={thick},
                        domain=-2.279018:3,      
                        samples=201,
                        smooth,   
                        clip=false,
                        axis equal image=true,
                        ]
                        \addplot[color=plot-blue-color] {sqrt(x^3-3*x+5)} node[right] {$E$};
                        \addplot[color=plot-blue-color] {-sqrt(x^3-3*x+5)};
                        \addplot[color=plot-red-color] coordinates {(-1.0, 3.65)(-1.0, -3.65)};
                        \draw [fill=black] (axis cs: -1.0, 2.65) circle (2pt);
                        \draw[color=black] (axis cs: -1.0, 3.0) node [left]{$P_1$};
                        \draw [fill=black] (axis cs: -1.0, -2.65) circle (2pt);
                        \draw[color=black] (axis cs: -1.0, -3.0) node [left]{$P_2$};
                    \end{axis}
                \end{tikzpicture}
            \end{figure}
        \end{column}
    \end{columns}
\end{frame}

\begin{frame}{Групповой закон - 3}%{Аффинные координаты}
    \begin{center}
        $E/K: y^2 = x^3 + Ax + B$
    \end{center}
    \begin{columns}
        \begin{column}{0.5\textwidth}
            \begin{equation*}
                \begin{split}
                    P_1 &= (x_1, y_1) \in E \\
                    P_2 &= (x_2, y_2) \in E \\
                    P_3 &= P_1 + P_2 = \left(x_3, y_3\right)
                \end{split}
            \end{equation*}
            \structure{Случай ${P_1} = {P_2}$, ${y_1} \ne 0$:}
            \begin{equation*}
                \begin{split}
                x_3 &= m^2 - 2 x_1 \\
                y_3 &= m\left( x_1 - x_3 \right) - y_1\\
                m   &= \frac{3x_1^2 + A}{2 y_1} \\ 
                \end{split}
            \end{equation*}
            \begin{center}
                \begin{tcolorbox}[enhanced,hbox,colback=box-blue-color!15,colframe=box-blue-color,title=Сложность,center title]
                    \begin{varwidth}{\textwidth}
                        \begin{center}
                            I + $4$M в $K$
                        \end{center}
                    \end{varwidth}
                \end{tcolorbox}	
            \end{center}
            
        \end{column}
        \begin{column}{0.5\textwidth}
            \begin{figure}[h!]
                \centering
                \begin{tikzpicture}[scale=0.85]
                    \begin{axis}[
                        xmin=-4,
                        xmax=5,
                        xtick=\empty,
                        ytick=\empty,
                        ymin=-5,
                        ymax=5,
                        xlabel={$x$},
                        ylabel={$y$},
                        scale only axis,
                        axis lines=middle,
                        style={thick},
                        domain=-2.279018:3,      
                        samples=201,
                        smooth,   
                        clip=false,
                        axis equal image=true,
                        ]
                        \addplot[color=plot-blue-color] {sqrt(x^3-3*x+5)} node[right] {$E$};
                        \addplot[color=plot-blue-color] {-sqrt(x^3-3*x+5)};
                        \addplot[color=plot-red-color] {2.65 + ((-3.0) + 3*(-1.2)^2)/(2*2.65) * (x - (-1.2))}  node[right] {$\ell$};
                        \addplot[color=plot-green-color] coordinates {(2.5, 4.5)(2.5, -4.5)};
                        \draw [fill=black] (axis cs: -1.2, 2.65) circle (2pt);
                        \draw[color=black] (axis cs: -1.0, 3.0) node [left]{$P_1$};
                        \draw [fill=black] (axis cs: 2.5, 3.6) circle (2pt);
                        \draw[color=black] (axis cs: 2.5, 3.9) node [left]{$P_3'$};
                        \draw [fill=black] (axis cs: 2.5, -3.6) circle (2pt);
                        \draw[color=black] (axis cs: 2.5, -3.9) node [left]{$P_3$};
                    \end{axis}
                \end{tikzpicture}
            \end{figure}
        \end{column}
    \end{columns}
\end{frame}

\begin{frame}{Групповой закон - 4}%{Аффинные координаты}
    \begin{tcolorbox}[colframe=title-and-section-color!120, colback=title-and-section-color!5, title=Теорема, center title]
    %Операция сложения на $E$ обладает свойствами:
    \begin{enumerate}
        \item $P_1 + P_2 = P_2 + P_1$ \hfill \textit{(коммутативность)}
        
        \item $P + \mathcal{O} = P$ $\forall P \in E$ \hfill \textit{($\exists$ нейтральный элемент)}
        
        \item $\forall P \in E$ $\exists P' \in E: P + P' = \mathcal{O}$ \hfill \textit{($\exists$ обратный элемент)}
        
        \item $\left( P_1 + P_2 \right) + P_3 = P_1 + \left( P_2 + P_3 \right)$ \hfill \textit{(ассоциативность)}
    \end{enumerate}
    \end{tcolorbox}
    
    \begin{itemize}
        \item $-P = (x,-y)$ для кривой в краткой форме
    \end{itemize}
    
    \structure{Вывод:} \\$
    E(K)$ -- аддитивная абелева группа
    \\
    $E(\mathbb{Q})$ -- конечно-порожденная группа
    \\
    $E(\mathbb{F}_q)$ -- конечная группа \structure{$\Rightarrow$} криптография на DLOG
\end{frame}

\begin{frame}{Быстрое умножение точки на число}
    \begin{columns}
        \begin{column}{0.5\textwidth}
          \[
          P \to \left[ k \right] \cdot P = \underbrace {P + P +  \ldots  + P}_{k{\text{-раз}}}
          \]
        \end{column}
        \begin{column}{0.5\textwidth}
            \begin{center}
                \begin{tcolorbox}[enhanced,hbox,colback=box-blue-color!15,colframe=box-blue-color,title=Сложность,center title]
                    \begin{varwidth}{\textwidth}
                        \begin{center}
                            $k-1$ сложений (наивно)
                        \end{center}
                    \end{varwidth}
                \end{tcolorbox}	
            \end{center}
        \end{column}
    \end{columns}
%
    \begin{columns}
        \begin{column}{0.5\textwidth}
            \structure{Бинарный метод:}\\
            $k = \sum\limits_{j = 0}^{\ell - 1} {{k_j}{2^j}} ,\quad {k_j} \in \left\{ {0,1} \right\}$\\
            \begin{enumerate}
                \item $Q \leftarrow \mathcal{O}$
                \item \structure{for} $j = \ell - 1$ \structure{to} $0$ \structure{by} $-1$:\\
                \quad$Q \leftarrow \left[ 2 \right]Q$ \\
                \quad \structure{if} ${k_j} = 1$:\\
                \quad\quad$Q \leftarrow Q + P$
                \item \structure{return} $Q$
            \end{enumerate}
        \end{column}
        \begin{column}{0.5\textwidth}
            \begin{center}
                \begin{tcolorbox}[enhanced,hbox,colback=box-blue-color!15,colframe=box-blue-color,title=Сложность,center title]
                    \begin{varwidth}{\textwidth}
                        %\begin{enumerate}
                            %\item
                            удвоений: $O(\lg k)$ \\
                            %\item
                            сложений: $\omega t(k)\sim O (\lg k)$\\
                            ($\omega t$ -- вес Хэмминга $k$)\\
                        %\end{enumerate}
                            \structure{всего}: $O\left( {\lg k} \right)$
                        %\begin{center}
                            %\item $O(\lg k)$ удвоений\\
                            %$\omega t(k)\sim O (\lg k)$ сложений\\
                            %($\omega t$~--~вес Хэмминга $k$)
                            %всего:$ \Rightarrow O \left( {\lg k} \right)$ операций сложения (дублирования).
                        %\end{center}
                    \end{varwidth}
                \end{tcolorbox}	
            \end{center}
        \end{column}
    \end{columns}
    %\structure{Быстрое возведение в степень (бинарный метод):}
\end{frame}

\begin{frame}{Протокол Диффи -- Хеллмана}{Выработка общего секретного ключа}
%    \begin{columns}
%        \begin{column}{0.4\textwidth}
%            \begin{flushright}
%               \structure{{\Large\faUserSecret}}    
%            \end{flushright} 
%        \end{column}
%        \begin{column}{0.2\textwidth}
%            \begin{center}
%                $\xleftrightarrow{\hspace*{0.5cm} P \in E(\mathbb{F}_q) \hspace*{0.5cm}}$
%            \end{center}
%        \end{column}
%        \begin{column}{0.4\textwidth}
%            \begin{flushleft}
%                 \structure{{\Large\faCat}}
%            \end{flushleft}
%        \end{column}
%    \end{columns}
%    
%    \begin{center}
%        %\structure{{\Large\faUserSecret} $\longrightarrow$ {\Large\faCat}}
%        %$\stackrel{P \in E(\mathbb{F}_q)}{\xleftrightarrow{\hspace*{3cm}}}$
%        \structure{{\Large\faUserSecret}} $\xleftrightarrow{\hspace*{1cm} P \in E(\mathbb{F}_q) \hspace*{1cm}}$ \structure{{\Large\faCat}}
%        \\
%        \structure{$a \in \mathbb{N}$} $\xrightarrow{\hspace*{1cm} [a] P \hspace*{1cm}}$
%        $\hspace*{1em}$
%        \\
%        $\hspace*{1em}$ $\xleftarrow{\hspace*{1cm} [b] P \hspace*{1cm}}$
%          \structure{$b \in \mathbb{N}$}
%         \\
%    \end{center}
%    %\begin{enumerate}
%    %    \item
%        %\structure{{\Large\faUserSecret} $\longrightarrow$ {\Large\faCat}}
%    %\end{enumerate}
%
    \begin{center}
    \begin{tikzpicture}[x=0.75pt,y=0.75pt,yscale=-1,xscale=1]
        %Straight Lines [id:da0832972187828418] 
        \draw    (80,35) -- (185,35) ;
        \draw [shift={(188,35)}, rotate = 540] [fill={rgb, 255:red, 0; green, 0; blue, 0 }  ][line width=0.08]  [draw opacity=0] (5.36,-2.57) -- (0,0) -- (5.36,2.57) -- cycle    ;
        \draw [shift={(77,35)}, rotate = 360] [fill={rgb, 255:red, 0; green, 0; blue, 0 }  ][line width=0.08]  [draw opacity=0] (5.36,-2.57) -- (0,0) -- (5.36,2.57) -- cycle    ;
        %Straight Lines [id:da17602726868869878] 
        \draw    (80,70) -- (185,70) ;
        \draw [shift={(188,70)}, rotate = 540] [fill={rgb, 255:red, 0; green, 0; blue, 0 }  ][line width=0.08]  [draw opacity=0] (5.36,-2.57) -- (0,0) -- (5.36,2.57) -- cycle    ;
        %Straight Lines [id:da5151276494097119] 
        \draw    (80,105) -- (185,105) ;
        \draw [shift={(79,105)}, rotate = 360] [fill={rgb, 255:red, 0; green, 0; blue, 0 }  ][line width=0.08]  [draw opacity=0] (5.36,-2.57) -- (0,0) -- (5.36,2.57) -- cycle    ;
        
        % Text Node
        \draw (50,24) node [anchor=north west][inner sep=0.75pt]   [align=left] { \structure{{\Large\faUserSecret}} };
        \draw (200,24) node [anchor=north west][inner sep=0.75pt]   [align=right] {\structure{{\Large\faCat}}};
        \draw (102,15) node [anchor=north west][inner sep=0.75pt]   [align=center] {$P \in E(\mathbb{F}_q)$};
        % Text Node
        \draw (35,62) node [anchor=north west][inner sep=0.75pt]   [align=left] { \structure{$a \in \mathbb{N}$} };
        \draw (120,50) node [anchor=north west][inner sep=0.75pt]   [align=left] {$[a] P$};
        % Text Node
        \draw (120,86) node [anchor=north west][inner sep=0.75pt]   [align=left] {$[b]P$};
        \draw (190,98) node [anchor=north west][inner sep=0.75pt]   [align=right] { \structure{$b \in \mathbb{N}$} };
        % Text Node
        \draw (-40,130) node [anchor=north west][inner sep=0.75pt]   [align=left] {\structure{\boxed{[ab]P = [a]([b]P)}}};
        \draw (190,130) node [anchor=north west][inner sep=0.75pt]   [align=left] {\structure{\boxed{[ab]P = [b]([a]P)}}};
    \end{tikzpicture}
\end{center}
\begin{itemize}
    \item Безопасность основана на сложности нахождения \structure{DLOG}: \[(P, [n]P) \mapsto n\]
\end{itemize}

\end{frame}

\begin{frame}{Оптимизация: проективные координаты}
    \[E: Y^2 Z = X^3+ A x Z^2 + B Z^3\]
    \begin{gather*}
        P_3 = P_1 + P_2 \\
        u = Y_2 Z_1 - Y_1 Z_2\\
        v = X_2 Z_1 - X_1 Z_2 \\
        X_3 = v(\underbrace{u^2 Z_1 Z_2 - v^3 - 2 v^2 X_1 Z_2}_w), \\
        Y_3 = u(X_1 v^2 Z_2 - w) - v^3 Z_2 Y_1 \\
        Z_3 = v^3 Z_1 Z_2
    \end{gather*}
    \begin{center}
        \begin{tcolorbox}[enhanced,hbox,colback=box-blue-color!15,colframe=box-blue-color,title=Сложность,center title]
            \begin{varwidth}{\textwidth}
                $12$M (проективные) \structure{vs} I + $3$M (аффинные)
            \end{varwidth}
        \end{tcolorbox}	
    \end{center}
\end{frame}

\begin{frame}{Оптимизация: особые формы кривой}
    \structure{Кривые Монтгомери:}
   \[
   B y^2 = x^3 + A x^2 + x
   \]
   \begin{center}
       \begin{tcolorbox}[enhanced,hbox,colback=box-blue-color!15,colframe=box-blue-color,title=Сложность,center title]
           \begin{varwidth}{\textwidth}
               удвоение+сложение: $6$M + $4$S
           \end{varwidth}
       \end{tcolorbox}	
   \end{center}
   \structure{Curve25519}: $B = 1, A = 486662, q=p=2^{255} - 19$.
   
   \structure{Также:} кривые Эдвардса, в форме Якоби и др.
\end{frame}

%\begin{frame}{Групповой закон}{Альтернативные координаты}
%    content...
%\end{frame}

\begin{frame}{Литература}
%\begin{itemize}
%    \item TODO: литература по эффективной арифметике ЭК
%    \item Ссылки на стандартизированные кривые исп. проективные координаты
%\end{itemize}
\nocite{Menezes1993}\nocite{Blake1999}\nocite{Washington2008}
\printbibliography

\begin{center}
    \begin{tcolorbox}[enhanced,hbox,colback=block-green-color-bg,colframe=subsection-color!120,title=Контакты,center title]
        \begin{varwidth}{\textwidth}
            \begin{center}
                \href{mailto:snovoselov@kantiana.ru}{snovoselov@kantiana.ru}
            \end{center}
        \end{varwidth}
    \end{tcolorbox}	
\end{center}
\end{frame}

\end{document}