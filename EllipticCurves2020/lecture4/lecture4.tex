\documentclass[12pt]{article}
\usepackage{amsmath,amssymb,amsthm}
\usepackage{mathtools}
\usepackage{algorithm}
\usepackage[noend]{algpseudocode} 
\usepackage{enumitem}

\usepackage{graphicx}

%---enable russian----

\usepackage[utf8]{inputenc} 
\usepackage[russian]{babel}


%---tikz----
\usepackage{tikz}
\usetikzlibrary{arrows, chains, matrix, positioning, scopes, patterns, shapes}
\usepackage{pgfplots, subfigure}

\usepackage{biblatex}

% MACROS 
% PROBABILITY SYMBOLS
\newcommand*\PROB\Pr 
\DeclareMathOperator*{\EXPECT}{\mathbb{E}}


% GROUPS/DISTRIBUTIONS/SETS/LISTS
\newcommand{\N}{{{\mathbb N}}}
\newcommand{\Z}{{{\mathbb Z}}}
\newcommand{\Q}{{{\mathbb Q}}}
\newcommand{\R}{{{\mathbb R}}}
\newcommand{\F}{{{\mathbb F}}}
\newcommand{\PP}{{{\mathbb P}}}
\newcommand*{\IZ}{\ensuremath{\mathbb{Z}}}
\newcommand*{\IN}{\ensuremath{\mathbb{N}}}
\newcommand*{\IR}{{{\mathbb R}}}
\newcommand{\Zp}{\ints_p} % Integers modulo p
\newcommand{\Zq}{\ints_q} % Integers modulo q
\newcommand{\Zn}{\ints_N} % Integers modulo N
\newcommand{\Zr}{\ensuremath{\mathbb{Z}/r\mathbb{Z}}} % Integers modulo N
\newcommand*{\dDR}{\mathrm{d}} %de-Rham-Differential (the d in dx, dy, dz and so on)
\newcommand{\transpose}{\mkern0.1mu^{\mathsf{t}}}
\newcommand*{\union}{\mathbin{\cup}}

% Landau 
\newcommand{\bigO}{\mathcal{O}}
\newcommand*{\OLandau}{\bigO}
\newcommand*{\WLandau}{\Omega}
\newcommand*{\xOLandau}{\widetilde{\OLandau}}
\newcommand*{\xWLandau}{\widetilde{\WLandau}}
\newcommand*{\TLandau}{\Theta}
\newcommand*{\xTLandau}{\widetilde{\TLandau}}
\newcommand{\smallo}{o} %technically, an omicron
\newcommand{\softO}{\widetilde{\bigO}}
\newcommand{\wLandau}{\omega}
\newcommand{\negl}{\mathrm{negl}} 

% Misc
\newcommand{\eps}{\varepsilon}
\newcommand{\inprod}[1]{\left\langle #1 \right\rangle}
\newcommand*\abs[1]{\left\lvert#1\right\rvert}
\newcommand*\norm[1]{\left\lVert#1\right\rVert}
\renewcommand{\char}[1]{\ensuremath{\mathrm{char}(#1)}}
\newcommand{\overbar}[1]{\overline{\mkern-1.0mu#1\mkern-1.0mu}}




 
\newcommand{\handout}[5]{
  \noindent
  \begin{center}
  \framebox{
    \vbox{
      \hbox to 5.78in { {\bf  } \hfill #2 }
      \vspace{4mm}
      \hbox to 5.78in { {\Large \hfill #5  \hfill} }
      \vspace{2mm}
      \hbox to 5.78in { {\em #3 \hfill #4} }
    }
  }
  \end{center}
  \vspace*{4mm}
}

%\newcommand{\lecture}[4]{\handout{#1}{#2}{#3}{Оформил #4}{Лекция #1}}
\newcommand{\lecture}[3]{\handout{#1}{#2}{#3}{Лекция #1}}

\newtheorem{theorem}{Теорема}
\newtheorem{corollary}[theorem]{Следствие}
\newtheorem{lemma}[theorem]{Лемма}
\newtheorem{observation}[theorem]{Observation}
\newtheorem{proposition}[theorem]{Предложение}

\theoremstyle{definition}
\newtheorem{definition}[theorem]{Определение}

\newtheorem{claim}[theorem]{Утверждение}
\newtheorem{fact}[theorem]{Факт}
\newtheorem{assumption}[theorem]{Предположение}

\theoremstyle{definition}
\newtheorem{examples}[theorem]{Примеры}

\theoremstyle{definition}
\newtheorem{example}[theorem]{Пример}
\newtheorem{remark}[theorem]{Замечание}

% 1-inch margins
\topmargin 0pt
\advance \topmargin by -\headheight
\advance \topmargin by -\headsep
\textheight 8.9in
\oddsidemargin 0pt
\evensidemargin \oddsidemargin
\marginparwidth 0.5in
\textwidth 6.5in

\parindent 0in
\parskip 1.5ex

%\addbibresource{books.bib}

\begin{document}
    
\lecture{№4 Алгоритм вычисления $E\left[ n \right]$ }{Осень 2020}{Лектор: Семен Новоселов}

$E$--эллиптическая кривая над полем $K = {\F_q}$, $\char{K} \ne 2,3$

В Лекции \#~3 мы определили

\begin{itemize}
	\item точки $n$-кручения $E[n] = \left\{ {P \in E\left( {\overbar{ K} } \right) : nP = \mathcal{O}} \right\}$
	
	\item $E\left[ n \right] \cong \Z/n\Z \times \Z/n\Z \Rightarrow E\left[ n \right]$ можно записать 
	$$
	E\left[ n \right] = \left\{ {\bigO,\;\left( {{x_1},\;{y_1}} \right),... \left( {{x_m},\;{y_m}} \right)} \right\} \text{ где } m=n^2-1
	$$
	$$ 
	\Rightarrow \text{поле, где лежит } E\left[ n \right], \text{( расширение }  K \text{) можно записать}
	$$
	$$
	{K_{E,n}} = K\left(x_1, y_1, \ldots, x_m, y_m \right),
	$$
	$$
	\left[ {{K_{E,n}}:\:K} \right] = d < \infty 
	$$
\end{itemize}

В этой лекции мы построим алгоритм вычисления $E\left[ n \right]$, основанный на факторизации \underline{многочленов деления}

\begin{itemize}
	\item ${\psi _m} \in \Z\left[ {x,y,A,B} \right]$ --- многочлены деления, заданные рекуррентными соотношениями (см. лекцию 3)
	
	\item ${\varphi _m} = x \cdot \psi _m^2 - {\psi _{m + 1}}{\psi _{m - 1}}$ \\
	${\omega _m} = \dfrac{1}{{4y}}\left( {{\psi _{m + 2}}\psi _{m - 1}^2 - {\psi _{m - 2}}\psi _{m + 1}^2} \right)$
	
	\item Сложение точки $P$ с самой собой $n$ раз:
	\begin{equation}\label{F_01}
	nP = \left( {\frac{{{\varphi _n}(x)}}{{\psi _n^2(x)}},\;\frac{{{\omega _n}( x )}}{{\psi _n^3(x,y)}}} \right) 
	\end{equation}
\end{itemize}

\begin{lemma}[Доказательство см. в Serge Lang "Elliptic Curves Diophantine analysis" II 2.3]
	\label{lemm_01}
	Многочлены $\varphi_n$ и $\psi_n^2 \in K[x]$ -- взаимно просты, если $\Delta(E) \ne 0$. \\
	Т.е. для $E$ -- эллиптической кривой, $\phi_n, \psi_n^2$ -- взаимно просты. 
\end{lemma}

\begin{corollary}
	\label{corol_02}
	Пусть $P \in ( {x,y}) \in E( {\overbar{K} } )$. Тогда $nP = \bigO \Leftrightarrow \psi _n^2(x) = 0$.
\end{corollary}
\begin{proof}
    Из \eqref{F_01} $x$ -- координата $P$ задаётся $\dfrac{{{\varphi_n}( x)}}{{\psi _n^2( x)}} \in K(x)$ \\
    (из Лекции 3: ${\varphi_n}(x),\;\psi _n^2(x) \in K[x]$ для фиксированной кривой $E$).
    
    $ \Rightarrow nP = \bigO \Leftrightarrow \dfrac{{{\varphi _n}( x )}}{{\psi _n^2( x )}}$ имеет полюс в $x$, $Z$ -- координата в проективных координатах. 
    
    Так как $\gcd( {{\varphi _n}( x ),\;\psi _n^2( x ))}  = 1 \Rightarrow \dfrac{{{\varphi _n}( x ) }}{{\psi _n^2( x )}}$ имеет полюс в  $\psi _n^2( x ) = 0$
\end{proof}

$$
\psi _n^2( x ) \in K[ X ] = {n^2}{x^{{n^2} - 1}} +  \ldots {\text{ мономы меньших степеней (см. Wash. \S 3.2) }}
$$

Если $n$~---~нечетно, то и ${\psi _n}( x ) \in K[ X ]$ (см. степень $x$ в $\psi _n^2( x $).

Факторизуем ${\psi _n}$ над ${\F_q}[ x ]$. 

\begin{equation}\label{F_02}
\psi_n = f_1 ... f_r, f_i \text{ -- неприводимые над } \F_q
\end{equation}

\begin{remark}
	Все ${f_i}$ -- различные (т.~е. все ${f_i}$ встречаются в разложении со степенью $1$)
\end{remark}

\begin{itemize}
	\item Всего ${n^2} - 1$ точек $E[ n] \ne \bigO$
	
	\item Для фиксированного ${x_i} \in \underbrace {\{ {{x_1} \ldots {x_{{n^2} - s}}}\}}_{\mathclap { x - {\text{координаты точек }}n - {\text{кручения}}} }$, существует две точки $P$, $P' \in E[ n ]$ с координатой $x_i$ (т.к. $E:\:{y^2} = {x^3} + Ax + B$). 
	
	\item Так как $\deg {\psi _n}( x ) = \dfrac{{{n^2} - 1}}{2} \Rightarrow {\psi _n}( x )$ имеет $\dfrac{{{n^2} - 1}}{2}$ корней в ${\overbar{ \F_q} }$ и каждый корень кратности $1$ (иначе мы имели бы меньше чем ${n^2} - 1$ точек $ \ne \mathcal{O}$ в $E[n]$). 
\end{itemize}

Покажем, что мы можем определить $d = [ {{K_{E,n}}:\:K} ] \ne 0$ фактора $2$ из разложения \ref{F_02}. 

\begin{theorem}
	Пусть $n$~---~простое $> 2$, $K = {\F_q}$, $n \ne char( K )$.
	
	${d_i} = \deg {f_i}$ в разложении~\eqref{F_02}.
	
	$\ell = \mathrm{lcm}( {\{ {{d_i}} \}_{i = 1}^m} )$
	
	$K'_{E,n} = K( {{x_1} \ldots {x_{{n^2} - 1}}}) $, ${x_i} - x$-координаты точек $n$-кручения. Тогда 
	$$[ {K'_{E,n}:\:K} ] = \ell
	$$
	
	Кроме того, $[ {{K_{E,n}}:\:K'_{E,n}} ] = 1$ либо $2$. То есть $d = \ell$ либо $2\ell$. 
\end{theorem}
\begin{proof}
    Из следствия~\ref{corol_02}: $x_i$~---~корни ${\psi _n} \Rightarrow K'_{E,n}$~---~поле разложения ${\psi _n}$ над $K$
    
    $ \Rightarrow $ $[ {K{'_{E,n}}:\:K} ] = \mathrm{lcm}( {{d_i}} )$.
    
    (Поле разложения $f_i$ над ${\F_q}:\:\F_{q^{d_i}}$. \\
    Поле разложения $f$ над ${\F_q}$~---~наименьшие расширение ${\F_q}$, содержащие все ${\F_{q^{d_i}}}$. \\
    $\F_{q^a} \subset {\F_{q^b}} \Leftrightarrow a | b $)
    
    Для второй части теоремы, положим ${K_{E,n}} \ne K'_{E,n}$
    
    $ \Rightarrow \exists \; {x_i}:\:{y_i} = \sqrt {x_i^3 + A{x_i} + B}  \notin K'_{E,n} = {\F_{{q^\ell}}}$
    
    $ \Rightarrow K{'_{E,n}}(y_i) = \F_{q^{2\ell}}$ и $\forall x \in \F_{q^\ell}$ $\mathop {{\text{ имеет квадратный корень в }}} \F_{q^\ell}$
    
    $ \Rightarrow \forall {y_i} \in \F_{q^{2\ell}}$.
    
    $d = [ {{K_{E,n}}:\:\bar K'_{E,n}} ] \cdot [ {K'_{E,n}:\:K} ].$
\end{proof}

Значит, чтобы определить $d = \ell$, нужно показать что 

$\forall {x_i} \in \bar K$~---~корень $\psi _n^2$, ${y_i} = \sqrt {x_i^3 + A{x_i} + b} \in  \F_{q^\ell}$. 

Если существует хотя бы один ${x_j}$ т. ч. ${y_j} \notin \F_{q^\ell}$, делаем вывод, что $d = 2\ell$.

Следующее определение обобщает символ Лежандра.

\begin{definition}
	$K = {\F_q}$, $x \in K$. Квадратичный характер $( {\frac{ \cdot }{K}} ) $~---~это 
	$$
	\left( \frac{x}{K} \right ) =
	\begin{cases}
		1,&{\text{ если }}\exists y \in K:\:{y^2} = x \\
		-1,& {\text{ если }}\not\exists y \in K:\:{y^2} = x \\
		0, & {\text{ если }}x = 0.
	\end{cases}
	$$
\end{definition}

Таким образом, чтобы определить $d = \ell$ или $d = 2\ell$, необходимо вычислить 
$$
\left(  {\frac{x_i^3 + A{x_i} + B}{\F_{q^\ell}}} \right )
$$
$\forall {x_i}$~---~корни $\psi _n^2$.

Так как ${x_i}$~---~корень какого-то ${f_i}$ из~\eqref{F_02}, ${d_1} = \deg {f_i}$, то \\
$y_i^2 = x_i^3 + A{x_i} + b \in \F_{q^{d_i}}$,\\
т.к. $f_i - \min$ многочлен $x_i$ над $K$, $ {d_i}|\ell \Rightarrow {x_i} \in \F_{q^{d_i}} $ -- кв. в $\F_{q^\ell} \Rightarrow $ ${y_i} \in \F_{q^\ell}$,\\
т.е. $x_i$~---~корень $f_i$, т.ч. ${2{d_i}}|\ell$, то ${y_i} \in \F_{q^\ell}$. Т.е. нам достаточно считать квадратные характеры тех $x_i$, которые являются корнями $f_i$ с $\deg {f_i}$, т.ч. $2d_i \not | \; \ell$. 

\begin{lemma}
	
Если $K = {\F_q}$ и $d = [ {{K_{E,n}}:\:K}]$, то ${q^d} \equiv 1\bmod n$. В частности, ${\operatorname{ord} ( {q,n} }) | c$
\end{lemma}

\begin{lemma}[A.L.\ van Tuyl ``The field of $N$-Torsion Points of an Elliptic Curve over a Finite Field'']

Пусть $f_i$~---~неприводимый многочлен в разложении ${\psi _n}$ \eqref{F_02}, т.ч. $2d_i | \ell$, ${d_i} = \deg f$. Положим
$$
{d^*} = \mathrm{lcm}( {\operatorname{ord} ( {q,n} ),\;{d_i}} ),
$$
$$
c = \left(\frac{x_i^3 + Ax + B}{\F_{q^{d_i}}} \right),{\text{ где }} f_i( x_i ) = 0.$$
Тогда
$$
d = 
\begin{cases}
\ell, &\text{ если } с=1 \text{ и } . d*|l  \\
2\ell & \text{ иначе. }
\end{cases}
$$
\end{lemma}

Эта лемма позволяет рассмотреть лишь один $f_i$ (и его корень $x_i$) для определения $d$. 

\section*{Алгоритм вычисления $d$ -- степени расширения $[{K_{E,n}}:K]$}

Вход: $n \geqslant 3$ -- нечётное, $q$, $A, B$ $( E:\:{y^2} = {x^3} + Ax + B,\;\;A,B \in \F_q )$

Выход: $d$. 

\begin{enumerate}
	\item Построить ${\psi _n} \in {\F_q}[x]$
	
	\item Факторизовать ${\psi _n} = {f_1} \ldots {f_r}$ над ${\F_q}[x]$
	
	\item $\ell: = \mathrm{lcm}( {{{\{ {\deg f}\}}_{i = 1,5}}} )$
	
	\item Выбрать ${f_i}$ т.ч. $2 \cdot \deg {f_i} \not | \; l$
	
	\item Вычислить $c = \left(\dfrac{x_i^3 + A{x_i} + B}{\F_{q^{d_i}}}\right)$, где $x_i$--корень $f_i$.
	
	\item if $c_i = - 1$:
	
	\quad return $d = 2\ell$
	
	\item ${d^*} = \mathrm( {\operatorname{ord} ( {q,n}),{d_i}})$
	
	if ${d^*} = \ell$ or $\ell = n \cdot {d^*}$:
	
	\quad return $d = \ell$
	
	return $d = 2\ell$ 
\end{enumerate}

\subsection*{Оценка сложности} 
Шаг $1$. $\deg \psi_n = n^2 - 1$. Грубо: $\operatorname{poly}(n)$. \\
Шаг $2$. В Sage Berlakamp-Zessenhaus: $\bigO( (\deg \psi_n)^3 + (\deg \psi_n)^2 \cdot \operatorname{lg}(n^2) \cdot \operatorname{lg}q  )$.\\
Шаг $5$. Наивно (baby step / giant step): $\bigO(\sqrt{q^{d_i}})$. Можно привести к вычислению символа Лежандра над $\F_q$.

Итого: $\operatorname{poly}(n) \;/\; \bigO(\operatorname{poly}(n) + \sqrt{q^{d_i}})$

\begin{remark}
	Алгоритм может быть адаптирован для вычисления самой группы точек $n$-кручения $E[ n]$, если $\forall {x_i}$ -- корень ${f_i}$, вычислить соответствующие ${y_i}$:\\
	для $n = 2, E[n]$ вычисляется разложением многочлена ${x^3} + Ax + B$ (см. лекцию~\#~3) \\
	для $n = 1, E[n] = \{ \bigO \}$. 
\end{remark}
\end{document}

