\documentclass[11pt]{exam}
%%%%%%%%%%%%%%%%%%%%%%%%%%%%%%%%
%\noprintanswers % pour enlever les réponses
%\printanswers

\unframedsolutions
\SolutionEmphasis{\itshape\small}
\renewcommand{\solutiontitle}{\noindent\textbf{A: }}
%%%%%%%%%%%%%%%%%%%%%%%%%%%%%%%%

\usepackage[T2A]{fontenc}
\usepackage[utf8]{inputenc}
\usepackage[english, russian]{babel}

\usepackage{graphicx}
\usepackage{url}
\usepackage{latexsym}
\usepackage{amscd,amsmath,amsthm}
\usepackage{mathtools}
\usepackage{amsfonts}
\usepackage{amssymb}
\usepackage[dvipsnames]{xcolor}
\usepackage{hyperref}


\usepackage[margin=0.73in]{geometry}
%\usepackage[top=1in, bottom=1in, left=1in, right=1in]{geometry}

%\usepackage{fullpage}


\usepackage{hyperref}
\usepackage{appendix}
\usepackage{enumerate}



\usepackage{algorithmicx, enumitem, algpseudocode, algorithm, caption}


%%%%%%%%%%%%%%%%%%%%%%%%%%%
%% THEOREMS
%%%%%%%%%%%%%%%%%%%%%%%%%%%


\newtheorem{theorem}{Theorem}[section]
\newtheorem{axiom}[theorem]{Axiom}
\newtheorem{conclusion}[theorem]{Conclusion}
\newtheorem{condition}[theorem]{Condition}
\newtheorem{conjecture}[theorem]{Conjecture}
\newtheorem{corollary}[theorem]{Corollary}
\newtheorem{criterion}[theorem]{Criterion}
\newtheorem{definition}[theorem]{Definition}
\newtheorem{lemma}[theorem]{Lemma}
\newtheorem{notation}[theorem]{Notation}
\newtheorem{proposition}[theorem]{Proposition}


\theoremstyle{definition}
\newtheorem{problem}{Problem}


\newcommand{\nc}{\newcommand}
\nc{\eps}{\varepsilon}
\nc{\RR}{{{\mathbb R}}}
\nc{\CC}{{{\mathbb C}}}
\nc{\FF}{{{\mathbb F}}}
\nc{\NN}{{{\mathbb N}}}
\nc{\ZZ}{{{\mathbb Z}}}
\nc{\PP}{{{\mathbb P}}}
\nc{\QQ}{{{\mathbb Q}}}
\nc{\UU}{{{\mathbb U}}}
\nc{\OO}{{{\mathbb O}}}
\nc{\EE}{{{\mathbb E}}}

\newcommand{\val}{\operatorname{val}}

\newcommand{\wt}{\ensuremath{\mathit{wt}}}
\newcommand{\Id}{\ensuremath{I}}
\newcommand{\transpose}{\mkern0.7mu^{\mathsf{ t}}}
\newcommand*{\ScProd}[2]{\ensuremath{\langle#1\mathbin{,}#2\rangle}} %Scalar Product

\pretolerance=1000

\newcommand*\abs[1]{\left\lvert#1\right\rvert}

%%%%%%%%%%%%%%%%%%%%%%%%%%%%%%%%
%%%%%%%%%%%%%%%%%%%%%%%%%%%%%%%%
%% DOCUMENT STARTS
%%%%%%%%%%%%%%%%%%%%%%%%%%%%%%%%
%%%%%%%%%%%%%%%%%%%%%%%%%%%%%%%%

\begin{document}
	{\noindent
		\textsc{БФУ им. И. Канта -- Теория кодирования и сжатия информации}
		\hfill {Е. Киршанова // 2022\\}
	\hrule
	\begin{center}
		{\Large\textbf{
				\textsc{Практика № 12} \\[5pt] {05.12.22}
		} } 
	\end{center}
	\hrule \vspace{5mm}
	
	\thispagestyle{empty}
	
	\vspace{0.2cm}
	
		
	
\section{Алгоритм декодирования кода Гоппы}
		Положим $y = (y_1, \ldots, y_n)$ - полученное искаженное сообщение кода Гоппы. Обозначим за $B = \{ i \vert e_i = 1\}$ -- позиции ошибок в $y$, $\abs{B} = t \leq \lfloor \frac{d-1}{2} \rfloor$. Обозначим далее
		\begin{align*}
				\sigma(x) &= \prod_{i \in B} (x - \alpha_i), \quad \deg \sigma = t \\
				\omega(x) & = \sum_{i \in B} \prod_{j \in B, j \neq i} (x - \alpha_j), \quad \deg \omega = t-1.
		\end{align*}
 	
 		Для кода Гоппы, заданного параметрами  $g(x) = x^2+x+1$, $q=2$, $L = \FF_{2^3} \cong \FF_2[x]/(x^3+x+1)  = \{ 0, 1, \alpha, \alpha^2, \alpha^3 = \alpha+1, \alpha^4 = \alpha^2 + \alpha, \alpha^5 = \alpha^2 + \alpha+1, \alpha^6 = \alpha^2+1 \}$, с помощью алгоритма, описанного ниже, декодируйте \\
 		
 		\def\arraystretch{1.5}
 		\setlength{\tabcolsep}{15pt}
 		\begin{tabular}{l  | c }
 			Шаманов Юрий & $y = (0, 1, 1, 0, 1, 0, 1, 1)$\\ \hline
 			Белов Андрей  & $y=(1, 1, 0, 0, 0, 0, 0, 1)$\\ \hline
 			Никита Мжачих   & $y=(1, 1, 1, 1, 1, 1, 0, 1)$\\ \hline
 			Клементий Конрат  &  $y=(1, 1, 0, 1, 1, 1, 1, 1)$\\  \hline
 			Попов Никита &$y = (1, 0, 0, 0, 1, 1, 1, 1)$ \\  \hline
 			Толпекин Максим &  $y=(1, 0, 1, 0, 0, 1, 1, 1)$\\  \hline
 			Чубань Артем  &  $y= (1, 1, 0, 0, 0, 0, 0, 1)$\\  \hline
 			Микрюков Данила & $y = (1, 0, 0, 1, 1, 1, 0, 1)$\\  \hline
 			Тронина София   & $y= (1, 0, 1, 0, 0, 0, 0, 1)$\\  \hline
 			Кунинец Артем  &$y= (0, 0, 0, 1, 0, 1, 0, 1)$ \\  \hline
 			Сацута Анатолий  & $y = (1, 0, 1, 1, 0, 0, 1, 1)$\\  \hline
 			Плюснина Арина & $y= (1, 0, 1, 0, 1, 0, 0, 0)$ \\  \hline
 			Меркулова Ольга &  $y= (1, 0, 0, 1, 1, 1, 0, 1)$\\  \hline
 			Воронов Алексадр & $y = (0, 1, 1, 1, 1, 1, 0, 1) $ \\  \hline
 			Кураленко Антон & $y= (1, 1, 0, 1, 1, 1, 1, 0) $\\  \hline 
 			Тарасов Егор & $y= (1, 1, 0, 0, 1, 0, 1, 1) $\\  \hline 
 		\end{tabular}\\
 	
 		\newpage
 		
 		\textbf{Алгоритм декодирования кода Гоппы}
 		\textit{
 		\begin{enumerate}
			\item Вычислить синдром $s(x) = \sum_{i =1}^n \frac{y_i}{x-\alpha_i}~\bmod g^2(x)$
			\item Используя сравнение $\sigma(x) s(x) \equiv \omega(x) \bmod g^2(x)$, найти многочлены $\sigma(x), \omega(x)$.
			\item Найти множество  $B = \{ i \vert e_i = 1\}$ по корням $\sigma(x)$ над $\FF_{q^m}$
			\item Вычислить вектор ошибок $e$, где $e_i = \frac{\omega(\alpha_i)}{\sigma'(\alpha_i)}$.
 		\end{enumerate}
		}
		Можете использовать следующие равенства по модулю $g^2(x)$  (FIXME: не совпадает последовательность элементов с $L$)
		\begin{equation*}
		\begin{aligned}[c]
			\frac{1}{x-\alpha_1} &\equiv x^3 + x \\ 
			\frac{1}{x-\alpha_2} &\equiv (\alpha^2 + \alpha)x^3 + (\alpha^2 + \alpha + 1)x^2 + (\alpha + 1)x + \alpha^2 + \alpha \\
			\frac{1}{x-\alpha_3} &\equiv \alpha x^3 + (\alpha + 1)x^2 + (\alpha^2 + 1)x + \alpha \\
			\frac{1}{x-\alpha_4} &\equiv (\alpha^2 + \alpha)x^3 + x^2 + (\alpha^2 + 1)x + \alpha^2  \\
		\end{aligned} \quad \quad 
		\begin{aligned}[c]
		\frac{1}{x-\alpha_5} &\equiv \alpha^2x^3 + (\alpha^2 + 1)x^2 + (\alpha^2 + \alpha + 1)x + \alpha^2 \\
		 \frac{1}{x-\alpha_6} &\equiv \alpha^2x^3 + x^2 + (\alpha + 1)x + \alpha \\
		 \frac{1}{x-\alpha_7}& \equiv \alpha x^3 + x^2 + (\alpha^2 + \alpha + 1)x + \alpha^2 + \alpha \\
		  \frac{1}{x-\alpha_8} &\equiv x^3 + x^2 \\
		\end{aligned}
		\end{equation*}
	
	

\end{document}
