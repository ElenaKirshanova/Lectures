\documentclass[11pt]{exam}
%%%%%%%%%%%%%%%%%%%%%%%%%%%%%%%%
%\noprintanswers % pour enlever les réponses
%\printanswers

\unframedsolutions
\SolutionEmphasis{\itshape\small}
\renewcommand{\solutiontitle}{\noindent\textbf{A: }}
%%%%%%%%%%%%%%%%%%%%%%%%%%%%%%%%

\usepackage[T2A]{fontenc}
\usepackage[utf8]{inputenc}
\usepackage[english, russian]{babel}

\usepackage{graphicx}
\usepackage{url}
\usepackage{latexsym}
\usepackage{amscd,amsmath,amsthm}
\usepackage{mathtools}
\usepackage{amsfonts}
\usepackage{amssymb}
\usepackage[dvipsnames]{xcolor}
\usepackage{hyperref}


\usepackage[margin=0.73in]{geometry}
%\usepackage[top=1in, bottom=1in, left=1in, right=1in]{geometry}

%\usepackage{fullpage}


\usepackage{hyperref}
\usepackage{appendix}
\usepackage{enumerate}



\usepackage{algorithmicx, enumitem, algpseudocode, algorithm, caption}


%%%%%%%%%%%%%%%%%%%%%%%%%%%
%% THEOREMS
%%%%%%%%%%%%%%%%%%%%%%%%%%%


\newtheorem{theorem}{Theorem}[section]
\newtheorem{axiom}[theorem]{Axiom}
\newtheorem{conclusion}[theorem]{Conclusion}
\newtheorem{condition}[theorem]{Condition}
\newtheorem{conjecture}[theorem]{Conjecture}
\newtheorem{corollary}[theorem]{Corollary}
\newtheorem{criterion}[theorem]{Criterion}
\newtheorem{definition}[theorem]{Definition}
\newtheorem{lemma}[theorem]{Lemma}
\newtheorem{notation}[theorem]{Notation}
\newtheorem{proposition}[theorem]{Proposition}


\theoremstyle{definition}
\newtheorem{problem}{Problem}


\newcommand{\nc}{\newcommand}
\nc{\eps}{\varepsilon}
\nc{\RR}{{{\mathbb R}}}
\nc{\CC}{{{\mathbb C}}}
\nc{\FF}{{{\mathbb F}}}
\nc{\NN}{{{\mathbb N}}}
\nc{\ZZ}{{{\mathbb Z}}}
\nc{\PP}{{{\mathbb P}}}
\nc{\QQ}{{{\mathbb Q}}}
\nc{\UU}{{{\mathbb U}}}
\nc{\OO}{{{\mathbb O}}}
\nc{\EE}{{{\mathbb E}}}

\newcommand{\val}{\operatorname{val}}

\newcommand{\wt}{\ensuremath{\mathit{wt}}}
\newcommand{\Id}{\ensuremath{I}}
\newcommand{\transpose}{\mkern0.7mu^{\mathsf{ t}}}
\newcommand*{\ScProd}[2]{\ensuremath{\langle#1\mathbin{,}#2\rangle}} %Scalar Product

\pretolerance=1000

\newcommand*\abs[1]{\left\lvert#1\right\rvert}

%%%%%%%%%%%%%%%%%%%%%%%%%%%%%%%%
%%%%%%%%%%%%%%%%%%%%%%%%%%%%%%%%
%% DOCUMENT STARTS
%%%%%%%%%%%%%%%%%%%%%%%%%%%%%%%%
%%%%%%%%%%%%%%%%%%%%%%%%%%%%%%%%

\begin{document}
	{\noindent
		\textsc{БФУ им. И. Канта -- Теория кодирования и сжатия информации}
		\hfill {Е. Киршанова // 2019--2020\\}
	\hrule
	\begin{center}
		{\Large\textbf{
				\textsc{Практика № 9} \\[5pt] {5.11.21}
		} } 
	\end{center}
	\hrule \vspace{5mm}
	
	\thispagestyle{empty}
	
	\vspace{0.2cm}
	
\section{Декодирование БЧХ-кода}
		Рассмотрим БЧХ-код длины $n=15$ размерности $k=9$ с минимальным расстоянием $d=5$. %Код задан порождающим многочленом $g(x) = x^8+x^7+x^6+x^4+1$, который факторизуется над $\FF_2$ как $g(x) = (x^4+x+1)\cdot (x^4+x^3+x^2+x+1)$. 
		Примитивный элемент $\alpha$ -- корень неприводимого над $\FF_2$ многочлена $x^4+x+1$.
	
		Примените расширенный алгоритм Евклида для декодирования слова (кол-во ошибок равно 2) $y$. \\
		
		
		\def\arraystretch{1.5}
		\setlength{\tabcolsep}{15pt}
		\begin{tabular}{l  | c }
			Шаманов Юрий & $y = (1, 0, 1, 1, 0, 1, 0, 1, 1, 0, 1, 0, 0, 1, 0)$\\ \hline
			Белов Андрей   & $y=(0, 0, 1, 0, 1, 1, 0, 0, 0, 0, 0, 0, 0, 0, 1)$\\ \hline
			Никита Мжачих  &  $y=(0, 0, 1, 0, 1, 1, 0, 0, 0, 0, 0, 0, 0, 0, 1)$\\  \hline
			Клементий Конрат  &$y = (1, 0, 1, 1, 0, 1, 0, 1, 1, 0, 1, 0, 0, 1, 0)$ \\  \hline
			Попов Никита &  $y=(0, 0, 1, 0, 1, 1, 0, 0, 0, 0, 0, 0, 0, 0, 1)$\\  \hline
			Толпекин Максим&  $y=(0, 0, 1, 0, 1, 1, 0, 0, 0, 0, 0, 0, 0, 0, 1)$\\  \hline
			Чубань Артем  & $y = (1, 0, 1, 1, 0, 1, 0, 1, 1, 0, 1, 0, 0, 1, 0)$\\  \hline
			Микрюков Данила  & $y=(0, 0, 1, 0, 1, 1, 0, 0, 0, 0, 0, 0, 0, 0, 1)$\\  \hline
			Тронина София  & $y=(0, 0, 1, 0, 1, 1, 0, 0, 0, 0, 0, 0, 0, 0, 1)$ \\  \hline
			Кунинец Артем  & $y = (1, 0, 1, 1, 0, 1, 0, 1, 1, 0, 1, 0, 0, 1, 0)$\\  \hline
			Кураленко Антон & $y=(0, 0, 1, 0, 1, 1, 0, 0, 0, 0, 0, 0, 0, 0, 1)$ \\  \hline
			Тарасов Егор &  $y=(0, 0, 1, 0, 1, 1, 0, 0, 0, 0, 0, 0, 0, 0, 1)$\\  \hline
			Сацута Анатолий & $y = (1, 0, 1, 1, 0, 1, 0, 1, 1, 0, 1, 0, 0, 1, 0)$ \\  \hline
			Плюснина Арина & $y=(0, 0, 1, 0, 1, 1, 0, 0, 0, 0, 0, 0, 0, 0, 1)$\\  \hline
			Меркулова Ольга & $y=(0, 0, 1, 0, 1, 1, 0, 0, 0, 0, 0, 0, 0, 0, 1)$\\  \hline
			Воронов Алексадр & $y=(1, 0, 1, 1, 0, 1, 0, 1, 1, 0, 1, 0, 0, 1, 0)$\\  \hline
		\end{tabular}
		
		
		%$y = (1, 0, 1, 1, 0, 1, 0, 1, 1, 0, 1, 0, 0, 1, 0)$ %$y=(0, 0, 1, 0, 1, 1, 0, 0, 0, 0, 0, 0, 0, 0, 1)$, $ y=(0, 0, 1, 0, 1, 1, 0, 0, 0, 0, 0, 0, 0, 0, 1)$
	
\end{document}
