\documentclass[12pt,a4paper]{scrartcl}

%\usepackage[T1]{fontenc}
%\usepackage[utf8x]{inputenc}
\usepackage{array}

%\usepackage[ngerman]{babel}
%\usepackage{ngerman}

%---enable russian----

\usepackage[T2A]{fontenc}
\usepackage[utf8]{inputenc}
\usepackage[russian,ngerman]{babel}


\usepackage{amsmath, amsfonts}
\newcounter{blatt}
\setcounter{blatt}{99}
\parindent = 0mm
\pagestyle{empty}
\setlength{\textheight}{26cm}

\usepackage{amssymb,epsfig,hyperref}
\usepackage{fancyhdr,pdfpages,graphicx}
\newcommand{\N}{\ensuremath{\mathbb{N}}\xspace}
\newcommand{\rem}[1]{}

\usepackage{enumerate}


%complexity classes
\newcommand*{\POLY}{{\mathcal{P}}}
\newcommand*{\NPOLY}{{\mathcal{NP}}}

\newcommand{\marrow}{\marginpar[\hfill$\longrightarrow$]{$\longleftarrow$}}
\newcommand{\todo}[1][]{\textbf{TODO} \marrow\textsf{#1}}

%
% Aufruf: \auf{Aufgabentext}
%
\newcounter{auf}
\newcommand{\auf}[2]
{
\stepcounter{auf}{\textbf{Задание} \textbf{\arabic{auf}} } ({#1} баллов) \vspace{3pt}

#2

% TODO: Uncomment page break here
%\bigskip
\newpage %\phantom{X} \newpage
}
%
% kleine Buchstaben zum durchnummerieren mit enumerate auf Level 1
%
%\renewcommand{\theenumi}{\alph{enumi}}
%\renewcommand{\labelenumi}{\theenumi)}

\newcommand{\nc}{\newcommand}
\nc{\eps}{\varepsilon}
\nc{\RR}{{{\mathbb R}}}
\nc{\CC}{{{\mathbb C}}}
\nc{\FF}{{{\mathbb F}}}
\nc{\NN}{{{\mathbb N}}}
\nc{\ZZ}{{{\mathbb Z}}}
\nc{\PP}{{{\mathbb P}}}
\nc{\QQ}{{{\mathbb Q}}}
\nc{\UU}{{{\mathbb U}}}
\nc{\OO}{{{\mathbb O}}}
\nc{\EE}{{{\mathbb E}}}



\newcommand{\set}[1]{\left\{#1\right\}}
\newcommand{\cset}[2]{\left\{#1 \middle\arrowvert #2\right\}}
\newcommand{\gor}{\; | \;}

\newcolumntype{L}[1]{>{\raggedright\arraybackslash}p{#1}} % linksbündig mit Breitenangabe
\newcolumntype{C}[1]{>{\centering\arraybackslash}p{#1}} % zentriert mit Breitenangabe
\newcolumntype{R}[1]{>{\raggedleft\arraybackslash}p{#1}} % rechtsbündig mit Breitenangabe


\begin{document}
Е.\ A.\ Киршанова \hfill БФУ им. И.Канта \\





\begin{center}
  \LARGE
  \underbar{Контрольная работа}\\[1ex]
  \Large
  по дисциплине \\[1ex]
  \textsc{Теория Кодироваия и Сжатия Информации}\\[2ex]
  \large
 26.12.2022\\
Работы принимаются до 13:20 по Калининграду
\end{center}
\vspace*{1.5cm}


Имя   :\\[-2.2ex] \rule{0.9\textwidth}{.2pt}\\[2ex]
Фамилия : \\ [-2.2ex] \rule{0.9\textwidth}{.2pt}\\[1ex]
%Studiengang:\\[-2.2ex] \rule{0.7\textwidth}{.2pt}\\[1ex]
%Geburtsdatum:\\[-2.2ex] \rule{0.7\textwidth}{.2pt}\\[1ex]

\vspace*{1.5cm}
\textbf{Требования:}
\begin{itemize}
  %\item Не следует разделять скреплённые листы.  Если вам не хватает места для ответа, попросите у экзаменатора дополнительные листы. Подпишите их и приложите их к контрольной, чётко указав, решения каких заданий содержатся на каком листе.
%Schreiben Sie die Lösung jeder Aufgabe direkt auf das Blatt mit der Aufgabenstellung. Es dürfen Vorder- und R"uckseite verwendet werden. Wenn der Platz nicht ausreicht, k"onnen die leeren Seiten am Ende der Klausur benutzt werden.
 \item Для записи ответов вы можете использовать обе стороны листов.
  \item Пишите \textbf{разборчиво}.
  \item Поясняйте свои ответы.
  \item Решения отправлять на почту elenakirshanova@gmail.com
    
    
  
\end{itemize}

%\vspace*{0.5cm}
\rule{\textwidth}{.2pt}
\vfill

\begin{center}
\renewcommand{\arraystretch}{1.3}
  \begin{tabular}{|p{5cm}||c|c|c|}
    \hline
    Задание & 1 & 2 & 3  \\
    \hline
    Баллы & \phantom{XXXXXX}/ 6 & \phantom{XXXXXX}/ 6 & \phantom{XXXXXX}/ 9  \\
    \hline
  \end{tabular}\\[5ex]

  \begin{tabular}{|p{4.2cm}|p{4.2cm}|p{4.2cm}|}
    \hline
    Контрольная & Бонусы & Общая  \\
    \hline
     &   &  \\[1ex]
    \hline
  \end{tabular}
  
\end{center}

\newpage 
 
\auf{$6 \times 1 $}{%

\begin{enumerate}[1]
	\setlength\itemsep{13em}
	\item Является множество $C$ линейным кодом? Ответ поясните. Если да, опишите его длину и мин. расстояние.
	\[
		C = \{ 0000, 1010, 1010, 1111 \}.
	\]
	\item Линейный код $C$ над $\FF_5$  задан проверочной матрицей
	\[
		H = \begin{bmatrix}
		1&2&0&1&4\\ 
		3&1&1&2&1
		\end{bmatrix}.
	\]
	Есть ли среди ниже перечисленных векторов $x$ кодовые слова из $C$? Если да, то какие? Ответ поясните. \\
	
	\begin{tabular}{ p{7cm} p{7cm}  }
	$x = \left[ \begin {array}{ccccc} 1&0&1&2&2\end {array} \right]  $ 	 & $x = \left[ \begin {array}{ccccc} 4&2&0&0&1\end {array} \right]  $ \\[3ex]
		$x = \left[ \begin {array}{ccccc} 1&1&3&1&0\end {array} \right] $ & $x = \left[ \begin {array}{ccccc} 0&1&3&3&0\end {array} \right] $ 
	\end{tabular}

	\item Каково максимальное число ошибок, исправляемое следующими линейными кодами?
	\begin{enumerate}
		\setlength\itemsep{1em}
		\item $[7,4,3]_2$ -- код Хэмминга,
		\item $[2^r, r, 2^{r-1}]$ -- код Адамара,
		\item Код Рида-Соломона над $\FF_{11}$ длины $n = |\FF^*_{11}|$, размерности $k = 4$.
	\end{enumerate}	

	\item Линейный код $C$ задан параметрами $[10, 5, 4]$. Каковы длина и размерность $C^\perp$ -- дуального к $C$ кода?

	\item Линейный код над $\FF_2$ задан проверочной матрицей 	
	\[
	H = \begin{bmatrix}
		1&1&0&1&0\\ 
		1&0&1&0&1
	\end{bmatrix}.
	\]
	Опишите порождающую матрицу для этого кода. 
	
	\item Линейный код над $\FF_3$ задан порождающей матрицей
		\[
	G = \begin{bmatrix}
		1&0&1&1\\ 
		0&1&1&-1
	\end{bmatrix}.
	\]
	Является ли этот код MDS кодом? \textit{Код является MDS кодом, если $k = n-d+1$}. Ответ обоснуйте.
	%See https://a.ravagnani.win.tue.nl/coding%20th/Coding_theory_notes.pdf for a solution
	
\end{enumerate}

}
\newpage


\auf{$6$}{%
	Пусть $C-[5,2]$--линейный код над $\FF_2$, заданный проверочной матрицей
	\[
	H = 
	\begin{pmatrix}
	1&0&1&1&0\\
	1&0&1&0&1\\
	0&1&1&0&1\\
	\end{pmatrix}
	\]  
	\begin{enumerate}
		\item Найдите порождающую матрицу кода.
		\item Определите минимальное расстояние кода с помощью проверочной матрицы.
		\item Какое количество ошибок может исправить этот код?
		
		и восстановите исходное сообщение.
		\item Какие вектора $y \in \FF_2^5$ не получится однозначно декодировать этим кодом?
		\item Опишите проверочную и порождающую матрицы кода, дуального к $C$.
		\item С помощью декодирования по синдрому декодируйте $y$.
	\end{enumerate}
\begin{center}


\def\arraystretch{1.5}
\setlength{\tabcolsep}{15pt}
\begin{tabular}{l  | c }
	Шаманов Юрий    & $y = (1, 0, 1, 1, 1)$\\ \hline
	Белов Андрей  &  $y =  (0, 1, 1, 1, 1)$\\  \hline
	Никита Мжачих&  $y=(0, 1, 1, 0, 0)$\\  \hline
	Клементий Конрат  & $y = (1, 0, 1, 0, 0)$\\  \hline
	Попов Никита & $y=(1, 1, 0, 0, 0)$ \\  \hline
	Толпекин Максим &  $y=(0, 1, 0, 1, 1)$\\  \hline
	Чубань Артем & $y = (1, 0, 0, 1, 1)$ \\  \hline
	Микрюков Данила  & $y=(1, 1, 1, 1, 1)$\\  \hline
	Тронина София  & $y=(1, 1, 0, 0, 1)$\\  \hline
	Кунинец Артем  & $y=(1, 1, 0, 1, 0)$\\  \hline
	Кураленко Антон  & $y=(0, 1, 1, 1, 1)$\\  \hline
	Тарасов Егор & $y=(1, 1, 1, 0, 1)$\\  \hline
	Сацута Анатолий  & $y=(1, 1, 1, 1, 0)$\\  \hline
	Плюснина Арина  & $y=(0, 0, 1, 1, 0)$\\  \hline
	Меркулова Ольга  & $y=(0, 0, 1, 0, 1)$\\  \hline
	Воронов Алексадр  & $y=(0, 0, 0, 1, 1)$\\  \hline
\end{tabular}
\end{center}
	
}

\newpage\ \newpage
 


\auf{$9$}{%
	
\textbf{Найдите свою фамилию в одном из заданий и выполните его.}

\section{БЧХ-код}
	
Рассмотрим БЧХ-код длины $n=15$ размерности $k=9$ с минимальным расстоянием $d=5$. %Код задан порождающим многочленом $g(x) = x^8+x^7+x^6+x^4+1$, который факторизуется над $\FF_2$ как $g(x) = (x^4+x+1)\cdot (x^4+x^3+x^2+x+1)$. 
Примитивный элемент $\alpha$ -- корень неприводимого над $\FF_2$ многочлена $x^4+x+1$.\\

\begin{enumerate}
	\item Сколько ошибок может исправить этот код?
	\item Примените расширенный алгоритм Евклида для декодирования слова $y$. Восстановите исходное сообщение.
\end{enumerate}


\def\arraystretch{1.5}
\setlength{\tabcolsep}{15pt}
\begin{tabular}{l  | c }
	Белов Андрей   & $y=(0, 0, 1, 0, 1, 1, 0, 0, 0, 0, 0, 0, 0, 0, 1)$\\ \hline
	Никита Мжачих  &  $y=(0, 0, 1, 0, 1, 1, 0, 0, 0, 0, 0, 0, 0, 0, 1)$\\  \hline
	Толпекин Максим&  $y=(0, 0, 1, 0, 1, 1, 0, 0, 0, 0, 0, 0, 0, 0, 1)$\\  \hline
	Кунинец Артем  & $y = (1, 0, 1, 1, 0, 1, 0, 1, 1, 0, 1, 0, 0, 1, 0)$\\  \hline
	Кураленко Антон & $y=(0, 0, 1, 0, 1, 1, 0, 0, 0, 0, 0, 0, 0, 0, 1)$ \\  \hline
	Тарасов Егор &  $y=(0, 0, 1, 0, 1, 1, 0, 0, 0, 0, 0, 0, 0, 0, 1)$\\  \hline
	Сацута Анатолий & $y = (1, 0, 1, 1, 0, 1, 0, 1, 1, 0, 1, 0, 0, 1, 0)$ \\  \hline
	Воронов Алексадр & $y=(1, 0, 1, 1, 0, 1, 0, 1, 1, 0, 1, 0, 0, 1, 0)$\\  \hline
\end{tabular}

\section{Код Рида-Соломона}

Код Рида-Соломона задан над $\FF_{11}$ множеством $S = \{ 2^0 = 1, 2^1 = 2, 2^2 = 4, 2^3 = 8, 2^4 = 5, 2^5 =10 , 2^6 = 9, 2^7 = 7, 2^8 =3 , 2^9 = 6   \}$. Код размерности $k=6$.

\begin{enumerate}
	\item Сколько ошибок может исправить этот код?
	\item Декодируйте слово $y$. Восстановите исходное сообщение.
\end{enumerate}

\def\arraystretch{1.5}
\setlength{\tabcolsep}{15pt}
\begin{tabular}{l  | c }
	Шаманов Юрий    & $y=(4, 1, 5, 1, 10, 1, 6, 1, 3, 6)$\\ \hline
	Попов Никита &  $y=(4, 1, 8, 5, 0, 9, 6, 7, 9, 5)$\\  \hline
	Чубань Артем &  $y=(2, 5, 7, 2, 0, 1, 5, 4, 5, 1)$\\  \hline
	Микрюков Данила   & $y = (3, 8, 0, 3, 5, 6, 3, 2, 8, 0)$\\  \hline
	Тронина София & $y=(2, 5, 2, 1, 9, 8, 4, 10, 5, 7)$ \\  \hline
	Плюснина Арина &  $y=(9, 6, 0, 7, 2, 6, 0, 1, 9, 10)$\\  \hline
	Меркулова Ольга & $y = (8, 5, 7, 6, 10, 2, 7, 0, 8, 9)$ \\  \hline
	Клементий Конрат & $y=(7, 3, 0, 0, 3, 9, 10, 1, 6, 9)$\\  \hline
\end{tabular}

}

\end{document}



