\documentclass[11pt]{exam}

%---enable russian----

\usepackage[utf8]{inputenc}
\usepackage[russian]{babel}



\usepackage[margin=0.73in]{geometry}
%\usepackage[top=1in, bottom=1in, left=1in, right=1in]{geometry}

\usepackage{graphicx}
\usepackage{url}
\usepackage{latexsym}
\usepackage{amscd,amsmath,amsthm}
\usepackage{mathtools}
\usepackage{amsfonts}
\usepackage{amssymb}
\usepackage[dvipsnames]{xcolor}
\usepackage{hyperref}

\usepackage{algorithmicx, enumitem, algpseudocode, algorithm, caption}
\usepackage{tikz}
\usetikzlibrary{automata}

%%%%%%%%%%%%%%%%%%%%%
% Handling comments and versions %%%
%%%%%%%%%%%%%%%%%%%%%

%\renewcommand{\comment}[1]{\texttt{[#1]}}


%%%%%%%%%%%%%%%%%%%%%%%%%%%
%% THEOREMS
%%%%%%%%%%%%%%%%%%%%%%%%%%%

\newtheorem{theorem}{Theorem}[section]
\newtheorem{axiom}[theorem]{Axiom}
\newtheorem{conclusion}[theorem]{Conclusion}
\newtheorem{condition}[theorem]{Condition}
\newtheorem{conjecture}[theorem]{Conjecture}
\newtheorem{corollary}[theorem]{Corollary}
\newtheorem{criterion}[theorem]{Criterion}
\newtheorem{definition}[theorem]{Definition}
\newtheorem{lemma}[theorem]{Lemma}
\newtheorem{notation}[theorem]{Notation}
\newtheorem{proposition}[theorem]{Proposition}


\theoremstyle{definition}
\newtheorem{problem}{Problem}


\newcommand{\nc}{\newcommand}
\nc{\eps}{\varepsilon}
\nc{\RR}{{{\mathbb R}}}
\nc{\CC}{{{\mathbb C}}}
\nc{\FF}{{{\mathbb F}}}
\nc{\NN}{{{\mathbb N}}}
\nc{\ZZ}{{{\mathbb Z}}}
\nc{\PP}{{{\mathbb P}}}
\nc{\QQ}{{{\mathbb Q}}}
\nc{\UU}{{{\mathbb U}}}
\nc{\OO}{{{\mathbb O}}}
\nc{\EE}{{{\mathbb E}}}

\newcommand{\val}{\operatorname{val}}
\newcommand{\wt}{\ensuremath{\mathit{wt}}}
\newcommand{\Id}{\ensuremath{I}}
\newcommand{\transpose}{\mkern0.7mu^{\mathsf{ t}}}
\newcommand*{\ScProd}[2]{\ensuremath{\langle#1\mathbin{,}#2\rangle}} %Scalar Product
\renewcommand{\char}{\ensuremath{\mathsf{char}}}

\DeclareMathOperator{\Vol}{Vol}

%\pretolerance=1000

%%%%%%%%%%%%%%%%%%%%%%%%%%%%%%%%
%%%%%%%%%%%%%%%%%%%%%%%%%%%%%%%%
%% DOCUMENT STARTS
%%%%%%%%%%%%%%%%%%%%%%%%%%%%%%%%
%%%%%%%%%%%%%%%%%%%%%%%%%%%%%%%%


\begin{document}	
	{\noindent
		\textsc{БФУ им. И. Канта -- Компьютерный практикум по криптографии на эллиптических кривых }\\[5pt]
		Преподаватель {С. Новоселов}   \hfill{Осень 2024\\}
	\hrule
	\begin{center}
		{\LARGE\textbf{
				Лабораторная работа № 1 \\[5pt]
		}} 
			Опубликована \textbf{27.09.2024} \\[5pt] 
			Дедлайн \textbf{10.10.2024}
		
	\end{center}
	\hrule \vspace{5mm}
	
	\thispagestyle{empty}
	
    Разработать программу в системе компьютерной алгебры Sage, реализующую следующие функции:

\begin{enumerate}
	\item \texttt{jInvariant}($a_1$, $a_2$, $a_3$, $a_4$, $a_6$), где $a_1$, $a_2$, $a_3$, $a_4$, $a_6$ -- коэффициенты кривой, заданной уравнением Вейерштрасса. Если кривая является эллиптической, функция возвращает $j$-инвариант кривой, иначе сообщение о том, что кривая сингулярна.
	\item \texttt{randIsomorphic}($a_1 = 0$, $a_2 = 0$, $a_3 = 0$, $a_4 = 0$, $a_6 = 0$), где $a_1$, $a_2$, $a_3$, $a_4$, $a_6$, $a$, $b$ -- коэффициенты эллиптической кривой $E_1$ в общем случае, или в случае $char(K)\neq 2, 3$. Функция возвращает коэффициенты кривой~$E_2$, изоморфной~$E_1$ над $\mathbb{Q}$ путём случайного выбора параметров~$(u, r, s, t)$. Если коэффициенты $a_1$, $a_2$, $a_3$, $a_4$, $a_6$ задают сингулярную кривую, функция терминирует с соответствующим сообщением.
	\item \texttt{isIsomorphic}($a_1$, $a_2$, $a_3$, $a_4$, $a_6$, $\_a_1$, $\_a_2$, $\_a_3$, $\_a_4$, $\_a_6$, $p$), где $a_1$, $a_2$, $a_3$, $a_4$, $a_6$ -- коэффициенты эллиптической кривой~$E_1$, $\_a_1$, $\_a_2$, $\_a_3$, $\_a_4$, $\_a_6$ -- коэффициенты эллиптической кривой $E_2$, $p$ -- простое число (означает кривые заданы над~$\mathbb{F}_p$) или $0$ (кривые заданы над~$\mathbb{Q}$). Функция определяет, являются ли кривые изоморфными над~$\mathbb{F}_p$ (или~$\mathbb{Q}$), и возвращает одно из значений $\in$ $\{isomorphic, non-isomorphic\}$. Если коэффициенты $a_1$, $a_2$, $a_3$, $a_4$, $a_6$ или $\_a_1$, $\_a_2$, $\_a_3$, $\_a_4$, $\_a_6$ задают сингулярную кривую, функция терминирует с соответствующим сообщением.
	\item \texttt{findExtension}($a_1$, $a_2$, $a_3$, $a_4$, $a_6$, $\_a_1$, $\_a_2$, $\_a_3$, $\_a_4$, $\_a_6$, $p$), коэффициенты эллиптической кривой $E_1$,
	$\_a_1$, $\_a_2$, $\_a_3$, $\_a_4$, $\_a_6$ -- коэффициенты эллиптической кривой $E_2$, заданные над~$\mathbb{F}_p$ ($p$ интерпретировать аналогично предыдущей функции).Функция определяет, над каким полем кривые~$E_1 \simeq E_2$ и возвращает степень расширения этого поля над~$\mathbb{F}_p$. Если коэффициенты $a_1$, $a_2$, $a_3$, $a_4$, $a_6$ или $\_a_1$, $\_a_2$, $\_a_3$, $\_a_4$, $\_a_6$ задают сингулярную кривую, функция терминирует с соответствующим сообщением.
\end{enumerate}

\section*{Требования к сдаче}
\begin{itemize}
	\item Исходный код должен содержать комментарии к каждой из функций с описанием входных и выходных параметров
	\item Лабораторную следует выполнять модификацией файла с тестами, заменяя строку\\\texttt{\# your code here.}\\на код, реализующий функцию.
	\item Функции должны работать на всех примерах, что проверяется запуском команды:
	\\\texttt{sage -t file\_with\_tests.sage}
	\item Студент должен понимать, что он написал, зачем, а также ответить на теоретические вопросы.
\end{itemize}
\end{document}
