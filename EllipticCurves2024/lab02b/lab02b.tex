\documentclass[11pt]{exam}

%---enable russian----

\usepackage[utf8]{inputenc}
\usepackage[russian]{babel}



\usepackage[margin=0.73in]{geometry}
%\usepackage[top=1in, bottom=1in, left=1in, right=1in]{geometry}

\usepackage{graphicx}
\usepackage{url}
\usepackage{latexsym}
\usepackage{amscd,amsmath,amsthm}
\usepackage{mathtools}
\usepackage{amsfonts}
\usepackage{amssymb}
\usepackage[dvipsnames]{xcolor}
\usepackage{hyperref}

\usepackage{algorithmicx, enumitem, algpseudocode, algorithm, caption}
\usepackage{tikz}
\usetikzlibrary{automata}

%%%%%%%%%%%%%%%%%%%%%
% Handling comments and versions %%%
%%%%%%%%%%%%%%%%%%%%%

%\renewcommand{\comment}[1]{\texttt{[#1]}}


%%%%%%%%%%%%%%%%%%%%%%%%%%%
%% THEOREMS
%%%%%%%%%%%%%%%%%%%%%%%%%%%

\newtheorem{theorem}{Theorem}[section]
\newtheorem{axiom}[theorem]{Axiom}
\newtheorem{conclusion}[theorem]{Conclusion}
\newtheorem{condition}[theorem]{Condition}
\newtheorem{conjecture}[theorem]{Conjecture}
\newtheorem{corollary}[theorem]{Corollary}
\newtheorem{criterion}[theorem]{Criterion}
\newtheorem{definition}[theorem]{Definition}
\newtheorem{lemma}[theorem]{Lemma}
\newtheorem{notation}[theorem]{Notation}
\newtheorem{proposition}[theorem]{Proposition}


\theoremstyle{definition}
\newtheorem{problem}{Problem}


\newcommand{\nc}{\newcommand}
\nc{\eps}{\varepsilon}
\nc{\RR}{{{\mathbb R}}}
\nc{\CC}{{{\mathbb C}}}
\nc{\FF}{{{\mathbb F}}}
\nc{\NN}{{{\mathbb N}}}
\nc{\ZZ}{{{\mathbb Z}}}
\nc{\PP}{{{\mathbb P}}}
\nc{\QQ}{{{\mathbb Q}}}
\nc{\UU}{{{\mathbb U}}}
\nc{\OO}{{{\mathbb O}}}
\nc{\EE}{{{\mathbb E}}}

\newcommand{\val}{\operatorname{val}}
\newcommand{\wt}{\ensuremath{\mathit{wt}}}
\newcommand{\Id}{\ensuremath{I}}
\newcommand{\transpose}{\mkern0.7mu^{\mathsf{ t}}}
\newcommand*{\ScProd}[2]{\ensuremath{\langle#1\mathbin{,}#2\rangle}} %Scalar Product
\renewcommand{\char}{\ensuremath{\mathsf{char}}}

\DeclareMathOperator{\Vol}{Vol}

%\pretolerance=1000

%%%%%%%%%%%%%%%%%%%%%%%%%%%%%%%%
%%%%%%%%%%%%%%%%%%%%%%%%%%%%%%%%
%% DOCUMENT STARTS
%%%%%%%%%%%%%%%%%%%%%%%%%%%%%%%%
%%%%%%%%%%%%%%%%%%%%%%%%%%%%%%%%


\begin{document}	
	{\noindent
		\textsc{БФУ им. И. Канта -- Компьютерный практикум по криптографии на эллиптических кривых }\\[5pt]
		Преподаватель {С. Новоселов}   \hfill{Осень 2024\\}
	\hrule
	\begin{center}
		{\LARGE\textbf{
				Лабораторная работа № 2b \\[5pt]
		}} 
			Опубликована \textbf{27.09.2024} \\[5pt] 
			Дедлайн \textbf{24.10.2024}
		
	\end{center}
	\hrule \vspace{5mm}
	
	\thispagestyle{empty}
	
	Разработать программу в системе компьютерной алгебры Sage для выработки общего ключа~$sk$ с удалённым сервером и обмена сообщениями зашифрованными симметричным шифром на ключе~$sk$.

	\begin{enumerate}
		\item Сервер доступен по адресу:\\
		\texttt{tasks.crypto-kantiana.com}\\
		порт: 10781
		
		\item Сообщения шифровать с помощью шифра Camellia в режиме OFB с начальным вектором (iv), полученным от сервера.
		
		\item Для работы с симметричной криптографией рекомендуется использовать модуль \texttt{cryptography} для Python 3. Если он не установлен, то установить с помощью команды:\\
		\texttt{pip install cryptography}
		
		\item В качестве общего секретного ключа использовать координату $x$ точки, переведенную в байтовую строку кодом:\\
		\texttt{int(x).to\_bytes(len(p.binary())/8, "big")}
		
		\item Все байтовые строки, передаваемые по сети должны кодироваться/декодироваться с помощью base64.
	\end{enumerate}

\section*{Требования к сдаче}
    \begin{itemize}
        \item При выработке общего ключа со своей стороны, использовать разработанные при выполнении предыдущих лабораторных методы (сложение точек, скалярное умножение и т.д.). Использование встроенных методов Sage для этой цели \textbf{не допускается}.
        \item Исходный код должен содержать комментарии к каждой из функций с описанием входных и выходных параметров
        \item Студент должен понимать, что он написал, зачем, а также ответить на теоретические вопросы.
    \end{itemize}
\end{document}
