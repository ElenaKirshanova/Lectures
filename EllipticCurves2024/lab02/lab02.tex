\documentclass[11pt]{exam}

%---enable russian----

\usepackage[utf8]{inputenc}
\usepackage[russian]{babel}



\usepackage[margin=0.73in]{geometry}
%\usepackage[top=1in, bottom=1in, left=1in, right=1in]{geometry}

\usepackage{graphicx}
\usepackage{url}
\usepackage{latexsym}
\usepackage{amscd,amsmath,amsthm}
\usepackage{mathtools}
\usepackage{amsfonts}
\usepackage{amssymb}
\usepackage[dvipsnames]{xcolor}
\usepackage{hyperref}

\usepackage{algorithmicx, enumitem, algpseudocode, algorithm, caption}
\usepackage{tikz}
\usetikzlibrary{automata}

%%%%%%%%%%%%%%%%%%%%%
% Handling comments and versions %%%
%%%%%%%%%%%%%%%%%%%%%

%\renewcommand{\comment}[1]{\texttt{[#1]}}


%%%%%%%%%%%%%%%%%%%%%%%%%%%
%% THEOREMS
%%%%%%%%%%%%%%%%%%%%%%%%%%%

\newtheorem{theorem}{Theorem}[section]
\newtheorem{axiom}[theorem]{Axiom}
\newtheorem{conclusion}[theorem]{Conclusion}
\newtheorem{condition}[theorem]{Condition}
\newtheorem{conjecture}[theorem]{Conjecture}
\newtheorem{corollary}[theorem]{Corollary}
\newtheorem{criterion}[theorem]{Criterion}
\newtheorem{definition}[theorem]{Definition}
\newtheorem{lemma}[theorem]{Lemma}
\newtheorem{notation}[theorem]{Notation}
\newtheorem{proposition}[theorem]{Proposition}


\theoremstyle{definition}
\newtheorem{problem}{Problem}


\newcommand{\nc}{\newcommand}
\nc{\eps}{\varepsilon}
\nc{\RR}{{{\mathbb R}}}
\nc{\CC}{{{\mathbb C}}}
\nc{\FF}{{{\mathbb F}}}
\nc{\NN}{{{\mathbb N}}}
\nc{\ZZ}{{{\mathbb Z}}}
\nc{\PP}{{{\mathbb P}}}
\nc{\QQ}{{{\mathbb Q}}}
\nc{\UU}{{{\mathbb U}}}
\nc{\OO}{{{\mathbb O}}}
\nc{\EE}{{{\mathbb E}}}

\newcommand{\val}{\operatorname{val}}
\newcommand{\wt}{\ensuremath{\mathit{wt}}}
\newcommand{\Id}{\ensuremath{I}}
\newcommand{\transpose}{\mkern0.7mu^{\mathsf{ t}}}
\newcommand*{\ScProd}[2]{\ensuremath{\langle#1\mathbin{,}#2\rangle}} %Scalar Product
\renewcommand{\char}{\ensuremath{\mathsf{char}}}

\DeclareMathOperator{\Vol}{Vol}

%\pretolerance=1000

%%%%%%%%%%%%%%%%%%%%%%%%%%%%%%%%
%%%%%%%%%%%%%%%%%%%%%%%%%%%%%%%%
%% DOCUMENT STARTS
%%%%%%%%%%%%%%%%%%%%%%%%%%%%%%%%
%%%%%%%%%%%%%%%%%%%%%%%%%%%%%%%%


\begin{document}	
	{\noindent
		\textsc{БФУ им. И. Канта -- Компьютерный практикум по криптографии на эллиптических кривых }\\[5pt]
		Преподаватель {С. Новоселов}   \hfill{Осень 2024\\}
	\hrule
	\begin{center}
		{\LARGE\textbf{
				Лабораторная работа № 2 \\[5pt]
		}} 
			Опубликована \textbf{27.09.2024} \\[5pt] 
			Дедлайн \textbf{17.10.2024}
		
	\end{center}
	\hrule \vspace{5mm}
	
	\thispagestyle{empty}
	
	Разработать программу в системе компьютерной алгебры Sage, реализующую следующие функции:
	
	\begin{enumerate}
		\item \texttt{Sum($\mathtt{a, b, q,  x1, y1, x2, y2}$)}, где $\mathtt{a, b}$ -- коэффициенты эллиптической кривой $E$, заданной над полем $\FF_q$, $q$ -- простое, $\neq 2,3$, $P_1 = \mathtt{(x1, y1)}, P_2 = \mathtt{(x2, y2)}$-- точки на $E$ ($y_i  = \text{infinity}$ для $P_i = \mathcal{O}$). Функция возвращает координаты $P_3 = \mathtt{(x3, y3)}= P_1 + P_2$. Если $P_1$ или $P_2$ не лежат на $E$, функция возвращает ошибку.
		
		\item \texttt{SumProj($\mathtt{a, b, q,  x1, y1, z1, x2, y2, z2}$)}, те же параметры и выходные данные, что и для функции \texttt{Sum($\mathtt{a, b, q,  x1, y1, x2, y2}$)}, но точки~$P_1, P_2 $ заданы в проективных координатах. Вычисления проводятся  также с проективными координатами.
		
		\item \texttt{Mul($\mathtt{a, b, q,  x1, y1, k}$)}, где $\mathtt{a, b}$ -- коэффициенты эллиптической кривой~$E$, заданной над полем~$\FF_q$, где~$q$ -- простое, $\neq 2,3$, $P_1 = (x_1, y_1) \in E$, $k \in \ZZ$. Функция возвращает координаты точки~$P_k = (x_k, y_k) = k \cdot P_1$. Если~$P_1 \notin E$, функция возвращает ошибку.
	\end{enumerate}

\section*{Требования к сдаче}
    \begin{itemize}
        \item Исходный код должен содержать комментарии к каждой из функций с описанием входных и выходных параметров
        \item Лабораторную следует выполнять модификацией файла с тестами, заменяя строку
        
        \texttt{\# your code here.}
        
        на код, реализующий функцию.
        \item Функции должны работать на всех примерах, что проверяется запуском команды:
        \\\texttt{sage -t file\_with\_tests.sage}
        \item Студент должен понимать, что он написал, зачем, а также ответить на теоретические вопросы.
    \end{itemize}
\end{document}
