\documentclass[usenames,dvipsnames,8pt,aspectratio=169]{beamer}
\usepackage{amsmath,amsfonts,amssymb}
\usepackage{mathtools}
\usepackage{etex} %for Windows
\usepackage[utf8]{inputenc}
\usepackage[english, russian]{babel} 
%\usepackage{microtype}			% Better interword spacing and additional kerning.
\usepackage{ellipsis}			% Adjusted space with \dots between two words.
\usepackage{graphicx}
\usepackage{pstricks}

\usepackage{xcolor}


\usepackage{changepage}

\usepackage{algorithm}
\usepackage{algpseudocode}
%\usepackage[]{algorithm2e}
%\usepackage{algorithmic}

%\usepackage{tcolorbox}


\usepackage{caption}
\usepackage{subcaption}
%\usepackage{stackengine}


\usepackage{tikz}
\usetikzlibrary{tikzmark,calc}
\usetikzlibrary{positioning, backgrounds}
\usetikzlibrary{arrows, chains, matrix, scopes, patterns, shapes, fit}
\usetikzlibrary{mindmap,trees,shadows}
\usetikzlibrary{decorations.pathreplacing}
%\usetikzlibrary{crypto.symbols}

\usepackage{pgfplots}

\pgfmathdeclarefunction{gauss}{2}{%
	\pgfmathparse{1/(#2*sqrt(2*pi))*exp(-((x-#1)^2)/(2*#2^2))}%
}


\tikzset{
	invisible/.style={opacity=0},
	visible on/.style={alt={#1{}{invisible}}},
	alt/.code args={<#1>#2#3}{%
		\alt<#1>{\pgfkeysalso{#2}}{\pgfkeysalso{#3}} % \pgfkeysalso doesn't change the path
	},
}

\newcommand\strikeout[2][]{%
	\begin{tabular}[b]{@{}c@{}} 
		\makebox(0,0)[cb]{{#1}} \\[-0.2\normalbaselineskip]
		\rlap{\color{Orange}\rule[0.5ex]{\widthof{#2}}{1.5pt}}#2
\end{tabular}}

\newcommand\Fontvi{\fontsize{11}{13.2}\selectfont}

\usepackage{listings} % for C++ code

\usepackage{braket}
%\usepackage[braket, qm]{qcircuit}



\usepackage[T1]{fontenc}
%\usepackage[sfdefault,scaled=.85]{FiraSans}
%\usepackage{newtxsf}
%\usepackage[nomap]{FiraMono}





\usefonttheme[onlymath]{serif}
\renewcommand\sfdefault{cmbr}

\renewcommand{\bfdefault}{sb}

\definecolor{CharCoalDark}{RGB}{13, 16, 19}
\definecolor{Orange}{RGB}{255, 165,0}
\definecolor{DarkOrange}{RGB}{255, 165,0}
\definecolor{LightSalmon}{RGB}{255, 160, 122}
\definecolor{LeafGreen}{RGB}{34, 139,  34}
\definecolor{Coral}{RGB}{255, 127, 80}
\definecolor{DarkTurquoise}{RGB}{0, 206, 209}

%\newtheorem{defRus}{Определение}
%\newtheorem{thmRus}{Теорема}
%s\newtheorem{corRus}{Следствие}


\setbeamercolor{background canvas}{bg=CharCoalDark}

\setbeamerfont{title}{series=\bfseries}
\setbeamercolor{title}{fg=Orange}
\setbeamercolor{section in toc}{fg=white}
\setbeamercolor{frametitle}{fg=Orange}
\setbeamercolor{normal text}{fg=white}
%\setbeamercolor{normal text}{fontsize=12pt}
\setbeamercolor{itemize item}{fg=Orange}
\setbeamercolor{enumerate item}{fg=Orange}
\setbeamercolor{enumerate item item}{fg=Orange}
\setbeamercolor{itemize item item}{fg=Orange}
\setbeamercolor{enumerate item}{fg=Orange}
\setbeamercolor{block title}{bg=DarkOrange,fg=white}
\setbeamerfont{block title}{series=\bfseries}

\setbeamertemplate{itemize item}[circle]
\setbeamertemplate{eumerate subitem}{\color{Orange}[$\checkmark$]}
\setbeamertemplate{itemize subitem}{\color{Orange}\Large$\textbullet$}
\setbeamertemplate{itemize subitem}{\color{Orange} \tiny $\blacksquare$}

% footnote without a marker
\newcommand\blfootnote[1]{%
	\begingroup
	\renewcommand\footnoterule{}
	\renewcommand\thefootnote{}\footnote{#1}%
	\addtocounter{footnote}{-1}%
	\endgroup
}

\addtobeamertemplate{footline}{%
	\setlength\unitlength{1ex}%
	\begin{picture}(0,0) 
	% \put{} defines the position of the frame
	\put(155,0){\makebox(0,0)[bl]{
			%\includegraphics[scale=0.65]{white_square}
			%\includegraphics[scale=0.65]{dark_square}
			\includegraphics[scale=0.65]{grey_circle}
	}}%
	\end{picture}%
}{}


\newcommand*{\Scale}[2][4]{\scalebox{#1}{\ensuremath{#2}}}%

\newcommand\Item[1][]{%
	\ifx\relax#1\relax  \item \else \item[#1] \fi
	\abovedisplayskip=0pt\abovedisplayshortskip=0pt~\vspace*{-\baselineskip}}

\pgfdeclareradialshading{ring}{\pgfpoint{0cm}{0cm}}%
{rgb(0cm)=(1,1,1);
	rgb(0.7cm)=(1,1,1);
	rgb(0.719cm)=(1,1,1);
	rgb(0.72cm)=(0.975,0,0);
	rgb(0.9cm)=(1,1,1)}

\usepackage[absolute,overlay]{textpos} %to clip to a corner
\newcommand\FrameText[1]{%
	\begin{textblock*}{\paperwidth}(\textwidth-35pt, 10 pt)
		\raggedright #1\hspace{.5em}
\end{textblock*}}

\makeatletter
\let\save@measuring@true\measuring@true
\def\measuring@true{%
	\save@measuring@true
	\def\beamer@sortzero##1{\beamer@ifnextcharospec{\beamer@sortzeroread{##1}}{}}%
	\def\beamer@sortzeroread##1<##2>{}%
	\def\beamer@finalnospec{}%
}
\makeatother

\AtBeginSection[]
{
	\begin{frame}<beamer>
		\frametitle{Outline}
		\tableofcontents[currentsection]
	\end{frame}
}

\titlegraphic{
	
	%\includegraphics[width=2.5cm]{stayhome}%
	%\includegraphics[width=4.0cm]{ens_logo_gray}
}
\title{Лекция №5 \\[10pt]
	Часть I. Криптографическая хэш-функция.}

\date{ Елена Киршанова \\  \textbf{Курс ``Основы криптографии''} \\  }



\setbeamertemplate{navigation symbols}{} %removes navigation

% proper highlightling of a code-snippet
\lstset{language=C++,
	keywordstyle=\color{magenta},
	stringstyle=\color{Goldenrod},
	commentstyle=\color{gray},
	breaklines=false,
	%morecomment=[l][\color{magenta}]{\#}
}

%\setlength{\parskip}{8pt}
\input{header} %all defs
\begin{document}
	
\begin{frame}
	\titlepage
\end{frame}

\begin{frame}{Хэш-функции}

\Large
\begin{center}
	{\color{Orange} Криптографическая хэш-функция ! = Хэш-функция} \\[10pt]
\end{center}

Фильтры Блума, контрольные суммы (sumXXX, fletcherXXX),  не являются криптографическими хэш-функциями.


\end{frame}

\begin{frame}{Криптографическая хэш-функция: определение}
\Large
	{\color{Orange}Криптографическая хэш-функция} -- \underline{двойка} полиномиальных алгоритмов $(\Gen, \Hash)$:
	\begin{enumerate}
		\itemsep 7pt
		\item Вероятностный $\Gen: s \leftarrow \Gen(1^\lambda)$  
		\item Детерминированный $\Hash_s: \{0,1\}^\star \rightarrow \{0,1\}^\ell$,
	\end{enumerate}

где $\Hash_s$ является  {\color{Orange} стойкой к коллизиям}: \\[5pt]

для заданного $s$,  не существует ppt алгоритма, который находит $x, x' (x!=x')$,
\[
	\Hash_s(x) = \Hash(x')
\]

{\color{Orange} Криптографическая} хэш-функция \underline{обязана} быть стойкой к коллизиям.\\[5pt]

Существует  \underline{много} коллизий для $\Hash_s$, но должно быть \underline{трудно} найти \\ любую коллизию.
\end{frame}

\begin{frame}{Свойство криптографической хэш-функции}
\Large 
\[
\Hash_s: \{0,1\}^\star \rightarrow \{0,1\}^\ell
\]

	{\color{Orange} I.} Стойкость к нахождению прообраза (или односторонность) \\[5pt]
	Дано: $(s, y \in \{0,1\}^\ell)$\\
	Найти: $x$, такой что \ $\Hash_s(x) =  y$ \\[5pt]
	{\color{Orange} Стойкая к коллизиям хэш-функция является стойкой к нахождению прообраза}
	

	\vspace{20pt}
	{\color{Orange} II.} Стойкость к нахождению $2$-го прообраза \\[5pt]
	Дано: $(s, x)$\\
	Найти: $x'!=x$, такой что\ $\Hash_s(x) =  \Hash_s(x')$ \\[5pt]
	{\color{Orange}  Стойкая к коллизиям хэш-функция является стойкой к нахождению \\ $2$-го прообраза}
	
	\vspace{20pt}
	
	Стойкость к коллизиям  $\implies$ 	{\color{Orange} II.}  $\implies$ 	{\color{Orange} I.}  
\end{frame}

\begin{frame}{Экзотические свойства хэш-функций}
	\LARGE
	В мире Bitcoin три свойства хэш-функции могу называться иначе:
	
	\begin{align*}
		\text{нахождение прообраза} & \rightarrow  \text{``hiding''} \\
		\text{нахождение 2-го прообраза} & \rightarrow  \text{``puzzle-friendliness''} \\
		\text{стойкость к коллизиям} & \rightarrow  \text{``collision resistance''} \\
	\end{align*}
	 
	
	То есть для реализации Bitcoin подойдет криптографическая хэш-функция.

\end{frame}

\begin{frame}{Атака на любую хэш-функцию: парадокс Дней рождений}
\large 
	Положим $h_1, h_2, \ldots, h_n \in \{0,1\}^\ell$ независимо случайное выбранные строки. Парадокс Дней рождений
	\[
	\text{Для } n = \bigO\left (\sqrt{ \abs{ \{0,1\}^\ell }} \right) = \bigO\left( 2^{\ell/2} \right) \quad  \Pr[\exists (i != j) \; : \; h_i = h_j] > 1/2.
	\]
	
	Алгоритм перебора находит коллизию после {\color{Orange} $\bigO(2^{\ell/2})$ } вычисленных хэшей: \\
	\begin{enumerate}
		\itemsep 10pt
		\item Выбрать $ 2^{\ell/2}$ случайных строк $m_1, \ldots, m_{2^{\ell/2}}$ 
		\item  Для каждой $m_i$ вычислить $h_i = \Hash_s(m_i)$, отсортировать пары $(h_i, m_i)$ по значению\ $h_i$
		\item  Найти в упорядоченном списке $h_i = h_j$. Коллизия: $(m_i, m_j)$.
	\end{enumerate}			

	 \vspace{10pt}
	 
	 Алгоритм успешен с константной вероятностью по парадоксу ДР.
	 \vfill
	\LARGE
	{\color{Orange} Вывод:} Требуем  $\ell \geq 160$.
\end{frame}


\begin{frame}{Хэш-функции: исторический дайджест}
\large
	\begin{enumerate}
		\itemsep7pt
		\item 1980e: MD4 (Message Digest) предложено R. Rivest.  {\color{Orange} $\ell = 128$} \\
		Статус: {\color{Orange} Взломана.} Коллизию можно найти в течение секунд
		
		\item 1990: MD5. {\color{Orange} $\ell = 128$} \\
		Статус: {\color{Orange} Взломана.} Коллизию можно найти в течение секунд
		\pause
		\item 1995: SHA-1 (Secure Hash Algorithm 1) {\color{Orange} $\ell = 160$ } \\
		Статус: {\color{Orange} Взломана}. См.\ \url{https://shattered.io/} два PDF файла с одинаковым значением SHA-. \\
		{\color{Orange} !:} всё еще используется некоторыми системами (GIT).
		\pause
		\item 2001: SHA-2 (SHA-256, SHA-384, SHA-512). {\color{Orange} $\ell=256, 348, 512$} \\
		Статус: {\color{Orange} Считается безопасной.}
		\pause
		\item 2012: SHA-3 (Keccak). SHA-3 {\color{Orange} $\ell = 224/256/348/512$.} \\
		Статус: {\color{Orange} Считается безопасной.} 
	\end{enumerate}
\pause
В России:

	\begin{enumerate}
			\item GOST R 34.11-94 and GOST 34.311-95. {\color{Orange} $\ell = 256$} \\
			 Статус: {\color{Orange} Считается устаревшей.} Коллизия за $2^{105}$ операций,
			\item  GOST R 34.11-2012. Стрибог {\color{Orange} $\ell = 256, 512$} \\
			 Статус: {\color{Orange} Считается безопасной.}  
	\end{enumerate}
\end{frame}


\end{document}
