\documentclass[usenames,dvipsnames,8pt,aspectratio=169]{beamer}
\usepackage{amsmath,amsfonts,amssymb}
\usepackage{mathtools}
\usepackage{etex} %for Windows
\usepackage[utf8]{inputenc}
\usepackage[english, russian]{babel} 
%\usepackage{microtype}			% Better interword spacing and additional kerning.
\usepackage{ellipsis}			% Adjusted space with \dots between two words.
\usepackage{graphicx}
\usepackage{pstricks}

\usepackage{xcolor}


\usepackage{changepage}

\usepackage{algorithm}
\usepackage{algpseudocode}
%\usepackage[]{algorithm2e}
%\usepackage{algorithmic}

%\usepackage{tcolorbox}


\usepackage{caption}
\usepackage{subcaption}
%\usepackage{stackengine}


\usepackage{tikz}
\usetikzlibrary{tikzmark,calc}
\usetikzlibrary{positioning, backgrounds}
\usetikzlibrary{arrows, chains, matrix, scopes, patterns, shapes, fit}
\usetikzlibrary{mindmap,trees,shadows}
\usetikzlibrary{decorations.pathreplacing}
%\usetikzlibrary{crypto.symbols}

\usepackage{pgfplots}

\pgfmathdeclarefunction{gauss}{2}{%
	\pgfmathparse{1/(#2*sqrt(2*pi))*exp(-((x-#1)^2)/(2*#2^2))}%
}


\tikzset{
	invisible/.style={opacity=0},
	visible on/.style={alt={#1{}{invisible}}},
	alt/.code args={<#1>#2#3}{%
		\alt<#1>{\pgfkeysalso{#2}}{\pgfkeysalso{#3}} % \pgfkeysalso doesn't change the path
	},
}

\newcommand\strikeout[2][]{%
	\begin{tabular}[b]{@{}c@{}} 
		\makebox(0,0)[cb]{{#1}} \\[-0.2\normalbaselineskip]
		\rlap{\color{Orange}\rule[0.5ex]{\widthof{#2}}{1.5pt}}#2
\end{tabular}}

\newcommand\Fontvi{\fontsize{11}{13.2}\selectfont}

\usepackage{listings} % for C++ code

\usepackage{braket}
%\usepackage[braket, qm]{qcircuit}



\usepackage[T1]{fontenc}
%\usepackage[sfdefault,scaled=.85]{FiraSans}
%\usepackage{newtxsf}
%\usepackage[nomap]{FiraMono}





\usefonttheme[onlymath]{serif}
\renewcommand\sfdefault{cmbr}

\renewcommand{\bfdefault}{sb}

\definecolor{CharCoalDark}{RGB}{13, 16, 19}
\definecolor{Orange}{RGB}{255, 165,0}
\definecolor{DarkOrange}{RGB}{255, 165,0}
\definecolor{LightSalmon}{RGB}{255, 160, 122}
\definecolor{LeafGreen}{RGB}{34, 139,  34}
\definecolor{Coral}{RGB}{255, 127, 80}
\definecolor{DarkTurquoise}{RGB}{0, 206, 209}

%\newtheorem{defRus}{Определение}
%\newtheorem{thmRus}{Теорема}
%s\newtheorem{corRus}{Следствие}


\setbeamercolor{background canvas}{bg=CharCoalDark}

\setbeamerfont{title}{series=\bfseries}
\setbeamercolor{title}{fg=Orange}
\setbeamercolor{section in toc}{fg=white}
\setbeamercolor{frametitle}{fg=Orange}
\setbeamercolor{normal text}{fg=white}
%\setbeamercolor{normal text}{fontsize=12pt}
\setbeamercolor{itemize item}{fg=Orange}
\setbeamercolor{enumerate item}{fg=Orange}
\setbeamercolor{enumerate item item}{fg=Orange}
\setbeamercolor{itemize item item}{fg=Orange}
\setbeamercolor{enumerate item}{fg=Orange}
\setbeamercolor{block title}{bg=DarkOrange,fg=white}
\setbeamerfont{block title}{series=\bfseries}

\setbeamertemplate{itemize item}[circle]
\setbeamertemplate{eumerate subitem}{\color{Orange}[$\checkmark$]}
\setbeamertemplate{itemize subitem}{\color{Orange}\Large$\textbullet$}
\setbeamertemplate{itemize subitem}{\color{Orange} \tiny $\blacksquare$}

% footnote without a marker
\newcommand\blfootnote[1]{%
	\begingroup
	\renewcommand\footnoterule{}
	\renewcommand\thefootnote{}\footnote{#1}%
	\addtocounter{footnote}{-1}%
	\endgroup
}

\addtobeamertemplate{footline}{%
	\setlength\unitlength{1ex}%
	\begin{picture}(0,0) 
	% \put{} defines the position of the frame
	\put(155,0){\makebox(0,0)[bl]{
			%\includegraphics[scale=0.65]{white_square}
			%\includegraphics[scale=0.65]{dark_square}
			\includegraphics[scale=0.65]{grey_circle}
	}}%
	\end{picture}%
}{}


\newcommand*{\Scale}[2][4]{\scalebox{#1}{\ensuremath{#2}}}%

\newcommand\Item[1][]{%
	\ifx\relax#1\relax  \item \else \item[#1] \fi
	\abovedisplayskip=0pt\abovedisplayshortskip=0pt~\vspace*{-\baselineskip}}

\pgfdeclareradialshading{ring}{\pgfpoint{0cm}{0cm}}%
{rgb(0cm)=(1,1,1);
	rgb(0.7cm)=(1,1,1);
	rgb(0.719cm)=(1,1,1);
	rgb(0.72cm)=(0.975,0,0);
	rgb(0.9cm)=(1,1,1)}

\usepackage[absolute,overlay]{textpos} %to clip to a corner
\newcommand\FrameText[1]{%
	\begin{textblock*}{\paperwidth}(\textwidth-35pt, 10 pt)
		\raggedright #1\hspace{.5em}
\end{textblock*}}

\makeatletter
\let\save@measuring@true\measuring@true
\def\measuring@true{%
	\save@measuring@true
	\def\beamer@sortzero##1{\beamer@ifnextcharospec{\beamer@sortzeroread{##1}}{}}%
	\def\beamer@sortzeroread##1<##2>{}%
	\def\beamer@finalnospec{}%
}
\makeatother

\AtBeginSection[]
{
	\begin{frame}<beamer>
		\frametitle{Outline}
		\tableofcontents[currentsection]
	\end{frame}
}

\titlegraphic{
	
	%\includegraphics[width=2.5cm]{stayhome}%
	%\includegraphics[width=4.0cm]{ens_logo_gray}
}
\title{Лекция №5 \\[10pt]
	Часть 3. Где используются хэш-функции}

\date{ Елена Киршанова \\  \textbf{Курс ``Основы криптографии''} \\  }



\setbeamertemplate{navigation symbols}{} %removes navigation

% proper highlightling of a code-snippet
\lstset{language=C++,
	keywordstyle=\color{magenta},
	stringstyle=\color{Goldenrod},
	commentstyle=\color{gray},
	breaklines=false,
	%morecomment=[l][\color{magenta}]{\#}
}

%\setlength{\parskip}{8pt}
\input{header} %all defs
\begin{document}
	
\begin{frame}
	\titlepage
\end{frame}

\begin{frame}{В этой лекции}
\Large
	\begin{itemize}
		\itemsep 10pt
		\item Построение MACa
		\item Протокол идентификации
		\item Доказательство работы (proof of work)
	\end{itemize}
\end{frame}

\begin{frame}

\begin{LARGE}
	
	
	\color{Orange}
	1. Построение MACa
	
\end{LARGE}
\end{frame}

\begin{frame}{Построение MACa}
\Large
{\color{Orange} Задача:} из хэш-функции 
\[H: \{0,1\}^\star \rightarrow \{0,1\}^{\ell}\] построить функцию генерации МАСа 
\[S: \keyS \times \{0,1\}^\star \rightarrow \{0,1\}^{\ell}.\]

\vspace{20pt}

Основная сложность: хэш-функция -- бесключевой примитив. 

\end{frame}

\begin{frame}{Конструкция ``Two-key Nest''}
\Large
Для сообщения $M = (m_1, m_2)$ и $h: \{0,1\}^\ell \times \mesS \rightarrow  \{0,1\}^\ell $ -- функции компрессии
\[
	S((k_1, k_2), M) = \Hash_{\text{NMAC}}(k_2 \; ||  \; \Hash_{\text{NMAC}}(k_1 \; || \; M)
\]

\begin{figure}
	\hspace{-60pt}
	\includegraphics[scale=0.35]{TwoKeyNest}
\end{figure}
\vspace{-30pt}
{\color{Orange} Теорема:} Если $h(\cdot, )$ и $h(, \cdot)$ -- псевдослучайная функция, то \\ Two-key Nest -- безопасный MAC.

\end{frame}

\begin{frame}

\begin{LARGE}
	
	
	\color{Orange}
	2. Протокол идентификации
	
\end{LARGE}
\end{frame}

\begin{frame}{Протокол идентификации:  определение}
\Large
$\text{Id } = (\KeyGen, \textsf{P}, \textsf{V})$ -- интерактивный протокол, состоящий из 

\begin{itemize}
	\itemsep 7pt
	\item $\KeyGen(1^\lambda) \rightarrow (\vk, \sk)$ 
	\item $\textsf{P}(\sk)$ -- Доказывающий (Prover)
	\item $\textsf{V}(\vk) \rightarrow \{0,1\}$ -- Проверяющий (Verifier)
\end{itemize}

\vspace{10pt}

{\color{Orange} Корректность:} $\textsf{V}(\vk) \rightarrow 1$ при общении с $\textsf{\sk}$, где $\KeyGen(1^\lambda) = (\vk, \sk)$.

\vspace{10pt}

{\color{Orange} Безопасность:}
\begin{tabular}{c c c}
	{\color{Orange} Челленджер $\mathcal{C}$ } & & {\color{Orange} $\mathcal{A}$ } \\ 
	$\KeyGen(1^\lambda) \rightarrow (\vk, \sk)$ & $\xrightarrow{\vk}$ & \\[5pt]
	 & $\leftarrow$ & \\[-2pt]
	 & $\rightarrow$ & \\[2pt]
	
\end{tabular}
\vspace{10pt}

Выигрыш $\mathcal{A}$ --  $\mathcal{C}$ возвращает ``1'' (Accept). \\[5pt]

$\text{Id }$ -- безопасный Id-протокол относительно прямых атак,  \\ если $\forall $ ppt $\mathcal{A}$  вероятность выигрыша  $\negl(\lambda)$.

\end{frame}

\begin{frame}{Протокол парольной идентификации}
\Large
\[
	\Hash: \{0,1\}^\star \rightarrow \{0,1\}^\ell - \text{ хэш-функция}
\]

\vspace{10pt}
\begin{center}
\begin{tabular}{c c c}
	 \multicolumn{3}{c}{$\KeyGen \rightarrow (\sk = pwd, \vk = \Hash(pwd))$}  \\[20pt] 
	$\textsf{P}(pwd)$  & & $\textsf{V}(\vk)$  \\[5pt]
	& $\xrightarrow{\quad pwd \quad }$ & \\
	&  & $\Hash(pwd) == \vk \; ? \; 1 \; : \; 0 $
\end{tabular}
\end{center}

\vspace{10pt}

$\Hash$ -- криптографическая хэш-функция $\implies$ протокол  \\ $\text{Id} = (\KeyGen,\textsf{P}, \textsf{V} )$ безопасен относительно прямых атак.

\end{frame}


\begin{frame}{Протокол парольной идентификации}
\Large
\[
\Hash: \{0,1\}^\star \rightarrow \{0,1\}^\ell - \text{ хэш-функция}
\]

\vspace{10pt}
\begin{center}
	\begin{tabular}{c c c}
		\multicolumn{3}{c}{$\KeyGen \rightarrow (\sk = pwd, \vk = [\Hash(pwd, {\color{Orange}salt}), {\color{Orange}salt \xleftarrow{\$} S}])$}  \\[20pt] 
		$\textsf{P}(pwd)$  & & $\textsf{V}(\vk = (h, {\color{Orange}salt}))$  \\[5pt]
		& $\xrightarrow{\quad pwd \quad }$ & \\
		&  & $\Hash(pwd, {\color{Orange}salt}) == h \; ? \; 1 \; : \; 0 $
	\end{tabular}
\end{center}

\vspace{10pt}

$\Hash$ -- криптографическая хэш-функция $\implies$ протокол  \\ $\text{Id} = (\KeyGen,\textsf{P}, \textsf{V} )$ безопасен относительно прямых атак.

\end{frame}



\begin{frame}

\begin{LARGE}
	
	
	\color{Orange}
	3. Доказательство работы (proof of work)
	
\end{LARGE}
\end{frame}




\begin{frame}{Хэш-функция в  BitCoin}
\Large
Важный примитив в BitCoin: {\color{Orange} Proof of Work (PoW) / Доказательство работы} \\[10pt]
Интуиция: вычислительная мощность пользователя $\implies$ пользователь должен доказать это результатом вычислений \\
\vspace{15pt}
\begin{itemize}
	\itemsep 1em
	\item PoW предложен Dwork \& Naor (1992) как противодействие спаму
	\item {\color{Orange} Идея:}  заставить пользователя решить пазл ``средней сложности''  (решение должно быть легко верифицировать)
\end{itemize}
\end{frame}

\begin{frame}{Хэш-функция в  BitCoin: конструкция PoW}

\large
Основной примитив: криптографическая хэш-функция  $\Hash:\{0,1\}^\star \rightarrow \{0,1\}^\ell$, сложность вычисления которой $T(\Hash)$.

\vspace{10pt}
	%\begin{center}
	\begin{tabular}{l c c c l}
		& Алиса  & & Боб &  \\
		& Доказывающий  & & Проверяющий &  \\[3pt]
		& \multirow{5}{*}{\includegraphics[scale=0.15]{Alice}} & & 
		\multirow{5}{*}{\includegraphics[scale=0.15]{Bob}} &  1. $x \in \{0,1\}^{\star}$   \\
		&  & \Huge $\xleftarrow{\mkern5mu x \mkern5mu}$ & &  3. Проверить: \\ [2pt]
		2. Выбрать $s \in \{0,1\}^\star$   & & &  & $\Hash(s||x)$  имеет $n$ 0ей?  \\[-6pt]
		т.ч.\ $\Hash(s||x)$ & & \Huge $\xrightarrow{\mkern5mu s \mkern5mu}$  &  &  {\color{Orange}  Время: $T(\Hash) $}  \\
		начинается с $n$ 0'ей & & &  &  \\[4pt]
		{\color{Orange}  Время: $2^n T(\Hash) $} & & &  & 
	\end{tabular}
%\end{center}

\vspace{15pt}
Для криптографической хэш-функции $\Hash$ Алиса не может найти $s$ \\ быстрее, чем перебором. Это атака на прообраз. 
\end{frame}

\end{document}
