\documentclass[usenames,dvipsnames,8pt,aspectratio=169]{beamer}
\usepackage{amsmath,amsfonts,amssymb}
\usepackage{mathtools}
\usepackage{etex} %for Windows
\usepackage[utf8]{inputenc}
\usepackage[english, russian]{babel} 
%\usepackage{microtype}			% Better interword spacing and additional kerning.
\usepackage{ellipsis}			% Adjusted space with \dots between two words.
\usepackage{graphicx}
\usepackage{pstricks}

\usepackage{xcolor}


\usepackage{changepage}

\usepackage{algorithm}
\usepackage{algpseudocode}
%\usepackage[]{algorithm2e}
%\usepackage{algorithmic}

%\usepackage{tcolorbox}


\usepackage{caption}
\usepackage{subcaption}
%\usepackage{stackengine}


\usepackage{tikz}
\usetikzlibrary{tikzmark,calc}
\usetikzlibrary{positioning, backgrounds}
\usetikzlibrary{arrows, chains, matrix, scopes, patterns, shapes, fit}
\usetikzlibrary{mindmap,trees,shadows}
\usetikzlibrary{decorations.pathreplacing}
%\usetikzlibrary{crypto.symbols}

\usepackage{pgfplots}

\pgfmathdeclarefunction{gauss}{2}{%
	\pgfmathparse{1/(#2*sqrt(2*pi))*exp(-((x-#1)^2)/(2*#2^2))}%
}


\tikzset{
	invisible/.style={opacity=0},
	visible on/.style={alt={#1{}{invisible}}},
	alt/.code args={<#1>#2#3}{%
		\alt<#1>{\pgfkeysalso{#2}}{\pgfkeysalso{#3}} % \pgfkeysalso doesn't change the path
	},
}

\newcommand\strikeout[2][]{%
	\begin{tabular}[b]{@{}c@{}} 
		\makebox(0,0)[cb]{{#1}} \\[-0.2\normalbaselineskip]
		\rlap{\color{Orange}\rule[0.5ex]{\widthof{#2}}{1.5pt}}#2
\end{tabular}}

\newcommand\Fontvi{\fontsize{11}{13.2}\selectfont}

\usepackage{listings} % for C++ code

\usepackage{braket}
%\usepackage[braket, qm]{qcircuit}



\usepackage[T1]{fontenc}
%\usepackage[sfdefault,scaled=.85]{FiraSans}
%\usepackage{newtxsf}
%\usepackage[nomap]{FiraMono}





\usefonttheme[onlymath]{serif}
\renewcommand\sfdefault{cmbr}

\renewcommand{\bfdefault}{sb}

\definecolor{CharCoalDark}{RGB}{13, 16, 19}
\definecolor{Orange}{RGB}{255, 165,0}
\definecolor{DarkOrange}{RGB}{255, 165,0}
\definecolor{LightSalmon}{RGB}{255, 160, 122}
\definecolor{LeafGreen}{RGB}{34, 139,  34}
\definecolor{Coral}{RGB}{255, 127, 80}
\definecolor{DarkTurquoise}{RGB}{0, 206, 209}

%\newtheorem{defRus}{Определение}
%\newtheorem{thmRus}{Теорема}
%s\newtheorem{corRus}{Следствие}


\setbeamercolor{background canvas}{bg=CharCoalDark}

\setbeamerfont{title}{series=\bfseries}
\setbeamercolor{title}{fg=Orange}
\setbeamercolor{section in toc}{fg=white}
\setbeamercolor{frametitle}{fg=Orange}
\setbeamercolor{normal text}{fg=white}
%\setbeamercolor{normal text}{fontsize=12pt}
\setbeamercolor{itemize item}{fg=Orange}
\setbeamercolor{enumerate item}{fg=Orange}
\setbeamercolor{enumerate item item}{fg=Orange}
\setbeamercolor{itemize item item}{fg=Orange}
\setbeamercolor{enumerate item}{fg=Orange}
\setbeamercolor{block title}{bg=DarkOrange,fg=white}
\setbeamerfont{block title}{series=\bfseries}

\setbeamertemplate{itemize item}[circle]
\setbeamertemplate{eumerate subitem}{\color{Orange}[$\checkmark$]}
\setbeamertemplate{itemize subitem}{\color{Orange}\Large$\textbullet$}
\setbeamertemplate{itemize subitem}{\color{Orange} \tiny $\blacksquare$}

% footnote without a marker
\newcommand\blfootnote[1]{%
	\begingroup
	\renewcommand\footnoterule{}
	\renewcommand\thefootnote{}\footnote{#1}%
	\addtocounter{footnote}{-1}%
	\endgroup
}

\newcommand*{\Scale}[2][4]{\scalebox{#1}{\ensuremath{#2}}}%

\newcommand\Item[1][]{%
	\ifx\relax#1\relax  \item \else \item[#1] \fi
	\abovedisplayskip=0pt\abovedisplayshortskip=0pt~\vspace*{-\baselineskip}}

\pgfdeclareradialshading{ring}{\pgfpoint{0cm}{0cm}}%
{rgb(0cm)=(1,1,1);
	rgb(0.7cm)=(1,1,1);
	rgb(0.719cm)=(1,1,1);
	rgb(0.72cm)=(0.975,0,0);
	rgb(0.9cm)=(1,1,1)}

\usepackage[absolute,overlay]{textpos} %to clip to a corner
\newcommand\FrameText[1]{%
	\begin{textblock*}{\paperwidth}(\textwidth-35pt, 10 pt)
		\raggedright #1\hspace{.5em}
\end{textblock*}}

\addtobeamertemplate{footline}{%
	\setlength\unitlength{1ex}%
	\begin{picture}(0,0) 
	% \put{} defines the position of the frame
	\put(155,0){\makebox(0,0)[bl]{
			%\includegraphics[scale=0.65]{white_square}
			%\includegraphics[scale=0.65]{dark_square}
			\includegraphics[scale=0.65]{grey_circle}
	}}%
	\end{picture}%
}{}

\makeatletter
\let\save@measuring@true\measuring@true
\def\measuring@true{%
	\save@measuring@true
	\def\beamer@sortzero##1{\beamer@ifnextcharospec{\beamer@sortzeroread{##1}}{}}%
	\def\beamer@sortzeroread##1<##2>{}%
	\def\beamer@finalnospec{}%
}
\makeatother

\AtBeginSection[]
{
	\begin{frame}<beamer>
		\frametitle{Outline}
		\tableofcontents[currentsection]
	\end{frame}
}


\title{Лекция №5 \\[10pt]
	Часть 4. Децентрализованное приватное отслеживание контактов}

\date{ Елена Киршанова \\  \textbf{Курс ``Основы криптографии''} \\  }



\setbeamertemplate{navigation symbols}{} %removes navigation

% proper highlightling of a code-snippet
\lstset{language=C++,
	keywordstyle=\color{magenta},
	stringstyle=\color{Goldenrod},
	commentstyle=\color{gray},
	breaklines=false,
	%morecomment=[l][\color{magenta}]{\#}
}

%\setlength{\parskip}{8pt}
\input{header} %all defs
\begin{document}
	
\begin{frame}
	\titlepage
\end{frame}

\begin{frame}{Цель}
	\Large 
		{\color{Orange}  Как определить круг общения зараженных? }\\[15pt]


	В мае 2020 	академическое сообщество предлагает мобильное приложение для  {\color{Orange} отслеживания  контактов (proximity tracing).}\\[10pt]
	
	Основная цель: быстрое оповещение людей, контактирующих с заражённым \\[10pt]
	
	{\color{Orange} Требование к приложению:} сохранение {\color{Orange}приватности} перемещения пользователей.  \\
	
	
	
	\vfill
	Подробности на \url{https://github.com/DP-3T/documents}
	
\end{frame}

\begin{frame}{Идея протокола}
	\large
	\begin{columns}[T]
		\begin{column}{0.5\textwidth}
			\begin{enumerate}
				\item приложение генерирует псевдослучайные ID (используется PRG)
				\item  ID постоянно транслируется по BlueTooth
				\item приложение регистрирует все полученные ID на расстоянии действия BlueTooth
				\item Если у пользователя с идентификатором ID$^\star$ диагностируется вирус, пользователь с  ID$^\star$ загружает список своих ID на сервер
				\item  приложение получает данные от сервера  и локально проверяет, был ли он в контакте с заражённым ID$^\star$ 
			\end{enumerate}
		\end{column}
	\begin{column}{0.5\textwidth}
		\vspace{-4em}
		\begin{tabular}{c c c}
			&\includegraphics[scale=0.15]{remote_cloud} &  \\[-30pt]
			\includegraphics[scale=0.2]{Alice_phone} & &\includegraphics[scale=0.2]{Bob_phone} \\
		\end{tabular}
	\end{column}
	\end{columns}
	
\end{frame}
\begin{frame}{Требования к безопасности}
	\Large 
	
	\begin{itemize}
		\item сервер  хранит только ID зараженных пользователей (никакая другая информация не записывается)
		\item невозможно отследить неинфицированных пользователей, зная только их ID
		\item невозможно получить какую-либо информацию по ID
	\end{itemize}
\end{frame}



\begin{frame}{Детали протокола I}
\large
\begin{enumerate}
	\itemsep 1em
	\item  {\color{Orange}Генерация ключа:}
	
	В день $t$, приложение генерирую псевдослучайное значение $S_t $ с помощью PRG:
	
	\[S_t \leftarrow \KeyGen(t)\]

	 
	 В последующие дни, секретные ключи $S_{t+i}$ генерируются с помощью {\color{Orange} криптографической хэш-функции} $\Hash$:
	 \[
	 	S_{t+i} = \Hash(S_{t+i-1})
	 \]
	  \vspace{5pt}
	  
	 \item {\color{Orange} Генерация эфемерного ID:}  \\
	 Каждый пользователь меняет своё $\EID$ $n$ раз в день. \\
	 В начале дня $t$, генерируются $n$ эфемерных $\EID$:
	 
	 \[
	 	\EID_1 || \ldots || \EID_n = \text{PRG} (  \Hash (S_t, \texttt{``broadcast key''}) ),
	 \]
	 
	  \vspace{5pt}
	 Каждое $\EID_i$ занимает16 байт (Bluetooth Low Energy beacons payload).
	 
\end{enumerate}
\end{frame}

\begin{frame}{Детали протокола II}
\large
\begin{enumerate}
	\itemsep 1em
	\setcounter{enumi}{2}
	\item {\color{Orange} Сохранение локальных данных:}
	Увидев любое $\EID$, приложение сохраняет
	\begin{itemize}
		\large
		\item $\EID$ 
		\item длительность контакта
		\item дистанцию
		\item дату (Сентябрь, 13)
	\end{itemize}

Вся информация занимает несколько десятков байт.

	\item {\color{Orange} Зараженный в день $t$ пользователь:}	
	\begin{itemize}
	\large
	\item Отправляет $S_t$ на сервер
	\item Сервер хранит $(S_t, t)$ и рассылает эти данные пользователем .
	\item Зараженный пользователь генерирует новый секретный ключ.
	\end{itemize}

\item {\color{Orange} Нотификаия пользователей:} 	
	\begin{itemize}
		\large
		\item Сервер рассылает $(S_t,t)$ зараженных пользователей
		\item Пользователь получает $(S_t,t)$  и вычисляет все эфемерные $\EID'$
		\item Пользователь проверяет было ли хотя бы одно из вычисленных \\ $\EID'$ записано в его локальной базе контактов.
	\end{itemize}
\end{enumerate}
\end{frame}


\begin{frame}{Полезные ссылки}
\large

Детали протокола и реализация:\\
 \url{https://github.com/DP-3T/documents/blob/master/DP3T\%20White\%20Paper.pdf}
 
 \vspace{20pt}
 
Анализ схемы и модификации:\\
\url{https://eprint.iacr.org/2020/418}

 \vspace{20pt}

Альтернативная конструкция:\\
\url{https://github.com/ROBERT-proximity-tracing/documents}


\end{frame}


\end{document}
