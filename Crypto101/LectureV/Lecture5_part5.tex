\documentclass[usenames,dvipsnames,8pt,aspectratio=169]{beamer}
\usepackage{amsmath,amsfonts,amssymb}
\usepackage{mathtools}
\usepackage{etex} %for Windows
\usepackage[utf8]{inputenc}
\usepackage[english, russian]{babel} 
%\usepackage{microtype}			% Better interword spacing and additional kerning.
\usepackage{ellipsis}			% Adjusted space with \dots between two words.
\usepackage{graphicx}
\usepackage{pstricks}

\usepackage{xcolor}


\usepackage{changepage}

\usepackage{algorithm}
\usepackage{algpseudocode}
%\usepackage[]{algorithm2e}
%\usepackage{algorithmic}

%\usepackage{tcolorbox}


\usepackage{caption}
\usepackage{subcaption}
%\usepackage{stackengine}


\usepackage{tikz}
\usetikzlibrary{tikzmark,calc}
\usetikzlibrary{positioning, backgrounds}
\usetikzlibrary{arrows, chains, matrix, scopes, patterns, shapes, fit}
\usetikzlibrary{mindmap,trees,shadows}
\usetikzlibrary{decorations.pathreplacing}
%\usetikzlibrary{crypto.symbols}

\usepackage{pgfplots}

\pgfmathdeclarefunction{gauss}{2}{%
	\pgfmathparse{1/(#2*sqrt(2*pi))*exp(-((x-#1)^2)/(2*#2^2))}%
}


\tikzset{
	invisible/.style={opacity=0},
	visible on/.style={alt={#1{}{invisible}}},
	alt/.code args={<#1>#2#3}{%
		\alt<#1>{\pgfkeysalso{#2}}{\pgfkeysalso{#3}} % \pgfkeysalso doesn't change the path
	},
}

\newcommand\strikeout[2][]{%
	\begin{tabular}[b]{@{}c@{}} 
		\makebox(0,0)[cb]{{#1}} \\[-0.2\normalbaselineskip]
		\rlap{\color{Orange}\rule[0.5ex]{\widthof{#2}}{1.5pt}}#2
\end{tabular}}

\newcommand\Fontvi{\fontsize{11}{13.2}\selectfont}

\usepackage{listings} % for C++ code

\usepackage{braket}
%\usepackage[braket, qm]{qcircuit}



\usepackage[T1]{fontenc}
%\usepackage[sfdefault,scaled=.85]{FiraSans}
%\usepackage{newtxsf}
%\usepackage[nomap]{FiraMono}





\usefonttheme[onlymath]{serif}
\renewcommand\sfdefault{cmbr}

\renewcommand{\bfdefault}{sb}

\definecolor{CharCoalDark}{RGB}{13, 16, 19}
\definecolor{Orange}{RGB}{255, 165,0}
\definecolor{DarkOrange}{RGB}{255, 165,0}
\definecolor{LightSalmon}{RGB}{255, 160, 122}
\definecolor{LeafGreen}{RGB}{34, 139,  34}
\definecolor{Coral}{RGB}{255, 127, 80}
\definecolor{DarkTurquoise}{RGB}{0, 206, 209}

%\newtheorem{defRus}{Определение}
%\newtheorem{thmRus}{Теорема}
%s\newtheorem{corRus}{Следствие}


\setbeamercolor{background canvas}{bg=CharCoalDark}

\setbeamerfont{title}{series=\bfseries}
\setbeamercolor{title}{fg=Orange}
\setbeamercolor{section in toc}{fg=white}
\setbeamercolor{frametitle}{fg=Orange}
\setbeamercolor{normal text}{fg=white}
%\setbeamercolor{normal text}{fontsize=12pt}
\setbeamercolor{itemize item}{fg=Orange}
\setbeamercolor{enumerate item}{fg=Orange}
\setbeamercolor{enumerate item item}{fg=Orange}
\setbeamercolor{itemize item item}{fg=Orange}
\setbeamercolor{enumerate item}{fg=Orange}
\setbeamercolor{block title}{bg=DarkOrange,fg=white}
\setbeamerfont{block title}{series=\bfseries}

\setbeamertemplate{itemize item}[circle]
\setbeamertemplate{eumerate subitem}{\color{Orange}[$\checkmark$]}
\setbeamertemplate{itemize subitem}{\color{Orange}\Large$\textbullet$}
\setbeamertemplate{itemize subitem}{\color{Orange} \tiny $\blacksquare$}

% footnote without a marker
\newcommand\blfootnote[1]{%
	\begingroup
	\renewcommand\footnoterule{}
	\renewcommand\thefootnote{}\footnote{#1}%
	\addtocounter{footnote}{-1}%
	\endgroup
}

\newcommand*{\Scale}[2][4]{\scalebox{#1}{\ensuremath{#2}}}%

\newcommand\Item[1][]{%
	\ifx\relax#1\relax  \item \else \item[#1] \fi
	\abovedisplayskip=0pt\abovedisplayshortskip=0pt~\vspace*{-\baselineskip}}

\pgfdeclareradialshading{ring}{\pgfpoint{0cm}{0cm}}%
{rgb(0cm)=(1,1,1);
	rgb(0.7cm)=(1,1,1);
	rgb(0.719cm)=(1,1,1);
	rgb(0.72cm)=(0.975,0,0);
	rgb(0.9cm)=(1,1,1)}

\usepackage[absolute,overlay]{textpos} %to clip to a corner
\newcommand\FrameText[1]{%
	\begin{textblock*}{\paperwidth}(\textwidth-35pt, 10 pt)
		\raggedright #1\hspace{.5em}
\end{textblock*}}

\addtobeamertemplate{footline}{%
	\setlength\unitlength{1ex}%
	\begin{picture}(0,0) 
	% \put{} defines the position of the frame
	\put(155,0){\makebox(0,0)[bl]{
			%\includegraphics[scale=0.65]{white_square}
			%\includegraphics[scale=0.65]{dark_square}
			\includegraphics[scale=0.65]{grey_circle}
	}}%
	\end{picture}%
}{}

\makeatletter
\let\save@measuring@true\measuring@true
\def\measuring@true{%
	\save@measuring@true
	\def\beamer@sortzero##1{\beamer@ifnextcharospec{\beamer@sortzeroread{##1}}{}}%
	\def\beamer@sortzeroread##1<##2>{}%
	\def\beamer@finalnospec{}%
}
\makeatother

\AtBeginSection[]
{
	\begin{frame}<beamer>
		\frametitle{Outline}
		\tableofcontents[currentsection]
	\end{frame}
}


\title{Лекция №5 \\[10pt]
	Часть 5. Случайный оракул (Random Oracle)}

\date{ Елена Киршанова \\  \textbf{Курс ``Основы криптографии''} \\  }



\setbeamertemplate{navigation symbols}{} %removes navigation

% proper highlightling of a code-snippet
\lstset{language=C++,
	keywordstyle=\color{magenta},
	stringstyle=\color{Goldenrod},
	commentstyle=\color{gray},
	breaklines=false,
	%morecomment=[l][\color{magenta}]{\#}
}

%\setlength{\parskip}{8pt}
\input{header} %all defs
\begin{document}
	
\begin{frame}
	\titlepage
\end{frame}

\begin{frame}{Хэш-функции }

	\LARGE
	 
	 {\color{Orange} Задача: } конвертировать секретную строку $s$ (пароль) в ключ {\color{Orange}$k$}  \\[8pt] 
	 
	В криптографических приложениях (MAC, подпись) ${\color{Orange}k} = \Hash(s)$.\\[8pt] 
	
	Для этого результат хэширования {\color{Orange}$k$} должен ``выглядеть'' случайно. \\[25pt] 
	
	Формально, если $I(x)$-- сторонняя информация от $x$, то распределения
	\[
		(I(s), \Hash(s))  \quad \quad 	(I(s), k \xleftarrow{\$} \mathcal{T})
	\]
	
	должно быть сложно отличить эффективному атакующему.
	
\end{frame}

\begin{frame}{Эвристика случайного оракула, Bellare--Rogaway'93}
\vspace{-60pt}
	\Large
	{\color{Orange} Идея:} моделировать хэш-функцию $\Hash: \mesS \rightarrow \mathcal{T}$ как {\color{Orange} случайную}  функцию из ${\color{Orange}\mathcal{O}} \xleftarrow{\$} \mathcal[\mesS, \mathcal{T}]$. \\[8pt]
	
	В док-ве безопасности челленджер {\color{Orange}$\mathcal{C}$} использует вместо {\color{Orange} конкретной} $\Hash$ функцию ${\color{Orange}\mathcal{O}}\xleftarrow{\$} \mathcal[\mesS, \mathcal{T}]$. \\[8pt]
	\begin{center}
		Функция {\color{Orange}$\mathcal{O}$} -- случайный оракул. \\[5pt]
		
		Док-во безопасности с  {\color{Orange}$\mathcal{O}$}  -- док-во в {\color{Orange} модели случайного оракула.}
		
	\end{center}

\textbf{Реализация случайного оракула: метод ленивой выборки.}
	

\end{frame}

\begin{frame}{Док-ва безопасности в модели случайного оракула}
	\Large
	\centering
	Криптопритив $\Pi$ использует хэш-функцию $\Hash: \mesS \rightarrow \mathcal{T}$. \\[15pt]
	
	\begin{columns}[T]
		\begin{column}{0.4\textwidth}
			Доказательство в {\color{Orange}стандартной} модели \\[15pt]
				\begin{tabular}{c c c}
					{\color{Orange} Челленджер $\mathcal{C}$ } & & {\color{Orange} $\mathcal{A}$ } \\ 
					& $\leftarrow$ & \\[-2pt]
					& $\rightarrow$ & \\[2pt]
				\end{tabular}
			\begin{tikzpicture}[overlay]
			\draw[fill=none, draw=white, opacity=0.5] (-4.8,-1.5) rectangle (-2.0,1.0); 
			\draw[fill=none, draw=white, opacity=0.5] (-1.0,-1.5) rectangle (0.5,1.0); 
			\end{tikzpicture}
		\end{column}
	\begin{column}{0.6\textwidth}
		Доказательство в модели {\color{Orange}случайного оракула} \\[25pt]
			\begin{tabular}{c c c}
				{\color{Orange} Челленджер $\mathcal{C}$ } & & {\color{Orange} $\mathcal{A}$ } \\ 
				& $\leftarrow$ & \\[-2pt]
				& $\rightarrow$ & \\[50pt]
				& {\color{Orange} Оракул $\mathcal{O}$ } &
			\end{tabular}
			\begin{tikzpicture}[overlay]
		\draw[fill=none, draw=white, opacity=0.5] (-6.3,-0.5) rectangle (-3.0,2.0); 
		\draw[fill=none, draw=white, opacity=0.5] (-1.0,-0.5) rectangle (0.5,2.0); 
		\end{tikzpicture}
	\end{column}
	\end{columns}

\vspace{20pt}

В модели случайного оракула каждая сторона, {\color{Orange}  $\mathcal{C}$} и {\color{Orange}  $\mathcal{A}$}, \\ имеют доступ к {\color{Orange} Оракул $\mathcal{O}$}.
\end{frame}

\begin{frame}{Пример: PRF  из хэш-функции}
 \Large
 \vspace{-10pt}
% Докажем, что  функция
 	\begin{align*}
 		\mathcal{F}: \keyS \times \mesS & \rightarrow \mathcal{T} \\
 								(k, m) & \rightarrow \Hash(k || m)
 	\end{align*}
% является псевдослучайной в модели случайного оракула.
 
 %\vspace{10pt}
 
 {\color{Orange} Теорема.} Если множество $\keyS$ достаточно большое, функция $\mathcal{F}$ является {\color{Orange} псевдослучайной} в {\color{Orange} модели случайного оракула}. 
 
 \vspace{25pt}
 
 %\begin{center}	
 	\begin{tabular}{c c c}
 		{\color{Orange} Челленджер $\mathcal{C}$ } &{\color{Orange} $ \quad \mathcal{O}$ }  & {\color{Orange} Атакующий $\mathcal{A}$ }\\ [5pt]
 		$b \xleftarrow{\$} \{0,1\}  $& &\\ [2pt]
 		$\mkern-45mu b ==0: f \xleftarrow{\$} \mathcal{F}[\mathcal{K}, \mathcal{M}]$  & &\\ [2pt]
 		$b ==1: k\xleftarrow{\$} \keyS, f = {\color{Orange}   \mathcal{O} }(k, \cdot )$  & &\\ [2pt]
 		& $ \xleftarrow{\quad x_i \quad} $  & $x_i \in \mathcal{K}  $\\[5pt]
 		& $\xrightarrow{f (x_i)}$  &\\[5pt]
 		& $\xleftarrow{\quad \hat{b} \quad}$ & \\ [5pt]
 	\end{tabular}
 	\begin{tikzpicture}[overlay]
 	\draw[fill=none, draw=white, opacity=0.5] (-9.8,-2.3) rectangle (-4.8,3.0); 
 	\draw[fill=none, draw=white, opacity=0.5] (-2.9,-2.3) rectangle (-0.1,3.0); 
 	\end{tikzpicture}
 %\end{center}	
 \end{frame}

\begin{frame}{Статус доказательств в модели со случайным оракулом}
	\Large
	
	\begin{itemize}
		\itemsep 10pt
		\item На практике, случайный оракул реализован с помощью криптографической хэш-функции (SHA-3)
		\item Если конструкция, доказанная безопасной в модели случайного оракула, оказывается ``взломанной'' На практике,  значит, выбранная  хэш-функция недостаточно ``хороша''.
		\item Модель случайного оракула -- {\color{Orange} эвристическая } модель, она может быть неверна для  всех конструкций (см. Canetti-Goldreich-Halevi. The Random Oracle Methodology, Revisited.)
		\item Современные конструкции цифровых подписей доказаны в модели случайного оракула
	\end{itemize}
\end{frame}


\end{document}
