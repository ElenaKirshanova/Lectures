\documentclass[usenames,dvipsnames,8pt,aspectratio=169]{beamer}
\usepackage{amsmath,amsfonts,amssymb}
\usepackage{mathtools}
\usepackage{etex} %for Windows
\usepackage[utf8]{inputenc}
\usepackage[english, russian]{babel} 

%\usepackage{microtype}			% Better interword spacing and additional kerning.
\usepackage{ellipsis}			% Adjusted space with \dots between two words.
\usepackage{graphicx}
\usepackage{pstricks}

\usepackage{xcolor}


\usepackage{changepage}

\usepackage{algorithm}
\usepackage{algpseudocode}
%\usepackage[]{algorithm2e}
%\usepackage{algorithmic}

%\usepackage{tcolorbox}

\addtobeamertemplate{footline}{%
	\setlength\unitlength{1ex}%
	\begin{picture}(0,0) 
	% \put{} defines the position of the frame
	\put(155,0){\makebox(0,0)[bl]{
			%\includegraphics[scale=0.65]{white_square}
			%\includegraphics[scale=0.65]{dark_square}
			\includegraphics[scale=0.65]{grey_circle}
	}}%
	\end{picture}%
}{}

\usepackage{tikz}
\usetikzlibrary{tikzmark,calc}
\usetikzlibrary{positioning, backgrounds}
\usetikzlibrary{arrows, chains, matrix, scopes, patterns, shapes, fit}
\usetikzlibrary{mindmap,trees,shadows}
\usetikzlibrary{decorations.pathreplacing}

\usepackage{pgfplots}

\pgfmathdeclarefunction{gauss}{2}{%
	\pgfmathparse{1/(#2*sqrt(2*pi))*exp(-((x-#1)^2)/(2*#2^2))}%
}


\tikzset{
	invisible/.style={opacity=0},
	visible on/.style={alt={#1{}{invisible}}},
	alt/.code args={<#1>#2#3}{%
		\alt<#1>{\pgfkeysalso{#2}}{\pgfkeysalso{#3}} % \pgfkeysalso doesn't change the path
	},
}

\newcommand\strikeout[2][]{%
	\begin{tabular}[b]{@{}c@{}} 
		\makebox(0,0)[cb]{{#1}} \\[-0.2\normalbaselineskip]
		\rlap{\color{Orange}\rule[0.5ex]{\widthof{#2}}{1.5pt}}#2
\end{tabular}}

\newcommand\Fontvi{\fontsize{11}{13.2}\selectfont}

\usepackage{listings} % for C++ code

\usepackage{braket}
%\usepackage[braket, qm]{qcircuit}



\usepackage[T1]{fontenc}
%\usepackage[sfdefault,scaled=.85]{FiraSans}
%\usepackage{newtxsf}
%\usepackage[nomap]{FiraMono}





\usefonttheme[onlymath]{serif}
\renewcommand\sfdefault{cmbr}

\renewcommand{\bfdefault}{sb}

\definecolor{CharCoalDark}{RGB}{13, 16, 19}
\definecolor{Orange}{RGB}{255, 165,0}
\definecolor{DarkOrange}{RGB}{255, 165,0}
\definecolor{LightSalmon}{RGB}{255, 160, 122}
\definecolor{LeafGreen}{RGB}{34, 139,  34}
\definecolor{Coral}{RGB}{255, 127, 80}
\definecolor{DarkTurquoise}{RGB}{0, 206, 209}

\definecolor{darkslateblue}{RGB}{72,61,139}

%\newtheorem{defRus}{Определение}
%\newtheorem{thmRus}{Теорема}
%s\newtheorem{corRus}{Следствие}

\def\darktheme{}
\ifdefined\darktheme
	\setbeamercolor{background canvas}{bg=CharCoalDark}
	\setbeamerfont{title}{series=\bfseries}
	\setbeamercolor{title}{fg=Orange}
	\setbeamercolor{section in toc}{fg=white}
	\setbeamercolor{frametitle}{fg=Orange}
	\setbeamercolor{normal text}{fg=white}
	%\setbeamercolor{normal text}{fontsize=12pt}
	\setbeamercolor{itemize item}{fg=Orange}
	\setbeamercolor{itemize item item}{fg=Orange}
	\setbeamercolor{enumerate item}{fg=Orange}
	\setbeamercolor{block title}{bg=DarkOrange,fg=white}
	\setbeamerfont{block title}{series=\bfseries}
	
	\setbeamertemplate{itemize item}[circle]
	%\setbeamertemplate{itemize subitem}[$\checkmark$]
	\setbeamertemplate{itemize subitem}{\color{Orange}\Large$\textbullet$}
	\setbeamertemplate{itemize subitem}{\color{Orange} \tiny $\blacksquare$}
\else
	\setbeamercolor{background canvas}{bg=white}
	\setbeamerfont{title}{series=\bfseries}
	\setbeamercolor{title}{fg=darkslateblue}
	\setbeamercolor{section in toc}{fg=black}
	\setbeamercolor{frametitle}{fg=darkslateblue}
	\setbeamercolor{normal text}{fg=black}
	%\setbeamercolor{normal text}{fontsize=9pt}
	\setbeamercolor{itemize item}{fg=darkslateblue}
	\setbeamercolor{itemize item item}{fg=darkslateblue}
	\setbeamercolor{enumerate item}{fg=darkslateblue}
	\setbeamercolor{block title}{bg=darkslateblue,fg=white}
	\setbeamerfont{block title}{series=\bfseries}
	
	\setbeamertemplate{itemize item}[circle]
	%\setbeamertemplate{itemize subitem}[$\checkmark$]
	\setbeamertemplate{itemize subitem}{\color{blue}\Large$\textbullet$}
	\setbeamertemplate{itemize subitem}{\color{blue} \tiny $\blacksquare$}

\fi

% footnote without a marker
\newcommand\blfootnote[1]{%
	\begingroup
	\renewcommand\footnoterule{}
	\renewcommand\thefootnote{}\footnote{#1}%
	\addtocounter{footnote}{-1}%
	\endgroup
}

\newcommand*{\Scale}[2][4]{\scalebox{#1}{\ensuremath{#2}}}%

\newcommand\Item[1][]{%
	\ifx\relax#1\relax  \item \else \item[#1] \fi
	\abovedisplayskip=0pt\abovedisplayshortskip=0pt~\vspace*{-\baselineskip}}

\pgfdeclareradialshading{ring}{\pgfpoint{0cm}{0cm}}%
{rgb(0cm)=(1,1,1);
	rgb(0.7cm)=(1,1,1);
	rgb(0.719cm)=(1,1,1);
	rgb(0.72cm)=(0.975,0,0);
	rgb(0.9cm)=(1,1,1)}

\usepackage[absolute,overlay]{textpos} %to clip to a corner
\newcommand\FrameText[1]{%
	\begin{textblock*}{\paperwidth}(\textwidth-35pt, 10 pt)
		\raggedright #1\hspace{.5em}
\end{textblock*}}

\newcommand\myeq{\stackrel{\mathclap{\normalfont\mbox{?}}}{=}}


\makeatletter
\let\save@measuring@true\measuring@true
\def\measuring@true{%
	\save@measuring@true
	\def\beamer@sortzero##1{\beamer@ifnextcharospec{\beamer@sortzeroread{##1}}{}}%
	\def\beamer@sortzeroread##1<##2>{}%
	\def\beamer@finalnospec{}%
}
\makeatother

\AtBeginSection[]
{
	\begin{frame}<beamer>
		\frametitle{Outline}
		\tableofcontents[currentsection]
	\end{frame}
}


%\institute{ENS Lyon}
\author{\\ [10pt]
}
\titlegraphic{
	
	%\includegraphics[width=2.5cm]{erc_logo_gray}\hspace*{2.5cm}~%
	%\includegraphics[width=4.0cm]{ens_logo_gray}
}
\title{Лекция №2 \\[10pt]
		Часть 3. PRG c произвольной областью значения}

\date{ Елена Киршанова \\  \textbf{Курс ``Основы криптографии''} \\  }


\setbeamertemplate{navigation symbols}{} %removes navigation

% proper highlightling of a code-snippet
\lstset{language=C++,
	keywordstyle=\color{magenta},
	stringstyle=\color{Goldenrod},
	commentstyle=\color{gray},
	breaklines=false,
	%morecomment=[l][\color{magenta}]{\#}
}

%\setlength{\parskip}{8pt}
\input{header} %all defs
\begin{document}
	
\begin{frame}
	\titlepage
\end{frame}

\begin{frame}{Потоковый шифр = OTP + PRG}
\LARGE

		 \[\mesS, \cipS = \{0,1\}^n, \; \keyS = \{0,1\}^{\ell}\]
		 \[G : \{0,1\}^{\ell}  \rightarrow \{0,1\}^{n} - \text{ PRG } \] \\
		\begin{itemize}
			\item $\KeyGen(1^{\ell}): $ \\[-20pt]
			\begin{flalign*} 
				s  &\xleftarrow{\$} \{0,1\}^\ell  \\
				k &= G(s)  & 
			\end{flalign*}
			\item $\Enc(k, m \in \{0,1\}^n): c = k \oplus m$ \\[10pt]
			\item $\Dec(k, c \in \{0,1\}^n): m = k \oplus c$ \\[10pt]
		\end{itemize}
	
{\color{Orange} Проблема:} фиксированный размер открытых текстов.
\end{frame}


\begin{frame}{Расширение области значений $G$}
\Large
\begin{align*}
	 \text{\color{Orange} Дано } \quad & G : \{0,1\}^{\ell}  \rightarrow \{0,1\}^{n} - \text{ PRG }  \\
	\text{\color{Orange}  Построить } &G': \{0,1\} ^{L}  \rightarrow \{0,1\}^{N}, \; N > n
\end{align*} 

\vspace{15pt}

Два метода композиции $G$:

\begin{enumerate}
	\item Параллельная конструкция
	\[
		G'(s_1, \ldots, s_n) = (G(s_1), \ldots, G(s_n))
	\]
	
	\item Последовательная конструкция (Blum-Micali)
	\begin{flalign*}
		G'(s) & =  \\
					& s_0 = s \\
					&\text{for } i = 1\text{ to  } k \\
					& \quad (r_i, s_i) = G(s_{i-1}) \\
					& \text{return } (r_1, \ldots, r_k, s_k)  &
	\end{flalign*}
	
\end{enumerate}
 
\end{frame}

\begin{frame}{Безопасность последовательной конструкции $G'$}
\LARGE
\vspace{-20pt}
\begin{align*}
 G : \; &\{0,1\}^{\ell}  \rightarrow \{0,1\}^{n}  \\
G': \; & \{0,1\} ^{\ell}  \rightarrow \{0,1\}^{4n} \quad (k=3) \\
\end{align*} 


{\color{Orange} Теорема.} $G$ -- безопасный PRG $\implies$ $G'$ -- безопасный PRG.\\[5pt]
\emph{Для любого ppt $\adv$, атакующего $G'$, найдется ppt алгоритм $\mathcal{B}$, атакующий $G$, такие что}
\[
	\mathsf{PRGadv} \left[  \mathcal{A}, G' \right ] = 3 \cdot \mathsf{PRGadv} \left[  \mathcal{B}, G \right ]
\]

\vspace{20pt}
Докажем с помощью {\color{Orange} гибридного } метода.

\end{frame}

\begin{frame}{Гибридный метод доказательства}
\Large
\[
\mathsf{PRGadv} \left[  \mathcal{A}, G' \right ] = 3 \cdot \mathsf{PRGadv} \left[  \mathcal{B}, G \right ]
\]

\vspace{10pt}

\begin{tikzpicture}

	\draw[-stealth] (-1,1) -- (0, 1) node[midway, above] {\Large $s$}; 
	\draw[fill=none, draw=white, opacity=0.5] (0,0) rectangle (2,2) node[pos=0.5]{$G$}; 
	\draw[-stealth] (2,1) -- (4, 1) node[midway, above] {\Large $s_1$}; 
	\draw[-stealth] (2,0.5) -- (2.5, 0.5) -- (2.5, -0.5) node[below] {\Large $r_1$}; 
	

	\draw[fill=none, draw=white, opacity=0.5] (4,0) rectangle (6,2) node[pos=0.5]{$G$}; 
	\draw[-stealth] (6,1) -- (8, 1) node[midway, above] {\Large $s_2$}; 
	\draw[-stealth] (6,0.5) -- (6.5, 0.5) -- (6.5, -0.5) node[below] {\Large $r_2$}; 
	
	\draw[fill=none, draw=white, opacity=0.5] (8,0) rectangle (10,2) node[pos=0.5]{$G$}; 
	\draw[-stealth] (10,1) -- (11, 1) node[midway, above] {\Large $s_3$}; 
	\draw[-stealth] (10,0.5) -- (10.5, 0.5) -- (10.5, -0.5) node[below] {\Large $r_3$}; 
\end{tikzpicture}

\vspace{10pt}

\begin{tabular}{c c c c  c c}
	{\color{Orange} Гибрид 0}  & & {\color{Orange} Гибрид 1} & & {\color{Orange} Гибрид 2} \\[10pt]
	Эксп. 0 $\xrightarrow{(r_1, r_2, r_3, s_3)}$ $\adv$  &  & 0.  $\xrightarrow{({\color{Orange}r_1}, r_2, r_3, s_3)}$ $\adv_1$  & & 0.  $\xrightarrow{({\color{Orange}r_1, r_2}, r_3, s_3)}$ $\mathcal{B}$ &  \\[10pt]
	Эксп. 1$\xrightarrow{ \hspace{20pt}  {\color{Orange} r } \hspace{20pt} }$ $\adv$ &  & 1. $\xrightarrow{ \hspace{20pt}  {\color{Orange} r } \hspace{20pt} }$ $\adv_1$ & & 1. $\xrightarrow{ \hspace{20pt}  {\color{Orange} r } \hspace{20pt} }$ $\mathcal{B}$ &  \\
\end{tabular}

\end{frame}

%
%
%
%\begin{frame}{Constrictions of a PRG: Salsa and ChaCha}
%\Large
%	\begin{itemize}
%		\item Salsa20,ChaCha20: proposed by D.Bernstein in 2005, 2008
%		\item used in many TLS cipher suits
%		\item Input: $256$-bit seed and a parameter $L$
%		\item Output: $(256 \cdot L)$-bit pseudorandom string
%	\end{itemize}
%	\vspace{20pt}
%	\pause
%	Two components
%	\begin{enumerate}
%		\item $\mathsf{pad}(s, j, 0)$: takes a seed $s$, a $64$-bit counter $j$ and a $64$-bit nonce\\
%		Output: $512$-bit block
%		\item a fixed public permutation $\pi: \{0,1\}^{512} \rightarrow \{0,1\}^{512}$
%	\end{enumerate}
%	\vspace{20pt}
%	See \url{https://cr.yp.to/chacha.html} for details
%\end{frame}
%
%\begin{frame}{ChaCha PRG}
%\begin{figure}
%	\includegraphics[width=\textwidth]{ChaCha20}
%\end{figure}
%
%Nonce -- the third parameter of $\mathsf{pad}(s, j, 0)$ is used to convert a PRG into a PRF (useful for encryption of multiple messages).
%\vfill
%\small
%{\color{gray}\textbf{picture is taken from D.Boneh, V.Shoup A Graduate Course in Applied Cryptography}} 
%\end{frame}
%
%\begin{frame}{(Somewhat) Broken PRGs}
%\LARGE
%\begin{enumerate}
%	\itemsep2em 
%	\item {\color{Orange}\textbf{linear congruential generators}} 
%	\begin{itemize}
%		\LARGE
%				\itemsep5pt  
%		\item had been used in glibc, Microsoft Visual Basic, Java
%		\item notorious for RANDU
%		\item \textbf{not cryptographically secure PRG!}
%	\end{itemize}
%
%	\item {\color{Orange}\textbf{RC4}} 
%	\begin{itemize}
%		\LARGE
%		\itemsep5pt 
%		\item proposed by R.Rivest  in 1987
%		\item used to be a part of TLS, 802.11b WEP
%		\item \textbf{not cryptographically secure PRG!}
%	\end{itemize}
%
%	\item {\color{Orange}\textbf{Linear feedback shift registers}}
%	\begin{itemize}
%			\LARGE
%			\itemsep5pt
%		\item used for protecting movies on DVD disks
%		\item weakly security  PRG (Trivium)
%	\end{itemize} 
%\end{enumerate}
%
%\end{frame}
%
%\begin{frame}{A Random Number Generator}
%	\Large
%	\begin{itemize}
%		\itemsep7pt
%		\item In practice, random bits are generated using a random number generator,  RNG
%		\item An RNG outputs a sequence of pseudorandom bits
%		\item Unlike PRG, an RNG take additional input (entropy source)
%		\item Example in Linux: $\mathsf{/dev/random}$
%		\item Entropy is usually taken from hardware (keyboard/mouse events, hardware interrupts, jitters).
%	\end{itemize}
%\end{frame}
%
%\begin{frame}{Application: Coin flipping}
%\LARGE
%Task:  throw a fair coin over without interaction 
%\begin{center}
%	\begin{tabular}{c c c c c}
%		 \multicolumn{5}{c}{$G: \{0,1\}^{\ell} \rightarrow \{0,1\}^{L}$}\\[10pt]
%		& Bob  & & Alice &  \\
%		 & \multirow{5}{*}{\includegraphics[scale=0.20]{Bob}} & &
%		\multirow{5}{*}{\includegraphics[scale=0.20]{Alice}} &  \\
%		&  &  & & $r \leftarrow \{0,1\}^{L}$  \\
%		&  & $\xleftarrow{r}$ & &  \\
%		Flips a coin &   & &  &  \\
%		$b \in \{0,1\}$&  & & &  \\
%		$s \in \{0,1\}^{\ell}$&  & & &  \\[15pt]
%		\multicolumn{5}{l}{$\mathsf{commit}(b, r, s)  = 
%			\begin{cases}
%			G(s), & b = 0\\
%			G(s) \oplus r, & b=1
%			\end{cases}
%			$}  \\
%		&  & $\xrightarrow{\mathsf{commit}(b, r, s)} $ & &  \\
%	\end{tabular}
%\end{center}
%
%\end{frame}

\end{document}