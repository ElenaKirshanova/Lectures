\documentclass[usenames,dvipsnames,8pt,aspectratio=169]{beamer}
\usepackage{amsmath,amsfonts,amssymb}
\usepackage{mathtools}
\usepackage{etex} %for Windows
\usepackage[utf8]{inputenc}
\usepackage[english, russian]{babel} 

%\usepackage{microtype}			% Better interword spacing and additional kerning.
\usepackage{ellipsis}			% Adjusted space with \dots between two words.
\usepackage{graphicx}
\usepackage{pstricks}

\usepackage{xcolor}


\usepackage{changepage}

\usepackage{algorithm}
\usepackage{algpseudocode}
%\usepackage[]{algorithm2e}
%\usepackage{algorithmic}

%\usepackage{tcolorbox}

\addtobeamertemplate{footline}{%
	\setlength\unitlength{1ex}%
	\begin{picture}(0,0) 
	% \put{} defines the position of the frame
	\put(155,0){\makebox(0,0)[bl]{
			%\includegraphics[scale=0.65]{white_square}
			%\includegraphics[scale=0.65]{dark_square}
			\includegraphics[scale=0.65]{grey_circle}
	}}%
	\end{picture}%
}{}

\usepackage{tikz}
\usetikzlibrary{tikzmark,calc}
\usetikzlibrary{positioning, backgrounds}
\usetikzlibrary{arrows, chains, matrix, scopes, patterns, shapes, fit}
\usetikzlibrary{mindmap,trees,shadows}
\usetikzlibrary{decorations.pathreplacing}

\usepackage{pgfplots}

\pgfmathdeclarefunction{gauss}{2}{%
	\pgfmathparse{1/(#2*sqrt(2*pi))*exp(-((x-#1)^2)/(2*#2^2))}%
}


\tikzset{
	invisible/.style={opacity=0},
	visible on/.style={alt={#1{}{invisible}}},
	alt/.code args={<#1>#2#3}{%
		\alt<#1>{\pgfkeysalso{#2}}{\pgfkeysalso{#3}} % \pgfkeysalso doesn't change the path
	},
}

\newcommand\strikeout[2][]{%
	\begin{tabular}[b]{@{}c@{}} 
		\makebox(0,0)[cb]{{#1}} \\[-0.2\normalbaselineskip]
		\rlap{\color{Orange}\rule[0.5ex]{\widthof{#2}}{1.5pt}}#2
\end{tabular}}

\newcommand\Fontvi{\fontsize{11}{13.2}\selectfont}

\usepackage{listings} % for C++ code

\usepackage{braket}
%\usepackage[braket, qm]{qcircuit}



\usepackage[T1]{fontenc}
%\usepackage[sfdefault,scaled=.85]{FiraSans}
%\usepackage{newtxsf}
%\usepackage[nomap]{FiraMono}





\usefonttheme[onlymath]{serif}
\renewcommand\sfdefault{cmbr}

\renewcommand{\bfdefault}{sb}

\definecolor{CharCoalDark}{RGB}{13, 16, 19}
\definecolor{Orange}{RGB}{255, 165,0}
\definecolor{DarkOrange}{RGB}{255, 165,0}
\definecolor{LightSalmon}{RGB}{255, 160, 122}
\definecolor{LeafGreen}{RGB}{34, 139,  34}
\definecolor{Coral}{RGB}{255, 127, 80}
\definecolor{DarkTurquoise}{RGB}{0, 206, 209}

\definecolor{darkslateblue}{RGB}{72,61,139}

%\newtheorem{defRus}{Определение}
%\newtheorem{thmRus}{Теорема}
%s\newtheorem{corRus}{Следствие}

\def\darktheme{}
\ifdefined\darktheme
	\setbeamercolor{background canvas}{bg=CharCoalDark}
	\setbeamerfont{title}{series=\bfseries}
	\setbeamercolor{title}{fg=Orange}
	\setbeamercolor{section in toc}{fg=white}
	\setbeamercolor{frametitle}{fg=Orange}
	\setbeamercolor{normal text}{fg=white}
	%\setbeamercolor{normal text}{fontsize=12pt}
	\setbeamercolor{itemize item}{fg=Orange}
	\setbeamercolor{itemize item item}{fg=Orange}
	\setbeamercolor{enumerate item}{fg=Orange}
	\setbeamercolor{block title}{bg=DarkOrange,fg=white}
	\setbeamerfont{block title}{series=\bfseries}
	
	\setbeamertemplate{itemize item}[circle]
	%\setbeamertemplate{itemize subitem}[$\checkmark$]
	\setbeamertemplate{itemize subitem}{\color{Orange}\Large$\textbullet$}
	\setbeamertemplate{itemize subitem}{\color{Orange} \tiny $\blacksquare$}
\else
	\setbeamercolor{background canvas}{bg=white}
	\setbeamerfont{title}{series=\bfseries}
	\setbeamercolor{title}{fg=darkslateblue}
	\setbeamercolor{section in toc}{fg=black}
	\setbeamercolor{frametitle}{fg=darkslateblue}
	\setbeamercolor{normal text}{fg=black}
	%\setbeamercolor{normal text}{fontsize=9pt}
	\setbeamercolor{itemize item}{fg=darkslateblue}
	\setbeamercolor{itemize item item}{fg=darkslateblue}
	\setbeamercolor{enumerate item}{fg=darkslateblue}
	\setbeamercolor{block title}{bg=darkslateblue,fg=white}
	\setbeamerfont{block title}{series=\bfseries}
	
	\setbeamertemplate{itemize item}[circle]
	%\setbeamertemplate{itemize subitem}[$\checkmark$]
	\setbeamertemplate{itemize subitem}{\color{blue}\Large$\textbullet$}
	\setbeamertemplate{itemize subitem}{\color{blue} \tiny $\blacksquare$}

\fi

% footnote without a marker
\newcommand\blfootnote[1]{%
	\begingroup
	\renewcommand\footnoterule{}
	\renewcommand\thefootnote{}\footnote{#1}%
	\addtocounter{footnote}{-1}%
	\endgroup
}

\newcommand*{\Scale}[2][4]{\scalebox{#1}{\ensuremath{#2}}}%

\newcommand\Item[1][]{%
	\ifx\relax#1\relax  \item \else \item[#1] \fi
	\abovedisplayskip=0pt\abovedisplayshortskip=0pt~\vspace*{-\baselineskip}}

\pgfdeclareradialshading{ring}{\pgfpoint{0cm}{0cm}}%
{rgb(0cm)=(1,1,1);
	rgb(0.7cm)=(1,1,1);
	rgb(0.719cm)=(1,1,1);
	rgb(0.72cm)=(0.975,0,0);
	rgb(0.9cm)=(1,1,1)}

\usepackage[absolute,overlay]{textpos} %to clip to a corner
\newcommand\FrameText[1]{%
	\begin{textblock*}{\paperwidth}(\textwidth-35pt, 10 pt)
		\raggedright #1\hspace{.5em}
\end{textblock*}}

\newcommand\myeq{\stackrel{\mathclap{\normalfont\mbox{?}}}{=}}


\makeatletter
\let\save@measuring@true\measuring@true
\def\measuring@true{%
	\save@measuring@true
	\def\beamer@sortzero##1{\beamer@ifnextcharospec{\beamer@sortzeroread{##1}}{}}%
	\def\beamer@sortzeroread##1<##2>{}%
	\def\beamer@finalnospec{}%
}
\makeatother

\AtBeginSection[]
{
	\begin{frame}<beamer>
		\frametitle{Outline}
		\tableofcontents[currentsection]
	\end{frame}
}


%\institute{ENS Lyon}
\author{\\ [10pt]
}
\titlegraphic{
	
	%\includegraphics[width=2.5cm]{erc_logo_gray}\hspace*{2.5cm}~%
	%\includegraphics[width=4.0cm]{ens_logo_gray}
}
\title{Лекция №2 \\[10pt]
		Часть 4. Современные псевдослучайные генераторы}

\date{ Елена Киршанова \\  \textbf{Курс ``Основы криптографии''} \\  }


\setbeamertemplate{navigation symbols}{} %removes navigation

% proper highlightling of a code-snippet
\lstset{language=C++,
	keywordstyle=\color{magenta},
	stringstyle=\color{Goldenrod},
	commentstyle=\color{gray},
	breaklines=false,
	%morecomment=[l][\color{magenta}]{\#}
}

%\setlength{\parskip}{8pt}
\input{header} %all defs
\begin{document}
	
\begin{frame}
	\titlepage
\end{frame}



\begin{frame}{Constrictions of a PRG: Salsa and ChaCha}
\Large
	\begin{itemize}
		\itemsep 10pt
		\item Salsa20,ChaCha20: предложены Д.Бернштайном в 2005, 2008
		\item один из предложенных к использованию PRG в портфолио eStream
		\item используется в интернет протоколах (TLS)
		\item Вход: $256$-битное нач. значение и параметр $L$
		\item Выход: $(256 \cdot L)$-битная псевдослучайная строка
		\item Детали алгоритма \url{https://cr.yp.to/chacha.html} 
	\end{itemize}

	
\end{frame}

\begin{frame}{ChaCha PRG (упрощенная версия)}
\LARGE

Два компонента:
\begin{enumerate}
	\itemsep 8pt
	\item функция $\texttt{pad}(s, j) : \{0,1\}^{256 + 64} \rightarrow \{0,1\}^{512}$
	\item фиксированная перестановка $\pi  : \{0,1\}^{512} \rightarrow \{0,1\}^{512} $
\end{enumerate}

\vspace{30pt}

Алгоритм:
\begin{enumerate}
	\item for  $j  =  0 \text{ to } L-1 $ 
	 \item $\quad h_j = \texttt{pad}(s, j) $
	 \item $\quad r_j = \pi(h_j) \oplus h_j$
	\item Выход $(r_0, \ldots r_{L-1})$
\end{enumerate}
%Nonce -- the third parameter of $\mathsf{pad}(s, j, 0)$ is used to convert a PRG into a PRF (useful for encryption of multiple messages).
%\vfill
%\small
%{\color{gray}\textbf{picture is taken from D.Boneh, V.Shoup A Graduate Course in Applied Cryptography}} 
\end{frame}

\begin{frame}{(Частично) Взломанные PRG}
\LARGE
\begin{enumerate}
	\itemsep1.5em 
	\item {\color{Orange}\textbf{Линейный конгруэнтный метод}} 
	\begin{itemize}
		\LARGE
				\itemsep5pt  
		\item использовался в glibc, Microsoft Visual Basic, Java
		\item notorious for RANDU
		\item \textbf{не является криптографическим PRG!}
	\end{itemize}

	\item {\color{Orange}\textbf{RC4}} 
	\begin{itemize}
		\LARGE
		\itemsep5pt 
		\item предложен Р.Ривестом в 1987
		\item использовался TLS, 802.11b WEP
		\item \textbf{не является криптографическим PRG!}
	\end{itemize}

	\item {\color{Orange}\textbf{Регистр сдвига с линейной обратной связью }}
	\begin{itemize}
			\LARGE
			\itemsep5pt
		\item использовался для защиты данных DVD дисков
		\item пример: Trivium (eStream)
	\end{itemize} 
\end{enumerate}

\end{frame}

\begin{frame}{Приложение: Подбрасывание монетки по телефону}
\LARGE
Задача:  выяснить результат подбрасывания монетки
\begin{center}
	\begin{tabular}{c c c c c}
		 \multicolumn{5}{c}{$G: \{0,1\}^{\ell} \rightarrow \{0,1\}^{L}$ -- PRG}\\[10pt]
		& Боб  & & Алиса &  \\
		 & \multirow{5}{*}{\includegraphics[scale=0.20]{Bob}} & &
		\multirow{5}{*}{\includegraphics[scale=0.20]{Alice}} &  \\
		&  &  & & $r \leftarrow \{0,1\}^{L}$  \\
		&  & $\xleftarrow{r}$ & &  \\
		$b \in \{0,1\}$&  & & &  \\
		$s \in \{0,1\}^{\ell}$&  & & &  \\[35pt]
		\multicolumn{5}{l}{$\mathsf{commit}(b, r, s)  = 
			\begin{cases}
			G(s), & b = 0\\
			G(s) \oplus r, & b=1
			\end{cases}
			$}  \\
		&  & $\xrightarrow{\mathsf{commit}(b, r, s)} $ & &  \\
	\end{tabular}
\end{center}

\end{frame}

\end{document}