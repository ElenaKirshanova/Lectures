\documentclass[usenames,dvipsnames,8pt,aspectratio=169]{beamer}
\usepackage{amsmath,amsfonts,amssymb}
\usepackage{mathtools}
\usepackage{etex} %for Windows
\usepackage[utf8]{inputenc}
\usepackage[english, russian]{babel} 

%\usepackage{microtype}			% Better interword spacing and additional kerning.
\usepackage{ellipsis}			% Adjusted space with \dots between two words.
\usepackage{graphicx}
\usepackage{pstricks}

\usepackage{xcolor}


\usepackage{changepage}

\usepackage{algorithm}
\usepackage{algpseudocode}
%\usepackage[]{algorithm2e}
%\usepackage{algorithmic}

%\usepackage{tcolorbox}

\addtobeamertemplate{footline}{%
	\setlength\unitlength{1ex}%
	\begin{picture}(0,0) 
	% \put{} defines the position of the frame
	\put(155,0){\makebox(0,0)[bl]{
			%\includegraphics[scale=0.65]{white_square}
			%\includegraphics[scale=0.65]{dark_square}
			\includegraphics[scale=0.65]{grey_circle}
	}}%
	\end{picture}%
}{}

\usepackage{tikz}
\usetikzlibrary{tikzmark,calc}
\usetikzlibrary{positioning, backgrounds}
\usetikzlibrary{arrows, chains, matrix, scopes, patterns, shapes, fit}
\usetikzlibrary{mindmap,trees,shadows}
\usetikzlibrary{decorations.pathreplacing}

\usepackage{pgfplots}

\pgfmathdeclarefunction{gauss}{2}{%
	\pgfmathparse{1/(#2*sqrt(2*pi))*exp(-((x-#1)^2)/(2*#2^2))}%
}


\tikzset{
	invisible/.style={opacity=0},
	visible on/.style={alt={#1{}{invisible}}},
	alt/.code args={<#1>#2#3}{%
		\alt<#1>{\pgfkeysalso{#2}}{\pgfkeysalso{#3}} % \pgfkeysalso doesn't change the path
	},
}

\newcommand\strikeout[2][]{%
	\begin{tabular}[b]{@{}c@{}} 
		\makebox(0,0)[cb]{{#1}} \\[-0.2\normalbaselineskip]
		\rlap{\color{Orange}\rule[0.5ex]{\widthof{#2}}{1.5pt}}#2
\end{tabular}}

\newcommand\Fontvi{\fontsize{11}{13.2}\selectfont}

\usepackage{listings} % for C++ code

\usepackage{braket}
%\usepackage[braket, qm]{qcircuit}



\usepackage[T1]{fontenc}
%\usepackage[sfdefault,scaled=.85]{FiraSans}
%\usepackage{newtxsf}
%\usepackage[nomap]{FiraMono}



%\renewtheorem{theorem}{Теорема}
%\renewtheorem{lemma}{Лемма}
%\renewtheorem{definition}{Определение}
%\renewtheorem{corollary}{Следствие}
%\renewtheorem{fact}{Факт}

\usefonttheme[onlymath]{serif}
\renewcommand\sfdefault{cmbr}

\renewcommand{\bfdefault}{sb}

\definecolor{CharCoalDark}{RGB}{13, 16, 19}
\definecolor{Orange}{RGB}{255, 165,0}
\definecolor{DarkOrange}{RGB}{255, 165,0}
\definecolor{LightSalmon}{RGB}{255, 160, 122}
\definecolor{LeafGreen}{RGB}{34, 139,  34}
\definecolor{Coral}{RGB}{255, 127, 80}
\definecolor{DarkTurquoise}{RGB}{0, 206, 209}

\definecolor{darkslateblue}{RGB}{72,61,139}

%\newtheorem{defRus}{Определение}
%\newtheorem{thmRus}{Теорема}
%s\newtheorem{corRus}{Следствие}

\def\darktheme{}
\ifdefined\darktheme
	\setbeamercolor{background canvas}{bg=CharCoalDark}
	\setbeamerfont{title}{series=\bfseries}
	\setbeamercolor{title}{fg=Orange}
	\setbeamercolor{section in toc}{fg=white}
	\setbeamercolor{frametitle}{fg=Orange}
	\setbeamercolor{normal text}{fg=white}
	%\setbeamercolor{normal text}{fontsize=12pt}
	\setbeamercolor{itemize item}{fg=Orange}
	\setbeamercolor{itemize item item}{fg=Orange}
	\setbeamercolor{enumerate item}{fg=Orange}
	\setbeamercolor{block title}{bg=DarkOrange,fg=white}
	\setbeamerfont{block title}{series=\bfseries}
	
	\setbeamertemplate{itemize item}[circle]
	%\setbeamertemplate{itemize subitem}[$\checkmark$]
	\setbeamertemplate{itemize subitem}{\color{Orange}\Large$\textbullet$}
	\setbeamertemplate{itemize subitem}{\color{Orange} \tiny $\blacksquare$}
\else
	\setbeamercolor{background canvas}{bg=white}
	\setbeamerfont{title}{series=\bfseries}
	\setbeamercolor{title}{fg=darkslateblue}
	\setbeamercolor{section in toc}{fg=black}
	\setbeamercolor{frametitle}{fg=darkslateblue}
	\setbeamercolor{normal text}{fg=black}
	%\setbeamercolor{normal text}{fontsize=9pt}
	\setbeamercolor{itemize item}{fg=darkslateblue}
	\setbeamercolor{itemize item item}{fg=darkslateblue}
	\setbeamercolor{enumerate item}{fg=darkslateblue}
	\setbeamercolor{block title}{bg=darkslateblue,fg=white}
	\setbeamerfont{block title}{series=\bfseries}
	
	\setbeamertemplate{itemize item}[circle]
	%\setbeamertemplate{itemize subitem}[$\checkmark$]
	\setbeamertemplate{itemize subitem}{\color{blue}\Large$\textbullet$}
	\setbeamertemplate{itemize subitem}{\color{blue} \tiny $\blacksquare$}

\fi

% footnote without a marker
\newcommand\blfootnote[1]{%
	\begingroup
	\renewcommand\footnoterule{}
	\renewcommand\thefootnote{}\footnote{#1}%
	\addtocounter{footnote}{-1}%
	\endgroup
}

\newcommand*{\Scale}[2][4]{\scalebox{#1}{\ensuremath{#2}}}%

\newcommand\Item[1][]{%
	\ifx\relax#1\relax  \item \else \item[#1] \fi
	\abovedisplayskip=0pt\abovedisplayshortskip=0pt~\vspace*{-\baselineskip}}

\pgfdeclareradialshading{ring}{\pgfpoint{0cm}{0cm}}%
{rgb(0cm)=(1,1,1);
	rgb(0.7cm)=(1,1,1);
	rgb(0.719cm)=(1,1,1);
	rgb(0.72cm)=(0.975,0,0);
	rgb(0.9cm)=(1,1,1)}

\usepackage[absolute,overlay]{textpos} %to clip to a corner
\newcommand\FrameText[1]{%
	\begin{textblock*}{\paperwidth}(\textwidth-35pt, 10 pt)
		\raggedright #1\hspace{.5em}
\end{textblock*}}

\makeatletter
\let\save@measuring@true\measuring@true
\def\measuring@true{%
	\save@measuring@true
	\def\beamer@sortzero##1{\beamer@ifnextcharospec{\beamer@sortzeroread{##1}}{}}%
	\def\beamer@sortzeroread##1<##2>{}%
	\def\beamer@finalnospec{}%
}
\makeatother

\AtBeginSection[]
{
	\begin{frame}<beamer>
		\frametitle{Outline}
		\tableofcontents[currentsection]
	\end{frame}
}


%\institute{ENS Lyon}
\author{\\ [10pt]
}
\titlegraphic{
	
	%\includegraphics[width=2.5cm]{erc_logo_gray}\hspace*{2.5cm}~%
	%\includegraphics[width=4.0cm]{ens_logo_gray}
}
\title{Лекция №1 \\[10pt]
		Часть 3. Абсолютная криптографическая стойкость. \\ Одноразовый блокнот.}

\date{ Елена Киршанова \\  \textbf{Курс ``Основы криптографии''} \\  }


\setbeamertemplate{navigation symbols}{} %removes navigation

% proper highlightling of a code-snippet
\lstset{language=C++,
	keywordstyle=\color{magenta},
	stringstyle=\color{Goldenrod},
	commentstyle=\color{gray},
	breaklines=false,
	%morecomment=[l][\color{magenta}]{\#}
}

%\setlength{\parskip}{8pt}
\input{header} %all defs
\begin{document}
	
\begin{frame}
	\titlepage
\end{frame}

\begin{frame}{Шифр Шеннона (Shannon's cipher)}
\Large
Положим $\keyS, \mesS, \cipS$ -- множества ключей, открытых текстов, шифр-текстов 
\begin{block}{Шифр Шеннона --}
			это тройка функций $\KeyGen, \Enc, \Dec$:
			\begin{align*}
				\Enc: \keyS \times \mesS & \rightarrow \cipS \\
				\Enc(k, m)  &= c  \\[10pt]
				\Dec: \keyS \times \cipS & \rightarrow \mesS \\
				\Dec(k, c) &= m,
			\end{align*}
			для которых выполняется 
			\[
				\Dec(k, \Enc(k,m)) == m \qquad \forall k \leftarrow \KeyGen, m \in \mesS 
			\]
\end{block}
\end{frame}

\begin{frame}{Абсолютная криптографическая стойкость (perfect secrecy)}
\Large
	\begin{itemize}
		\itemsep 5pt
		\item На каждом из этих множеств зададим распределение:  \\ $\Pr[M = m]$ -- вероятность выбора $m \in \mesS$.
		\item Аналогично для $K \in \keyS, C \in \cipS$.
	\end{itemize}

	\begin{block}{Абсолютная криптографическая стойкость}
		Шифр-схема $\Pi = (\KeyGen, \Enc, \Dec)$ обладает \emph{абсолютной криптографической стойкостью}, если для любого распределения над $\mesS$
		\[
			\Pr \left[ M= m | C = c\right] = \Pr\left[M = m\right] \quad \forall m \in \mesS, c \in \cipS.
		\]
	\end{block}
	\textbf{Интуиция:} шифр-текст $c$ не содержит никакой информации об \\ открытом тексте $m$.
%	Equivalent definition:
%	\[
%		\Pr\left[\Enc(k, m_0) = c\right] = \Pr\left[\Enc(k, m_1) = c\right] \quad \forall m_0, m_1 \in \mesS, c \in \cipS.
%	\]
\end{frame}

\begin{frame}{Шифр-текст не зависит от открытого текста}
	\Large
	\vspace{-50pt}
	  Шифр-схема $\Pi = (\Enc, \Dec)$, определённая над $\keyS, \mesS, \cipS$, абсолютно стойка тогда и только тогда, когда 

	\[
		\Pr[C = c \; |\; M =m  ] = \Pr[C = c] \qquad \forall m \in \mesS, c \in \cipS.
	\]
	

\end{frame}

\begin{frame}{Шифр-текст не отличимы друг от друга}
	\Large
	\vspace{-50pt}
	Шифр-схема $\Pi = (\Enc, \Dec)$, определённая над $\keyS, \mesS, \cipS$, абсолютно стойка тогда и только тогда, когда для любых $m_0, m_1 \in \mesS$ выполняется
	\[
	\Pr[C = c \; | \; M = m_0] = \Pr[C = C \; | \; M = m_1]
	\]
\end{frame}

\begin{frame}{Одноразовый блокнот (One-time pad) или шифр Вернама }
\LARGE
\vspace{-40pt}
\begin{block}{Одноразовый блокнот}
	Положим $\mesS, \keyS, \cipS = \{0,1\}^n$.
	\begin{itemize}
		\item $\KeyGen(1^{\lambda}): k \leftarrow \{0,1\}^n$ \\[10pt]
		\item $\Enc(k, m \in \{0,1\}^n): c = k \oplus m$ \\[10pt]
		\item $\Dec(k, c \in \{0,1\}^n): m = k \oplus c$ \\[10pt]
	\end{itemize}
\end{block}

{\color{Orange} Теорема.}
	Одноразовый блокнот является абсолютно  стойким.


\end{frame}

\begin{frame}{Недостаток абсолютной стойкости }
\LARGE
{\color{Orange} Теорема.} Положим $\Pi = (\KeyGen, \Enc, \Dec)$  -- абсолютно стойкая шифр-схема. Тогда $\abs{\keyS} \geq \abs{\mesS}$
\vspace{40pt}

\textbf{Интуиция:} Абсолютно стойкие схемы неэффективны.
\end{frame}

\begin{frame}{Теорема Шэннона (1949)}
\LARGE
\vspace{-70pt}
Положим $\Pi = (\KeyGen, \Enc, \Dec)$  -- шифр-схема с $\abs{\keyS}=\abs{\mesS} =\abs{\cipS} $. \\[3pt]
Тогда $\Pi$ --  абсолютно стойкая тогда и только тогда, когда\\[6pt]
\begin{enumerate}
	\itemsep 7pt
	\item $\KeyGen$ выбирает $k \in \keyS$ с вероятностью $\frac{1}{\abs{\keyS}}$ для всех $k$
	\item $\forall m \in \mesS, c \in \cipS \;$ существует единственный $ k \in \keyS$ : $c = \Enc(k, m)$.
\end{enumerate}
		
\end{frame}

\begin{frame}{Одноразовый блокнот на практике}
	\Large
	\begin{itemize}
		\itemsep 12pt
		\item Правительственная «горячая линия» между Вашингтоном и Москвой в 60-x\\[4pt]
		\url{https://en.wikipedia.org/wiki/Moscow\%E2\%80\%93Washington_hotline}
		
		\item Вьетнамские войны\\[4pt]
		\url{https://eprint.iacr.org/2016/1136.pdf}
	\end{itemize}
	
\end{frame}
%
%\begin{frame}{Price to pay for perfect secrecy}
%\Large
%	\begin{block}{Size of $\keyS$}
%		Let $\Pi$ be perfectly secure. Then
%		\[
%			|\keyS| \geq |\mesS|
%		\]
%	\end{block}
%
%	\vspace{10pt}
%	
%	\begin{block}{Shannon's theorem}
%		Let $\Pi = (\KeyGen, \Enc, \Dec)$ satisfy $|\keyS| = |\mesS| = |\cipS|$. Then $\Pi$ is perfectly secure iff
%		\begin{enumerate}
%			\item $\KeyGen$ chooses $k \leftarrow$ uniformly at random with prob.\ $\frac{1}{|\keyS|}$
%			\item For all $m \in \mesS$, $c \in \cipS$, $\exists!$ $k \in \keyS:\,$ $c = \Enc(k,m)$.
%		\end{enumerate}
%	\end{block}
%	
%	\vspace{15pt}
%	\large
%	\centering
%	See proofs in Katz\&Lindell \textit{Introduction to Modern Cryptography}
%\end{frame}
%
%\begin{frame}{Computational Security}
%\Large
%	{\color{Orange}\textbf{Perfect Secrecy}}
%	\begin{itemize}
%		\item Information-theoretic (strong) security against {\color{Orange}\textbf{unbounded}} adversary
%		\item Impractical key space size
%	\end{itemize}
%		\vspace{15pt}
%	{\color{Orange}\textbf{Computational Security}}
%	\begin{itemize}
%		\item We usually use keys of sizes $128, 256$ bits
%		\item Security against {\color{Orange}\textbf{ppt}} adversaries
%		\item Unbounded adversary with access to plaintext-ciphertext pairs $(m_i, c_i)$ can launch an exhaustive search for $k \in \keyS$ s.t.\ $\Enc(k, m_i) == c_i \; \forall i$.
%	\end{itemize}
%\end{frame}
%
%\begin{frame}{Pseudorandom Generators (PRGs)}
%\LARGE 
%	Idea: `Stretch' truly random $\ell$-bits seed $s$ into a longer $L$-bits `random looking' string  
%	\vspace{15pt}
%	Define
%	\LARGE
%	\begin{align*}
%		G : \{0,1\}^{\ell} & \rightarrow \{0,1\}^{L}:	\\
%		s & \mapsto G(s) 
%	\end{align*}
%	
%	{\color{Orange}\textbf{Intuition:}} an ppt adversary cannot tell the difference between $G(s)$ and $r \leftarrow \{0,1\}^L$.
%	
%\end{frame}
%
%\begin{frame}{Statistical tests}
%\Large 
%A {\color{Orange}\textbf{Is $\Pi$ statistical test}} on $\{0,1\}^n$ is an algorithm $A$ s.t. $A(x)$ outputs $0$ (=``not random'') or $1$ (=``random'') \\[10pt]
%\LARGE
%\begin{enumerate}
%	\item $A(x) = 1 \quad \abs{\# 0(x) - \# 1(x)} \leq 10 \sqrt{n}$  \\ [10pt]
%	\item $A(x) = 1 \quad \abs{\# 00(x) - n/4} \leq 10 \sqrt{n}$  \\ [10pt]
%	\item $A(x) = 1 \quad \max \mathsf{len}\{1...1(x) \} \leq 10 \log n$  \\ [10pt]
%\end{enumerate}
%
%\end{frame}
%
%\begin{frame}{Secure PRG}
%\Large
%	\begin{block}{Formal definition}
%	A PRG $G$ is an efficient deterministic algorithm that given a seed $s \in \mathcal{S} = \{0,1\}^{\ell}$ outputs an $r \in \mathcal{R} = \{0,1\}^L$ 
%	
%\end{block}
%	{\color{Orange}\textbf{Attack game for PRG:}} \\
%	\vspace{15pt}
%	%For a given PRG $G$ and a given adversary $\mathcal{A}$, define 	{\color{Orange}\textbf{Experiment 0:}} and 	{\color{Orange}\textbf{Experiment 1:}}: \\
%	{\color{Orange}\textbf{Experiment 0:}}
%	The Challenger computes $r \in \mathcal{R}$ as
%	\begin{align*}
%		 \Huge	s &\leftarrow \mathcal{S} \\
%		 \Huge	r & \leftarrow G(s)
%	\end{align*}
%	\pause
%	{\color{Orange}\textbf{Experiment 1:}}
%	The Challenger computes $r \in \mathcal{R}$ as
%	\begin{align*}
%		 \Huge	r & \leftarrow \{0,1\}^{L}
%	\end{align*}
%	Let $W_b$ be the event that $\mathcal{A}$ outputs $b$. \\
%	 $\mathcal{A}$'s advantage: $	\mathsf{PRGadv} \left[  \mathcal{A}, G \right ] = |\Pr[W_0] - \Pr[W_1]|$. \\[5pt]
%	 
%	 A PRG $G$ is {\color{Orange}\textbf{secure}} if $	\mathsf{PRGadv}$ is negligible for all ppt $\mathcal{A}$.
%	 
%
%\end{frame}
%
%\begin{frame}{Stream Cipher = OTP with keys output by a PRG}
%\LARGE
%	Let $G: \{0,1\}^{\ell} \rightarrow \{0,1\}^L$ \\[5pt]
%	
%	The {\color{Orange}\textbf{stream cipher}} $\Pi = (\KeyGen, \Enc, \Dec)$ is defined over
%	\begin{itemize}
%		\item $\keyS = \{0,1\}^{\ell}$ \\
%		\item $\mesS = \{0,1\}^{L}$ \\
%		\item $\cipS = \{0,1\}^{L}$ \\
%	\end{itemize}
%	\vspace{10pt}
%	For $s \in \{0,1\}^{\ell}$
%	\begin{itemize}
%		\item $\Enc(s, m \in \{0,1\}^L): c = G(s) \oplus m$ \\[10pt]
%		\item $\Dec(s, c \in \{0,1\}^L): m = G(s)\oplus c$ \\
%	\end{itemize}
%
%\end{frame}
%
%\begin{frame}{Semantic Security}
%		\begin{tikzpicture}
%			\draw[-stealth, thick] (-1.8,-1.0) -- (-1.0,-1.0) node[above,midway]{\Huge $b$};
%			\draw[fill=CharCoalDark, draw=white, opacity=0.5] (-1.0,-2.0) rectangle (1.8,1.8)  node[color=white, opacity=1,align=center, pos=0.5]{
%				\Large  {\color{Orange}{\underline{Challenger}} }  \\[18pt]
%				\Huge $k \leftarrow \keyS$
%			};
%			\draw[stealth-, thick] (1.8,0.5) -- (6.5,0.5) node[above,midway]{\Large $m_0, m_1 \in \mesS,  |m_0|=|m_1|$};
%			\draw[-stealth, thick] (1.8,-0.5) -- (6.5,-0.5) node[above,midway]{\Large $c \leftarrow \Enc(k,m)$};
%			\draw[fill=CharCoalDark, draw=white, opacity=0.5] (6.5,-2.0) rectangle (8.0,1.8)  node[color=white,pos=0.8] {
%				\Large  {\color{Orange}{\underline{$\mathcal{A}$}} }
%			};
%			\draw[-stealth, thick] (8.0,-1.0) -- (9.0,-1.0) node[above,midway]{\Huge $b'$};
%		\end{tikzpicture}
%		\Large
%		\vspace{15pt} \\
%		\centering
%		
%		Let $W_b$ be the event that $\mathcal{A}$ outputs $b$. \\
%		$\mathcal{A}$'s advantage: $	\mathsf{Adv} \left[  \mathcal{A}, \Enc \right ] = |\Pr[W_0] - \Pr[W_1]|$. \\[5pt]
%		
%		 $\Pi$ is {\color{Orange}\textbf{semantically secure}} if $	\mathsf{PRGadv} = \negl(\lambda)$  for all ppt $\mathcal{A}$.
%\end{frame}
%
%%\begin{frame}{Unpredictability of a PRG}
%%	content...
%%\end{frame}
%
%\begin{frame}{Semantic Security of $\Pi$}
%\LARGE
%	\begin{theorem}
%		If $G$ is a secure PRG, then the stream cipher $\Pi$ constructed from $G$ is semantically secure.
%	\end{theorem}
%\vspace{20pt}
%{\color{Orange}\textbf{Proof strategy: }} for any adversary $\mathcal{A}$ against semantic security, $\exists$ an adversary $\mathcal{B}$ against $G$.
%\end{frame}
%
%\begin{frame}{QUESTION!}
%	\LARGE
%	
%	Let $G$ be a PRG that the last bit of the output is always 0. \\[10pt]
%	Let $\Pi$ be a stream cipher constructed from $G$.\\[15pt]
%	\centering
%	{\color{Orange}\textbf{Is $\Pi$ semantically secure?}} 
%\end{frame}
%
%
%
%\begin{frame}{Constrictions of a PRG: Salsa and ChaCha}
%\Large
%	\begin{itemize}
%		\item Salsa20,ChaCha20: proposed by D.Bernstein in 2005, 2008
%		\item used in many TLS cipher suits
%		\item Input: $256$-bit seed and a parameter $L$
%		\item Output: $(256 \cdot L)$-bit pseudorandom string
%	\end{itemize}
%	\vspace{20pt}
%	\pause
%	Two components
%	\begin{enumerate}
%		\item $\mathsf{pad}(s, j, 0)$: takes a seed $s$, a $64$-bit counter $j$ and a $64$-bit nonce\\
%		Output: $512$-bit block
%		\item a fixed public permutation $\pi: \{0,1\}^{512} \rightarrow \{0,1\}^{512}$
%	\end{enumerate}
%	\vspace{20pt}
%	See \url{https://cr.yp.to/chacha.html} for details
%\end{frame}
%
%\begin{frame}{ChaCha PRG}
%\begin{figure}
%	\includegraphics[width=\textwidth]{ChaCha20}
%\end{figure}
%
%Nonce -- the third parameter of $\mathsf{pad}(s, j, 0)$ is used to convert a PRG into a PRF (useful for encryption of multiple messages).
%\vfill
%\small
%{\color{gray}\textbf{picture is taken from D.Boneh, V.Shoup A Graduate Course in Applied Cryptography}} 
%\end{frame}
%
%\begin{frame}{(Somewhat) Broken PRGs}
%\LARGE
%\begin{enumerate}
%	\itemsep2em 
%	\item {\color{Orange}\textbf{linear congruential generators}} 
%	\begin{itemize}
%		\LARGE
%				\itemsep5pt  
%		\item had been used in glibc, Microsoft Visual Basic, Java
%		\item notorious for RANDU
%		\item \textbf{not cryptographically secure PRG!}
%	\end{itemize}
%
%	\item {\color{Orange}\textbf{RC4}} 
%	\begin{itemize}
%		\LARGE
%		\itemsep5pt 
%		\item proposed by R.Rivest  in 1987
%		\item used to be a part of TLS, 802.11b WEP
%		\item \textbf{not cryptographically secure PRG!}
%	\end{itemize}
%
%	\item {\color{Orange}\textbf{Linear feedback shift registers}}
%	\begin{itemize}
%			\LARGE
%			\itemsep5pt
%		\item used for protecting movies on DVD disks
%		\item weakly security  PRG (Trivium)
%	\end{itemize} 
%\end{enumerate}
%
%\end{frame}
%
%\begin{frame}{A Random Number Generator}
%	\Large
%	\begin{itemize}
%		\itemsep7pt
%		\item In practice, random bits are generated using a random number generator,  RNG
%		\item An RNG outputs a sequence of pseudorandom bits
%		\item Unlike PRG, an RNG take additional input (entropy source)
%		\item Example in Linux: $\mathsf{/dev/random}$
%		\item Entropy is usually taken from hardware (keyboard/mouse events, hardware interrupts, jitters).
%	\end{itemize}
%\end{frame}
%
%\begin{frame}{Application: Coin flipping}
%\LARGE
%Task:  throw a fair coin over without interaction 
%\begin{center}
%	\begin{tabular}{c c c c c}
%		 \multicolumn{5}{c}{$G: \{0,1\}^{\ell} \rightarrow \{0,1\}^{L}$}\\[10pt]
%		& Bob  & & Alice &  \\
%		 & \multirow{5}{*}{\includegraphics[scale=0.20]{Bob}} & &
%		\multirow{5}{*}{\includegraphics[scale=0.20]{Alice}} &  \\
%		&  &  & & $r \leftarrow \{0,1\}^{L}$  \\
%		&  & $\xleftarrow{r}$ & &  \\
%		Flips a coin &   & &  &  \\
%		$b \in \{0,1\}$&  & & &  \\
%		$s \in \{0,1\}^{\ell}$&  & & &  \\[15pt]
%		\multicolumn{5}{l}{$\mathsf{commit}(b, r, s)  = 
%			\begin{cases}
%			G(s), & b = 0\\
%			G(s) \oplus r, & b=1
%			\end{cases}
%			$}  \\
%		&  & $\xrightarrow{\mathsf{commit}(b, r, s)} $ & &  \\
%	\end{tabular}
%\end{center}
%
%\end{frame}

\end{document}