\documentclass[usenames,dvipsnames,8pt,aspectratio=169]{beamer}
\usepackage{amsmath,amsfonts,amssymb}
\usepackage{mathtools}
\usepackage{etex} %for Windows
\usepackage[utf8]{inputenc}
\usepackage[english, russian]{babel} 
%\usepackage{microtype}			% Better interword spacing and additional kerning.
\usepackage{ellipsis}			% Adjusted space with \dots between two words.
\usepackage{graphicx}
\usepackage{pstricks}

\usepackage{xcolor}


\usepackage{changepage}

\usepackage{algorithm}
\usepackage{algpseudocode}
%\usepackage[]{algorithm2e}
%\usepackage{algorithmic}

%\usepackage{tcolorbox}


\usepackage{caption}
\usepackage{subcaption}
%\usepackage{stackengine}


\usepackage{tikz}
\usetikzlibrary{tikzmark,calc}
\usetikzlibrary{positioning, backgrounds}
\usetikzlibrary{arrows, chains, matrix, scopes, patterns, shapes, fit}
\usetikzlibrary{mindmap,trees,shadows}
\usetikzlibrary{decorations.pathreplacing}
%\usetikzlibrary{crypto.symbols}

\usepackage{pgfplots}

\pgfmathdeclarefunction{gauss}{2}{%
	\pgfmathparse{1/(#2*sqrt(2*pi))*exp(-((x-#1)^2)/(2*#2^2))}%
}


\tikzset{
	invisible/.style={opacity=0},
	visible on/.style={alt={#1{}{invisible}}},
	alt/.code args={<#1>#2#3}{%
		\alt<#1>{\pgfkeysalso{#2}}{\pgfkeysalso{#3}} % \pgfkeysalso doesn't change the path
	},
}

\newcommand\strikeout[2][]{%
	\begin{tabular}[b]{@{}c@{}} 
		\makebox(0,0)[cb]{{#1}} \\[-0.2\normalbaselineskip]
		\rlap{\color{Orange}\rule[0.5ex]{\widthof{#2}}{1.5pt}}#2
\end{tabular}}

\newcommand\Fontvi{\fontsize{11}{13.2}\selectfont}

\usepackage{listings} % for C++ code

\usepackage{braket}
%\usepackage[braket, qm]{qcircuit}



\usepackage[T1]{fontenc}
%\usepackage[sfdefault,scaled=.85]{FiraSans}
%\usepackage{newtxsf}
%\usepackage[nomap]{FiraMono}

\usepackage{fontawesome} % for ruble sign




\usefonttheme[onlymath]{serif}
\renewcommand\sfdefault{cmbr}

\renewcommand{\bfdefault}{sb}

\definecolor{CharCoalDark}{RGB}{13, 16, 19}
\definecolor{Orange}{RGB}{255, 165,0}
\definecolor{DarkOrange}{RGB}{255, 165,0}
\definecolor{LightSalmon}{RGB}{255, 160, 122}
\definecolor{LeafGreen}{RGB}{34, 139,  34}
\definecolor{Coral}{RGB}{255, 127, 80}
\definecolor{DarkTurquoise}{RGB}{0, 206, 209}

%\newtheorem{defRus}{Определение}
%\newtheorem{thmRus}{Теорема}
%s\newtheorem{corRus}{Следствие}


\setbeamercolor{background canvas}{bg=CharCoalDark}

\setbeamerfont{title}{series=\bfseries}
\setbeamercolor{title}{fg=Orange}
\setbeamercolor{section in toc}{fg=white}
\setbeamercolor{frametitle}{fg=Orange}
\setbeamercolor{normal text}{fg=white}
%\setbeamercolor{normal text}{fontsize=12pt}
\setbeamercolor{itemize item}{fg=Orange}
\setbeamercolor{enumerate item}{fg=Orange}
\setbeamercolor{enumerate item item}{fg=Orange}
\setbeamercolor{itemize item item}{fg=Orange}
\setbeamercolor{enumerate item}{fg=Orange}
\setbeamercolor{block title}{bg=DarkOrange,fg=white}
\setbeamerfont{block title}{series=\bfseries}

\setbeamertemplate{itemize item}[circle]
\setbeamertemplate{eumerate subitem}{\color{Orange}[$\checkmark$]}
\setbeamertemplate{itemize subitem}{\color{Orange}\Large$\textbullet$}
\setbeamertemplate{itemize subitem}{\color{Orange} \tiny $\blacksquare$}

% footnote without a marker
\newcommand\blfootnote[1]{%
	\begingroup
	\renewcommand\footnoterule{}
	\renewcommand\thefootnote{}\footnote{#1}%
	\addtocounter{footnote}{-1}%
	\endgroup
}

\newcommand*{\Scale}[2][4]{\scalebox{#1}{\ensuremath{#2}}}%

\newcommand\Item[1][]{%
	\ifx\relax#1\relax  \item \else \item[#1] \fi
	\abovedisplayskip=0pt\abovedisplayshortskip=0pt~\vspace*{-\baselineskip}}

\addtobeamertemplate{footline}{%
	\setlength\unitlength{1ex}%
	\begin{picture}(0,0) 
	% \put{} defines the position of the frame
	\put(155,0){\makebox(0,0)[bl]{
			%\includegraphics[scale=0.65]{white_square}
			%\includegraphics[scale=0.65]{dark_square}
			\includegraphics[scale=0.65]{grey_circle}
	}}%
	\end{picture}%
}{}

\newcommand{\AxisRotator}[1][rotate=0]{%
	\tikz [x=0.45cm,y=1.2cm,line width=.2ex,-stealth,#1] \draw[color=Orange] (0,0) arc (-150:150:2 and 1);%
}

\usepackage[absolute,overlay]{textpos} %to clip to a corner
\newcommand\FrameText[1]{%
	\begin{textblock*}{\paperwidth}(\textwidth-35pt, 10 pt)
		\raggedright #1\hspace{.5em}
\end{textblock*}}

\makeatletter
\let\save@measuring@true\measuring@true
\def\measuring@true{%
	\save@measuring@true
	\def\beamer@sortzero##1{\beamer@ifnextcharospec{\beamer@sortzeroread{##1}}{}}%
	\def\beamer@sortzeroread##1<##2>{}%
	\def\beamer@finalnospec{}%
}
\makeatother

\AtBeginSection[]
{
	\begin{frame}<beamer>
		\frametitle{Outline}
		\tableofcontents[currentsection]
	\end{frame}
}



\titlegraphic{
	
	%\includegraphics[width=2.5cm]{erc_logo_gray}\hspace*{2.5cm}~%
	%\includegraphics[width=4.0cm]{ens_logo_gray}
}
\title{Лекция №4 \\[10pt]
	Коды аутентификации сообщений.}

\date{ Елена Киршанова \\  \textbf{Курс ``Основы криптографии''} \\  }



\setbeamertemplate{navigation symbols}{} %removes navigation

% proper highlightling of a code-snippet
\lstset{language=C++,
	keywordstyle=\color{magenta},
	stringstyle=\color{Goldenrod},
	commentstyle=\color{gray},
	breaklines=false,
	%morecomment=[l][\color{magenta}]{\#}
}

%\setlength{\parskip}{8pt}
\input{header} %all defs
\begin{document}
	
\begin{frame}
	\titlepage
\end{frame}


\begin{frame}{Конфиденциальность \& целостность}
\LARGE 

\begin{itemize}
	\itemsep 1em
	\item В предыдущих лекциях: {\color{Orange} конфиденциальность} сообщений
	\item В этой лекции:  {\color{Orange} целостность}
\end{itemize}

\vspace{20pt}
\LARGE {\color{Orange}Криптопрититив}: Код Аутентификации Сообщения (или Имитовставка)  \\[5pt] 
Message Authentication Code (MAC) 


\end{frame}

\begin{frame}{Мотивация}
\Large
	\begin{center}
		\begin{tabular}{c c c c c}
			& Алиса  & & Боб &  \\
			& \multirow{5}{*}{\includegraphics[scale=0.15]{Alice}} & &
			\multirow{5}{*}{\includegraphics[scale=0.15]{Bob}} &  \\
			&  & \Huge $\xrightarrow{\text{``Переслать \texttt{1000 \faRub } на карту XXXX'' }}$ & &  \\[20pt]
			\multicolumn{5}{c}{\includegraphics[scale=0.15]{Eve}} \\
			\multicolumn{5}{c}{Ева}
		\end{tabular}
	\end{center}

	
\end{frame}


\begin{frame}{MAC: определение}
\Large
Цель: отправить сообщение $m$ от Алису к Бобу так, что бы злоумышленник не смог модифицировать $m$, оставаясь незамеченным \\[10pt]
\begin{center}
	\begin{tabular}{c c c c c}
		& Alice  & & Bob &  \\
		& \multirow{5}{*}{\includegraphics[scale=0.18]{Alice}} & &
		\multirow{5}{*}{\includegraphics[scale=0.18]{Bob}} &  \\
		&  & \Huge $\xrightarrow{\mkern25mu m, \; \mactag \mkern25mu}$ & &  \\ 
		\LARGE $k \in \keyS$ &   & &  & $k \in \keyS$ \\[10pt]
		$\mactag \leftarrow S(k,m)$   & & &  &    \\ [20pt] 
		\multicolumn{5}{r}{$V(k,m, \mactag) \in \{ 0, 1\}$ }  \\
	\end{tabular}
\end{center}
\pause
\large
{\color{Orange} Код Аутентификации Сообщения } состоит из 3-х ppt алгоритмов:
\vspace{-2pt}
\begin{itemize}
	\item Генерация ключа: $\KeyGen(1^\lambda): k \leftarrow \keyS$ \\[4pt]
	\item Генерация тага: $S(k, m): \mactag \leftarrow \tagS$ \\[4pt]
	\item Верификация тага: $V(k, m, \mactag): \{0, 1\}  $
\end{itemize}
\end{frame}

\begin{frame}{MAC: корректность и безопасность}
\Large
{\color{Orange} Код Аутентификации Сообщения } состоит из 3-х ppt алгоритмов:
\vspace{-2pt}
\begin{itemize}
	\item Генерация ключа: $\KeyGen(1^\lambda): k \leftarrow \keyS$ \\[4pt]
	\item Генерация тага: $S(k, m): \mactag \leftarrow \tagS$ \\[4pt]
	\item Верификация тага: $V(k, m, \mactag): \{0, 1\}  $
\end{itemize}

\vspace{10pt}
{\color{Orange} Корректность:} \LARGE 
$ V(k, m, S(k,m)) = 1 \; \forall \; m \in \mesS, k \in \keyS $

\vspace{10pt}
{\color{Orange}Безопасность:}
Для ppt атакующего $\adv$, имеющего пары $\{(m_1, t_1), \ldots, (m_N, t_N) \}$ для $m_i$, выбранных им самим, $\adv$ не может сгенерировать новую пару  $(m,t)$
\[(m,t) \notin \{(m_1, t_1), \ldots, (m_N, t_N) \}\]

\vspace{10pt}

Поэтому безопасная длина тага: $96, 128, 256 $ бит.
\end{frame}

\begin{frame}{MAC: безопасность}

\Large 
\begin{center}
	$I = (\KeyGen, S, V)$ --  MAC \\[10pt]
	
	\begin{tabular}{c c c}
		{\color{Orange} Челленджер $\mathcal{C}$ } & & {\color{Orange} Атакующий $\mathcal{A}$ }\\ [5pt]
		$k \leftarrow \KeyGen(1^\lambda)$ & $\xrightarrow{\quad \lambda \quad}$  &\\[5pt]
		& $\xleftarrow{\quad  m_i \quad}$  &$m_i\leftarrow \mesS $\\ [5pt]
		
		$t_i \leftarrow S(k, m_i)$ & $\xrightarrow{\quad t_i \quad}$ &\\ [5pt]

		& $\xleftarrow{\quad  (m^\star, t^\star) \quad}$ & \\ [5pt]
	\end{tabular}
	\begin{tikzpicture}[overlay]
	%\draw[fill=none, draw=white, opacity=0.5] (-8.5,-2.3) rectangle (-5.0,2.7); 
	%\draw[fill=none, draw=white, opacity=0.5] (-3.0,-2.3) rectangle (0.0,2.7); 
	\end{tikzpicture}
\end{center}

\vspace{5pt}
$\mathtt{W_{I, \adv}}$ -- событие $Ver(k, m^\star, t^\star) == 1$. \\ [4pt]
$\mathtt{MacAdv} = \Pr[\mathtt{W_{I, \adv}}] $ -- выигрыш  $\adv$. \\ [4pt]
\color{Orange} Схема MAC $I$ безопасна, если для любого ppt $\adv:$  \[\mathtt{MacAdv} = \negl(\lambda).\]
\end{frame}


\begin{frame}{Конструкции MAC}
\Large
\begin{enumerate}
	\itemsep1em
	\item Блок-шифр  (AES, ГОСТ), псевдослучайная ф-ия $f$  -- примеры конструкции MAC для 16-байтных сообщений
	\begin{align*}
			S(m) &:= f(k, m) \rightarrow t \\
			V(k, m, t) & :=f(k, m) == t \; ? \;  1 : 0 
	\end{align*}

	
	\item Для более длинных сообщений:  \\[10pt]
		-- CBC-MAC (целостность банковских транзакций) \\ 
		Стандарты: ANSI, FIPS 186-3, ГОСТ \\[5pt]
		-- NMAC (Nested MAC) \\[5pt]
		-- HMAC (SSL, IPSec)
\end{enumerate}

\end{frame}

\begin{frame}{ECBC-MAC }
\Large
 $m = (m_1, m_2, m_3, ...)$, $F-$ блок-шифр (AES, ГОСТ), $|m_i|$  --  128 бит
	\begin{figure}
		\includegraphics[width=0.75\textwidth]{CBC_MAC}
	\end{figure}
\vspace{-90pt}

\Large 
$V((k, k1), m)$ выполняет идентичные шаги.\\[10pt]

Отличия от CBC-mode:
\begin{itemize}
	\itemsep 5pt
	\item $k, k1$ - два {\color{Orange} разных} ключа для $F $
	\item Не используется IV
	\item Выходное значение \textbf{tag} может быть урезано (тем самым уменьшая безопасность схемы, стандарты ANSI X9.19, ISO/IEC 9797)
\end{itemize}

\end{frame}


\begin{frame}{Вложенный (Nested) MAC}
	\Large
	\vspace{-25pt}
	$m = (m_1, m_2, m_3, ...)$, $\textbf{F}-$ блок-шифр (AES, ГОСТ), $|m_i|$ -- $n$ бит, $k  \in \{0,1\}^\kappa$.
	\begin{figure}
		\includegraphics[width=0.75\textwidth]{NMAC}
	\end{figure}
\vspace{-40pt}
$
	\text{fpad} \in \{0,1\}^{n - \kappa} -\text{любое фиксированное значение}
$
\end{frame}

\begin{frame}{Набивка (padding)}
\Large
\begin{center}
Что если $\mathbf{m} = (m_1, m_2, ...)$ не кратно длине блока?
\end{center}
Пример {\color{Orange} плохой} набивки : добивать $0'$-ми.
\[m = (\star \star \star  ), \quad |m|< 128 \text{бит}\] 

Набивка такого $m$:
\[m_{\text{padded}}=(\star \star \star  \; 0 \ldots 0  ). \] 

Тогда для сообщения $m' = (m||0)$:
\[S(k,m_{\text{padded}}) = S(k, m')\]

\vspace{10pt}
\centering
Для $\mathbf{m} \neq \mathbf{m'}$, должно выполнятся $\mathbf{m}||\text{pad} \neq \mathbf{m'}||\text{pad}$. \\[5pt]
 Набивка должна быть функцией $1 \leftrightarrow 1$.
	
\end{frame}

\begin{frame}{Набивка (padding)}
	\Large
	\begin{enumerate}
		\itemsep1em
		\item {\color{Orange} ISO.} Набивка: $10\ldots0$. `1' означает начало набивки. \\
		Добавляется фиктивный блок, если $|m| < $ длина блока
		\item {\color{Orange} NIST, GOST}  
		\begin{figure}
			\captionsetup[subfigure]{labelformat=empty}
			\begin{subfigure}{.5\textwidth}
				\hspace{-30pt}
				\includegraphics[width=.95\textwidth]{CBC_Mac_Padding}
				\caption{\hspace{-30pt}  \Large $m_4$< длина блока}
			\end{subfigure}%
		 \begin{subfigure}{.5\textwidth}
		\hspace{-36pt}
			\includegraphics[width=.95\textwidth]{CBC_Mac_Padding_1}
			\caption{\hspace{-50pt}  \Large $m_4$= длина блока}
		\end{subfigure}%
		\end{figure}
		\vspace{1em}
		Преимущества: нет фиктивного блока, нет дополнительного вызова F.
		 
	\end{enumerate}
\end{frame}


\begin{frame}{Двухфакторная аутентификация}

\Large 
Двухфакторная аутентификация = пользователь предоставляет то, что он {\color{Orange} знает} (пароль) + то, чем он {\color{Orange} владеет} (смартфон) 

\[
I = (\KeyGen, S, V) - \text{ MAC }  
\]

\vspace{10pt}
%	\begin{center}
	\begin{tabular}{c c c c c}
		 Пользователь &  & &  & Сервер \\
		 \multirow{5}{*}{\includegraphics[scale=0.12]{Alice}} & & &
		 & \multirow{5}{*}{\includegraphics[scale=0.12]{Bob}}  \\
		&  $k \leftarrow \KeyGen()$ &  & &  \\
		&  &  $\xrightarrow{ \quad k \quad}$ & &  \\[10pt]
		&  $p = S(k, T)$ &  & &  \\
		& {\small $T$ -- текущая дата, время} & $\xrightarrow{ \quad p \quad}$   & &  \\
		& &  &  & $V(k, p, T)$ \\
	\end{tabular}

\vspace{10pt}
$p$ -- временной одноразовый пароль (timed one-time password).
%\end{center}


\end{frame}

\end{document}