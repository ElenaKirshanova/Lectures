\documentclass[usenames,dvipsnames,8pt,aspectratio=169]{beamer}
\usepackage{amsmath,amsfonts,amssymb}
\usepackage{mathtools}
\usepackage{etex} %for Windows
\usepackage[utf8]{inputenc}
\usepackage[english, russian]{babel} 
%\usepackage{microtype}			% Better interword spacing and additional kerning.
\usepackage{ellipsis}			% Adjusted space with \dots between two words.
\usepackage{graphicx}
\usepackage{pstricks}

\usepackage{xcolor}


\usepackage{changepage}

\usepackage{algorithm}
\usepackage{algpseudocode}
%\usepackage[]{algorithm2e}
%\usepackage{algorithmic}

%\usepackage{tcolorbox}


\usepackage{caption}
\usepackage{subcaption}
\usepackage[normalem]{ulem} %to strike out text
\usepackage{cancel}
%\usepackage{stackengine}


\usepackage{tikz}
\usetikzlibrary{tikzmark,calc}
\usetikzlibrary{positioning, backgrounds}
\usetikzlibrary{arrows, chains, matrix, scopes, patterns, shapes, fit}
\usetikzlibrary{mindmap,trees,shadows}
\usetikzlibrary{decorations.pathreplacing}
%\usetikzlibrary{crypto.symbols}

\usepackage{pgfplots}

\pgfmathdeclarefunction{gauss}{2}{%
	\pgfmathparse{1/(#2*sqrt(2*pi))*exp(-((x-#1)^2)/(2*#2^2))}%
}


\tikzset{
	invisible/.style={opacity=0},
	visible on/.style={alt={#1{}{invisible}}},
	alt/.code args={<#1>#2#3}{%
		\alt<#1>{\pgfkeysalso{#2}}{\pgfkeysalso{#3}} % \pgfkeysalso doesn't change the path
	},
}

\newcommand\strikeout[2][]{%
	\begin{tabular}[b]{@{}c@{}} 
		\makebox(0,0)[cb]{{#1}} \\[-0.2\normalbaselineskip]
		\rlap{\color{Orange}\rule[0.5ex]{\widthof{#2}}{1.5pt}}#2
\end{tabular}}

\newcommand\Fontvi{\fontsize{11}{13.2}\selectfont}

\usepackage{listings} % for C++ code

\usepackage{braket}
%\usepackage[braket, qm]{qcircuit}



\usepackage[T1]{fontenc}
%\usepackage[sfdefault,scaled=.85]{FiraSans}
%\usepackage{newtxsf}
%\usepackage[nomap]{FiraMono}





\usefonttheme[onlymath]{serif}
\renewcommand\sfdefault{cmbr}

\renewcommand{\bfdefault}{sb}

\definecolor{CharCoalDark}{RGB}{13, 16, 19}
\definecolor{Orange}{RGB}{255, 165,0}
\definecolor{DarkOrange}{RGB}{255, 165,0}
\definecolor{LightSalmon}{RGB}{255, 160, 122}
\definecolor{LeafGreen}{RGB}{34, 139,  34}
\definecolor{Coral}{RGB}{255, 127, 80}
\definecolor{DarkTurquoise}{RGB}{0, 206, 209}

%\newtheorem{defRus}{Определение}
%\newtheorem{thmRus}{Теорема}
%s\newtheorem{corRus}{Следствие}


\setbeamercolor{background canvas}{bg=CharCoalDark}

\setbeamerfont{title}{series=\bfseries}
\setbeamercolor{title}{fg=Orange}
\setbeamercolor{section in toc}{fg=white}
\setbeamercolor{frametitle}{fg=Orange}
\setbeamercolor{normal text}{fg=white}
%\setbeamercolor{normal text}{fontsize=12pt}
\setbeamercolor{itemize item}{fg=Orange}
\setbeamercolor{enumerate item}{fg=Orange}
\setbeamercolor{enumerate item item}{fg=Orange}
\setbeamercolor{itemize item item}{fg=Orange}
\setbeamercolor{enumerate item}{fg=Orange}
\setbeamercolor{block title}{bg=DarkOrange,fg=white}
\setbeamerfont{block title}{series=\bfseries}

\setbeamertemplate{itemize item}[circle]
\setbeamertemplate{eumerate subitem}{\color{Orange}[$\checkmark$]}
\setbeamertemplate{itemize subitem}{\color{Orange}\Large$\textbullet$}
\setbeamertemplate{itemize subitem}{\color{Orange} \tiny $\blacksquare$}

% footnote without a marker
\newcommand\blfootnote[1]{%
	\begingroup
	\renewcommand\footnoterule{}
	\renewcommand\thefootnote{}\footnote{#1}%
	\addtocounter{footnote}{-1}%
	\endgroup
}

\newcommand*{\Scale}[2][4]{\scalebox{#1}{\ensuremath{#2}}}%

\newcommand\Item[1][]{%
	\ifx\relax#1\relax  \item \else \item[#1] \fi
	\abovedisplayskip=0pt\abovedisplayshortskip=0pt~\vspace*{-\baselineskip}}

\pgfdeclareradialshading{ring}{\pgfpoint{0cm}{0cm}}%
{rgb(0cm)=(1,1,1);
	rgb(0.7cm)=(1,1,1);
	rgb(0.719cm)=(1,1,1);
	rgb(0.72cm)=(0.975,0,0);
	rgb(0.9cm)=(1,1,1)}

\usepackage[absolute,overlay]{textpos} %to clip to a corner
\newcommand\FrameText[1]{%
	\begin{textblock*}{\paperwidth}(\textwidth-35pt, 10 pt)
		\raggedright #1\hspace{.5em}
\end{textblock*}}

\makeatletter
\let\save@measuring@true\measuring@true
\def\measuring@true{%
	\save@measuring@true
	\def\beamer@sortzero##1{\beamer@ifnextcharospec{\beamer@sortzeroread{##1}}{}}%
	\def\beamer@sortzeroread##1<##2>{}%
	\def\beamer@finalnospec{}%
}
\makeatother

\AtBeginSection[]
{
	\begin{frame}<beamer>
		\frametitle{Outline}
		\tableofcontents[currentsection]
	\end{frame}
}

\addtobeamertemplate{footline}{%
	\setlength\unitlength{1ex}%
	\begin{picture}(0,0) 
	% \put{} defines the position of the frame
	\put(155,0){\makebox(0,0)[bl]{
			%\includegraphics[scale=0.65]{white_square}
			%\includegraphics[scale=0.65]{dark_square}
			\includegraphics[scale=0.65]{grey_circle}
	}}%
	\end{picture}%
}{}


\newcommand{\AxisRotator}[1][rotate=0]{%
	\tikz [x=0.4cm,y=1.0cm,line width=.2ex,-stealth,#1] \draw[color=Orange] (0,0) arc (-150:150:2 and 1);%
}

\title{Лекция №8 \\[10pt]
	Часть 3. Подпись Шнорра }

\date{ Елена Киршанова \\  \textbf{Курс ``Основы криптографии''} \\  }



\setbeamertemplate{navigation symbols}{} %removes navigation

% proper highlightling of a code-snippet
\lstset{language=C++,
	keywordstyle=\color{magenta},
	stringstyle=\color{Goldenrod},
	commentstyle=\color{gray},
	breaklines=false,
	%morecomment=[l][\color{magenta}]{\#}
}

%\setlength{\parskip}{8pt}
\input{header} %all defs
\begin{document}
	
\begin{frame}
	\titlepage
\end{frame}

\begin{frame}{Подпись Шнорра}
	\Large
	\begin{itemize}
		\itemsep 10pt
		\item основана на задаче дискретного логарифма в $\Z_p$
		\item более компактные ключи и подписи, чем в RSA
		\item лежит в основе стандартов ECDSA, ГОСТ Р 34.10-2012
		\item конструкция подписи Шнорра строится на безопасной схеме идентификации
	\end{itemize}
\end{frame}

\begin{frame}{Парадигмы }
	\LARGE
	
	\begin{itemize}
		\itemsep 10pt
		\item {\color{Orange} Hash-and-Sign.} Пример: RSA \\[5pt]
			\begin{itemize}
				\itemsep 5pt
				\Large
				\item $\Pi = (\KeyGen, \Sign, \Ver)$ -- схема подписи
				\item $\Hash$ -- криптографическая хэш-функция
				\item Подпись для сообщение $m$:  $\Sign(\Hash(m))$ 
			\end{itemize}
		\item {\color{Orange} Эвристика Фиат-Шамира. } Пример: подпись Шнорра
			\[
		\hspace{-40pt}	\text{Док-во с нулевым разглашением}  + { \Hash  случайный оракул } \implies \text{ Схема подписи}
			\]
			\item {\color{Orange} Подписи на хэш-функциях.} Пример: дерево Меркля.
	\end{itemize}
\end{frame}

\begin{frame}{Доказательство с нулевым разглашением о знании dlog}
\Large
%\vspace{10pt}
\begin{center}
	Пусть $g$ -- образующий $\Z_p^\ast$ \\[3pt]
	{\color{Orange} Цель:} $\mathcal{P}$ должен доказать $\mathcal{V}$, что он знает $x$ (= dlog $g^x$), не выдавая значение $x$.\\[20pt] 
	
	\begin{tabular}{c c c}
		{\color{Orange} Доказывающий $\mathcal{P}$ } & & {\color{Orange} Проверяющий $\mathcal{V}$ }\\ [5pt]
		$x, g^x$ & & $g^x$\\ [2pt]
		$k\xleftarrow{\$} \Z_p^\ast$& $\xrightarrow{\; g^k \;} $  &\\ 
		$\sigma =  \Sign(m, \sk)$ &$\xleftarrow{ \; c \; }$  &$c\xleftarrow{\$} \Z_p^\ast$\\[3pt]
		& $\xrightarrow{k+x\cdot c} $& \\ [5pt]
		& & $g^{k+x\cdot c}  == (g^{k}) (g^{x})^c$\\ [5pt]
	\end{tabular}
	\begin{tikzpicture}[overlay]
	\draw[fill=none, draw=white, opacity=0.5] (-9.0,-2.0) rectangle (-5.3,2.5); 
	\draw[fill=none, draw=white, opacity=0.5] (-3.8,-2.) rectangle (0.0,2.5); 
	%\node at (-3.8, 0.3) {\AxisRotator};
	\end{tikzpicture}
\end{center}


\end{frame}


\begin{frame}{Подпись Шнорра: конструкция}
\Large 
 Пусть $g$ -- образующий $\Z_p^\ast$ \\[3pt]
$\Hash: \{0,1\}^\ast \rightarrow \Z_p^\ast$ -- криптографическая хэш-функция
\vspace{10pt}
\begin{columns}[T]
	\begin{column}{0.45\textwidth}
		{\color{Orange} I. $\mathsf{\KeyGen}(1^\lambda):$}
		\begin{enumerate}
			\itemsep4pt
			\item $\sk = x \xleftarrow{\$} \Z_p^\ast$
			\item $\vk = g^x$
		\end{enumerate}
		\vspace{10pt}
		{\color{Orange} II. $\mathsf{Sign}(\sk, m):$}
		\begin{enumerate}
			\itemsep4pt
			\item $ k \xleftarrow{\$} \Z_p^\ast$
			\item $r = g^k$ \\
			\item $c = \Hash(\vk, r, m)$
			\item $\mathtt{return} \; \sigma = (k+ cx, c)$
		\end{enumerate}
	\end{column}
	\begin{column}{0.45\textwidth}
		{\color{Orange} III. $\mathsf{Ver}(\vk, m, \sigma = (s, c)):$}
		\begin{enumerate}
			\itemsep4pt
			\item $r' = g^s /\vk^c $
			\item $\mathtt{return}(r'==\Hash(\vk, r', m))$ \\
		\end{enumerate}
	\end{column}
	\end{columns}
\end{frame}

\begin{frame}{Подпись Шнорра: безопасность (скетч)}
\Large
\vspace{-5pt}
{\color{Orange}  Теорема:} Подпись  Шнорра  безопасна в  модели {\color{Orange} UF-CMA}, при условии {\color{Orange} сложности задачи dlog} и $\Hash$ -- {\color{Orange}  случайный оракул.} \\[10pt]


\begin{center}
	\begin{tabular}{c c c}
		{\color{Orange} Челленджер $\mathcal{C}$ } & & {\color{Orange} Атакующий $\mathcal{A}$ }\\ [5pt]
		$(\vk, \sk)\leftarrow \mathsf{\KeyGen(1^\lambda)}$ & & \\[2pt]
		\pause
		{\color{Orange}$\vk = g^x$ }& $\xrightarrow{\vk }$ &   \\ [4pt]
		$\sigma ( s = k+cx , c)\leftarrow \mathsf{\Sign}$ & $\xleftarrow{m} $  & \\[2pt] 
		{\color{Orange}$s, c \xleftarrow{\$} \Z_p^\ast $}& & \\
		\pause
		{\color{Orange}$\Hash(\vk, g^s / \vk^c, m) =: c$}&$\xrightarrow{\sigma = (s,c) }$ & \\
		\pause
		&  $\xleftarrow{(m_1 , \sigma_1 = (s_1, c_1)) }$ & \\ [5pt]
	\end{tabular}
	%\begin{tikzpicture}[overlay]
	%\draw[fill=none, draw=white, opacity=0.5] (-9.5,-2.0) rectangle (-4.7,2.0); 
	%\draw[fill=none, draw=white, opacity=0.5] (-3.2,-2.) rectangle (0.0,2.0); 
	%\end{tikzpicture}
	
	\pause
	 $\Ver(m_1, \sigma_1, \vk) = \mathsf{accept} \implies$ $\adv$ должен был запросить \\ $\Hash(\vk, g^{s_1}/ \vk^{c_1}, m_1)$. Иначе $\adv$ смог обратить $\Hash$.
\end{center}

\end{frame}


\begin{frame}{Подпись Шнорра: безопасность (скетч)}
\Large
\vspace{-5pt}
{\color{Orange}  Теорема:} Подпись  Шнорра  безопасна в  модели {\color{Orange} UF-CMA}, при условии {\color{Orange} сложности задачи dlog} и $\Hash$ -- {\color{Orange}  случайный оракул.} \\[10pt]


\begin{center}
	\begin{tabular}{c c c}
		{\color{Orange} Челленджер $\mathcal{C}$ } & & {\color{Orange} Атакующий $\mathcal{A}$ }\\ [5pt]
		$(\vk, \sk)\leftarrow \mathsf{\KeyGen(1^\lambda)}$ & & \\[2pt]
		{\color{Orange}$\vk = g^x$ }& $\xrightarrow{\vk }$ &   \\ [4pt]
		$\sigma ( s = k+cx , c)\leftarrow \mathsf{\Sign}$ & $\xleftarrow{m} $  & \\[2pt] 
		{\color{Orange}$s, c \xleftarrow{\$} \Z_p^\ast $}& & \\
		{\color{Orange}$\Hash(\vk, g^s / \vk^c, m) =: c$}&$\xrightarrow{\sigma = (s,c) }$ & \\
		&  $\xleftarrow{(m_1 , \sigma_1 = (s_1, c_1)) }$ & \\ [5pt]
		\multicolumn{3}{c}{{\color{Orange} \hspace{20pt} RESTART (a.k.a. Rewinding)}} \\[5pt]
		& $\xleftarrow{(m_2 , \sigma_2) }$ & \\ [5pt]
	\end{tabular}
	%\begin{tikzpicture}[overlay]
	%\draw[fill=none, draw=white, opacity=0.5] (-9.5,-2.0) rectangle (-4.7,2.0); 
	%\draw[fill=none, draw=white, opacity=0.5] (-3.2,-2.) rectangle (0.0,2.0); 
	%\end{tikzpicture}
\end{center}


\end{frame}

%\begin{frame}{Подпись Шнорра: безопасность}
%\Large
%
%{\color{Orange}  Теорема:} Схема подписи  Шнорра  безопасна в  модели {\color{Orange} UF-CMA}, если {\color{Orange}  предположение сложности задачи dlog корректно} и $\Hash$ является \\ {\color{Orange}  случайным оракулом.} \\[10pt]
%
%
%\vfill
%\LARGE
%{\color{Orange} Случайное значение $k \xleftarrow{\$} \Z_p^\ast$ в алгоритме $\mathsf{Sign}(\sk, m)$ обязано быть выбрано заново для нового $m$.}

%
%\begin{center}
%	\begin{tabular}{c c c}
%		{\color{Orange} Челленджер $\mathcal{C}$ } & & {\color{Orange} Атакующий $\mathcal{A}$ }\\ [5pt]
%		$\vk = g^x$ & $\xrightarrow{\vk }$& \\ [2pt]
%		& $\xleftarrow{m} $  &\\ 
%		&$\xrightarrow{\sigma }$  & \\
%		& $\xleftarrow{(m' , \sigma' ) }$ & \\ [5pt]
%	\end{tabular}
%	\begin{tikzpicture}[overlay]
%	\draw[fill=none, draw=white, opacity=0.5] (-9.5,-2.0) rectangle (-4.7,2.0); 
%	\draw[fill=none, draw=white, opacity=0.5] (-3.2,-2.) rectangle (0.0,2.0); 
%	\end{tikzpicture}
%\end{center}


%\end{frame}


%
%\begin{frame}{Usages of signatures schemes}
%	\Large
%	Practical signature schemes 
%	\begin{enumerate}
%		\itemsep 10pt
%		\item RSA \\
%		long keys, signatures; fast verification
%		\item ECDSA, GOST\\
%		short keys, signatures; slower verification
%	\end{enumerate}
%\vspace{10pt}
%RSA is good for {\color{Orange} Certificates}, ECDSA/GOST are good for e-mails. \\[10pt]
%
%Certificates  bind a public key to an identity.
%\end{frame}
%
%\begin{frame}{Certificates and PKI}
%\begin{center}
%	\begin{tabular}{l c c c l}
%		& \Large Alice & &\Large Certificate Authority (CA)&  \\ 
%		&  $\pk_{\texttt{A}}$  & & $\vk_{\texttt{CA}}$, $\sk_{\texttt{CA}}$ & \\
%		& \multirow{5}{*}{\includegraphics[scale=0.15]{Alice}} & & &  \\  \pause
%		& & $\xrightarrow[\texttt{id, email, } \pk_{\texttt{A}}]{\text {Certificate Signing Request}}$ & & \\  \pause
%		& & & Certifying Alice’s identity & \\
%		& & & Creating a certificate $\texttt{cert}$& \\
%		& & & Signing $\texttt{cert}$& \\
%		& & $\xleftarrow{\LARGE \texttt{cert}, \; \sigma = \Sign(\sk_{\texttt{CA}}, \texttt{cert})}$ & & \\  \pause
%	\end{tabular}
%\end{center}
%
%\vspace{15pt}
%\Large
%Anyone, who needs to communicate securely with Alice, first runs $\Ver(\vk_{\texttt{CA}} \texttt{cert},\sigma )$. If verification passes, $\pk_{\texttt{A}}$ can be used to communicate with Alice.\\[10pt]
%
%Example: X.509 certificate
%\end{frame}
%
%\begin{frame}{Certificate chains}
%	\Large
%	\begin{center}
%	
%	\begin{align*}
%			&{\LARGE \text{Root CAs } \quad \quad\xrightarrow{\hspace{2em}} \quad \quad \text{Intermediate CA}} \\
%			&\text{certifies Interm.\ CAs}  \hspace{46pt} \text{certifies clients} 
%	\end{align*}
%	\end{center}
%	\vspace{20pt}
%	There are currently thousands of intermediate CAs operating on the Internet \\[10pt]
%	To avoid malicious CAs: {\color{Orange} certificate pinning:} \\[7pt]
%	\begin{enumerate}
%		\item Every browser maintains a pinning database: \\
%		($\texttt{domain}$, $\texttt{hash}_0$, $\texttt{hash}_1, \ldots$ )
%		\item The data for each record is provided by the domain owner
%		\item 	When the browser connects to a domain, domain sends its certificate chain $\texttt{cert}_0, 
%		\texttt{cert}_1, \ldots$
%		\item The browser computes $\Hash(\texttt{cert}_i)$ and verifies against $\texttt{hash}_i$. 
%	\end{enumerate}
%	
%\end{frame}


\end{document}
