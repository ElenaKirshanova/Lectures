\documentclass[usenames,dvipsnames,8pt,aspectratio=169]{beamer}
\usepackage{amsmath,amsfonts,amssymb}
\usepackage{mathtools}
\usepackage{etex} %for Windows
\usepackage[utf8]{inputenc}
\usepackage[english, russian]{babel} 
%\usepackage{microtype}			% Better interword spacing and additional kerning.
\usepackage{ellipsis}			% Adjusted space with \dots between two words.
\usepackage{graphicx}
\usepackage{pstricks}

\usepackage{xcolor}


\usepackage{changepage}

\usepackage{algorithm}
\usepackage{algpseudocode}
%\usepackage[]{algorithm2e}
%\usepackage{algorithmic}

%\usepackage{tcolorbox}


\usepackage{caption}
\usepackage{subcaption}
%\usepackage{stackengine}


\usepackage{tikz}
\usetikzlibrary{tikzmark,calc}
\usetikzlibrary{positioning, backgrounds}
\usetikzlibrary{arrows, chains, matrix, scopes, patterns, shapes, fit}
\usetikzlibrary{mindmap,trees,shadows}
\usetikzlibrary{decorations.pathreplacing}
%\usetikzlibrary{crypto.symbols}

\usepackage{pgfplots}

\pgfmathdeclarefunction{gauss}{2}{%
	\pgfmathparse{1/(#2*sqrt(2*pi))*exp(-((x-#1)^2)/(2*#2^2))}%
}


\tikzset{
	invisible/.style={opacity=0},
	visible on/.style={alt={#1{}{invisible}}},
	alt/.code args={<#1>#2#3}{%
		\alt<#1>{\pgfkeysalso{#2}}{\pgfkeysalso{#3}} % \pgfkeysalso doesn't change the path
	},
}

\newcommand\strikeout[2][]{%
	\begin{tabular}[b]{@{}c@{}} 
		\makebox(0,0)[cb]{{#1}} \\[-0.2\normalbaselineskip]
		\rlap{\color{Orange}\rule[0.5ex]{\widthof{#2}}{1.5pt}}#2
\end{tabular}}

\newcommand\Fontvi{\fontsize{11}{13.2}\selectfont}

\usepackage{listings} % for C++ code

\usepackage{braket}
%\usepackage[braket, qm]{qcircuit}



\usepackage[T1]{fontenc}
%\usepackage[sfdefault,scaled=.85]{FiraSans}
%\usepackage{newtxsf}
%\usepackage[nomap]{FiraMono}





\usefonttheme[onlymath]{serif}
\renewcommand\sfdefault{cmbr}

\renewcommand{\bfdefault}{sb}

\definecolor{CharCoalDark}{RGB}{13, 16, 19}
\definecolor{Orange}{RGB}{255, 165,0}
\definecolor{DarkOrange}{RGB}{255, 165,0}
\definecolor{LightSalmon}{RGB}{255, 160, 122}
\definecolor{LeafGreen}{RGB}{34, 139,  34}
\definecolor{Coral}{RGB}{255, 127, 80}
\definecolor{DarkTurquoise}{RGB}{0, 206, 209}

%\newtheorem{defRus}{Определение}
%\newtheorem{thmRus}{Теорема}
%s\newtheorem{corRus}{Следствие}


\setbeamercolor{background canvas}{bg=CharCoalDark}

\setbeamerfont{title}{series=\bfseries}
\setbeamercolor{title}{fg=Orange}
\setbeamercolor{section in toc}{fg=white}
\setbeamercolor{frametitle}{fg=Orange}
\setbeamercolor{normal text}{fg=white}
%\setbeamercolor{normal text}{fontsize=12pt}
\setbeamercolor{itemize item}{fg=Orange}
\setbeamercolor{enumerate item}{fg=Orange}
\setbeamercolor{enumerate item item}{fg=Orange}
\setbeamercolor{itemize item item}{fg=Orange}
\setbeamercolor{enumerate item}{fg=Orange}
\setbeamercolor{block title}{bg=DarkOrange,fg=white}
\setbeamerfont{block title}{series=\bfseries}

\setbeamertemplate{itemize item}[circle]
\setbeamertemplate{eumerate subitem}{\color{Orange}[$\checkmark$]}
\setbeamertemplate{itemize subitem}{\color{Orange}\Large$\textbullet$}
\setbeamertemplate{itemize subitem}{\color{Orange} \tiny $\blacksquare$}

% footnote without a marker
\newcommand\blfootnote[1]{%
	\begingroup
	\renewcommand\footnoterule{}
	\renewcommand\thefootnote{}\footnote{#1}%
	\addtocounter{footnote}{-1}%
	\endgroup
}

\newcommand*{\Scale}[2][4]{\scalebox{#1}{\ensuremath{#2}}}%

\newcommand\Item[1][]{%
	\ifx\relax#1\relax  \item \else \item[#1] \fi
	\abovedisplayskip=0pt\abovedisplayshortskip=0pt~\vspace*{-\baselineskip}}

\pgfdeclareradialshading{ring}{\pgfpoint{0cm}{0cm}}%
{rgb(0cm)=(1,1,1);
	rgb(0.7cm)=(1,1,1);
	rgb(0.719cm)=(1,1,1);
	rgb(0.72cm)=(0.975,0,0);
	rgb(0.9cm)=(1,1,1)}

\usepackage[absolute,overlay]{textpos} %to clip to a corner
\newcommand\FrameText[1]{%
	\begin{textblock*}{\paperwidth}(\textwidth-35pt, 10 pt)
		\raggedright #1\hspace{.5em}
\end{textblock*}}

\makeatletter
\let\save@measuring@true\measuring@true
\def\measuring@true{%
	\save@measuring@true
	\def\beamer@sortzero##1{\beamer@ifnextcharospec{\beamer@sortzeroread{##1}}{}}%
	\def\beamer@sortzeroread##1<##2>{}%
	\def\beamer@finalnospec{}%
}
\makeatother

\AtBeginSection[]
{
	\begin{frame}<beamer>
		\frametitle{Outline}
		\tableofcontents[currentsection]
	\end{frame}
}

\addtobeamertemplate{footline}{%
	\setlength\unitlength{1ex}%
	\begin{picture}(0,0) 
	% \put{} defines the position of the frame
	\put(155,0){\makebox(0,0)[bl]{
			%\includegraphics[scale=0.65]{white_square}
			%\includegraphics[scale=0.65]{dark_square}
			\includegraphics[scale=0.65]{grey_circle}
	}}%
	\end{picture}%
}{}


\newcommand{\AxisRotator}[1][rotate=0]{%
	\tikz [x=0.4cm,y=1.0cm,line width=.2ex,-stealth,#1] \draw[color=Orange] (0,0) arc (-150:150:2 and 1);%
}

\title{Лекция №8 \\[10pt]
	Часть 2. Арифметика в кольце целых чисел. Подпись RSA }

\date{ Елена Киршанова \\  \textbf{Курс ``Основы криптографии''} \\  }



\setbeamertemplate{navigation symbols}{} %removes navigation

% proper highlightling of a code-snippet
\lstset{language=C++,
	keywordstyle=\color{magenta},
	stringstyle=\color{Goldenrod},
	commentstyle=\color{gray},
	breaklines=false,
	%morecomment=[l][\color{magenta}]{\#}
}

%\setlength{\parskip}{8pt}
\input{header} %all defs
\begin{document}
	
\begin{frame}
	\titlepage
\end{frame}


\begin{frame}{Арифметика в кольце целых}
\Large 
{\color{Orange} Положим $N = p \cdot q$, где $p, q$ -- большие простые числа}

\begin{itemize}
	\item $\ZN = \left\{ 0, 1, \ldots, N-1 \right\}$ -- {\color{Orange} кольцо}
	\item Элементы $\ZN$ складываются и умножаются по модулю $N$. \\
	Пример: $N = 15$
	\begin{align*}
	11+6 \bmod N &= \rem(17, 15) = 2 \\
	6\cdot 7  \bmod N &= \rem(42, 15) = 12
	\end{align*}
	
	\item Не для всякого ненулевого $x \in \ZN$ существует обратимый! \\[3pt]
	Множество обратимых элементов: $\ZN^{\ast} =\{x \in \ZN \; | \;  \text{НОД}(x,N) = 1\}$.\\[5pt]

	Пример: $3, 6, 9, 5, 10, 12 \notin \ZN^{\ast}$. \\[3pt]
	$\ZN^{\ast} = \{1, 2, 4, 7, 8, 11, 13, 14\}$.
\end{itemize}

\end{frame}

\begin{frame}{Структура $\ZN^\ast$ }
\Large 
\begin{itemize}
	\itemsep 10pt
	\item  $\phi(N) = |\ZN^\ast|$ -- {\color{Orange}функция  Эйлера} \\[5pt]
	--  $N$ -- prime, $\phi(N) = N-1$ \\[3pt]
	--  $N = p_1^{e_1} \cdot p_n^{e_n}$, $\phi(N)=  N \cdot \prod_{i} \left(1 - \frac{1}{p_i}\right)$. \\[3pt]
	-- for $N = p \cdot q$, {\color{Orange}$\phi(N) = (p-1)(q-1)$.} \\[3pt]
	Пример: $|\ZN^{\ast}| = |\{1, 2, 4, 7, 8, 11, 13, 14\}| = 2 \cdot 4 = 8$.
	
	\item {\color{Orange}Теорема Эйлера:} для всех $a \in \ZN^{\ast}$
	\[
	{\color{Orange}	\LARGE a^{\phi(n)} = 1 \bmod N }
	\]
	{\large Теорема Ферма: $a^{p-1} = 1 \bmod p$ для простого $p$.}
\end{itemize}
\end{frame}

\begin{frame}{Простые и трудные задачи в $\ZN$}
\Large 
\begin{center}
{\color{Orange}  $N = p \cdot q$},  $p, q$ -- по $\approx$ 1024 бит каждое.\\
\end{center}
	В $\ZN$ следующие операции  {\color{Orange} эффективные}  \\[5pt]
	-- сложение, умножение, нахождение обратного (если существует, проверить несуществование)\\[3pt]
	--  возведение в степерь $g^r \bmod N$ \\[14pt]
	Сегодня  {\color{Orange} считаются трудными} задачи \\[5pt]
	 -- {\color{Orange} факторизация}: нахождение $p, q$  \\[3pt]
	 -- вычисление квадратного корня в $\ZN$ (эквивалентно факторизации) \\[3pt]
	 -- вычисление $e$-го корня в  $\ZN$ при $\gcd(e, \phi(N)) = 1$\\
	 %-- dlog (TODO)
\end{frame}

\begin{frame}{Генерация ключей в подписи RSA}
\Large
Возьмём $\ell>2$--целое и $e>2$ -- нечетное целое (на практике $e=3$  или $e=65537$) \\[8pt]
{\color{Orange} $\mathsf{RSAGen}(\ell, e):$}
\begin{enumerate}
	\itemsep5pt
	\item Сгенерировать $\ell$-битное целое $p$ т.ч.\ $\gcd(p-1, e)=1$
	\item Сгенерировать  $\ell$-битное цело $q \neq p$ т.ч.\ $\gcd(q-1, e)=1$
	\item $N= p \cdot q$, $\phi(N) = (p-1)(q-1)$
	\item $d = e^{-1} \bmod \phi(N)$
	\item Вывод: $\vk = (N, e), \sk=(N, d)$
\end{enumerate}
%\vspace{7pt}
\large
\pause
\begin{itemize}
	\item  $\exists$ ppt алгоритм генерации простых чисел ($\exists$  ppt алгоритмы теста на простоту)
	\item Шаг 4 корректен:  $d \in \ZN^\ast$, т.к. \[ \hspace{-20pt} \gcd(p-1, e) = \gcd(q-1, e) = 1 \implies \gcd((p-1)(q-1), e)=1.\]
	\item на $p,q$ должны удовлетворять  {\color{Orange} большому числу условий} 
\end{itemize}
\centering
\Large Не пытайтесь повторить $\mathsf{RSAGen}$ самостоятельно. 
\end{frame}

\begin{frame}{RSA Signature Generation and Verification}
\large
$\Hash: \{0,1\}^\ast \rightarrow \ZN^\ast$-- криптографическая хэш-функция
\vspace{10pt}
\begin{columns}[t]
	\begin{column}{0.45\textwidth}
		
{\color{Orange} I. $\mathsf{RSASign}(\sk=(N, d), m):$}
\begin{enumerate}
	\itemsep5pt
	\item $y = \Hash(m) \in \ZN^\ast$
	\item $\sigma = y^d \bmod N$
\end{enumerate}
\vspace{10pt}
\pause
{\color{Orange} II. $\mathsf{RSAVerify}(\vk=(N, e), m, \sigma):$}
\begin{enumerate}
	\itemsep5pt
	\item $y' = \sigma^e \bmod N$
	\item $\mathtt{return}(y'==\Hash(m))$ \\
\end{enumerate}
	\end{column}
	\begin{column}{0.55\textwidth}
		\pause
		{\color{Orange} Корректность:}
		Для $N=pq$ и $e,d$ т.ч.\ $ed = 1 \bmod \phi(N)$ и для всех $x \in \Z$
		{\color{Orange} 
		\[
			x^{ed} = x \bmod N
		\] }
	\pause
		Док-во: для $k \in \Z$
		\begin{align*}
		&ed = 1 + k \phi(N) = 1+k(p-1)(q-1) \\  \pause
		& x^{p-1} = x \bmod p \quad \text{(Ferma't thm.)}\\ \pause
		& x^{ed} = x^{1+k(p-1)(q-1)} = \\
		& x \cdot (x^{p-1})^{q-1} = x \bmod p \\ \pause
		&\text{Аналог.}, x^{ed}  = x \bmod q \\ \pause
		& \implies  p, q \, |\, x^{ed} - x \\
		& \implies   x^{ed} = x \bmod p \cdot q
		\end{align*}
	\end{column}
\end{columns}
\LARGE
%\vspace*{-40pt}
%{\color{Orange}  $(y^d)^e = y^{ed} = y \bmod N$ } \\[20pt]
\vspace{10pt}
\centering
\vfill
Без $\Hash$ схема тривиально взламывается!
\end{frame}

\begin{frame}{Безопасность подписи RSA}
\Large

{\color{Orange}  RSA предположение сложности:}
Не существует ppt алгоритма, который, получив на вход $(N, m, m^e)$ для случайного $m \in \ZN^\ast$, выдаёт $m$.\\[10pt]

{\centering Факторизация $N \implies $ вычисление $e$-го корня. \\[5pt] Обратная редукция не доказана! }  \\[15pt]

Самый быстрый сегодня алгоритм факторизации -- General Number Field sievie -- со сложностью
\[
	\sim e^{(\lg N)^{1/3}}
\]


{\color{Orange}  Теорема:} Схема подписи ($\mathsf{RSAGen},$ $\mathsf{RSASign},$ $\mathsf{RSAVerify}$) безопасна в \\ модели {\color{Orange} UF-CMA}, если {\color{Orange}  предположение RSA корректно} и $\Hash$ является \\ {\color{Orange}  случайным оракулом.} \\[10pt]
 
\end{frame}

\begin{frame}{Стандарты}
\Large 
Вариант RSA подписи станартизованы под именем PKCS (Public Key Cryptography Standards)   \\
\url{https://en.wikipedia.org/wiki/PKCS} \\[10pt]

Версия 2.2 (последняя)  PKCS \#1 включает \\[10pt]
\begin{itemize}
	\itemsep10pt
	\item RSASSA-PSS \\
	SSA = Signature Scheme with Appendix \\[5pt]
	PSS =  Probabilistic Signature Scheme
	\item  RSASSA-PKCS1-v1\_5 (существуеют атаки)
\end{itemize}

\end{frame}

%\begin{frame}{Usages of signatures schemes}
%	\Large
%	Practical signature schemes 
%	\begin{enumerate}
%		\itemsep 10pt
%		\item RSA \\
%		long keys, signatures; fast verification
%		\item ECDSA, GOST\\
%		short keys, signatures; slower verification
%	\end{enumerate}
%\vspace{10pt}
%RSA is good for {\color{Orange} Certificates}, ECDSA/GOST are good for e-mails. \\[10pt]
%
%Certificates  bind a public key to an identity.
%\end{frame}

%\begin{frame}{Certificates and PKI}
%\begin{center}
%	\begin{tabular}{l c c c l}
%		& \Large Alice & &\Large Certificate Authority (CA)&  \\ 
%		&  $\pk_{\texttt{A}}$  & & $\vk_{\texttt{CA}}$, $\sk_{\texttt{CA}}$ & \\
%		& \multirow{5}{*}{\includegraphics[scale=0.15]{Alice}} & & &  \\  \pause
%		& & $\xrightarrow[\texttt{id, email, } \pk_{\texttt{A}}]{\text {Certificate Signing Request}}$ & & \\  \pause
%		& & & Certifying Alice’s identity & \\
%		& & & Creating a certificate $\texttt{cert}$& \\
%		& & & Signing $\texttt{cert}$& \\
%		& & $\xleftarrow{\LARGE \texttt{cert}, \; \sigma = \Sign(\sk_{\texttt{CA}}, \texttt{cert})}$ & & \\  \pause
%	\end{tabular}
%\end{center}
%
%\vspace{15pt}
%\Large
%Anyone, who needs to communicate securely with Alice, first runs $\Ver(\vk_{\texttt{CA}} \texttt{cert},\sigma )$. If verification passes, $\pk_{\texttt{A}}$ can be used to communicate with Alice.\\[10pt]
%
%Example: X.509 certificate
%\end{frame}
%
%\begin{frame}{Certificate chains}
%	\Large
%	\begin{center}
%	
%	\begin{align*}
%			&{\LARGE \text{Root CAs } \quad \quad\xrightarrow{\hspace{2em}} \quad \quad \text{Intermediate CA}} \\
%			&\text{certifies Interm.\ CAs}  \hspace{46pt} \text{certifies clients} 
%	\end{align*}
%	\end{center}
%	\vspace{20pt}
%	There are currently thousands of intermediate CAs operating on the Internet \\[10pt]
%	To avoid malicious CAs: {\color{Orange} certificate pinning:} \\[7pt]
%	\begin{enumerate}
%		\item Every browser maintains a pinning database: \\
%		($\texttt{domain}$, $\texttt{hash}_0$, $\texttt{hash}_1, \ldots$ )
%		\item The data for each record is provided by the domain owner
%		\item 	When the browser connects to a domain, domain sends its certificate chain $\texttt{cert}_0, 
%		\texttt{cert}_1, \ldots$
%		\item The browser computes $\Hash(\texttt{cert}_i)$ and verifies against $\texttt{hash}_i$. 
%	\end{enumerate}
%	
%\end{frame}


\end{document}
