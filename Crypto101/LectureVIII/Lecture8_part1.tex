\documentclass[usenames,dvipsnames,8pt,aspectratio=169]{beamer}
\usepackage{amsmath,amsfonts,amssymb}
\usepackage{mathtools}
\usepackage{etex} %for Windows
\usepackage[utf8]{inputenc}
\usepackage[english, russian]{babel} 
%\usepackage{microtype}			% Better interword spacing and additional kerning.
\usepackage{ellipsis}			% Adjusted space with \dots between two words.
\usepackage{graphicx}
\usepackage{pstricks}

\usepackage{xcolor}


\usepackage{changepage}

\usepackage{algorithm}
\usepackage{algpseudocode}
%\usepackage[]{algorithm2e}
%\usepackage{algorithmic}

%\usepackage{tcolorbox}


\usepackage{caption}
\usepackage{subcaption}
%\usepackage{stackengine}


\usepackage{tikz}
\usetikzlibrary{tikzmark,calc}
\usetikzlibrary{positioning, backgrounds}
\usetikzlibrary{arrows, chains, matrix, scopes, patterns, shapes, fit}
\usetikzlibrary{mindmap,trees,shadows}
\usetikzlibrary{decorations.pathreplacing}
%\usetikzlibrary{crypto.symbols}

\usepackage{pgfplots}

\pgfmathdeclarefunction{gauss}{2}{%
	\pgfmathparse{1/(#2*sqrt(2*pi))*exp(-((x-#1)^2)/(2*#2^2))}%
}


\tikzset{
	invisible/.style={opacity=0},
	visible on/.style={alt={#1{}{invisible}}},
	alt/.code args={<#1>#2#3}{%
		\alt<#1>{\pgfkeysalso{#2}}{\pgfkeysalso{#3}} % \pgfkeysalso doesn't change the path
	},
}

\newcommand\strikeout[2][]{%
	\begin{tabular}[b]{@{}c@{}} 
		\makebox(0,0)[cb]{{#1}} \\[-0.2\normalbaselineskip]
		\rlap{\color{Orange}\rule[0.5ex]{\widthof{#2}}{1.5pt}}#2
\end{tabular}}

\newcommand\Fontvi{\fontsize{11}{13.2}\selectfont}

\usepackage{listings} % for C++ code

\usepackage{braket}
%\usepackage[braket, qm]{qcircuit}



\usepackage[T1]{fontenc}
%\usepackage[sfdefault,scaled=.85]{FiraSans}
%\usepackage{newtxsf}
%\usepackage[nomap]{FiraMono}





\usefonttheme[onlymath]{serif}
\renewcommand\sfdefault{cmbr}

\renewcommand{\bfdefault}{sb}

\definecolor{CharCoalDark}{RGB}{13, 16, 19}
\definecolor{Orange}{RGB}{255, 165,0}
\definecolor{DarkOrange}{RGB}{255, 165,0}
\definecolor{LightSalmon}{RGB}{255, 160, 122}
\definecolor{LeafGreen}{RGB}{34, 139,  34}
\definecolor{Coral}{RGB}{255, 127, 80}
\definecolor{DarkTurquoise}{RGB}{0, 206, 209}

%\newtheorem{defRus}{Определение}
%\newtheorem{thmRus}{Теорема}
%s\newtheorem{corRus}{Следствие}


\setbeamercolor{background canvas}{bg=CharCoalDark}

\setbeamerfont{title}{series=\bfseries}
\setbeamercolor{title}{fg=Orange}
\setbeamercolor{section in toc}{fg=white}
\setbeamercolor{frametitle}{fg=Orange}
\setbeamercolor{normal text}{fg=white}
%\setbeamercolor{normal text}{fontsize=12pt}
\setbeamercolor{itemize item}{fg=Orange}
\setbeamercolor{enumerate item}{fg=Orange}
\setbeamercolor{enumerate item item}{fg=Orange}
\setbeamercolor{itemize item item}{fg=Orange}
\setbeamercolor{enumerate item}{fg=Orange}
\setbeamercolor{block title}{bg=DarkOrange,fg=white}
\setbeamerfont{block title}{series=\bfseries}

\setbeamertemplate{itemize item}[circle]
\setbeamertemplate{eumerate subitem}{\color{Orange}[$\checkmark$]}
\setbeamertemplate{itemize subitem}{\color{Orange}\Large$\textbullet$}
\setbeamertemplate{itemize subitem}{\color{Orange} \tiny $\blacksquare$}

% footnote without a marker
\newcommand\blfootnote[1]{%
	\begingroup
	\renewcommand\footnoterule{}
	\renewcommand\thefootnote{}\footnote{#1}%
	\addtocounter{footnote}{-1}%
	\endgroup
}

\newcommand*{\Scale}[2][4]{\scalebox{#1}{\ensuremath{#2}}}%

\newcommand\Item[1][]{%
	\ifx\relax#1\relax  \item \else \item[#1] \fi
	\abovedisplayskip=0pt\abovedisplayshortskip=0pt~\vspace*{-\baselineskip}}

\pgfdeclareradialshading{ring}{\pgfpoint{0cm}{0cm}}%
{rgb(0cm)=(1,1,1);
	rgb(0.7cm)=(1,1,1);
	rgb(0.719cm)=(1,1,1);
	rgb(0.72cm)=(0.975,0,0);
	rgb(0.9cm)=(1,1,1)}

\usepackage[absolute,overlay]{textpos} %to clip to a corner
\newcommand\FrameText[1]{%
	\begin{textblock*}{\paperwidth}(\textwidth-35pt, 10 pt)
		\raggedright #1\hspace{.5em}
\end{textblock*}}

\makeatletter
\let\save@measuring@true\measuring@true
\def\measuring@true{%
	\save@measuring@true
	\def\beamer@sortzero##1{\beamer@ifnextcharospec{\beamer@sortzeroread{##1}}{}}%
	\def\beamer@sortzeroread##1<##2>{}%
	\def\beamer@finalnospec{}%
}
\makeatother

\AtBeginSection[]
{
	\begin{frame}<beamer>
		\frametitle{Outline}
		\tableofcontents[currentsection]
	\end{frame}
}

\addtobeamertemplate{footline}{%
	\setlength\unitlength{1ex}%
	\begin{picture}(0,0) 
	% \put{} defines the position of the frame
	\put(155,0){\makebox(0,0)[bl]{
			%\includegraphics[scale=0.65]{white_square}
			%\includegraphics[scale=0.65]{dark_square}
			\includegraphics[scale=0.65]{grey_circle}
	}}%
	\end{picture}%
}{}


\newcommand{\AxisRotator}[1][rotate=0]{%
	\tikz [x=0.4cm,y=1.0cm,line width=.2ex,-stealth,#1] \draw[color=Orange] (0,0) arc (-150:150:2 and 1);%
}

\title{Лекция №8 \\[10pt]
	Часть 1. Схема криптографической подписи. Определение. }

\date{ Елена Киршанова \\  \textbf{Курс ``Основы криптографии''} \\  }



\setbeamertemplate{navigation symbols}{} %removes navigation

% proper highlightling of a code-snippet
\lstset{language=C++,
	keywordstyle=\color{magenta},
	stringstyle=\color{Goldenrod},
	commentstyle=\color{gray},
	breaklines=false,
	%morecomment=[l][\color{magenta}]{\#}
}

%\setlength{\parskip}{8pt}
\input{header} %all defs
\begin{document}
	
\begin{frame}
	\titlepage
\end{frame}

\begin{frame}{Схема цифровой подписи: зачем нужна?}
	\large 
	\begin{center}
Обмен ключами + Симметрическое шифрование =  {\color{Orange}{конфидециальность} } 
	\end{center}
Но
	\begin{itemize}
		\itemsep 10pt
		\item протокол обмена ключами Диффи-Хэллмана{\color{Orange} подвержен активным атакам} 
		\item  он не обеспечивает {\color{Orange} целостность} передаваемых данных
		\item он не обеспечивает {\color{Orange} аутентификацию}
	\end{itemize}

\vspace{15pt}

Наша цель: обеспечить целостность и аутентификацию, как схема MAC,  но в асимметрическом  {\color{Orange} открытым} мире.\\[10pt]

%\centering

\Large Асимметрическая версия МАС'a -- {\color{Orange} криптогрфическая схема подписи}
\end{frame}



\begin{frame}{Схема подписи: определение}
\Large

{\color{Orange}{Схема подписи}} состоит из трех ppt алгоритмов
\begin{itemize}
	\itemsep 10pt
	\item Генерация ключа: $(\sk, \vk) \leftarrow \KeyGen(1^\lambda)$ \\
	$\vk$ -- ключ верификации (открытый), $\sk$ -- подписывающий ключ (секретный)
	\item Генерация подписи: $\sigma \leftarrow \Sign(m, \sk)$
	\item Верификация: $\Ver(m, \sigma, \vk)$ выдает $\{\mathsf{accept}, \mathsf{reject}  \}$.
\end{itemize}
\vspace{15pt}
Здесь,  $m \in \mesS$ -- сообщение, которое должно быть подписано\\[10pt]

{\color{Orange}Корректность}: $\forall m, \forall (\sk, \vk) \leftarrow \KeyGen():$ 
\[
	\Ver(m, \Sign(m, \sk), \vk)  = \mathsf{accept}
\]
\end{frame}

\begin{frame}{Схема подписи: безопасность}
	\Large
	2 типа атак {\color{Orange} на выбранное сообщение}: \\[10pt]
	{\color{Orange}I. Экзистенциальная подделка (Existential forgery)} \\
	\large 
		\begin{itemize}
			\item атакующий может запросить подписать любое сообщение $m$
			\item ppt атакующий не должен сформировать корректную пару $(m, \sigma)$ для нового $m$, т.е.,  для $m$, для которого он не запрашивал подпись
		\end{itemize}
	\Large 
	\pause
	{\color{Orange}II. Сильная экзистенциальная подделка (Strong Existential forgery)} \\
	\large 
	\begin{itemize}
		\item атакующий может запросить подписать любое сообщение $m$
		\item  ppt атакующий не должен сформировать корректную пару $(m, \sigma)$  {\color{Orange}  даже для сообщений, на которые была запрошена подпись}, i.e., $(m, \sigma')$ является \\ корректной атакой, даже если злоумышленник знает $(m, \sigma)$.
	\end{itemize}

\vspace{10pt}

Из цифровой подписи, безопасной в первой (слабой) модели, можно \\ сделать схему подписи, безопасную во второй, более сильной модели
\end{frame}

\begin{frame}{Безопасность UF-CMA (Unforgeability under Chosen Message Attack)}
\Large

\begin{center}
	$\Pi = (\KeyGen, \Sign, \Ver)$ --  схема подписи \\[20pt]
	
	\begin{tabular}{c c c}
		{\color{Orange} Челленджер $\mathcal{C}$ } & & {\color{Orange} Атакующий $\mathcal{A}$ }\\ [5pt]
		$(\vk, \sk ) \leftarrow \KeyGen(1^\lambda)$ & & $\mesS \gets \emptyset$\\ [2pt]
		 & $\xleftarrow{m} $  &\\ 
		$\sigma =  \Sign(m, \sk)$ &$\xrightarrow{\sigma }$  & $\mesS \gets \mesS \cup \{m \}$\\[10pt]
		& $\xleftarrow{(m' , \sigma' ) }$ & $m' \notin \mesS$ \\ [5pt]
	\end{tabular}
	\begin{tikzpicture}[overlay]
	\draw[fill=none, draw=white, opacity=0.5] (-9.5,-2.0) rectangle (-4.7,2.0); 
	\draw[fill=none, draw=white, opacity=0.5] (-3.2,-2.) rectangle (0.0,2.0); 
	\node at (-3.8, 0.3) {\AxisRotator};
	\end{tikzpicture}
\end{center}

\vspace{15pt}

$\mathtt{W_{\Pi, \adv}}$ -- событие $\Ver(\vk, \sigma') = \mathsf{accept}$. \\ [4pt]
$\mathtt{And} = \Pr[\mathtt{W_{\Pi, \adv}}] $ -- выигрыш  $\adv$. \\ [4pt]
\color{Orange} Схема подписи $\Pi$ безопасна в модели UF-CMA, если $\forall$ ppt $\adv:$  \[\mathtt{SigAdv} = \negl(\lambda).\]

\end{frame}

\begin{frame}{Еще о безопасности}
\Large
	\begin{itemize}
		\itemsep 15pt
		\item {\color{Orange} Hеотказ от авторства} (Non-repudiation) \\
		 -- Подписывающий ``привязан'' к своим подписям. \\
		 
		\item {\color{Orange} Стойкость относительно дубликатов подписывающих ключей (Duplicate Signature Key Selection,DSKS)}\\
		-- Атакующий, имея $(m, \sigma)$, может сгенерировать пару$(\vk', \sk')$, т.ч.\ $(m, \sigma)$ -- корректная пара относительно $(\vk', \sk')$.\\
		-- Для предотвращения такой атаки подписывающий конкатинирует сообщение со своим открытым ключом
	\end{itemize}
\end{frame}

\begin{frame}{Применение цифровой подписи: обновление софта}

\begin{figure}
	\hspace{-60pt}
	\includegraphics[width=0.9\textwidth]{Signature_software}
\end{figure}
\end{frame}

\begin{frame}{Схемы подписи на практике}
\Large
\begin{enumerate}
	\itemsep10pt
	\item {\color{Orange} RSA} \\
	-- основана на сложности задачи {\color{Orange} факторизации}\\
	-- быстрая процедура $\Ver$, медленные $\Sign, \KeyGen$; длинный ключи, подписи
	
	\item {\color{Orange} (EC)DSA= (Elliptic Curve) Digital Signature Algorithm} \\
	-- основана на сложности задачи  {\color{Orange} dlog}\\
	-- медленнее $\Ver$, быстрее $\Sign, \KeyGen$\\
	-- в ECDSA меньшего размера ключи и подписи
	\pause
	\item {\color{Orange} ГОСТ Р 34.10-2012 }  \\
	-- аналогичен ECDSA \\
	-- старый ГОСТ Р 34.10-94 аналогичен DSA
\end{enumerate}
\pause
\centering

Размеры ключей (в битах):\\[5pt]
\begin{tabular}{c | c| c}
	Security lvl. & ECDSA / ГОСТ'12 & RSA/DSA \\ \hline
	80 & 160 & 1024 \\
	128 & 256 & 3072 \\
	256 & 512 & 15360
\end{tabular}
\end{frame}

%\begin{frame}{Math crash course I: arithmetic in a ring}
%\Large 
%{\color{Orange} Let $N = p \cdot q$, where $p, q$ are large primes}
%
%\begin{itemize}
%	\item $\ZN = \left\{ 0, 1, \ldots, n-1 \right\}$ -- {\color{Orange} ring}
%	\item elements in $\ZN$ are added and multiplied modulo $N$, i.e., for $x, y \in \ZN$
%	Ex.: $N = 15$
%	\begin{align*}
%	11+6 \bmod N &= \rem(17, 15) = 2 \\
%	6\cdot 7  \bmod N &= \rem(42, 15) = 12
%	\end{align*}
%	\pause
%	\item Not every non-zero $x \in \ZN$ has inverse!
%	The set of invertible elements is denoted $\ZN^{\ast} =\{x \in \ZN \; | \;  \gcd(x,N) == 1\}$.\\
%	{\large $\gcd $ -- greatest common divisor (НОД)}.\\[5pt]
%	Ex.: $3, 6, 9, 5, 10, 12 \notin \ZN^{\ast}$. \\
%	$\ZN^{\ast} = \{1, 2, 4, 7, 8, 11, 13, 14\}$.
%\end{itemize}
%
%\end{frame}
%
%\begin{frame}{Math crash course I: structure of $\ZN^\ast$ }
%\Large 
%\begin{itemize}
%	\itemsep 10pt
%	\item Denote $\phi(N) = |\ZN^\ast|$. $\phi(N)$ is known as {\color{Orange}Euler function} \\
%	-- if $N$ -- prime, $\phi(N) = N-1$ (see prev.\ lecture)\\
%	-- if $N = p_1^{e_1} \cdot p_n^{e_n}$, $\phi(N)=  N \cdot \prod_{i} \left(1 - \frac{1}{p_i}\right)$. \\
%	-- for $N = p \cdot q$, {\color{Orange}$\phi(N) = (p-1)(q-1)$.} \\
%	Ex.: $|\ZN^{\ast}| = |\{1, 2, 4, 7, 8, 11, 13, 14\}| = 2 \cdot 4 = 8$.
%	\pause
%	\item {\color{Orange}Euler's theorem:} for all $a \in \ZN^{\ast}$
%	\[
%	{\color{Orange}	a^{\phi(n)} = 1 \bmod N }
%	\]
%	{\large recall Fermat's: $a^{p-1} = 1 \bmod p$ for $p$-prime.}
%\end{itemize}
%\end{frame}
%
%\begin{frame}{Easy and hard problems in $\ZN$}
%\Large 
%\begin{center}
%{\color{Orange}  $N = p \cdot q$},  $p, q$ are of $\approx$ 1024 bits each.\\
%\end{center}
%	In $\ZN$ it is  {\color{Orange} easy} to \\[5pt]
%	-- add, multiply, find inverse (if exists, or check if does not)\\
%	--  compute $g^r \bmod N$ \\[14pt]
%	It is {\color{Orange} believed to be  hard} to \\[5pt]
%	 -- find $p, q$ \\
%	 -- compute square roots in $\ZN$ (as hard as factoring) \\
%	 -- compute $e\nth$ roots module $N$ when $\gcd(e, \phi(N)) = 1$\\
%	 %-- dlog (TODO)
%\end{frame}
%
%\begin{frame}{RSA Key Generation}
%\Large
%Let $\ell>2$ be an integer and $e>2$ be an odd integer. \\[8pt]
%{\color{Orange} $\mathsf{RSAGen}(\ell, e):$}
%\begin{enumerate}
%	\itemsep5pt
%	\item Generate an $\ell$-bit integer $p$ s.t.\ $\gcd(p-1, e)=1$
%	\item Generate an $\ell$-bit integer $q \neq p$ s.t.\ $\gcd(q-1, e)=1$
%	\item $N= p \cdot q$, $\phi(N) = (p-1)(q-1)$
%	\item $d = e^{-1} \bmod \phi(N)$
%	\item Output $\vk = (N, e), \sk=(N, d)$
%\end{enumerate}
%\vspace{10pt}
%\large
%\pause
%\begin{itemize}
%	\item there exist efficient probabilistic algorithms to generate primes
%	\item Step 4 is correct since $d \in \ZN^\ast$ since \[\gcd(p-1, e) = \gcd(q-1, e) = 1 \implies \gcd((p-1)(q-1), e)=1.\]
%	\item there is plenty of conditions on $p,q$ to make the above secure
%\end{itemize}
%\centering
%\Large Do not try to implement $\mathsf{RSAGen}$ yourself. 
%\end{frame}
%
%\begin{frame}{RSA Signature Generation and Verification}
%\large
%$\Hash: \{0,1\}^\ast \rightarrow \ZN^\ast$-- a cryptographic hash-function
%\vspace{10pt}
%\begin{columns}[t]
%	\begin{column}{0.45\textwidth}
%		
%{\color{Orange} I. $\mathsf{RSASign}(\sk=(N, d), m):$}
%\begin{enumerate}
%	\itemsep5pt
%	\item $y = \Hash(m) \in \ZN^\ast$
%	\item $\sigma = y^d \bmod N$
%\end{enumerate}
%\vspace{10pt}
%\pause
%{\color{Orange} II. $\mathsf{RSAVerify}(\vk=(N, e), m, \sigma):$}
%\begin{enumerate}
%	\itemsep5pt
%	\item $y' = \sigma^e \bmod N$
%	\item $\mathtt{return}(y'==\Hash(m))$ \\
%\end{enumerate}
%	\end{column}
%	\begin{column}{0.55\textwidth}
%		\pause
%		{\color{Orange} Correctness:}
%		For $N=pq$ and $e,d$ s.t.\ $ed = 1 \bmod \phi(N)$ and for all $x \in \Z$
%		{\color{Orange} 
%		\[
%			x^{ed} = x \bmod N
%		\] }
%	\pause
%		Proof: for $k \in \Z$
%		\begin{align*}
%		&ed = 1 + k \phi(N) = 1+k(p-1)(q-1) \\  \pause
%		& x^{p-1} = x \bmod p \quad \text{(Ferma't thm.)}\\ \pause
%		& x^{ed} = x^{1+k(p-1)(q-1)} = \\
%		& x \cdot (x^{p-1})^{q-1} = x \bmod p \\ \pause
%		&\text{Analogously}, x^{ed}  = x \bmod q \\ \pause
%		& \implies  p, q \, |\, x^{ed} - x \\
%		& \implies   x^{ed} = x \bmod p \cdot q
%		\end{align*}
%	\end{column}
%\end{columns}
%\LARGE
%\vspace*{-40pt}
%{\color{Orange}  $(y^d)^e = y^{ed} = y \bmod N$ } \\[20pt]
%
%\centering
%\vfill
%Without $\Hash$ the scheme is trivially insecure!
%\end{frame}
%
%\begin{frame}{RSA security}
%\Large
%\begin{itemize}
%	\item $\mathsf{RSAGen}(\ell, e) \rightarrow (\vk = (N, e), \sk=(N, d))$
%	\item $\mathsf{RSASign}(\sk, m) \rightarrow \sigma = \Hash(m)^e \bmod N$
%	\item $\mathsf{RSAVerify}(\vk, m, \sigma) \rightarrow \{0,1\}$
%\end{itemize}
%
%{\color{Orange}  RSA Assumption:}
%There does not exist a ppt adversary that given $(N, m, m^e)$ for a random $m \in \ZN^\ast$, outputs $m$.\\[10pt]
%\pause
%{\large Factoring $N \implies $ computing $e\nth$--roots. \\ The inverse is not known! }  \\[10pt]
%\pause
%{\color{Orange}  Theorem:} The signature scheme ($\mathsf{RSAGen},$ $\mathsf{RSASign},$ $\mathsf{RSAVerify}$) is secure in {\color{Orange}  exsistensial forgery CMA} model under {\color{Orange}  the RSA Assumption} and the assumption that $\Hash$ is a {\color{Orange}  Random Oracle}. \\[10pt]
%\pause
%Informally, {\color{Orange} a Random Oracle model} is an heuristic way to say that $\Hash$ behaves like a black-box that replies with random (but consistent) outputs.
% 
%\end{frame}
%
%\begin{frame}{Standards}
%\Large 
%Variants of RSA Signatures are standardized at PKCS (Public Key Cryptography Standards)  by RSA Security LLC
%\url{https://en.wikipedia.org/wiki/PKCS} \\[10pt]
%
%Version 2.2 (latest) of PKCS \#1 includes \\[10pt]
%\begin{itemize}
%	\itemsep10pt
%	\item RSASSA-PSS \\
%	SSA = Signature Scheme with Appendix \\
%	PSS =  Probabilistic Signature Scheme
%	\item  RSASSA-PKCS1-v1\_5 (attacks exist)
%\end{itemize}
%
%The standards describe data type conversions, how to represent the data, hash functions, etc.
%
%\end{frame}
%
%\begin{frame}{Usages of signatures schemes}
%	\Large
%	Practical signature schemes 
%	\begin{enumerate}
%		\itemsep 10pt
%		\item RSA \\
%		long keys, signatures; fast verification
%		\item ECDSA, GOST\\
%		short keys, signatures; slower verification
%	\end{enumerate}
%\vspace{10pt}
%RSA is good for {\color{Orange} Certificates}, ECDSA/GOST are good for e-mails. \\[10pt]
%
%Certificates  bind a public key to an identity.
%\end{frame}
%
%\begin{frame}{Certificates and PKI}
%\begin{center}
%	\begin{tabular}{l c c c l}
%		& \Large Alice & &\Large Certificate Authority (CA)&  \\ 
%		&  $\pk_{\texttt{A}}$  & & $\vk_{\texttt{CA}}$, $\sk_{\texttt{CA}}$ & \\
%		& \multirow{5}{*}{\includegraphics[scale=0.15]{Alice}} & & &  \\  \pause
%		& & $\xrightarrow[\texttt{id, email, } \pk_{\texttt{A}}]{\text {Certificate Signing Request}}$ & & \\  \pause
%		& & & Certifying Alice’s identity & \\
%		& & & Creating a certificate $\texttt{cert}$& \\
%		& & & Signing $\texttt{cert}$& \\
%		& & $\xleftarrow{\LARGE \texttt{cert}, \; \sigma = \Sign(\sk_{\texttt{CA}}, \texttt{cert})}$ & & \\  \pause
%	\end{tabular}
%\end{center}
%
%\vspace{15pt}
%\Large
%Anyone, who needs to communicate securely with Alice, first runs $\Ver(\vk_{\texttt{CA}} \texttt{cert},\sigma )$. If verification passes, $\pk_{\texttt{A}}$ can be used to communicate with Alice.\\[10pt]
%
%Example: X.509 certificate
%\end{frame}
%
%\begin{frame}{Certificate chains}
%	\Large
%	\begin{center}
%	
%	\begin{align*}
%			&{\LARGE \text{Root CAs } \quad \quad\xrightarrow{\hspace{2em}} \quad \quad \text{Intermediate CA}} \\
%			&\text{certifies Interm.\ CAs}  \hspace{46pt} \text{certifies clients} 
%	\end{align*}
%	\end{center}
%	\vspace{20pt}
%	There are currently thousands of intermediate CAs operating on the Internet \\[10pt]
%	To avoid malicious CAs: {\color{Orange} certificate pinning:} \\[7pt]
%	\begin{enumerate}
%		\item Every browser maintains a pinning database: \\
%		($\texttt{domain}$, $\texttt{hash}_0$, $\texttt{hash}_1, \ldots$ )
%		\item The data for each record is provided by the domain owner
%		\item 	When the browser connects to a domain, domain sends its certificate chain $\texttt{cert}_0, 
%		\texttt{cert}_1, \ldots$
%		\item The browser computes $\Hash(\texttt{cert}_i)$ and verifies against $\texttt{hash}_i$. 
%	\end{enumerate}
%	
%\end{frame}
%
%\begin{frame}{The last PA}
%
%	\Large
%	Task: implement $\KeyGen, \Sign, \Ver$ for RSA and ECDSA. \\[10pt]
%	Compare the speed of $\Sign, \Ver$ for 1000 messages (standard comparison)
%\end{frame}

\end{document}
