\documentclass[usenames,dvipsnames,8pt,aspectratio=169]{beamer}
\usepackage{amsmath,amsfonts,amssymb}
\usepackage{mathtools}
\usepackage{etex} %for Windows
\usepackage[utf8]{inputenc}
\usepackage[english, russian]{babel} 
%\usepackage{microtype}			% Better interword spacing and additional kerning.
\usepackage{ellipsis}			% Adjusted space with \dots between two words.
\usepackage{graphicx}
\usepackage{pstricks}

\usepackage{xcolor}


\usepackage{changepage}

\usepackage{algorithm}
\usepackage{algpseudocode}
%\usepackage[]{algorithm2e}
%\usepackage{algorithmic}

%\usepackage{tcolorbox}


\usepackage{caption}
\usepackage{subcaption}
\usepackage[normalem]{ulem} %to strike out text
\usepackage{cancel}
%\usepackage{stackengine}


\usepackage{tikz}
\usetikzlibrary{tikzmark,calc}
\usetikzlibrary{positioning, backgrounds}
\usetikzlibrary{arrows, chains, matrix, scopes, patterns, shapes, fit}
\usetikzlibrary{mindmap,trees,shadows}
\usetikzlibrary{decorations.pathreplacing}
%\usetikzlibrary{crypto.symbols}

\usepackage{pgfplots}

\pgfmathdeclarefunction{gauss}{2}{%
	\pgfmathparse{1/(#2*sqrt(2*pi))*exp(-((x-#1)^2)/(2*#2^2))}%
}


\tikzset{
	invisible/.style={opacity=0},
	visible on/.style={alt={#1{}{invisible}}},
	alt/.code args={<#1>#2#3}{%
		\alt<#1>{\pgfkeysalso{#2}}{\pgfkeysalso{#3}} % \pgfkeysalso doesn't change the path
	},
}

\newcommand\strikeout[2][]{%
	\begin{tabular}[b]{@{}c@{}} 
		\makebox(0,0)[cb]{{#1}} \\[-0.2\normalbaselineskip]
		\rlap{\color{Orange}\rule[0.5ex]{\widthof{#2}}{1.5pt}}#2
\end{tabular}}

\newcommand\Fontvi{\fontsize{11}{13.2}\selectfont}

\usepackage{listings} % for C++ code

\usepackage{braket}
%\usepackage[braket, qm]{qcircuit}



\usepackage[T1]{fontenc}
%\usepackage[sfdefault,scaled=.85]{FiraSans}
%\usepackage{newtxsf}
%\usepackage[nomap]{FiraMono}





\usefonttheme[onlymath]{serif}
\renewcommand\sfdefault{cmbr}

\renewcommand{\bfdefault}{sb}

\definecolor{CharCoalDark}{RGB}{13, 16, 19}
\definecolor{Orange}{RGB}{255, 165,0}
\definecolor{DarkOrange}{RGB}{255, 165,0}
\definecolor{LightSalmon}{RGB}{255, 160, 122}
\definecolor{LeafGreen}{RGB}{34, 139,  34}
\definecolor{Coral}{RGB}{255, 127, 80}
\definecolor{DarkTurquoise}{RGB}{0, 206, 209}

%\newtheorem{defRus}{Определение}
%\newtheorem{thmRus}{Теорема}
%s\newtheorem{corRus}{Следствие}


\setbeamercolor{background canvas}{bg=CharCoalDark}

\setbeamerfont{title}{series=\bfseries}
\setbeamercolor{title}{fg=Orange}
\setbeamercolor{section in toc}{fg=white}
\setbeamercolor{frametitle}{fg=Orange}
\setbeamercolor{normal text}{fg=white}
%\setbeamercolor{normal text}{fontsize=12pt}
\setbeamercolor{itemize item}{fg=Orange}
\setbeamercolor{enumerate item}{fg=Orange}
\setbeamercolor{enumerate item item}{fg=Orange}
\setbeamercolor{itemize item item}{fg=Orange}
\setbeamercolor{enumerate item}{fg=Orange}
\setbeamercolor{block title}{bg=DarkOrange,fg=white}
\setbeamerfont{block title}{series=\bfseries}

\setbeamertemplate{itemize item}[circle]
\setbeamertemplate{eumerate subitem}{\color{Orange}[$\checkmark$]}
\setbeamertemplate{itemize subitem}{\color{Orange}\Large$\textbullet$}
\setbeamertemplate{itemize subitem}{\color{Orange} \tiny $\blacksquare$}

% footnote without a marker
\newcommand\blfootnote[1]{%
	\begingroup
	\renewcommand\footnoterule{}
	\renewcommand\thefootnote{}\footnote{#1}%
	\addtocounter{footnote}{-1}%
	\endgroup
}

\newcommand*{\Scale}[2][4]{\scalebox{#1}{\ensuremath{#2}}}%

\newcommand\Item[1][]{%
	\ifx\relax#1\relax  \item \else \item[#1] \fi
	\abovedisplayskip=0pt\abovedisplayshortskip=0pt~\vspace*{-\baselineskip}}

\pgfdeclareradialshading{ring}{\pgfpoint{0cm}{0cm}}%
{rgb(0cm)=(1,1,1);
	rgb(0.7cm)=(1,1,1);
	rgb(0.719cm)=(1,1,1);
	rgb(0.72cm)=(0.975,0,0);
	rgb(0.9cm)=(1,1,1)}

\usepackage[absolute,overlay]{textpos} %to clip to a corner
\newcommand\FrameText[1]{%
	\begin{textblock*}{\paperwidth}(\textwidth-35pt, 10 pt)
		\raggedright #1\hspace{.5em}
\end{textblock*}}

\makeatletter
\let\save@measuring@true\measuring@true
\def\measuring@true{%
	\save@measuring@true
	\def\beamer@sortzero##1{\beamer@ifnextcharospec{\beamer@sortzeroread{##1}}{}}%
	\def\beamer@sortzeroread##1<##2>{}%
	\def\beamer@finalnospec{}%
}
\makeatother

\AtBeginSection[]
{
	\begin{frame}<beamer>
		\frametitle{Outline}
		\tableofcontents[currentsection]
	\end{frame}
}

\addtobeamertemplate{footline}{%
	\setlength\unitlength{1ex}%
	\begin{picture}(0,0) 
	% \put{} defines the position of the frame
	\put(155,0){\makebox(0,0)[bl]{
			%\includegraphics[scale=0.65]{white_square}
			%\includegraphics[scale=0.65]{dark_square}
			\includegraphics[scale=0.65]{grey_circle}
	}}%
	\end{picture}%
}{}


\newcommand{\AxisRotator}[1][rotate=0]{%
	\tikz [x=0.4cm,y=1.0cm,line width=.2ex,-stealth,#1] \draw[color=Orange] (0,0) arc (-150:150:2 and 1);%
}

\title{Лекция №8 \\[10pt]
	Часть 4. Слепая подпись. Сертификаты }

\date{ Елена Киршанова \\  \textbf{Курс ``Основы криптографии''} \\  }



\setbeamertemplate{navigation symbols}{} %removes navigation

% proper highlightling of a code-snippet
\lstset{language=C++,
	keywordstyle=\color{magenta},
	stringstyle=\color{Goldenrod},
	commentstyle=\color{gray},
	breaklines=false,
	%morecomment=[l][\color{magenta}]{\#}
}

%\setlength{\parskip}{8pt}
\input{header} %all defs
\begin{document}
	
\begin{frame}
	\titlepage
\end{frame}

\begin{frame}{ Слепая подпись RSA}

\Large
{\color{Orange} Сценарий:} {\color{Orange} A }  необходима подпись банка ({\color{Orange} Б }) на осуществлении транзакции. При этом {\color{Orange} Б } не должен знать о том, что именно он подписывает. {\color{Orange} Б } осуществляет {\color{Orange} слепую подпись.} \\[5pt]

$\Hash: \{0,1\}^\ast \rightarrow \Z_p^\ast$ -- криптографическая хэш-функция \\
$ m $ -- транзакция

\begin{center}
	\begin{tabular}{c c c}
		{\color{Orange} A } & & {\color{Orange} Б }\\ [5pt]
		& &  $\vk = (N, d), \sk= (N, w)\leftarrow \mathtt{RSA.KeyGen()}$\\
		$r \xleftarrow{\$} \Z_N$ & &  \\
		$m' = \Hash(m) \cdot r^e$  & $\xrightarrow{ \Huge \; m' \;}$&  $\sigma' = (m')^d$ \\
		$\sigma = \sigma' / r $& $\xleftarrow{ \Huge \; \sigma' \;}$&  \\
	\end{tabular}
	
\end{center}
	
\begin{itemize}
		\item {\color{Orange} A } получает корректную подпись для $m'$ \\[25pt]
		\item Значение $r^e$ маскирует исходное сообщение ({\color{Orange} Б} не знает ничего о $m$)
\end{itemize}
	
\end{frame}

\begin{frame}{Слепая подпись }

\Large

\begin{itemize}
	\itemsep10pt
	\item Слепая подпись безопасна, если ppt атакующий, зная $\vk$ и $Q$ слепых подписей, не может сгенерировать $Q+1$ пару  $(m, \sigma)$
	\item Безопасность слепой подписи RSA основана на задача 1MRSA: \\
	\begin{itemize}
		\itemsep5pt
		\large 
		\item {\color{Orange} A} получает значения $w = \hat{v}^{1/e}$ от Челленджера для $\hat{v}$ по своему выбору
		\item Задача {\color{Orange} A}: вычислить $v^{1/e}$ для какого-либо одного $v \neq \hat{v}$, выбранного Челленджером
	\end{itemize}	
	
	\item Слепая подпись может быть построена на основе подписи Шнорра
	\item Используются в протоколах Anonymous Credentials, E-cash. 
\end{itemize}


\end{frame}

\begin{frame}{Применение криптографических подписей}
	\Large
	Подписи на практике
	\begin{enumerate}
		\itemsep 10pt
		\item RSA \\
		длинные ключи и подписи; быстрая верификация
		\item ECDSA, ГОСТ\\
		длинные ключи и подписи; верификация медленнее
	\end{enumerate}
\vspace{10pt}
RSA хороша для {\color{Orange} Сертификатов}, ECDSA/ГОСТ -- для имейлов. \\[10pt]


\end{frame}

\begin{frame}{Сертификаты}
\Large 
Задача сертификата:  связать публичный ключ с человеком/машиной.
\begin{center}
	\begin{tabular}{l c c c l}
		& \Large Алиса & &\Large Сертификационный центр (CA)&  \\ 
		&  $\pk_{\texttt{A}}$  & & $\vk_{\texttt{CA}}$, $\sk_{\texttt{CA}}$ & \\
		& \multirow{5}{*}{\includegraphics[scale=0.15]{Alice}} & & &  \\  \pause
		& & $\xrightarrow[\texttt{id, email, } \pk_{\texttt{A}}]{\text {Certificate Signing Request}}$ & & \\  \pause
		& & & \Large Сертификация id Алисы: & \\
		& & & Создание $\texttt{cert}$& \\
		& & & Подпись $\texttt{cert}$& \\
		& & $\xleftarrow{\LARGE \texttt{cert}, \; \sigma = \Sign(\sk_{\texttt{CA}}, \texttt{cert})}$ & & \\  
	\end{tabular}
\end{center}

\vspace{15pt}
\Large
Для осуществления аутентифицированного канала с Алисой \\
верифицируется $\texttt{cert}$:
$\Ver(\vk_{\texttt{CA}} \texttt{cert},\sigma )$. \\[5pt]

Если $\Ver(\vk_{\texttt{CA}} \texttt{cert},\sigma ) = \mathsf{accept}$, то $\pk_{\texttt{A}}$ используется для \\ коммуникации с Алисой.\\[8pt]

Пример: сертификат X.509 certificate
\end{frame}

\begin{frame}{Цепочки сертификатов}
	\Large
	\begin{center}
	
	\begin{align*}
			&{\LARGE \text{Корневой CA } \hspace{70pt} \xrightarrow{\hspace{2em}} \quad \quad \text{Промежуточный CA}} \\
			&\text{сертифицирует промежуточные CA}  \hspace{37pt} \text{сертифицирует клиентов} 
	\end{align*}
	\end{center}
	\vspace{15pt}
	Сегодня мы имеем тысячи промежуточный CAs \\[10pt]
	Для противостояния CAs: {\color{Orange} certificate pinning:} \\[7pt]
	\begin{enumerate}
		\item Каждый браузер (клиент) поддерживает базу данных (pinning database): \\
		($\texttt{domain}$, $\texttt{hash}_0$, $\texttt{hash}_1, \ldots$ )
		\item Данные для БД предоставляет домен
		\item Когда браузер стучится в домен, домен высылает ему цепочку\\ своих сертификатов $\texttt{cert}_0, 
		\texttt{cert}_1, \ldots$
		\item Браузер вычисляет $\Hash(\texttt{cert}_i)$ и верифицирует $\texttt{hash}_i$. 
	\end{enumerate}
	
\end{frame}

\begin{frame}{Пример цепочки сертификатов }
\begin{figure}
	\includegraphics[scale=0.45]{cert1}
\end{figure}


\end{frame}


\end{document}
