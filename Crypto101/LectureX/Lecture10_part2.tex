\documentclass[usenames,dvipsnames,8pt,aspectratio=169]{beamer}
\usepackage{amsmath,amsfonts,amssymb}
\usepackage{mathtools}
\usepackage{etex} %for Windows
\usepackage[utf8]{inputenc}
\usepackage[english, russian]{babel} 
%\usepackage{microtype}			% Better interword spacing and additional kerning.
\usepackage{ellipsis}			% Adjusted space with \dots between two words.
\usepackage{graphicx}
\usepackage{pstricks}

\usepackage{xcolor}


\usepackage{changepage}

\usepackage{algorithm}
\usepackage{algpseudocode}
%\usepackage[]{algorithm2e}
%\usepackage{algorithmic}

%\usepackage{tcolorbox}


\usepackage{caption}
\usepackage{subcaption}
%\usepackage{stackengine}


\usepackage{tikz}
\usetikzlibrary{tikzmark,calc}
\usetikzlibrary{positioning, backgrounds}
\usetikzlibrary{arrows, chains, matrix, scopes, patterns, shapes, fit}
\usetikzlibrary{mindmap,trees,shadows}
\usetikzlibrary{decorations.pathreplacing}
%\usetikzlibrary{crypto.symbols}

\usepackage{pgfplots}

\pgfmathdeclarefunction{gauss}{2}{%
	\pgfmathparse{1/(#2*sqrt(2*pi))*exp(-((x-#1)^2)/(2*#2^2))}%
}


\tikzset{
	invisible/.style={opacity=0},
	visible on/.style={alt={#1{}{invisible}}},
	alt/.code args={<#1>#2#3}{%
		\alt<#1>{\pgfkeysalso{#2}}{\pgfkeysalso{#3}} % \pgfkeysalso doesn't change the path
	},
}

\newcommand\strikeout[2][]{%
	\begin{tabular}[b]{@{}c@{}} 
		\makebox(0,0)[cb]{{#1}} \\[-0.2\normalbaselineskip]
		\rlap{\color{Orange}\rule[0.5ex]{\widthof{#2}}{1.5pt}}#2
\end{tabular}}

\newcommand\Fontvi{\fontsize{11}{13.2}\selectfont}

\usepackage{listings} % for C++ code

\usepackage{braket}
%\usepackage[braket, qm]{qcircuit}



\usepackage[T1]{fontenc}
%\usepackage[sfdefault,scaled=.85]{FiraSans}
%\usepackage{newtxsf}
%\usepackage[nomap]{FiraMono}


\usepackage{changepage}



\usefonttheme[onlymath]{serif}
\renewcommand\sfdefault{cmbr}

\renewcommand{\bfdefault}{sb}

\definecolor{CharCoalDark}{RGB}{13, 16, 19}
\definecolor{Orange}{RGB}{255, 165,0}
\definecolor{DarkOrange}{RGB}{255, 165,0}
\definecolor{LightSalmon}{RGB}{255, 160, 122}
\definecolor{MapleWood}{RGB}{241, 195, 142}
\definecolor{Redwood}{RGB}{210, 173, 169}
\definecolor{LeafGreen}{RGB}{34, 139,  34}
\definecolor{Coral}{RGB}{255, 127, 80}
\definecolor{DarkTurquoise}{RGB}{0, 206, 209}

%\newtheorem{defRus}{Определение}
%\newtheorem{thmRus}{Теорема}
%s\newtheorem{corRus}{Следствие}


\setbeamercolor{background canvas}{bg=CharCoalDark}

\setbeamerfont{title}{series=\bfseries}
\setbeamercolor{title}{fg=Orange}
\setbeamercolor{section in toc}{fg=white}
\setbeamercolor{frametitle}{fg=Orange}
\setbeamercolor{normal text}{fg=white}
%\setbeamercolor{normal text}{fontsize=12pt}
\setbeamercolor{itemize item}{fg=Orange}
\setbeamercolor{enumerate item}{fg=Orange}
\setbeamercolor{enumerate item item}{fg=Orange}
\setbeamercolor{itemize item item}{fg=Orange}
\setbeamercolor{enumerate item}{fg=Orange}
\setbeamercolor{block title}{bg=DarkOrange,fg=white}
\setbeamerfont{block title}{series=\bfseries}

\setbeamertemplate{itemize item}[circle]
\setbeamertemplate{eumerate subitem}{\color{Orange}[$\checkmark$]}
\setbeamertemplate{itemize subitem}{\color{Orange}\Large$\textbullet$}
\setbeamertemplate{itemize subitem}{\color{Orange} \tiny $\blacksquare$}

% footnote without a marker
\newcommand\blfootnote[1]{%
	\begingroup
	\renewcommand\footnoterule{}
	\renewcommand\thefootnote{}\footnote{#1}%
	\addtocounter{footnote}{-1}%
	\endgroup
}


\usepackage[absolute,overlay]{textpos} %to clip to a corner
\newcommand\FrameText[1]{%
	\begin{textblock*}{\paperwidth}(\textwidth-35pt, 10 pt)
		\raggedright #1\hspace{.5em}
\end{textblock*}}

\makeatletter
\let\save@measuring@true\measuring@true
\def\measuring@true{%
	\save@measuring@true
	\def\beamer@sortzero##1{\beamer@ifnextcharospec{\beamer@sortzeroread{##1}}{}}%
	\def\beamer@sortzeroread##1<##2>{}%
	\def\beamer@finalnospec{}%
}
\makeatother

\AtBeginSection[]
{
	\begin{frame}<beamer>
		\frametitle{Outline}
		\tableofcontents[currentsection]
	\end{frame}
}

\addtobeamertemplate{footline}{%
	\setlength\unitlength{1ex}%
	\begin{picture}(0,0) 
	% \put{} defines the position of the frame
	\put(155,0){\makebox(0,0)[bl]{
			%\includegraphics[scale=0.65]{white_square}
			%\includegraphics[scale=0.65]{dark_square}
			\includegraphics[scale=0.65]{grey_circle}
	}}%
	\end{picture}%
}{}

\title{Лекция №10 \\[10pt]
	Часть 2. Протоколы сквозного шифрования. Протокол Signal.}

\date{ Елена Киршанова \\  \textbf{Курс ``Основы криптографии''} \\  }



\setbeamertemplate{navigation symbols}{} %removes navigation

% proper highlightling of a code-snippet
\lstset{language=C++,
	keywordstyle=\color{magenta},
	stringstyle=\color{Goldenrod},
	commentstyle=\color{gray},
	breaklines=false,
	%morecomment=[l][\color{magenta}]{\#}
}


\input{header} %all defs
\begin{document}
	
\begin{frame}
	\titlepage
\end{frame}



\begin{frame}{Безопасный обмен сообщениями}
	\Large
	{\color{Orange} Безопасный мессенджер (Secure Messaging, SM)} позволяет группе людей обмениваться сообщениями. При этом обеспечиваются\\[10pt]
	\large

	\begin{itemize}
		\itemsep5pt
		\item {\color{Orange} Корректность}
		\item {\color{Orange} Конфиденциальность:} атакующий не получает никакой информации о переданных сообщениях, в случае если ни одна из сторон не скомпрометирована
		\item {\color{Orange} Аутентификация:} атакующий не может изменить, продублировать или вставить новое сообщение
		\item  {\color{Orange} Немедленное дешифрование}
		\item {\color{Orange}  Стойкость к потере сообщений:} в случае утраты сообщения, коммуникация продолжается
		\item {\color{Orange} Прямая секретность (forward secrecy):} все сообщения, переданные \emph{до} компрометации остаются конфиденциальны
		\item {\color{Orange} Безопасность после компрометации: } стороны могут возобновить конфиденциальность \emph{после} компрометации
	\end{itemize}
	\centering
\end{frame}

\begin{frame}{Протокол Signal}
	\Large
	{\color{Orange}Протокол Signal} предложен компанией Open Whisper Systems. Пример протокола безопасного мессенджера. \\[10pt]
	\large
	\begin{itemize}
		\itemsep 7pt
		\item используется в приложениях Signal, WhatsApp, Facebook Messenger, Skype
		\item каждое сообщение шифруется и аутентифицируется с помощью свежего симметрического ключа
		\item удовлетворяет вышеперечисленным критериям безопасности 
	\end{itemize}
	\vspace{30pt}

\normalsize 
Описание протокола: \\ \url{https://signal.org/docs/specifications/doubleratchet/doubleratchet.pdf} \\[5pt]
Его анализ: \url{https://eprint.iacr.org/2018/1037.pdf} \\[5pt]
Лекция Y.Dodis: \url{https://www.youtube.com/watch?v=az-tMBjLcks}

\end{frame}

\begin{frame}{Primitives for SM Protocols}
\Large
\[
	\left.
	\begin{array}{rrr}
		\text{Корректность}\\
		\text{Конфиденциальность}\\
		\text{Аутентификация}\\
		\text{Немедленное дешифровани} \\
		\text{Стойкость к потере сообщенийe}\\
		\text{Прямая секретность }
	\end{array}
	\right\}\text{AEAD (симметрический примитив)}
	\]
AEAD -- Authenticated Encryption with Associated Data \\[20pt]
\[
\hspace{-70pt}\left.
\begin{array}{rrr}
\text{Прямая секретность }\\
\text{Безопасность после компрометации} 
\end{array}
\right\}\text{CKA (асимм. примитив)}
\]
CKA -- Непрерывный обмен ключами (Continuous Key Agreement)
\end{frame}

\begin{frame}{Signal: симметрическая часть}
\LARGE
	\begin{center}
		\begin{tabular}{c c c }
			{\LARGE \color{Orange}\texttt{A}}&    & {\LARGE \color{Orange}\texttt{B}}  \\
			{\color{Orange}\texttt{Отправляющий}}&  {\LARGE $\leftarrow k \rightarrow $ }  & {\color{Orange}\texttt{Получатель}}   \\[10pt]
			$s_{\text{A},0} \leftarrow \mathtt{InitSender}(k)$& & $s_{\text{B},0} \leftarrow \mathtt{InitReceiver}(k)$ \\[10pt]
			$s_{\text{A},1}, c_1 \leftarrow \mathtt{Send}(m_1)$& $\xrightarrow{\hspace{5pt}c_1\hspace{5pt}}$ & \\
			$s_{\text{A},2}, c_2 \leftarrow \mathtt{Send}(m_2)$& $\xrightarrow{\hspace{5pt}c_2\hspace{5pt}}$ & \\
			& & $s_{\text{B},1}, m_1 \leftarrow \mathtt{Receive}(c_1)$ \\
			& & $s_{\text{B},2}, m_2 \leftarrow \mathtt{Receive}(c_2)$ \\
			& $\vdots $& \\
		\end{tabular}
	\end{center} 
\begin{itemize}
	\item $\mathtt{Send}$ -- шифрование, $\mathtt{Receive}$ -- дешифрование
	\item $s_{\text{A},i}$ -- $i$-ое состояние A
	\item $s_{\text{B},i}$ -- $i$-ое состояние B
	\item состояния должны оставаться конфиденциальными
	\item шифр-тексты $c_i$ не обязаны приходит к \texttt{B} в корректном порядке
\end{itemize}

\end{frame}

\begin{frame}{Signal: AEAD + PRG}
\Large 
$\Enc,\Dec$ -- процедуры шифрования/дешифрование в AEAD (см. Лекцию № 6)
\[
\text{G}: \{0,1\}^n \rightarrow \{0,1\}^{2n} - \text{псевдослучайный генератор (см. Лекцию № 2)}
\]
	\vspace{-10pt}
	\begin{columns}
		\hspace{-30pt}
		\begin{column}{0.5\textwidth}
			\centering
			\includegraphics[width=0.8\textwidth]{SendSignal}
		\end{column}
		\hspace{-80pt}
		\begin{column}{0.5\textwidth}
			\centering
			\pause 
			\includegraphics[width=0.8\textwidth]{ReceiveSignal}
		\end{column}
	\end{columns}
\end{frame}

\begin{frame}{Signal: AEAD + PRG}
\Large 
$\Enc, \Dec$ -- процедуры шифрования/дешифрование в AEAD (см. Лекцию № 6)
\[
\text{G}: \{0,1\}^n \rightarrow \{0,1\}^{2n} \text{псевдо-случайный генератор (см. Лекцию № 2)}
\]
\vspace{-20pt}
\begin{columns}[T]
	\hspace{-30pt}
	\begin{column}{0.5\textwidth}
		\centering
		\includegraphics[width=0.9\textwidth]{SendSignalNoC2}
	\end{column}
	\hspace{-70pt}
	\begin{column}{0.5\textwidth}
		\centering
		\pause 
		\includegraphics[width=0.80\textwidth]{ReceiveSignalNoC2}
	\end{column}
\end{columns}
\centering
\vspace{10pt}
%All w$_i$ are erased when no further needed.
\end{frame}

\begin{frame}{Signal: ассиметрическая часть}
	\Large Общий ключ $k$ генерируется в результате протокола обмена ключами\\
	Signal использует {\color{Orange} непрерывный протокол обмена ключами} Диффи-Хэллмана
	 
	\begin{center}
		\begin{tabular}{c c c }
			{\LARGE \color{Orange}\texttt{A}}&    & {\LARGE \color{Orange}\texttt{B}}  \\[2pt]
			$x_1 $& $\xrightarrow{\hspace{15pt} g^{x_1} \hspace{15pt}}$ & \\[5pt]
			& $\xleftarrow{\hspace{15pt} g^{x_2} \hspace{15pt}}$ & $x_2$ \\
			{\color{Orange}$g^{x_1x_2}$} & & {\color{Orange}$g^{x_1x_2}$} \pause \\[5pt]
			$x_3 $& $\xrightarrow{\hspace{15pt} g^{x_3} \hspace{15pt}}$ & \\
			{\color{Orange}$g^{x_2x_3}$} & & {\color{Orange}$g^{x_2x_3}$ }\\[5pt]
			& $\xleftarrow{\hspace{15pt} g^{x_4} \hspace{15pt}}$ & $x_4$  \\
			{\color{Orange}$g^{x_3x_4}$} & & {\color{Orange}$g^{x_3x_4}$ }\\
			& $\vdots $& \\
		\end{tabular}
	\end{center}
\vspace{-10pt}
\normalfont
\begin{itemize}
	\item Во время $i$ общий ключ $ = g^{x_i x_{i-1}}$
	\item Новый общий ключ генерируется всякий раз, когда одна из \\ сторон меняет роль с {\color{Orange}\texttt{Получателя}} на {\color{Orange}\texttt{Отправителя}} 
	\item Если ключ $g^{x_i x_{i-1}}$ скомпрометирован (атакующий знает $x_i$), стороны восстанавливают конфиденциальность спустя 2 раунда
\end{itemize}
	
\end{frame}

\begin{frame}{Замечания}
\Large

\begin{itemize}
	\itemsep 10pt
	\item Безопасность (нетривильно) основана на стойкость {\color{Orange} обмена ключами/KEM'a,} примитива {\color{Orange} AEAD},  {\color{Orange} PRG} 
	\item На практике \\
	\hspace{10pt} обмен ключами: Диффи-Хэллман на эллиптической кривой \\[3pt]
	\hspace{10pt} AEAD:  CBC + HMAC \\[3pt]
	\hspace{10pt} вместо PRG используется SHA-256 \\
	\item формальное доказательство безопасности чатов (>2 участников) \\ под вопросом
\end{itemize}
\end{frame}

\end{document}
