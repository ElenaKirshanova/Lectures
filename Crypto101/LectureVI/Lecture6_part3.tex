\documentclass[usenames,dvipsnames,8pt,aspectratio=169]{beamer}
\usepackage{amsmath,amsfonts,amssymb}
\usepackage{mathtools}
\usepackage{etex} %for Windows
\usepackage[utf8]{inputenc}
\usepackage[english, russian]{babel} 
%\usepackage{microtype}			% Better interword spacing and additional kerning.
\usepackage{ellipsis}			% Adjusted space with \dots between two words.
\usepackage{graphicx}
\usepackage{pstricks}

\usepackage{xcolor}


\usepackage{changepage}

\usepackage{algorithm}
\usepackage{algpseudocode}
%\usepackage[]{algorithm2e}
%\usepackage{algorithmic}

%\usepackage{tcolorbox}


\usepackage{caption}
\usepackage{subcaption}
%\usepackage{stackengine}


\usepackage{tikz}
\usetikzlibrary{tikzmark,calc}
\usetikzlibrary{positioning, backgrounds}
\usetikzlibrary{arrows, chains, matrix, scopes, patterns, shapes, fit}
\usetikzlibrary{mindmap,trees,shadows}
\usetikzlibrary{decorations.pathreplacing}
%\usetikzlibrary{crypto.symbols}

\usepackage{pgfplots}

\pgfmathdeclarefunction{gauss}{2}{%
	\pgfmathparse{1/(#2*sqrt(2*pi))*exp(-((x-#1)^2)/(2*#2^2))}%
}


\tikzset{
	invisible/.style={opacity=0},
	visible on/.style={alt={#1{}{invisible}}},
	alt/.code args={<#1>#2#3}{%
		\alt<#1>{\pgfkeysalso{#2}}{\pgfkeysalso{#3}} % \pgfkeysalso doesn't change the path
	},
}

\newcommand\strikeout[2][]{%
	\begin{tabular}[b]{@{}c@{}} 
		\makebox(0,0)[cb]{{#1}} \\[-0.2\normalbaselineskip]
		\rlap{\color{Orange}\rule[0.5ex]{\widthof{#2}}{1.5pt}}#2
\end{tabular}}

\newcommand\Fontvi{\fontsize{11}{13.2}\selectfont}

\usepackage{listings} % for C++ code

\usepackage{braket}
%\usepackage[braket, qm]{qcircuit}



\usepackage[T1]{fontenc}
%\usepackage[sfdefault,scaled=.85]{FiraSans}
%\usepackage{newtxsf}
%\usepackage[nomap]{FiraMono}





\usefonttheme[onlymath]{serif}
\renewcommand\sfdefault{cmbr}

\renewcommand{\bfdefault}{sb}

\definecolor{CharCoalDark}{RGB}{13, 16, 19}
\definecolor{Orange}{RGB}{255, 165,0}
\definecolor{DarkOrange}{RGB}{255, 165,0}
\definecolor{LightSalmon}{RGB}{255, 160, 122}
\definecolor{LeafGreen}{RGB}{34, 139,  34}
\definecolor{Coral}{RGB}{255, 127, 80}
\definecolor{DarkTurquoise}{RGB}{0, 206, 209}

%\newtheorem{defRus}{Определение}
%\newtheorem{thmRus}{Теорема}
%s\newtheorem{corRus}{Следствие}


\setbeamercolor{background canvas}{bg=CharCoalDark}

\setbeamerfont{title}{series=\bfseries}
\setbeamercolor{title}{fg=Orange}
\setbeamercolor{section in toc}{fg=white}
\setbeamercolor{frametitle}{fg=Orange}
\setbeamercolor{normal text}{fg=white}
%\setbeamercolor{normal text}{fontsize=12pt}
\setbeamercolor{itemize item}{fg=Orange}
\setbeamercolor{enumerate item}{fg=Orange}
\setbeamercolor{enumerate item item}{fg=Orange}
\setbeamercolor{itemize item item}{fg=Orange}
\setbeamercolor{enumerate item}{fg=Orange}
\setbeamercolor{block title}{bg=DarkOrange,fg=white}
\setbeamerfont{block title}{series=\bfseries}

\setbeamertemplate{itemize item}[circle]
\setbeamertemplate{eumerate subitem}{\color{Orange}[$\checkmark$]}
\setbeamertemplate{itemize subitem}{\color{Orange}\Large$\textbullet$}
\setbeamertemplate{itemize subitem}{\color{Orange} \tiny $\blacksquare$}

% footnote without a marker
\newcommand\blfootnote[1]{%
	\begingroup
	\renewcommand\footnoterule{}
	\renewcommand\thefootnote{}\footnote{#1}%
	\addtocounter{footnote}{-1}%
	\endgroup
}

\newcommand*{\Scale}[2][4]{\scalebox{#1}{\ensuremath{#2}}}%

\newcommand\Item[1][]{%
	\ifx\relax#1\relax  \item \else \item[#1] \fi
	\abovedisplayskip=0pt\abovedisplayshortskip=0pt~\vspace*{-\baselineskip}}

\pgfdeclareradialshading{ring}{\pgfpoint{0cm}{0cm}}%
{rgb(0cm)=(1,1,1);
	rgb(0.7cm)=(1,1,1);
	rgb(0.719cm)=(1,1,1);
	rgb(0.72cm)=(0.975,0,0);
	rgb(0.9cm)=(1,1,1)}

\addtobeamertemplate{footline}{%
	\setlength\unitlength{1ex}%
	\begin{picture}(0,0) 
	% \put{} defines the position of the frame
	\put(155,0){\makebox(0,0)[bl]{
			%\includegraphics[scale=0.65]{white_square}
			%\includegraphics[scale=0.65]{dark_square}
			\includegraphics[scale=0.65]{grey_circle}
	}}%
	\end{picture}%
}{}


\usepackage[absolute,overlay]{textpos} %to clip to a corner
\newcommand\FrameText[1]{%
	\begin{textblock*}{\paperwidth}(\textwidth-35pt, 10 pt)
		\raggedright #1\hspace{.5em}
\end{textblock*}}

\makeatletter
\let\save@measuring@true\measuring@true
\def\measuring@true{%
	\save@measuring@true
	\def\beamer@sortzero##1{\beamer@ifnextcharospec{\beamer@sortzeroread{##1}}{}}%
	\def\beamer@sortzeroread##1<##2>{}%
	\def\beamer@finalnospec{}%
}
\makeatother

\AtBeginSection[]
{
	\begin{frame}<beamer>
		\frametitle{Outline}
		\tableofcontents[currentsection]
	\end{frame}
}


\title{Лекция №6 \\[10pt]
	Часть 3. Шифрование с аутентификацией и ассоциированными данными.}

\date{ Елена Киршанова \\  \textbf{Курс ``Основы криптографии''} \\  }



\setbeamertemplate{navigation symbols}{} %removes navigation

% proper highlightling of a code-snippet
\lstset{language=C++,
	keywordstyle=\color{magenta},
	stringstyle=\color{Goldenrod},
	commentstyle=\color{gray},
	breaklines=false,
	%morecomment=[l][\color{magenta}]{\#}
}

%\setlength{\parskip}{8pt}
\input{header} %all defs

\newcommand{\AxisRotator}[1][rotate=0]{%
	\tikz [x=0.5cm,y=1.9cm,line width=.2ex,-stealth,#1] \draw[color=Orange] (0,0) arc (-150:150:2 and 1);%
}

\begin{document}
	
\begin{frame}
	\titlepage
\end{frame}

\begin{frame}{AEAD: \underline{A}uthenticated \underline{E}ncryption with \underline{A}ssociated \underline{D}ata }
	\Large
	Часто не всё сообщение должно быть зашифровано.
	
	\vspace{15pt}
	
	Пример: $[\texttt{header} || \texttt{payload} ]$ в интернет протоколах.\\
	
	\LARGE
	\[
	\underbrace{
	\left[
	\texttt{Associated data} ||
	\texttt{Encrypted data}
	\right]
	}_\text{Аутентификация}
	\]
	
	
	
	\vspace{25pt}

	Самое популярное AEAD: {\color{Orange}AES-GCM AEAD}
	
\end{frame}

\begin{frame}{AES-GCM AEAD}
\Large Сообщение $m=(m_1, \ldots, m_s)$
\only<1>
{
	\begin{figure}
		\includegraphics[width=0.7\linewidth]{AES_GCM_AEAD_1}
	\end{figure}
}
\only<2>
{
	\begin{figure}
		\includegraphics[width=0.7\linewidth]{AES_GCM_AEAD_2}
	\end{figure}
}
\only<3>
{
	\begin{figure}
		\includegraphics[width=0.7\linewidth]{AES_GCM_AEAD_3}
	\end{figure}
}
\only<4->
{
	\begin{figure}
		\includegraphics[width=0.7\linewidth]{AES_GCM_AEAD_full}
	\end{figure}
Выход $(c_1, \ldots, c_s, c_{s+1})$
}
\end{frame}

\begin{frame}{AES-GCM AEAD}
\begin{figure}
	\includegraphics[width=0.65\linewidth]{AES_GCM_AEAD_full}
\end{figure}

\Large
\begin{itemize}
	\item Используется лишь один ключ
	\item MAC: конструкция Картера-Вегмана на основе GHASH
	\item Дешифрование: \\
	1. Проверка MACа \\
	2. $\Dec(c_1, \ldots, c_s)$
\end{itemize}

\end{frame}

\begin{frame}{Конструкция Картера-Вегмана / Carter-Wegman MAC }
\vspace{-30pt}
\begin{figure}
	\includegraphics[width=\textwidth]{Carter-Wegman}
\end{figure}
\end{frame}
\begin{frame}{AEAD в TLS 1.3}
	\LARGE
	\begin{center}
		\begin{tabular}{c c c }
			\texttt{Браузер}&  {\color{Orange}{Фаза 1} Рукопожатие}   & \texttt{Веб-сервер}  \\
			& {\Large Ассиметрическое Шифрование}  & \\ 
			& {\Large Формирование ключей}  & \\ 
			$k_{b\rightarrow s}$&  & $k_{b\rightarrow s}$\\ 
			$k_{s\rightarrow b}$&  & $k_{s\rightarrow b}$  \\ [10pt]
			&  {\color{Orange}{Фаза 2} TLS протокол передачи данных}   &  \\
			& {\Large AEAD}  & \\ 
		\end{tabular}
	\end{center}
\end{frame}

%\begin{frame}{TLS протокол передачи данных}
%\Large Данные = $\left[m_1, \ldots, m_s\right]$
%\LARGE
%\begin{center}
%	\begin{tabular}{c c c }
%\texttt{Браузер}&   & \texttt{Веб-сервер}  \\
%$k_{b\rightarrow s}$&  & $k_{b\rightarrow s}$\\ 
%$k_{s\rightarrow b}$&  & $k_{s\rightarrow b}$  \\ 
%$ctr_{b\rightarrow s}$&  & $ctr_{b\rightarrow s}$\\ 
%$ctr_{s\rightarrow b}$& {\large$\underbrace{
%		\left[
%		\texttt{Meta data} || \;
%		m_i \; || \;
%		\texttt{Nonce}
%		\right]}$}  & $ctr_{s\rightarrow b}$  \\ 
%\multicolumn{3}{c}{ $\xrightarrow{ \hspace{1em}  \Huge \text{AES-GCM-AEAD}(k_{b\rightarrow s}) \hspace{3.4em} }$  }  \pause \\
%$ctr_{b\rightarrow s}++$&  & $ctr_{b\rightarrow s}++$\\ 
%\end{tabular}
%\end{center}
%
%\pause 
%\vspace{10pt}
%\texttt{Meta data} включает: record on the phase (1 or 2), TLS Version, \texttt{len}$(c)$ \\
%Counters $ctr$ are used to prevent replay attacks
%
%\end{frame}
%
%\begin{frame}{Programming Assignment \# 5}
%\Large
%OpenSSL provides interfaces to GCM, CCM AEs via \texttt{EVP}\\[15pt]
%
%
%This PA:  to implement Encryption and Decryption Interfaces for any two Authenticated Encryption
%\begin{itemize}
%	\item GCM 
%	\item CCM 
%	\item ChaCha20-Poly1305
%\end{itemize}
%
%\vspace{10pt}
%See \url{https://wiki.openssl.org/index.php/EVP_Authenticated_Encryption_and_Decryption} for code
%
%
%\end{frame}

\end{document}
