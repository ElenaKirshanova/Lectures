\documentclass[usenames,dvipsnames,8pt,aspectratio=169]{beamer}
\usepackage{amsmath,amsfonts,amssymb}
\usepackage{mathtools}
\usepackage{etex} %for Windows
\usepackage[utf8]{inputenc}
\usepackage[english, russian]{babel} 
%\usepackage{microtype}			% Better interword spacing and additional kerning.
\usepackage{ellipsis}			% Adjusted space with \dots between two words.
\usepackage{graphicx}
\usepackage{pstricks}

\usepackage{xcolor}


\usepackage{changepage}

\usepackage{algorithm}
\usepackage{algpseudocode}
%\usepackage[]{algorithm2e}
%\usepackage{algorithmic}

%\usepackage{tcolorbox}


\usepackage{caption}
\usepackage{subcaption}
%\usepackage{stackengine}


\usepackage{tikz}
\usetikzlibrary{tikzmark,calc}
\usetikzlibrary{positioning, backgrounds}
\usetikzlibrary{arrows, chains, matrix, scopes, patterns, shapes, fit}
\usetikzlibrary{mindmap,trees,shadows}
\usetikzlibrary{decorations.pathreplacing}
%\usetikzlibrary{crypto.symbols}

\usepackage{pgfplots}

\pgfmathdeclarefunction{gauss}{2}{%
	\pgfmathparse{1/(#2*sqrt(2*pi))*exp(-((x-#1)^2)/(2*#2^2))}%
}


\tikzset{
	invisible/.style={opacity=0},
	visible on/.style={alt={#1{}{invisible}}},
	alt/.code args={<#1>#2#3}{%
		\alt<#1>{\pgfkeysalso{#2}}{\pgfkeysalso{#3}} % \pgfkeysalso doesn't change the path
	},
}

\newcommand\strikeout[2][]{%
	\begin{tabular}[b]{@{}c@{}} 
		\makebox(0,0)[cb]{{#1}} \\[-0.2\normalbaselineskip]
		\rlap{\color{Orange}\rule[0.5ex]{\widthof{#2}}{1.5pt}}#2
\end{tabular}}

\newcommand\Fontvi{\fontsize{11}{13.2}\selectfont}

\usepackage{listings} % for C++ code

\usepackage{braket}
%\usepackage[braket, qm]{qcircuit}



\usepackage[T1]{fontenc}
%\usepackage[sfdefault,scaled=.85]{FiraSans}
%\usepackage{newtxsf}
%\usepackage[nomap]{FiraMono}





\usefonttheme[onlymath]{serif}
\renewcommand\sfdefault{cmbr}

\renewcommand{\bfdefault}{sb}

\definecolor{CharCoalDark}{RGB}{13, 16, 19}
\definecolor{Orange}{RGB}{255, 165,0}
\definecolor{DarkOrange}{RGB}{255, 165,0}
\definecolor{LightSalmon}{RGB}{255, 160, 122}
\definecolor{LeafGreen}{RGB}{34, 139,  34}
\definecolor{Coral}{RGB}{255, 127, 80}
\definecolor{DarkTurquoise}{RGB}{0, 206, 209}

%\newtheorem{defRus}{Определение}
%\newtheorem{thmRus}{Теорема}
%s\newtheorem{corRus}{Следствие}


\setbeamercolor{background canvas}{bg=CharCoalDark}

\setbeamerfont{title}{series=\bfseries}
\setbeamercolor{title}{fg=Orange}
\setbeamercolor{section in toc}{fg=white}
\setbeamercolor{frametitle}{fg=Orange}
\setbeamercolor{normal text}{fg=white}
%\setbeamercolor{normal text}{fontsize=12pt}
\setbeamercolor{itemize item}{fg=Orange}
\setbeamercolor{enumerate item}{fg=Orange}
\setbeamercolor{enumerate item item}{fg=Orange}
\setbeamercolor{itemize item item}{fg=Orange}
\setbeamercolor{enumerate item}{fg=Orange}
\setbeamercolor{block title}{bg=DarkOrange,fg=white}
\setbeamerfont{block title}{series=\bfseries}

\setbeamertemplate{itemize item}[circle]
\setbeamertemplate{eumerate subitem}{\color{Orange}[$\checkmark$]}
\setbeamertemplate{itemize subitem}{\color{Orange}\Large$\textbullet$}
\setbeamertemplate{itemize subitem}{\color{Orange} \tiny $\blacksquare$}

% footnote without a marker
\newcommand\blfootnote[1]{%
	\begingroup
	\renewcommand\footnoterule{}
	\renewcommand\thefootnote{}\footnote{#1}%
	\addtocounter{footnote}{-1}%
	\endgroup
}

\newcommand*{\Scale}[2][4]{\scalebox{#1}{\ensuremath{#2}}}%

\newcommand\Item[1][]{%
	\ifx\relax#1\relax  \item \else \item[#1] \fi
	\abovedisplayskip=0pt\abovedisplayshortskip=0pt~\vspace*{-\baselineskip}}

\pgfdeclareradialshading{ring}{\pgfpoint{0cm}{0cm}}%
{rgb(0cm)=(1,1,1);
	rgb(0.7cm)=(1,1,1);
	rgb(0.719cm)=(1,1,1);
	rgb(0.72cm)=(0.975,0,0);
	rgb(0.9cm)=(1,1,1)}

\addtobeamertemplate{footline}{%
	\setlength\unitlength{1ex}%
	\begin{picture}(0,0) 
	% \put{} defines the position of the frame
	\put(155,0){\makebox(0,0)[bl]{
			%\includegraphics[scale=0.65]{white_square}
			%\includegraphics[scale=0.65]{dark_square}
			\includegraphics[scale=0.65]{grey_circle}
	}}%
	\end{picture}%
}{}


\usepackage[absolute,overlay]{textpos} %to clip to a corner
\newcommand\FrameText[1]{%
	\begin{textblock*}{\paperwidth}(\textwidth-35pt, 10 pt)
		\raggedright #1\hspace{.5em}
\end{textblock*}}

\makeatletter
\let\save@measuring@true\measuring@true
\def\measuring@true{%
	\save@measuring@true
	\def\beamer@sortzero##1{\beamer@ifnextcharospec{\beamer@sortzeroread{##1}}{}}%
	\def\beamer@sortzeroread##1<##2>{}%
	\def\beamer@finalnospec{}%
}
\makeatother

\AtBeginSection[]
{
	\begin{frame}<beamer>
		\frametitle{Outline}
		\tableofcontents[currentsection]
	\end{frame}
}


\title{Лекция №6 \\[10pt]
	Часть 2. Шифрование с аутентификацией.  Конструкции. }

\date{ Елена Киршанова \\  \textbf{Курс ``Основы криптографии''} \\  }



\setbeamertemplate{navigation symbols}{} %removes navigation

% proper highlightling of a code-snippet
\lstset{language=C++,
	keywordstyle=\color{magenta},
	stringstyle=\color{Goldenrod},
	commentstyle=\color{gray},
	breaklines=false,
	%morecomment=[l][\color{magenta}]{\#}
}

%\setlength{\parskip}{8pt}
\input{header} %all defs

\newcommand{\AxisRotator}[1][rotate=0]{%
	\tikz [x=0.5cm,y=1.9cm,line width=.2ex,-stealth,#1] \draw[color=Orange] (0,0) arc (-150:150:2 and 1);%
}

\begin{document}
	
\begin{frame}
	\titlepage
\end{frame}

\begin{frame}{Атака на выбранный шифр-текст}
	\Large 
	 В реальных протоколах атакующий может получить доступ к (частичной) функции {\color{Orange}  дешифрования}.  \\[10pt]
	 
	 Пример: атакующий может знать ответ на вопрос ``корректно ли сформирован шифр-текст'' (атака Bleichenbacher на RSA) \\[10pt]
	 
	 Симметрическое шифрование ``из коробки'' (CBC, CTR)  {\color{Orange}  не являются}  стойкими к таким атакам.  {\color{Orange} Причина:} измененный шифр-текст является  {\color{Orange}  корректным } шифр-текстом. \\[10pt]
	 
	 \centering
	 
	 {\color{Orange} Решение: } шифрование с аутентификацией
	 
	 
\end{frame}



\begin{frame}{Шифрование с аутентификацией стойко к CCA атаке на выбранный шифр-текст}
	\large
	\begin{center}
		
		\begin{tabular}{c c c}
			{\color{Orange} Челленджер $\mathcal{C}$ } & & {\color{Orange} Атакующий $\mathcal{A}$ }\\ [5pt]
			$k \leftarrow \KeyGen(1^\lambda)$ & $\xrightarrow{\quad \Huge \lambda \quad}$  &\\[5pt]
			$b \xleftarrow{\$} \{0,1\}  $& &\\ [5pt]
			& $\xleftarrow{\; \Huge m_{0, i}, m_{1, i}  \;}$  &$m_{0, i}, m_{1,i} \leftarrow \mesS $\\ [2pt]
			
			$c_i \leftarrow \Enc(k, m_{b, i})$ &  &\\ [2pt]
			$C = C \cup \{c_i\} $ & $\xrightarrow{\quad c_i \quad}$  &\\ [8pt]
			
			& $\xleftarrow{\; c_j \notin C   \;}$  &\\ [5pt]
			$\hat{m_j} \leftarrow \Dec(k, c_{j})$  &  $\xrightarrow{\quad c_i \quad}$ & \\[2pt]
			& $\xleftarrow{\quad \hat{b} \quad}$ & \\ [5pt]
		\end{tabular}
		\begin{tikzpicture}[overlay]
		\draw[fill=none, draw=white, opacity=0.5] (-8.3,-2.5) rectangle (-5.0,3.2); 
		\draw[fill=none, draw=white, opacity=0.5] (-3.0,-2.5) rectangle (0.0,3.2); 
		\node at (-3.8, -0.1) {\AxisRotator};
		\end{tikzpicture}
	\end{center}
	
	 {\color{Orange} Теорема:} 
	 \[
	 \begin{rcases*}
	 1. \text{\;  CPA безопасная схема} \quad \\
	 2. \text{\; целостность шифр-текста}  \quad
	 \end{rcases*} \implies \text{\; CCA безопасная схема}
	 \]
	 
\end{frame}

\begin{frame}{Целостность шифр-текста}
\Large 

	\begin{center}
	$\Pi = (\KeyGen, \Enc, \Dec)$ -- шифрование с аутентификацией.
	
	\vspace{20pt}
	
	\begin{tabular}{c c c}
		{\color{Orange} Челленджер $\mathcal{C}$ } & & {\color{Orange} Атакующий $\mathcal{A}$ }\\ [5pt]
		$k \leftarrow \KeyGen(1^\lambda)$ & &\\[5pt]
		& $\xleftarrow{\; \Huge m_{i}  \;}$  &$m_{i}\leftarrow \mesS $\\ [2pt]
		
		$c_i \leftarrow \Enc(k, m_{i})$ & $\xrightarrow{\; c_i \;}$ & $C = C \cup \{c_i\}$\\ [2pt]
		
		& $\xleftarrow{\; \hat{c} \notin C   \;}$  &\\ [5pt]
	\end{tabular}
	\begin{tikzpicture}[overlay]
	\draw[fill=none, draw=white, opacity=0.5] (-7.5,-1.5) rectangle (-4.2,2.0); 
	\draw[fill=none, draw=white, opacity=0.5] (-3.0,-1.5) rectangle (0.0,2.0); 
	\end{tikzpicture}
\end{center}


\vspace{15pt}

Выигрыш $\mathcal{A}$ -- событие $\Dec(\hat{c})$ возвращает не $\bot$.



\end{frame}

\begin{frame}{Конструкции шифрований с аутентификацией}
\Large 
\centering
{\color{Orange} AE = Безопасное шифрование + Криптографический MAC} \\[10pt]

Два ключа: Ключ шифрования $k_E$, ключ MACa $K_M$ \\
 
\begin{center}
	Две основные парадигмы:
\end{center}
\vspace{10pt}

\begin{columns}
	\begin{column}{0.5\linewidth}
	{\color{Orange} I. Encrypt-then-MAC}
	\begin{enumerate}
		\item $c = \Enc(k_E, m)$
		\item $t = \MAC(k_M, c)$
		\item return $(c,t)$
	\end{enumerate}
Пример: IPSec 
	\end{column} 
	\begin{column}{0.5\linewidth}
	 {\color{Orange} II. MAC-then-Encrypt}
		\begin{enumerate}
			\item $t = \MAC(k_M, n) $
			\item $c = \Enc(k_E, m||t)$
			\item return $c$
		\end{enumerate}
		Пример: SSL 
\end{column}
\end{columns}

\vspace{10pt}

\pause
\begin{itemize}
	\item Encrypt-then-MAC всегда даёт AE
	\item MAC-then-Encrypt даёт AE, если в $\Enc$ используются режимы \\ шифрования CTR/CBC
	\item Другие комбинации $\MAC$ / $\Enc$ обычно не дают безопасное AE
\end{itemize}

\end{frame}

\begin{frame}{AE стандарты}
\LARGE
\begin{enumerate}
	\itemsep 10 pt
	\item \textbf{GCM (Galois Counter Mode)}. {\Large {\color{Orange}Encrypt-then-MAC}  \\
		Шифрование: CTR mode + быстрый Mac (Carter-Wegman Mac).  \\
		Применение: TLS \\
		Преимущество: скорость
	 }
 \pause
 	\item \textbf{CCM}.  {\Large {\color{Orange}MAC-then-Encrypt}  \\
 		Шифрование: CBC MAC (AES)+ CTR mode (AES)  \\
 		Применение: 802.11i \\
 		Преимущество: компактный код
 	}
 \pause
 	\item \textbf{ChaCha20-Poly1305.} {\Large {\color{Orange} Encrypt-then-MAC}  \\
 		Шифрование: ChaCha20 (Enc) + Poly1305 MAC \\
 		Применение: TLS \\
 		Преимущество: скорость
 	}
\end{enumerate}
\end{frame}
\end{document}