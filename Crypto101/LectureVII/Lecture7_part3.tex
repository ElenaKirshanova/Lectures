\documentclass[usenames,dvipsnames,8pt,aspectratio=169]{beamer}
\usepackage{amsmath,amsfonts,amssymb}
\usepackage{mathtools}
\usepackage{etex} %for Windows
\usepackage[utf8]{inputenc}
\usepackage[english, russian]{babel} 
%\usepackage{microtype}			% Better interword spacing and additional kerning.
\usepackage{ellipsis}			% Adjusted space with \dots between two words.
\usepackage{graphicx}
\usepackage{pstricks}

\usepackage{xcolor}


\usepackage{changepage}

\usepackage{algorithm}
\usepackage{algpseudocode}
%\usepackage[]{algorithm2e}
%\usepackage{algorithmic}

%\usepackage{tcolorbox}


\usepackage{caption}
\usepackage{subcaption}
%\usepackage{stackengine}


\usepackage{tikz}
\usetikzlibrary{tikzmark,calc}
\usetikzlibrary{positioning, backgrounds}
\usetikzlibrary{arrows, chains, matrix, scopes, patterns, shapes, fit}
\usetikzlibrary{mindmap,trees,shadows}
\usetikzlibrary{decorations.pathreplacing}
%\usetikzlibrary{crypto.symbols}

\usepackage{pgfplots}

\pgfmathdeclarefunction{gauss}{2}{%
	\pgfmathparse{1/(#2*sqrt(2*pi))*exp(-((x-#1)^2)/(2*#2^2))}%
}


\tikzset{
	invisible/.style={opacity=0},
	visible on/.style={alt={#1{}{invisible}}},
	alt/.code args={<#1>#2#3}{%
		\alt<#1>{\pgfkeysalso{#2}}{\pgfkeysalso{#3}} % \pgfkeysalso doesn't change the path
	},
}

\newcommand\strikeout[2][]{%
	\begin{tabular}[b]{@{}c@{}} 
		\makebox(0,0)[cb]{{#1}} \\[-0.2\normalbaselineskip]
		\rlap{\color{Orange}\rule[0.5ex]{\widthof{#2}}{1.5pt}}#2
\end{tabular}}

\addtobeamertemplate{footline}{%
	\setlength\unitlength{1ex}%
	\begin{picture}(0,0) 
	% \put{} defines the position of the frame
	\put(155,0){\makebox(0,0)[bl]{
			%\includegraphics[scale=0.65]{white_square}
			%\includegraphics[scale=0.65]{dark_square}
			\includegraphics[scale=0.65]{grey_circle}
	}}%
	\end{picture}%
}{}

\newcommand\Fontvi{\fontsize{11}{13.2}\selectfont}

\usepackage{listings} % for C++ code

\usepackage{braket}
%\usepackage[braket, qm]{qcircuit}



\usepackage[T1]{fontenc}
%\usepackage[sfdefault,scaled=.85]{FiraSans}
%\usepackage{newtxsf}
%\usepackage[nomap]{FiraMono}





\usefonttheme[onlymath]{serif}
\renewcommand\sfdefault{cmbr}

\renewcommand{\bfdefault}{sb}

\definecolor{CharCoalDark}{RGB}{13, 16, 19}
\definecolor{Orange}{RGB}{255, 165,0}
\definecolor{DarkOrange}{RGB}{255, 165,0}
\definecolor{LightSalmon}{RGB}{255, 160, 122}
\definecolor{LeafGreen}{RGB}{34, 139,  34}
\definecolor{Coral}{RGB}{255, 127, 80}
\definecolor{DarkTurquoise}{RGB}{0, 206, 209}

%\newtheorem{defRus}{Определение}
%\newtheorem{thmRus}{Теорема}
%s\newtheorem{corRus}{Следствие}


\setbeamercolor{background canvas}{bg=CharCoalDark}

\setbeamerfont{title}{series=\bfseries}
\setbeamercolor{title}{fg=Orange}
\setbeamercolor{section in toc}{fg=white}
\setbeamercolor{frametitle}{fg=Orange}
\setbeamercolor{normal text}{fg=white}
%\setbeamercolor{normal text}{fontsize=12pt}
\setbeamercolor{itemize item}{fg=Orange}
\setbeamercolor{enumerate item}{fg=Orange}
\setbeamercolor{enumerate item item}{fg=Orange}
\setbeamercolor{itemize item item}{fg=Orange}
\setbeamercolor{enumerate item}{fg=Orange}
\setbeamercolor{block title}{bg=DarkOrange,fg=white}
\setbeamerfont{block title}{series=\bfseries}

\setbeamertemplate{itemize item}[circle]
\setbeamertemplate{eumerate subitem}{\color{Orange}[$\checkmark$]}
\setbeamertemplate{itemize subitem}{\color{Orange}\Large$\textbullet$}
\setbeamertemplate{itemize subitem}{\color{Orange} \tiny $\blacksquare$}

% footnote without a marker
\newcommand\blfootnote[1]{%
	\begingroup
	\renewcommand\footnoterule{}
	\renewcommand\thefootnote{}\footnote{#1}%
	\addtocounter{footnote}{-1}%
	\endgroup
}

\newcommand*{\Scale}[2][4]{\scalebox{#1}{\ensuremath{#2}}}%

\newcommand\Item[1][]{%
	\ifx\relax#1\relax  \item \else \item[#1] \fi
	\abovedisplayskip=0pt\abovedisplayshortskip=0pt~\vspace*{-\baselineskip}}

\pgfdeclareradialshading{ring}{\pgfpoint{0cm}{0cm}}%
{rgb(0cm)=(1,1,1);
	rgb(0.7cm)=(1,1,1);
	rgb(0.719cm)=(1,1,1);
	rgb(0.72cm)=(0.975,0,0);
	rgb(0.9cm)=(1,1,1)}

\usepackage[absolute,overlay]{textpos} %to clip to a corner
\newcommand\FrameText[1]{%
	\begin{textblock*}{\paperwidth}(\textwidth-35pt, 10 pt)
		\raggedright #1\hspace{.5em}
\end{textblock*}}

\makeatletter
\let\save@measuring@true\measuring@true
\def\measuring@true{%
	\save@measuring@true
	\def\beamer@sortzero##1{\beamer@ifnextcharospec{\beamer@sortzeroread{##1}}{}}%
	\def\beamer@sortzeroread##1<##2>{}%
	\def\beamer@finalnospec{}%
}
\makeatother

\AtBeginSection[]
{
	\begin{frame}<beamer>
		\frametitle{Outline}
		\tableofcontents[currentsection]
	\end{frame}
}


%\institute{ENS Lyon}
\title{Лекция №7 \\[10pt]
	Часть 3. Задача Диффи-Хэллмана. Протокол Диффи-Хэллмана}

\date{ Елена Киршанова \\  \textbf{Курс ``Основы криптографии''} \\  }




\setbeamertemplate{navigation symbols}{} %removes navigation

% proper highlightling of a code-snippet
\lstset{language=C++,
	keywordstyle=\color{magenta},
	stringstyle=\color{Goldenrod},
	commentstyle=\color{gray},
	breaklines=false,
	%morecomment=[l][\color{magenta}]{\#}
}

%\setlength{\parskip}{8pt}
\input{header} %all defs
\begin{document}
	
\begin{frame}
	\titlepage
\end{frame}



\begin{frame}{Трудные (сегодня) задачи в $\Zp$}
	\large
	\begin{enumerate}
		\itemsep 10pt
		\item {\color{Orange}{Задача дискретного логарифма} (dlog):} \\[3pt]
			Для $g$ -- образующего $\Zp^\ast$ и $x \in \Zp^\ast$, найти $r$ т.ч.\ $g^r = x \bmod p$ \\[5pt]
			Пример: Для $\langle 3 \rangle =  \Z_7^\ast$ и $5$ найти $r = 5$ ($3^5 = 5$).

		\item  {\color{Orange}{Задача Diffie-Hellman (вычислительная версия)} (CDH):} \\[3pt]
			Для $g$ -- образующего $\Zp^\ast$,  $x = g^r \in \Zp^\ast$, $y = g^t \in \Zp^\ast$ найти $z = g^{r\cdot t} \bmod p$. \\[5pt]
			Пример: Для $\langle 3 \rangle  = \Z_7^\ast$ и $x = 2, y = 6 $ найти $z = 3^5 = 5 \bmod p$.
			
		\item {\color{Orange}{Задача принятия решения  Diffie-Hellman} (DDH):} \\
		Для $g$ -- образующего $\Zp^\ast$, $a, b, c \xleftarrow{\$} \Zp^\ast$, отличить тройки
		\[
			(g^a, g^b, g^{ab}) \quad \quad (g^a, g^b, g^c)
		\]
		%\pause
		%\item If one can solve dlog, one can also solve CDH
		%\item The other direction is not known in general
		
		
		%\item {\color{Orange}{The hardness of both problems depend on the choice of $p$!} Not every prime gives a hard instance of dlog/CDH. }
	\end{enumerate}
\Large
\centering
\pause
\vspace{20pt}
{\color{Orange}
	DDH $\leq$ CDH $\leq$ DLOG \\[5pt]
}
A $\leq$ B = ``решение для $B$ дает решение для $A$'' \\[5pt]

В общем случае редукции в обратную сторону не известны.
\end{frame}

\begin{frame}{Сложность задач DLOG/CHD/DDH?}
	\Large 
	Лучший из известных на сегодня алгоритмов для вычисления DLOG в $\Zp^\ast$ для $n=\lceil\log p\rceil$: {\color{Orange}General Number Field Sieve}  работает за суб-экспоненциальное время  {\Huge \color{Orange}\[e^{n^{1/3}}\]}

	Для уровня безопасности {\color{Orange} $\lambda=128$} бит, необходимо взять
	\[
		n \approx {\color{Orange}3072} \text{ бит}
	\]
	%\vspace{10pt}
	
	Вместо группы $\Zp^\star$ на практике используется группа рациональных \\ точек эллиптической кривой. \\[5pt]
	{\color{Orange}Преимущество:}  известны лишь {\color{Orange} экспоненциальные} (от порядка \\ группы) алгоритмы для задачи dlog.
\end{frame}

\begin{frame}{Обмен ключами Диффи-Хэллмана}
\Large
\begin{center}
%Зафиксируем большое простое $p$, и $\langle g \rangle = \Zp^\ast$
$\mathtt{GenParam} \rightarrow (g, p)$, где $p$ -- большое простое число и$\langle g \rangle = \Zp^\ast$
\large 
	\begin{center}
		\begin{tabular}{l c c c l}
			& Алиса  & & Боб &  \\
			& \multirow{5}{*}{\includegraphics[scale=0.15]{Alice}} & & 
			\multirow{5}{*}{\includegraphics[scale=0.15]{Bob}} &    \\
			\pause
			$a \leftarrow \{2, \ldots, p-2 \}$ & & &  &  $b \leftarrow \{2, \ldots, p-2 \}$ \\
			\pause
			$k_\text{A} = g^a \bmod p$ & & $\xrightarrow{ \hspace{10pt } \Huge k_\text{A} \hspace{10pt } }$  &  & $k_\text{B} = g^b \bmod p$ \\
			\pause
		& & $\xleftarrow{ \hspace{10pt } \Huge k_\text{B} \hspace{10pt } }$  &  & \\
		\pause
		$k_\text{AB} =   k_\text{B}^a$ & &  &  & $k_\text{AB} =   k_\text{A}^b$ 
		\end{tabular}
	\end{center}
\vspace{15pt}
{\color{Orange}$k_\text{AB}$} -- общий ключ \\[10pt]
\end{center}
{\color{Orange}Корректность:} $k_\text{B}^a = (g^b)^a = g^{ab} = (g^a)^b=  k_\text{A}^b$.\\[5pt]

{\color{Orange}Безопасность (неформально): } атакующий видит траскрипт $g^a, g^b$. \\
Для того, чтобы различить $g^{ab}$ от случайного элемента $\Zp^\ast$, он \\ должен решить  задачу  {\color{Orange}DHH.}
\end{frame}

\begin{frame}{Безопасность протокола Диффи-Хэллмана}
\Large
\vspace{-35pt}

{\color{Orange} Теорема.} Протокол Диффи-Хэллмана безопасен относительно пассивного атакующего под предположением сложности задачи {\color{Orange}DHH.}

\vspace{15pt}

\begin{center}

	\begin{tabular}{c c c}
	{\color{Orange} Челленджер $\mathcal{C}$ } & & {\color{Orange} Атакующий $\mathcal{A}$ }\\ [5pt]
	$\mathtt{trans}  = (g^a, g^b), k_0 = g^{ab}$ & &\\ [2pt]
	$k_1 = g^c \rightarrow \Zp^\ast$ &  &\\ 
	$b \xleftarrow{\$} \{0,1\}  $&$\xrightarrow{(k_b, \mathtt{trans} )}$  &\\ 
	& $\xleftarrow{\quad \hat{b} \quad}$ & \\ [5pt]
\end{tabular}
\begin{tikzpicture}[overlay]
\draw[fill=none, draw=white, opacity=0.5] (-10.0,-2.) rectangle (-5.0,2.0); 
\draw[fill=none, draw=white, opacity=0.5] (-3.0,-2.) rectangle (0.0,2.0); 
\end{tikzpicture}
\end{center}
\end{frame}

\begin{frame}{Активная атака ``человек по середине'' (Man-in-the-middle attack)}
\Large
Протокол Диффи-Хэллмана в ``чистом виде'' подвержен {\color{Orange} активным} атакам.


\large
\begin{center}
	\begin{tabular}{l c c c c c l}
		& Alice  & & Eve & & Bob &  \\
		& \multirow{5}{*}{\includegraphics[scale=0.10]{Alice}} & & 
		\multirow{5}{*}{\includegraphics[scale=0.10]{Eve}} & &
		\multirow{5}{*}{\includegraphics[scale=0.10]{Bob}} &    \\
		
		$k_\text{A} = g^a \bmod p$ & & $\xrightarrow{ \hspace{5pt } \Huge k_\text{A} \hspace{5pt } }$  &  & $\xrightarrow{ \hspace{5pt } \Huge g^{a'} \hspace{5pt} }$ &  &$k_\text{B} = g^b \bmod p$ \\
		
		 & & $\xleftarrow{ \hspace{5pt } \Huge g^{b'} \hspace{5pt } }$  &  & $\xleftarrow{ \hspace{5pt } \Huge k_\text{B} \hspace{5pt} }$ &  &\\[20pt]
		 
		 &\Huge  $g^{ab'}$ & \multicolumn{3}{c} {\Huge $g^{ab'}$, $g^{a'b}$} & \Huge $g^{a'b}$ &\\
		\end{tabular}
\end{center}
\centering
\vfill
Фикс: использовать (ассиметрическую) аутентификацию.  \\ См. следующую лекцию
\end{frame}

\begin{frame}{Протокол Диффи-Хэллмана на практике}
\Large
\begin{itemize}
	\itemsep 10pt
	\item Атака ``человек по середине'' предотвращается с помощью цифровой подписи
	\item В реальных приложениях протокол ДХ вместо группы $\Zp^*$ использует группу рац, точек эллиптической кривой. Соответствующая задача:  {\color{Orange}{EDDH.}} 
	\item Построить подходящую группу для задач DLOG, CDH, DDH {\color{Orange}{нетривиальна!}}  Не придумывайте свою, используйте стандарты.
	\item См. \url{https://safecurves.cr.yp.to/} для выбора хорошей кривой
	\item ГОСТа для протокола обмена ключами нет, есть RFC \url{https://www.ietf.org/rfc/rfc5246.txt} и рекомендации.
\end{itemize}

\end{frame}



\end{document}
