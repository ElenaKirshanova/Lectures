\documentclass[usenames,dvipsnames,8pt,aspectratio=169]{beamer}
\usepackage{amsmath,amsfonts,amssymb}
\usepackage{mathtools}
\usepackage{etex} %for Windows
\usepackage[utf8]{inputenc}
\usepackage[english, russian]{babel} 
%\usepackage{microtype}			% Better interword spacing and additional kerning.
\usepackage{ellipsis}			% Adjusted space with \dots between two words.
\usepackage{graphicx}
\usepackage{pstricks}

\usepackage{xcolor}


\usepackage{changepage}

\usepackage{algorithm}
\usepackage{algpseudocode}
%\usepackage[]{algorithm2e}
%\usepackage{algorithmic}

%\usepackage{tcolorbox}


\usepackage{caption}
\usepackage{subcaption}
%\usepackage{stackengine}


\usepackage{tikz}
\usetikzlibrary{tikzmark,calc}
\usetikzlibrary{positioning, backgrounds}
\usetikzlibrary{arrows, chains, matrix, scopes, patterns, shapes, fit}
\usetikzlibrary{mindmap,trees,shadows}
\usetikzlibrary{decorations.pathreplacing}
%\usetikzlibrary{crypto.symbols}

\usepackage{pgfplots}

\pgfmathdeclarefunction{gauss}{2}{%
	\pgfmathparse{1/(#2*sqrt(2*pi))*exp(-((x-#1)^2)/(2*#2^2))}%
}


\tikzset{
	invisible/.style={opacity=0},
	visible on/.style={alt={#1{}{invisible}}},
	alt/.code args={<#1>#2#3}{%
		\alt<#1>{\pgfkeysalso{#2}}{\pgfkeysalso{#3}} % \pgfkeysalso doesn't change the path
	},
}

\newcommand\strikeout[2][]{%
	\begin{tabular}[b]{@{}c@{}} 
		\makebox(0,0)[cb]{{#1}} \\[-0.2\normalbaselineskip]
		\rlap{\color{Orange}\rule[0.5ex]{\widthof{#2}}{1.5pt}}#2
\end{tabular}}

\addtobeamertemplate{footline}{%
	\setlength\unitlength{1ex}%
	\begin{picture}(0,0) 
	% \put{} defines the position of the frame
	\put(155,0){\makebox(0,0)[bl]{
			%\includegraphics[scale=0.65]{white_square}
			%\includegraphics[scale=0.65]{dark_square}
			\includegraphics[scale=0.65]{grey_circle}
	}}%
	\end{picture}%
}{}

\newcommand\Fontvi{\fontsize{11}{13.2}\selectfont}

\usepackage{listings} % for C++ code

\usepackage{braket}
%\usepackage[braket, qm]{qcircuit}



\usepackage[T1]{fontenc}
%\usepackage[sfdefault,scaled=.85]{FiraSans}
%\usepackage{newtxsf}
%\usepackage[nomap]{FiraMono}





\usefonttheme[onlymath]{serif}
\renewcommand\sfdefault{cmbr}

\renewcommand{\bfdefault}{sb}

\definecolor{CharCoalDark}{RGB}{13, 16, 19}
\definecolor{Orange}{RGB}{255, 165,0}
\definecolor{DarkOrange}{RGB}{255, 165,0}
\definecolor{LightSalmon}{RGB}{255, 160, 122}
\definecolor{LeafGreen}{RGB}{34, 139,  34}
\definecolor{Coral}{RGB}{255, 127, 80}
\definecolor{DarkTurquoise}{RGB}{0, 206, 209}

%\newtheorem{defRus}{Определение}
%\newtheorem{thmRus}{Теорема}
%s\newtheorem{corRus}{Следствие}


\setbeamercolor{background canvas}{bg=CharCoalDark}

\setbeamerfont{title}{series=\bfseries}
\setbeamercolor{title}{fg=Orange}
\setbeamercolor{section in toc}{fg=white}
\setbeamercolor{frametitle}{fg=Orange}
\setbeamercolor{normal text}{fg=white}
%\setbeamercolor{normal text}{fontsize=12pt}
\setbeamercolor{itemize item}{fg=Orange}
\setbeamercolor{enumerate item}{fg=Orange}
\setbeamercolor{enumerate item item}{fg=Orange}
\setbeamercolor{itemize item item}{fg=Orange}
\setbeamercolor{enumerate item}{fg=Orange}
\setbeamercolor{block title}{bg=DarkOrange,fg=white}
\setbeamerfont{block title}{series=\bfseries}

\setbeamertemplate{itemize item}[circle]
\setbeamertemplate{eumerate subitem}{\color{Orange}[$\checkmark$]}
\setbeamertemplate{itemize subitem}{\color{Orange}\Large$\textbullet$}
\setbeamertemplate{itemize subitem}{\color{Orange} \tiny $\blacksquare$}

% footnote without a marker
\newcommand\blfootnote[1]{%
	\begingroup
	\renewcommand\footnoterule{}
	\renewcommand\thefootnote{}\footnote{#1}%
	\addtocounter{footnote}{-1}%
	\endgroup
}

\newcommand*{\Scale}[2][4]{\scalebox{#1}{\ensuremath{#2}}}%

\newcommand\Item[1][]{%
	\ifx\relax#1\relax  \item \else \item[#1] \fi
	\abovedisplayskip=0pt\abovedisplayshortskip=0pt~\vspace*{-\baselineskip}}

\pgfdeclareradialshading{ring}{\pgfpoint{0cm}{0cm}}%
{rgb(0cm)=(1,1,1);
	rgb(0.7cm)=(1,1,1);
	rgb(0.719cm)=(1,1,1);
	rgb(0.72cm)=(0.975,0,0);
	rgb(0.9cm)=(1,1,1)}

\usepackage[absolute,overlay]{textpos} %to clip to a corner
\newcommand\FrameText[1]{%
	\begin{textblock*}{\paperwidth}(\textwidth-35pt, 10 pt)
		\raggedright #1\hspace{.5em}
\end{textblock*}}

\makeatletter
\let\save@measuring@true\measuring@true
\def\measuring@true{%
	\save@measuring@true
	\def\beamer@sortzero##1{\beamer@ifnextcharospec{\beamer@sortzeroread{##1}}{}}%
	\def\beamer@sortzeroread##1<##2>{}%
	\def\beamer@finalnospec{}%
}
\makeatother

\AtBeginSection[]
{
	\begin{frame}<beamer>
		\frametitle{Outline}
		\tableofcontents[currentsection]
	\end{frame}
}


%\institute{ENS Lyon}
\title{Лекция №7 \\[10pt]
	Часть 4. PAKE. Парольный обмен ключами с аутентификацией}

\date{ Елена Киршанова \\  \textbf{Курс ``Основы криптографии''} \\  }




\setbeamertemplate{navigation symbols}{} %removes navigation

% proper highlightling of a code-snippet
\lstset{language=C++,
	keywordstyle=\color{magenta},
	stringstyle=\color{Goldenrod},
	commentstyle=\color{gray},
	breaklines=false,
	%morecomment=[l][\color{magenta}]{\#}
}

%\setlength{\parskip}{8pt}
\input{header} %all defs
\begin{document}
	
\begin{frame}
	\titlepage
\end{frame}



\begin{frame}{Password Authenticated Key Exchange (PAKE): мотивация}
	\Large
	\begin{itemize}
		\itemsep 10pt
		\item Классический Диффи-Хэллмана (ДХ) протокол уязвим к активной атаке
		\item Безопасная реализация ДХ подразумевает аутентифицированный канал 
		\item Аутентификация:  либо { \color{Orange}доверенная сторона}, либо  {\color{Orange} парольные решения }
	\end{itemize}
%\centering 

\vspace{20pt}

{\color{Orange} Сценарий}: две стороны, сервер и клиент, знают пароль $pwd$ \\[10pt]
{\color{Orange} Задача}: обезопасить клиента от атакующего, представляющегося \\ сервером. 
\end{frame}

\begin{frame}{Упрощенная версия $\text{PAKE}_0$}
\vspace{-20pt}
	\Large
	\begin{center}
		%Зафиксируем большое простое $p$, и $\langle g \rangle = \Zp^\ast$
		$\Hash$-- криптографическая хэш-функция. $R, S$--множество нонсов \\[5pt]
		$\mathtt{id}_{A}$ -- идентификатор Алисы, $\mathtt{id}_{B}$ -- идентификатор Боба\\[10pt]
		
		\large 
		\begin{center}
			\begin{tabular}{l c c c l}
				& Алиса (клиент)  & & Боб (сервер) &  \\
				& \multirow{5}{*}{\includegraphics[scale=0.15]{Alice}} & {\huge $pwd$} &  \hspace{-20pt}
				\multirow{5}{*}{\includegraphics[scale=0.15]{Bob}} &    \\[10pt]
				$r \xleftarrow{\$} R$ & & {\huge $\xrightarrow{\quad r \quad }$}&  & \\
				 & &  {\huge $\xleftarrow{\quad s \quad }$}&  & \hspace{-30pt} $s \xleftarrow{\$} S$ \\
				${\color{Orange}k} =\Hash(\mathtt{id}_{A}, \mathtt{id}_{B}, r, s, pwd) $ & &  &  & \hspace{-30pt} ${\color{Orange}k} =\Hash(\mathtt{id}_{A}, \mathtt{id}_{B}, r, s, pwd) $ \\
				\end{tabular}
		\end{center}
		\vspace{15pt}
		{\color{Orange}$k$} -- общий ключ \\[10pt]
	\end{center}

	{\color{Orange}Проблема: } Если $pwd$ легко угадывается, пассивный злоумышленник  \\ может легко провести атаку перебором на {\color{Orange}$k$}

\end{frame}
\begin{frame}{Упрощенная версия $\text{PAKE}_1$}
\Large
\vspace{-20pt}
\begin{center}
	$\Hash$-- криптографическая хэш-функция \\[2pt]
	$\mathtt{id}_{A}$ -- идентификатор Алисы, $\mathtt{id}_{B}$ -- идентификатор Боба\\[2pt]
	$\mathtt{GenParam} \rightarrow (g, p)$, где $p$ -- большое простое число и$\langle g \rangle = \Zp^\ast$
	\large 
	\begin{center}
		\begin{tabular}{l c c c l}
			& Алиса (клиент)  & & Боб (сервер) &  \\
			& \multirow{2}{*}{\includegraphics[scale=0.15]{Alice}} & {\huge $pwd$} &  \hspace{-20pt}
			\multirow{2}{*}{\includegraphics[scale=0.15]{Bob}} &    \\[10pt]
			$ a \leftarrow \{2, \ldots, p-2 \}$ & & {\Huge $\xrightarrow{\quad u  \quad }$}&  &\hspace{-30pt} $b \leftarrow \{2, \ldots, p-2 \}$ \\
			\huge $w = v^a$& &  {\Huge $\xleftarrow{\quad v  \quad }$}&  & \hspace{-30pt}\huge $v = u^b$ \\[20pt]
			\multicolumn{5}{c}{ \LARGE ${\color{Orange}k} =\Hash(\mathtt{id}_{A}, \mathtt{id}_{B}, u, v, w, pwd) $}   \\
		\end{tabular}
	\end{center}
	%\vspace{15pt}
\end{center}
	$\text{PAKE}_1$ безопасен относительно {\color{Orange} пассивных} злоумышленников  при \\ условии сложности задачи CDH. \\[5pt]
	$\text{PAKE}_1$  небезопасен относительно {\color{Orange}активного} атакующего, выдающего \\ себя за сервер
\end{frame}


\begin{frame}{ Версия $\text{PAKE}_2$}
\Large
\vspace{-20pt}
\begin{center}
	$\Hash$-- криптографическая хэш-функция  \\[3pt]
	$\mathtt{id}_{A}$ -- идентификатор Алисы, $\mathtt{id}_{B}$ -- идентификатор Боба\\[3pt]
	Зафиксируем большое простое $p$, и $\langle g \rangle = \Zp^\ast$, $\color{Orange} \alpha, \beta \in \Zp^\ast $\\
	\large 
	\begin{center}
		\begin{tabular}{l c c c l}
			& Алиса (клиент)  & & Боб (сервер) &  \\
			& \multirow{2}{*}{\includegraphics[scale=0.15]{Alice}} & { ${\color{Orange}pwd} \in \Zp^\ast $} &  \hspace{-20pt}
			\multirow{2}{*}{\includegraphics[scale=0.15]{Bob}} &    \\[10pt]
			$ a \leftarrow \{2, \ldots, p-2 \}$ & & &  &\hspace{-30pt} $b \leftarrow \{2, \ldots, p-2 \}$ \\[3pt]
			\huge ${\color{Orange} u = g^a \cdot \alpha^{pwd}} $ & & {\Huge $\xrightarrow{\quad u \quad }$}& & \hspace{-30pt}  \huge ${\color{Orange} v = g^b \cdot \beta^{pwd}} $ \\[3pt]
			\huge $ {\color{Orange}w = (v/\beta^{pwd})^a}$& &  {\Huge $\xleftarrow{\quad v \quad }$}&  & \huge \hspace{-30pt} ${\color{Orange}w = (u/\alpha^{pwd})^b}$ \\[17pt]
			\multicolumn{5}{c}{ \LARGE ${\color{Orange}k} =\Hash(\mathtt{id}_{A}, \mathtt{id}_{B}, u, v, w, pwd) $}   \\
		\end{tabular}
	\end{center}
	%
\end{center}

\vspace{20pt}

$\text{PAKE}_2$ безопасен относительно {\color{Orange} активных} злоумышленников \\ при условии сложности задачи CDH \\


\end{frame}

\begin{frame}{PAKE на практике}
	\Large
	\begin{itemize}
		\itemsep 10pt
			\item Пароли на сервере хранятся в виде $g^{\Hash(\text{salt}, pwd)}$ (атакующий, получив это значение, не может эффективно вычислить $pwd$)
			\item Популярный пример PAKE: Secure Remote Password (SRP) (iCloud Key Vault)
			\item Другой пример: OPAQUE \url{https://tools.ietf.org/html/draft-krawczyk-cfrg-opaque-00} -- криптографический стойкий и эффективный PAKE
	\end{itemize}
\end{frame}


\end{document}
