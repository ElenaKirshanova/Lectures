\documentclass[usenames,dvipsnames,8pt,aspectratio=169]{beamer}
\usepackage{amsmath,amsfonts,amssymb}
\usepackage{mathtools}
\usepackage{etex} %for Windows
\usepackage[utf8]{inputenc}
\usepackage[english, russian]{babel} 
%\usepackage{microtype}			% Better interword spacing and additional kerning.
\usepackage{ellipsis}			% Adjusted space with \dots between two words.
\usepackage{graphicx}
\usepackage{pstricks}

\usepackage{xcolor}


\usepackage{changepage}

\usepackage{algorithm}
\usepackage{algpseudocode}
%\usepackage[]{algorithm2e}
%\usepackage{algorithmic}

%\usepackage{tcolorbox}


\usepackage{caption}
\usepackage{subcaption}
%\usepackage{stackengine}


\usepackage{tikz}
\usetikzlibrary{tikzmark,calc}
\usetikzlibrary{positioning, backgrounds}
\usetikzlibrary{arrows, chains, matrix, scopes, patterns, shapes, fit}
\usetikzlibrary{mindmap,trees,shadows}
\usetikzlibrary{decorations.pathreplacing}
%\usetikzlibrary{crypto.symbols}

\usepackage{pgfplots}

\pgfmathdeclarefunction{gauss}{2}{%
	\pgfmathparse{1/(#2*sqrt(2*pi))*exp(-((x-#1)^2)/(2*#2^2))}%
}


\tikzset{
	invisible/.style={opacity=0},
	visible on/.style={alt={#1{}{invisible}}},
	alt/.code args={<#1>#2#3}{%
		\alt<#1>{\pgfkeysalso{#2}}{\pgfkeysalso{#3}} % \pgfkeysalso doesn't change the path
	},
}

\newcommand\strikeout[2][]{%
	\begin{tabular}[b]{@{}c@{}} 
		\makebox(0,0)[cb]{{#1}} \\[-0.2\normalbaselineskip]
		\rlap{\color{Orange}\rule[0.5ex]{\widthof{#2}}{1.5pt}}#2
\end{tabular}}

\addtobeamertemplate{footline}{%
	\setlength\unitlength{1ex}%
	\begin{picture}(0,0) 
	% \put{} defines the position of the frame
	\put(155,0){\makebox(0,0)[bl]{
			%\includegraphics[scale=0.65]{white_square}
			%\includegraphics[scale=0.65]{dark_square}
			\includegraphics[scale=0.65]{grey_circle}
	}}%
	\end{picture}%
}{}

\newcommand\Fontvi{\fontsize{11}{13.2}\selectfont}

\usepackage{listings} % for C++ code

\usepackage{braket}
%\usepackage[braket, qm]{qcircuit}



\usepackage[T1]{fontenc}
%\usepackage[sfdefault,scaled=.85]{FiraSans}
%\usepackage{newtxsf}
%\usepackage[nomap]{FiraMono}





\usefonttheme[onlymath]{serif}
\renewcommand\sfdefault{cmbr}

\renewcommand{\bfdefault}{sb}

\definecolor{CharCoalDark}{RGB}{13, 16, 19}
\definecolor{Orange}{RGB}{255, 165,0}
\definecolor{DarkOrange}{RGB}{255, 165,0}
\definecolor{LightSalmon}{RGB}{255, 160, 122}
\definecolor{LeafGreen}{RGB}{34, 139,  34}
\definecolor{Coral}{RGB}{255, 127, 80}
\definecolor{DarkTurquoise}{RGB}{0, 206, 209}

%\newtheorem{defRus}{Определение}
%\newtheorem{thmRus}{Теорема}
%s\newtheorem{corRus}{Следствие}


\setbeamercolor{background canvas}{bg=CharCoalDark}

\setbeamerfont{title}{series=\bfseries}
\setbeamercolor{title}{fg=Orange}
\setbeamercolor{section in toc}{fg=white}
\setbeamercolor{frametitle}{fg=Orange}
\setbeamercolor{normal text}{fg=white}
%\setbeamercolor{normal text}{fontsize=12pt}
\setbeamercolor{itemize item}{fg=Orange}
\setbeamercolor{enumerate item}{fg=Orange}
\setbeamercolor{enumerate item item}{fg=Orange}
\setbeamercolor{itemize item item}{fg=Orange}
\setbeamercolor{enumerate item}{fg=Orange}
\setbeamercolor{block title}{bg=DarkOrange,fg=white}
\setbeamerfont{block title}{series=\bfseries}

\setbeamertemplate{itemize item}[circle]
\setbeamertemplate{eumerate subitem}{\color{Orange}[$\checkmark$]}
\setbeamertemplate{itemize subitem}{\color{Orange}\Large$\textbullet$}
\setbeamertemplate{itemize subitem}{\color{Orange} \tiny $\blacksquare$}

% footnote without a marker
\newcommand\blfootnote[1]{%
	\begingroup
	\renewcommand\footnoterule{}
	\renewcommand\thefootnote{}\footnote{#1}%
	\addtocounter{footnote}{-1}%
	\endgroup
}

\newcommand*{\Scale}[2][4]{\scalebox{#1}{\ensuremath{#2}}}%

\newcommand\Item[1][]{%
	\ifx\relax#1\relax  \item \else \item[#1] \fi
	\abovedisplayskip=0pt\abovedisplayshortskip=0pt~\vspace*{-\baselineskip}}

\pgfdeclareradialshading{ring}{\pgfpoint{0cm}{0cm}}%
{rgb(0cm)=(1,1,1);
	rgb(0.7cm)=(1,1,1);
	rgb(0.719cm)=(1,1,1);
	rgb(0.72cm)=(0.975,0,0);
	rgb(0.9cm)=(1,1,1)}

\usepackage[absolute,overlay]{textpos} %to clip to a corner
\newcommand\FrameText[1]{%
	\begin{textblock*}{\paperwidth}(\textwidth-35pt, 10 pt)
		\raggedright #1\hspace{.5em}
\end{textblock*}}

\makeatletter
\let\save@measuring@true\measuring@true
\def\measuring@true{%
	\save@measuring@true
	\def\beamer@sortzero##1{\beamer@ifnextcharospec{\beamer@sortzeroread{##1}}{}}%
	\def\beamer@sortzeroread##1<##2>{}%
	\def\beamer@finalnospec{}%
}
\makeatother

\AtBeginSection[]
{
	\begin{frame}<beamer>
		\frametitle{Outline}
		\tableofcontents[currentsection]
	\end{frame}
}


%\institute{ENS Lyon}
\title{Лекция №7 \\[10pt]
	Часть 2. Крэш-курс по конечным полям.}

\date{ Елена Киршанова \\  \textbf{Курс ``Основы криптографии''} \\  }




\setbeamertemplate{navigation symbols}{} %removes navigation

% proper highlightling of a code-snippet
\lstset{language=C++,
	keywordstyle=\color{magenta},
	stringstyle=\color{Goldenrod},
	commentstyle=\color{gray},
	breaklines=false,
	%morecomment=[l][\color{magenta}]{\#}
}

%\setlength{\parskip}{8pt}
\input{header} %all defs
\begin{document}
	
\begin{frame}
	\titlepage
\end{frame}



\begin{frame}{Модульная арифметика}
	\Large 
	{\color{Orange} Возьмем $p$ -- большое простое число ($\sim$\ 2000 бит)}
	\begin{itemize}
		\item $\Zp = \left\{ 0, 1, \ldots, p-1 \right\}$ -- конечное поле
		\item умножение/сложение в $\Zp$ производится по модулю $p$, т.е.\ для $x, y \in \Zp$
	\begin{align*}
			x+y \bmod p &= \rem(x+y, p) \\
			x\cdot y \bmod p &= \rem(x\cdot y, p)
	\end{align*}
	{\hspace{90pt} \large $\rem $ -- остаток от целочисл.\ деления}\\[10pt]
	\pause
	
	Пример: $p  = 7, \Zp = \{0, 1, 2, 3, 4, 5, 6\}$
	\begin{align*}
	5+6 \bmod p &= \rem(11, 7) = 4 \\
	3\cdot 3  \bmod p &= \rem(9, 7) = 2
	\end{align*}
	\pause
	\item Для ненулевого $x \in \Zp$, $\exists x^{-1} \in \Zp$: $x \cdot x^{-1} = 1 \bmod p$\\
	Пример.:  $1^{-1} = 1$, $2^{-1} = 4$, $3^{-1} = 5$,  $4^{-1} = 2$, $5^{-1} = 3$, $6^{-1} = 6$ в $\Z_7$
	\item Множество обратимых элементов $\Zp^{\ast} = \Zp \setminus \{0\}$
	\end{itemize}

\end{frame}

\begin{frame}{Структура $\Zp^\ast$ }
\Large
\begin{itemize}
	\itemsep 10pt
	\item {\color{Orange} Теорма Ферма:} $g^{p-1} = 1 \bmod p \; \; \forall 0 \neq g \in \Zp$\\[5pt]
	Пример.: $2^6 = 64 = 1 \bmod 7$
	\item $\Zp^\ast$ --  {\color{Orange} циклическая группа}, т.e., \\
	$\exists g \in \Zp$ s.t.\ $\Zp^\ast = \{1 = g^0, g^1, g^2, \ldots, g^{p-2}\}$ \\[5pt]
	Пример: $\Z_7^\ast = \{1, 3, 3^2 = 2, 3^3 = 6, 3^4 = 4, 3^5 = 5\}$ 

	\item Не каждый элемент является образующим  $\Zp$, но мы знаем, как его эффективно отыскать

	\item {\color{Orange} Порядок} $g \in \Zp$, $\ord(g)$ -- {\color{Orange} наименьшее} положительное $a$ т.ч.\ $g^a = 1$ \\[5pt]
	Для любого образующего $g$, $\ord(g) = p-1$.
\end{itemize}
\end{frame}

\begin{frame}{Вычисления в $\Zp$ }
	\large
	Эффективность арифметики в $\Zp$ измеряется в $n = \lceil \log p \rceil$
	\begin{itemize}
		\itemsep 5pt
		\item Сложение: $\bigO(n)$ битовых операций
		\item Умножение: $\bigO(n^2)$ (или $\bigO(n^{1.7})$) битовых операций
		\item Нахождение обратного: $\bigO(n^2)$ битовых операций
		\pause
		\item Возведение в степень, $x^r$:  $\bigO(\log r)$ умножений в $\Zp$ (быстрое возведение в степень)\\
		\begin{itemize}
			\Large
			\item $y \leftarrow g$, $z \leftarrow 1$
			\item \texttt{for $i$ in $[0, n]$:}
			\item \hspace{10pt}  \texttt{if $r[i]==1$ :} $z \leftarrow z \cdot y$
			\item \hspace{10pt}  $y \leftarrow y^2$
			\item \texttt{return} $z$
		\end{itemize}
		\pause 
		Пример: вычислим $g^r$ для $r = 23 = (10111)_2$. I.e., $g^{23} = g^{16+4+2+1}$. \\
		\[
			g^1 \rightarrow g^{1+2} \rightarrow g^{1+2+4} \rightarrow  g^{1+2+4} \rightarrow g^{1+2+4+16} 
		\]
	
	\end{itemize}

\vspace{15pt}
\centering
\Large
Эти операции {\color{Orange} {эффективны}} в $\Zp$
\end{frame}



\end{document}
