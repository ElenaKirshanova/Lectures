\documentclass[usenames,dvipsnames,8pt,aspectratio=169]{beamer}
\usepackage{amsmath,amsfonts,amssymb}
\usepackage{mathtools}
\usepackage{etex} %for Windows
\usepackage[utf8]{inputenc}
\usepackage[english, russian]{babel} 
%\usepackage{microtype}			% Better interword spacing and additional kerning.
\usepackage{ellipsis}			% Adjusted space with \dots between two words.
\usepackage{graphicx}
\usepackage{pstricks}

\usepackage{xcolor}


\usepackage{changepage}

\usepackage{algorithm}
\usepackage{algpseudocode}
%\usepackage[]{algorithm2e}
%\usepackage{algorithmic}

%\usepackage{tcolorbox}


\usepackage{caption}
\usepackage{subcaption}
%\usepackage{stackengine}


\usepackage{tikz}
\usetikzlibrary{tikzmark,calc}
\usetikzlibrary{positioning, backgrounds}
\usetikzlibrary{arrows, chains, matrix, scopes, patterns, shapes, fit}
\usetikzlibrary{mindmap,trees,shadows}
\usetikzlibrary{decorations.pathreplacing}
%\usetikzlibrary{crypto.symbols}

\usepackage{pgfplots}

\pgfmathdeclarefunction{gauss}{2}{%
	\pgfmathparse{1/(#2*sqrt(2*pi))*exp(-((x-#1)^2)/(2*#2^2))}%
}


\tikzset{
	invisible/.style={opacity=0},
	visible on/.style={alt={#1{}{invisible}}},
	alt/.code args={<#1>#2#3}{%
		\alt<#1>{\pgfkeysalso{#2}}{\pgfkeysalso{#3}} % \pgfkeysalso doesn't change the path
	},
}

\newcommand\strikeout[2][]{%
	\begin{tabular}[b]{@{}c@{}} 
		\makebox(0,0)[cb]{{#1}} \\[-0.2\normalbaselineskip]
		\rlap{\color{Orange}\rule[0.5ex]{\widthof{#2}}{1.5pt}}#2
\end{tabular}}

\addtobeamertemplate{footline}{%
	\setlength\unitlength{1ex}%
	\begin{picture}(0,0) 
	% \put{} defines the position of the frame
	\put(155,0){\makebox(0,0)[bl]{
			%\includegraphics[scale=0.65]{white_square}
			%\includegraphics[scale=0.65]{dark_square}
			\includegraphics[scale=0.65]{grey_circle}
	}}%
	\end{picture}%
}{}

\newcommand\Fontvi{\fontsize{11}{13.2}\selectfont}

\usepackage{listings} % for C++ code

\usepackage{braket}
%\usepackage[braket, qm]{qcircuit}



\usepackage[T1]{fontenc}
%\usepackage[sfdefault,scaled=.85]{FiraSans}
%\usepackage{newtxsf}
%\usepackage[nomap]{FiraMono}





\usefonttheme[onlymath]{serif}
\renewcommand\sfdefault{cmbr}

\renewcommand{\bfdefault}{sb}

\definecolor{CharCoalDark}{RGB}{13, 16, 19}
\definecolor{Orange}{RGB}{255, 165,0}
\definecolor{DarkOrange}{RGB}{255, 165,0}
\definecolor{LightSalmon}{RGB}{255, 160, 122}
\definecolor{LeafGreen}{RGB}{34, 139,  34}
\definecolor{Coral}{RGB}{255, 127, 80}
\definecolor{DarkTurquoise}{RGB}{0, 206, 209}

%\newtheorem{defRus}{Определение}
%\newtheorem{thmRus}{Теорема}
%s\newtheorem{corRus}{Следствие}


\setbeamercolor{background canvas}{bg=CharCoalDark}

\setbeamerfont{title}{series=\bfseries}
\setbeamercolor{title}{fg=Orange}
\setbeamercolor{section in toc}{fg=white}
\setbeamercolor{frametitle}{fg=Orange}
\setbeamercolor{normal text}{fg=white}
%\setbeamercolor{normal text}{fontsize=12pt}
\setbeamercolor{itemize item}{fg=Orange}
\setbeamercolor{enumerate item}{fg=Orange}
\setbeamercolor{enumerate item item}{fg=Orange}
\setbeamercolor{itemize item item}{fg=Orange}
\setbeamercolor{enumerate item}{fg=Orange}
\setbeamercolor{block title}{bg=DarkOrange,fg=white}
\setbeamerfont{block title}{series=\bfseries}

\setbeamertemplate{itemize item}[circle]
\setbeamertemplate{eumerate subitem}{\color{Orange}[$\checkmark$]}
\setbeamertemplate{itemize subitem}{\color{Orange}\Large$\textbullet$}
\setbeamertemplate{itemize subitem}{\color{Orange} \tiny $\blacksquare$}

% footnote without a marker
\newcommand\blfootnote[1]{%
	\begingroup
	\renewcommand\footnoterule{}
	\renewcommand\thefootnote{}\footnote{#1}%
	\addtocounter{footnote}{-1}%
	\endgroup
}

\newcommand*{\Scale}[2][4]{\scalebox{#1}{\ensuremath{#2}}}%

\newcommand\Item[1][]{%
	\ifx\relax#1\relax  \item \else \item[#1] \fi
	\abovedisplayskip=0pt\abovedisplayshortskip=0pt~\vspace*{-\baselineskip}}

\pgfdeclareradialshading{ring}{\pgfpoint{0cm}{0cm}}%
{rgb(0cm)=(1,1,1);
	rgb(0.7cm)=(1,1,1);
	rgb(0.719cm)=(1,1,1);
	rgb(0.72cm)=(0.975,0,0);
	rgb(0.9cm)=(1,1,1)}

\usepackage[absolute,overlay]{textpos} %to clip to a corner
\newcommand\FrameText[1]{%
	\begin{textblock*}{\paperwidth}(\textwidth-35pt, 10 pt)
		\raggedright #1\hspace{.5em}
\end{textblock*}}

\makeatletter
\let\save@measuring@true\measuring@true
\def\measuring@true{%
	\save@measuring@true
	\def\beamer@sortzero##1{\beamer@ifnextcharospec{\beamer@sortzeroread{##1}}{}}%
	\def\beamer@sortzeroread##1<##2>{}%
	\def\beamer@finalnospec{}%
}
\makeatother

\AtBeginSection[]
{
	\begin{frame}<beamer>
		\frametitle{Outline}
		\tableofcontents[currentsection]
	\end{frame}
}


%\institute{ENS Lyon}
\title{Лекция №7 \\[10pt]
	Часть 1. Протокол обмена ключами.}

\date{ Елена Киршанова \\  \textbf{Курс ``Основы криптографии''} \\  }




\setbeamertemplate{navigation symbols}{} %removes navigation

% proper highlightling of a code-snippet
\lstset{language=C++,
	keywordstyle=\color{magenta},
	stringstyle=\color{Goldenrod},
	commentstyle=\color{gray},
	breaklines=false,
	%morecomment=[l][\color{magenta}]{\#}
}

%\setlength{\parskip}{8pt}
\input{header} %all defs
\begin{document}
	
\begin{frame}
	\titlepage
\end{frame}

\begin{frame}{До сих пор...}
	\Large 
	\begin{itemize}
		\item  {\color{Orange}Псевдослучайные генераторы} 
		\item {\color{Orange} Потоковые шифры} 
		\item {\color{Orange} Блок-шифры } 
		\item {\color{Orange} Коды Аутентификации сообщений (MAC)}
		\item {\color{Orange} Хэш-функции}  
		\item {\color{Orange} Шифрование с аутентификацией }  
	\end{itemize}

\vspace{10pt}
\centering
 
Кроме псевдослучайных генераторов, \textbf{все} примитивы подразумевают наличие {\color{Orange} общего секретного  ключа.}\\[10pt]
Как двум сторонам сгенерировать секретный ключ? \\[10pt]



\end{frame}

\begin{frame}{Обмен ключами Диффи-Хэллмана (Diffie-Hellman Key Exchange)}
\centering
\begin{columns}
	\begin{column}{0.5\textwidth}
		\includegraphics[width=0.65\linewidth]{Whit_Diffie}\\
		Whitfield Diffie  \textcopyright Wikipedia
	\end{column}
	\begin{column}{0.5\textwidth}
	\includegraphics[width=0.6\linewidth]{Martin-Hellman}\\
	Martin Hellman  	\small \textcopyright Wikipedia
\end{column}
\end{columns}
\vspace{15pt}
\large
	\begin{itemize}
		\item протокол предложен  Whitfield Diffie, Martin Hellman в 1976 \\
		\textit{``New Directions in Cryptography''}
		\item используется в современных протоколах (TLS, Signal)
		\item в основе  трудность {\color{Orange} задачи дискретного логарифма}
	\end{itemize}
\end{frame}

\begin{frame}{Обмен ключами: определение}
\Large
\vspace{-40pt}
{\color{Orange} Сценарий:} Алиса и Боб имеют общий канал связи. Их цель: сгенерировать общий секретный ключ \\[10pt]

{\color{Orange} Злоумышленник } -- пассивный участник, просматривающий канал связи \\[20pt]


Протокол {\color{Orange} обмена ключами} -- ppt интерактивный алгоритм, состоящий из двух алгоритмов: \\
\begin{enumerate}
	\item $\mathtt{GenParam}(1^\lambda) \rightarrow \mathtt{param}$ 
	\item $\mathtt{KeyGen}(\mathtt{param}) \rightarrow k \in \keyS$ -- интерактивный протокол. \\
	
\end{enumerate} 

Передаваемая информация по каналу -- $\mathtt{trans}$
	
\end{frame}

\begin{frame}{Обмен ключами: безопасность}
\Large

		\begin{center}
	$\Pi = (\mathtt{GenParam}, \mathtt{KeyGen})$ --  протокол обмена ключами \\[20pt]
	

		
		\begin{tabular}{c c c}
			{\color{Orange} Челленджер $\mathcal{C}$ } & & {\color{Orange} Атакующий $\mathcal{A}$ }\\ [5pt]
			$k_0, \mathtt{trans} \leftarrow \Pi(1^\lambda)$ & &\\ [2pt]
			$k_1 \xleftarrow{\$} \keyS$ &  &\\ 
			$b \xleftarrow{\$} \{0,1\}  $&$\xrightarrow{(k_b, \mathtt{trans} )}$  &\\ 
			& $\xleftarrow{\quad \hat{b} \quad}$ & \\ [5pt]
		\end{tabular}
		\begin{tikzpicture}[overlay]
		\draw[fill=none, draw=white, opacity=0.5] (-8.8,-2.) rectangle (-5.0,2.0); 
		\draw[fill=none, draw=white, opacity=0.5] (-3.0,-2.) rectangle (0.0,2.0); 
		\end{tikzpicture}
	\end{center}

\vspace{10pt}

$\mathtt{W_{\Pi, \adv}}$ -- событие $\hat{b} == b$. \\ [4pt]
$\mathtt{Adv} = \abs{\Pr[\mathtt{W_{\Pi, \adv}}]  - \frac{1}{2}}$ -- выигрыш  $\adv$. \\ [4pt]
\color{Orange} Протокол обмена ключами $\Pi$ безопасен относительно \\ пассивного атакующего, если для любого ppt $\adv:$  \[\mathtt{Adv} = \negl(\lambda).\]

\end{frame}





\end{document}
