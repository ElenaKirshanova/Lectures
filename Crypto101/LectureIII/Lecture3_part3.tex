\documentclass[usenames,dvipsnames,8pt,aspectratio=169]{beamer}
\usepackage{amsmath,amsfonts,amssymb}
\usepackage{mathtools}
\usepackage{etex} %for Windows
\usepackage[utf8]{inputenc}
\usepackage[english, russian]{babel} 
%\usepackage{microtype}			% Better interword spacing and additional kerning.
\usepackage{ellipsis}			% Adjusted space with \dots between two words.
\usepackage{graphicx}
\usepackage{pstricks}

\usepackage{xcolor}


\usepackage{changepage}

\usepackage{algorithm}
\usepackage{algpseudocode}
%\usepackage[]{algorithm2e}
%\usepackage{algorithmic}

%\usepackage{tcolorbox}



\addtobeamertemplate{footline}{%
	\setlength\unitlength{1ex}%
	\begin{picture}(0,0) 
	% \put{} defines the position of the frame
	\put(155,0){\makebox(0,0)[bl]{
			%\includegraphics[scale=0.65]{white_square}
			%\includegraphics[scale=0.65]{dark_square}
			\includegraphics[scale=0.65]{grey_circle}
	}}%
	\end{picture}%
}{}



\usepackage{tikz}
\usetikzlibrary{tikzmark,calc}
\usetikzlibrary{positioning, backgrounds}
\usetikzlibrary{arrows, chains, matrix, scopes, patterns, shapes, fit}
\usetikzlibrary{mindmap,trees,shadows}
\usetikzlibrary{decorations.pathreplacing}
\usetikzlibrary{crypto.symbols}

\usepackage{pgfplots}

\pgfmathdeclarefunction{gauss}{2}{%
	\pgfmathparse{1/(#2*sqrt(2*pi))*exp(-((x-#1)^2)/(2*#2^2))}%
}


\tikzset{
	invisible/.style={opacity=0},
	visible on/.style={alt={#1{}{invisible}}},
	alt/.code args={<#1>#2#3}{%
		\alt<#1>{\pgfkeysalso{#2}}{\pgfkeysalso{#3}} % \pgfkeysalso doesn't change the path
	},
}

\newcommand\strikeout[2][]{%
	\begin{tabular}[b]{@{}c@{}} 
		\makebox(0,0)[cb]{{#1}} \\[-0.2\normalbaselineskip]
		\rlap{\color{Orange}\rule[0.5ex]{\widthof{#2}}{1.5pt}}#2
\end{tabular}}

\newcommand\Fontvi{\fontsize{11}{13.2}\selectfont}

\usepackage{listings} % for C++ code

\usepackage{braket}
%\usepackage[braket, qm]{qcircuit}



\usepackage[T1]{fontenc}
%\usepackage[sfdefault,scaled=.85]{FiraSans}
%\usepackage{newtxsf}
%\usepackage[nomap]{FiraMono}





\usefonttheme[onlymath]{serif}
\renewcommand\sfdefault{cmbr}

\renewcommand{\bfdefault}{sb}

\definecolor{CharCoalDark}{RGB}{13, 16, 19}
\definecolor{Orange}{RGB}{255, 165,0}
\definecolor{DarkOrange}{RGB}{255, 165,0}
\definecolor{LightSalmon}{RGB}{255, 160, 122}
\definecolor{LeafGreen}{RGB}{34, 139,  34}
\definecolor{Coral}{RGB}{255, 127, 80}
\definecolor{DarkTurquoise}{RGB}{0, 206, 209}

%\newtheorem{defRus}{Определение}
%\newtheorem{thmRus}{Теорема}
%s\newtheorem{corRus}{Следствие}


\setbeamercolor{background canvas}{bg=CharCoalDark}

\setbeamerfont{title}{series=\bfseries}
\setbeamercolor{title}{fg=Orange}
\setbeamercolor{section in toc}{fg=white}
\setbeamercolor{frametitle}{fg=Orange}
\setbeamercolor{normal text}{fg=white}
%\setbeamercolor{normal text}{fontsize=12pt}
\setbeamercolor{itemize item}{fg=Orange}
\setbeamercolor{enumerate item}{fg=Orange}
\setbeamercolor{enumerate item item}{fg=Orange}
\setbeamercolor{itemize item item}{fg=Orange}
\setbeamercolor{enumerate item}{fg=Orange}
\setbeamercolor{block title}{bg=DarkOrange,fg=white}
\setbeamerfont{block title}{series=\bfseries}

\setbeamertemplate{itemize item}[circle]
\setbeamertemplate{eumerate subitem}{\color{Orange}[$\checkmark$]}
\setbeamertemplate{itemize subitem}{\color{Orange}\Large$\textbullet$}
\setbeamertemplate{itemize subitem}{\color{Orange} \tiny $\blacksquare$}

% footnote without a marker
\newcommand\blfootnote[1]{%
	\begingroup
	\renewcommand\footnoterule{}
	\renewcommand\thefootnote{}\footnote{#1}%
	\addtocounter{footnote}{-1}%
	\endgroup
}

\newcommand*{\Scale}[2][4]{\scalebox{#1}{\ensuremath{#2}}}%

\newcommand\Item[1][]{%
	\ifx\relax#1\relax  \item \else \item[#1] \fi
	\abovedisplayskip=0pt\abovedisplayshortskip=0pt~\vspace*{-\baselineskip}}

\pgfdeclareradialshading{ring}{\pgfpoint{0cm}{0cm}}%
{rgb(0cm)=(1,1,1);
	rgb(0.7cm)=(1,1,1);
	rgb(0.719cm)=(1,1,1);
	rgb(0.72cm)=(0.975,0,0);
	rgb(0.9cm)=(1,1,1)}

\usepackage[absolute,overlay]{textpos} %to clip to a corner
\newcommand\FrameText[1]{%
	\begin{textblock*}{\paperwidth}(\textwidth-35pt, 10 pt)
		\raggedright #1\hspace{.5em}
\end{textblock*}}

\makeatletter
\let\save@measuring@true\measuring@true
\def\measuring@true{%
	\save@measuring@true
	\def\beamer@sortzero##1{\beamer@ifnextcharospec{\beamer@sortzeroread{##1}}{}}%
	\def\beamer@sortzeroread##1<##2>{}%
	\def\beamer@finalnospec{}%
}
\makeatother

\AtBeginSection[]
{
	\begin{frame}<beamer>
		\frametitle{Outline}
		\tableofcontents[currentsection]
	\end{frame}
}


\title{Лекция №3 \\[10pt]
	Часть 3. Режим шифрования (modes of operation)}

\date{ Елена Киршанова \\  \textbf{Курс ``Основы криптографии''} \\  }


\setbeamertemplate{navigation symbols}{} %removes navigation

% proper highlightling of a code-snippet
\lstset{language=C++,
	keywordstyle=\color{magenta},
	stringstyle=\color{Goldenrod},
	commentstyle=\color{gray},
	breaklines=false,
	%morecomment=[l][\color{magenta}]{\#}
}

%\setlength{\parskip}{8pt}
\input{header} %all defs
\begin{document}
	
\begin{frame}
	\titlepage
\end{frame}


\begin{frame}
	\Huge
	\centering
	{\color{Orange}Как правильно использовать блочный шифр для шифрования сообщений?} \\[10pt]
	
\end{frame}

\begin{frame}{Режимы шифрования}
\LARGE
\centering
\begin{enumerate} 
	\itemsep 10pt
	\item Режим электронной кодовой книги или режим простой замены (Electronic Block Code, EBC)
	\item Режим сцепления блоков шифротекста (Cipher Block Chain, CBC)
	\item Режим счётчика (Counter mode, CTR)
\end{enumerate}

\end{frame}

\begin{frame}{Электронная кодовая книга, Electronic Block Code (EBC)}

\Large
Пусть $m = (m_1, m_2, m_3, ...), m_i \in \{0,1\}^n$ -- открытый текст. \\[10pt]
Наивный способ  использования блочного шифра $B$
\begin{figure}
			\includegraphics[width=0.7\textwidth]{EBC}
	\end{figure}
\LARGE
Это {\color{Orange} небезопасный!} метод: \quad
если $m_1 = m_2$, то $c_1 = c_2$.

\end{frame}

\begin{frame}{Небезопасность EBC}
\centering
\Huge Если $m_1 = m_2$, то $c_1 = c_2$ \\[20pt]
\large
\begin{figure}
	\includegraphics[width=0.22\textwidth]{Tux} \quad 
	\includegraphics[width=0.22\textwidth]{Tux_ecb} \quad 
	\includegraphics[width=0.22\textwidth]{Tux_secure}
\end{figure}

\vfill
\small
{\color{gray} \textsuperscript{\textcopyright} Wikipedia} 
\end{frame}

\begin{frame}{Режим сцепления блоков, Cipher Block Chain (CBC)}
\Large
{\color{Orange} IV} -- инициализирующий (начальный) вектор -- случайная строка $n$-бит

\begin{figure}
	\includegraphics[width=0.7\textwidth]{CBC}
\end{figure}
IV передается с шифр-текстом (публично известно).
\end{frame}

\begin{frame}{Безопасность режима сцепления блоков}
\Large
\begin{itemize}
\itemsep 10pt
\item Начальное значение IV должно быть {\color{Orange} случайным} (если атакующий может предсказать IV, шифрование CBC небезопасно).\\ 
См. атаку на TLS 1.1.
 
\item{\color{Orange} IV необходимо обновлять}  \\

\end{itemize}
\end{frame}

\begin{frame}{Безопасность режима сцепления блоков}
\Large
	Положим, мы используем одно и тоже IV для длинного сообщения $m=(m_1, \ldots, m_t)$ при $t> 2^{n/2}$.
	\vspace{-10pt}
	\begin{figure}
		\includegraphics[width=0.75\textwidth]{CBC_1}
	\end{figure}
\vspace{-40pt}
{\color{Orange}Парадокс Дней Рождений:} имея ~$2^{n/2}$ блоков шифр-текста $c_i$, \\ с большой вероятностью мы увидим два одинаковых $c_i$.
\Large
\[
	c_1 \oplus m_2 == c_3 \oplus m_4
\]
Далее применяются статистические атаки на $m$.
\end{frame}

\begin{frame}{Парадокс Дней Рождений}
\Large
	{\color{Orange} Определить вероятность того, что в комнате из 30 человек двое родились в один день.}\\
	\pause
	\vspace{30pt}
	\[
		\left(1 - \frac{1}{365}\right)  \left(1 - \frac{2}{365}\right) \left(1 - \frac{3}{365}\right) \cdot \ldots \cdot \left(1 - \frac{29}{365}\right)  \approx 0.294
	\]
	
	\vspace{30pt}
	Значит, с вероятностью $1-0.294>0.7$ найдутся двое таких людей.
	
\end{frame}

\begin{frame}{Обобщение Парадокса Дней Рождений}
\Large
Для $m$ человек и $N$ возможных дней рождений, вероятность того, что все $m$ человек имеют разные дни рождения:
\[
	P : = \prod_{i=1}^{m-1} \left(  1  -\frac{i}{N} \right) \approx e^{-m^2/2N}
\]

Для $m = \sqrt{2N \ln 2}$, $P \approx 1/2$.  Вероятность $P$ быстро увеличивается при росте $m$. \\[10pt]

\pause
Для блок-шифра с длиной блока $n$, имеем $2^{n}$ всевозможных блоков шифр-текстов.\\
При {\color{Orange} $m = \bigO(2^{n/2})$ } шифр-блоков $c_i$'s, некоторые два из них равны с константной вероятностью.\\[15pt]

Для режима CBC: $c_i == c_j$ при $m=(m_1, \ldots, m_t)$, {\color{Orange} $t \approx 2^{n/2}$}:
\Large
\[
{\color{Orange} c_{i-1} \oplus m_{i} == c_{j-1} \oplus m_{j}}
\]

\end{frame}


\begin{frame}{Набивка (Padding) для режима CBC}
\Large
CBC подразумевает, что все блоки $m_i$ фиксированной длины. Для этого используется ``набивка''.
\begin{figure}
	\includegraphics[width=0.70\textwidth]{CBC_Padding}
\end{figure}
Обычно $\ell$-байтная набивка состоит из $\ell$ копий of $\ell$. \\
Набивка из $5$ байт: $5|5|5|5|5$. \\
Если $m$ занимает меньше $n$-бит, добавляется фиктивный (dummy) блок.
\end{frame}

\begin{frame}{Режим счетчика, Counter Mode (CTR)}
\Large
Один из самых популярных режимов шифрования \\
Здесь IV - начальное значение счетчика. \\
Счетчик увеличивается для каждого нового блока. \\
\begin{figure}
	\includegraphics[width=0.75\textwidth]{CTR}
\end{figure}
\vspace{-20pt}
CTR создает потоковое шифрование из блок-шифра \\

\end{frame}

\begin{frame}{Как выглядит IV}
	\begin{figure}
		\includegraphics[width=0.70\textwidth]{ShapeOfIV}
	\end{figure}
\vspace*{-50pt}
\large
\begin{itemize}
\item Nonce (нонс) должен быть псевдослучайным (64-битный выход PRG) и \\ не должен повторяться для одного и того же ключа $k$\\
\item Счетчик увеличивается для каждого нового блока
\item Значение счетчика не передается в протоколах, обеспечивающих \\ последовательную доставку пакетов (https)
\item Нонс обновляется после $2^{64}$ зашифрованных блоков.

\end{itemize}
\end{frame}


\begin{frame}{Преимущества режима Counter Mode}
%\vspace{-40pt}
\begin{figure}
	\includegraphics[width=0.55\textwidth]{CTR}
		\includegraphics[width=0.5\textwidth]{ShapeOfIV}
\end{figure}
\vspace{-20pt}
\Large
\begin{itemize}
	\item Нонс известен шифрующей и дешифрующей сторонам
	\item Простая процедура дешифрования
	\item Лёгкая параллелизация (в отличие от CBC)
	\item Не нужно использовать набивку
	\item 64-битный нонс (CTR) vs.\ 128-битный IV
\end{itemize}

\end{frame}




\end{document}