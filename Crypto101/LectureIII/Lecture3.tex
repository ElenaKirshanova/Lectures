\documentclass[usenames,dvipsnames,8pt,aspectratio=169]{beamer}
\usepackage{amsmath,amsfonts,amssymb}
\usepackage{mathtools}
\usepackage{etex} %for Windows
\usepackage[utf8]{inputenc}
\usepackage[english, russian]{babel} 
%\usepackage{microtype}			% Better interword spacing and additional kerning.
\usepackage{ellipsis}			% Adjusted space with \dots between two words.
\usepackage{graphicx}
\usepackage{pstricks}

\usepackage{xcolor}


\usepackage{changepage}

\usepackage{algorithm}
\usepackage{algpseudocode}
%\usepackage[]{algorithm2e}
%\usepackage{algorithmic}

%\usepackage{tcolorbox}

\addtobeamertemplate{footline}{%
	\setlength\unitlength{1ex}%
	\begin{picture}(0,0) 
	% \put{} defines the position of the frame
	\put(155,0){\makebox(0,0)[bl]{
			%\includegraphics[scale=0.65]{white_square}
			%\includegraphics[scale=0.65]{dark_square}
			\includegraphics[scale=0.65]{grey_circle}
	}}%
	\end{picture}%
}{}




\usepackage{tikz}
\usetikzlibrary{tikzmark,calc}
\usetikzlibrary{positioning, backgrounds}
\usetikzlibrary{arrows, chains, matrix, scopes, patterns, shapes, fit}
\usetikzlibrary{mindmap,trees,shadows}
\usetikzlibrary{decorations.pathreplacing}
\usetikzlibrary{crypto.symbols}

\usepackage{pgfplots}






%\newcommand\strikeout[2][]{%
%	\begin{tabular}[b]{@{}c@{}} 
%		\makebox(0,0)[cb]{{#1}} \\[-0.2\normalbaselineskip]
%		\rlap{\color{Orange}\rule[0.5ex]{\widthof{#2}}{1.5pt}}#2
%\end{tabular}}



\usepackage{listings} % for C++ code

\usepackage{braket}
%\usepackage[braket, qm]{qcircuit}



\usepackage[T1]{fontenc}
%\usepackage[sfdefault,scaled=.85]{FiraSans}
%\usepackage{newtxsf}
%\usepackage[nomap]{FiraMono}





\usefonttheme[onlymath]{serif}
\renewcommand\sfdefault{cmbr}

\renewcommand{\bfdefault}{sb}

\definecolor{CharCoalDark}{RGB}{13, 16, 19}
\definecolor{Orange}{RGB}{255, 165,0}
\definecolor{DarkOrange}{RGB}{255, 165,0}
\definecolor{LightSalmon}{RGB}{255, 160, 122}
\definecolor{LeafGreen}{RGB}{34, 139,  34}
\definecolor{Coral}{RGB}{255, 127, 80}
\definecolor{DarkTurquoise}{RGB}{0, 206, 209}

%\newtheorem{defRus}{Определение}
%\newtheorem{thmRus}{Теорема}
%s\newtheorem{corRus}{Следствие}


\setbeamercolor{background canvas}{bg=CharCoalDark}

\setbeamerfont{title}{series=\bfseries}
\setbeamercolor{title}{fg=Orange}
\setbeamercolor{section in toc}{fg=white}
\setbeamercolor{frametitle}{fg=Orange}
\setbeamercolor{normal text}{fg=white}
%\setbeamercolor{normal text}{fontsize=12pt}
\setbeamercolor{itemize item}{fg=Orange}
\setbeamercolor{enumerate item}{fg=Orange}
\setbeamercolor{enumerate item item}{fg=Orange}
\setbeamercolor{itemize item item}{fg=Orange}
\setbeamercolor{enumerate item}{fg=Orange}
\setbeamercolor{block title}{bg=DarkOrange,fg=white}
\setbeamerfont{block title}{series=\bfseries}

\setbeamertemplate{itemize item}[circle]
\setbeamertemplate{eumerate subitem}{\color{Orange}[$\checkmark$]}
\setbeamertemplate{itemize subitem}{\color{Orange}\Large$\textbullet$}
\setbeamertemplate{itemize subitem}{\color{Orange} \tiny $\blacksquare$}

% footnote without a marker
\newcommand\blfootnote[1]{%
	\begingroup
	\renewcommand\footnoterule{}
	\renewcommand\thefootnote{}\footnote{#1}%
	\addtocounter{footnote}{-1}%
	\endgroup
}

\newcommand*{\Scale}[2][4]{\scalebox{#1}{\ensuremath{#2}}}%

\newcommand\Item[1][]{%
	\ifx\relax#1\relax  \item \else \item[#1] \fi
	\abovedisplayskip=0pt\abovedisplayshortskip=0pt~\vspace*{-\baselineskip}}

%\pgfdeclareradialshading{ring}{\pgfpoint{0cm}{0cm}}%
%{rgb(0cm)=(1,1,1);
%	rgb(0.7cm)=(1,1,1);
%	rgb(0.719cm)=(1,1,1);
%	rgb(0.72cm)=(0.975,0,0);
%	rgb(0.9cm)=(1,1,1)}
%
%\usepackage[absolute,overlay]{textpos} %to clip to a corner
%\newcommand\FrameText[1]{%
%	\begin{textblock*}{\paperwidth}(\textwidth-35pt, 10 pt)
%		\raggedright #1\hspace{.5em}
%\end{textblock*}}

%\makeatletter
%\let\save@measuring@true\measuring@true
%\def\measuring@true{%
%	\save@measuring@true
%	\def\beamer@sortzero##1{\beamer@ifnextcharospec{\beamer@sortzeroread{##1}}{}}%
%	\def\beamer@sortzeroread##1<##2>{}%
%	\def\beamer@finalnospec{}%
%}
%\makeatother


\title{Лекция №3 \\[10pt]
	Часть 1. Блочный шифр.}

\date{ Елена Киршанова \\  \textbf{Курс ``Основы криптографии''} \\  }


\setbeamertemplate{navigation symbols}{} %removes navigation

% proper highlightling of a code-snippet
\lstset{language=C++,
	keywordstyle=\color{magenta},
	stringstyle=\color{Goldenrod},
	commentstyle=\color{gray},
	breaklines=false,
	%morecomment=[l][\color{magenta}]{\#}
}

%\setlength{\parskip}{8pt}
\input{header} %all defs
\begin{document}
	
\begin{frame}
	\titlepage
\end{frame}


\begin{frame}{Блок-шифры: мотивация}
\LARGE 
	Симметричные шифр-схемы могут быть основаны на
	
	\begin{enumerate}
		\itemsep 10pt
		\item Потоковом шифре 
		\item Блочном  шифре
	\end{enumerate}

	\vspace{20pt}
	Блочные шифры лежат в основе {\color{Orange} шифрования с аутентификацией.}
\end{frame}


\begin{frame}{Блок-шифр: определение}
\LARGE

		{\color{Orange}\textbf{Блок-шифр}} -- это {\color{Orange} детерминированный} шифр $(\KeyGen, \Enc, \Dec)$ с  $\keyS, \mathcal{X}:=\mesS = \cipS, $ и эффективной функцией
		\[
			\Enc = f(k, \cdot): \mathcal{X} \rightarrow \mathcal{X}.
		\]
		
При этом выполняется
\begin{itemize}
	\item корректность $\implies f(k, \cdot)$ -- биекция для всех $k \in \keyS$ \\
	\item $|\mathcal{X}| < \infty$. \\
\end{itemize}

\vspace{20pt}

То есть, $f(k, \cdot)$ -- перестановка на $\mathcal{X}$. \\[10pt]


\end{frame}

\begin{frame}{Шифрование блока}
	\begin{tikzpicture}
	%\draw[-stealth, thick] (-1.8,-1.0) -- (-1.0,-1.0) node[above,midway]{\Huge $b$};
	\draw[fill=CharCoalDark, draw=white, opacity=0.5] (-1.0,0.0) rectangle (2.0,1.0)  node[color=white, opacity=1,align=center, pos=0.5]{
		\LARGE  {Откр. текст}  
	} node[above, midway, yshift=17pt]{\Large $n$ бит};
	\draw[-stealth, thick] (2.0,0.5) -- (3.0,0.5);

	\draw[fill=CharCoalDark, draw=white, opacity=0.5] (3.0,0.0) rectangle (5.0,1.0)  node[color=white, opacity=1,align=center, pos=0.5]{
		\LARGE  {$f$}  
	};
	\draw[-stealth, thick] (5.0,0.5) -- (6.0,0.5);
	
	\draw[fill=CharCoalDark, draw=white, opacity=0.5] (2.5,2.2) rectangle (5.5,3.0)  node[color=white, opacity=1,align=center, pos=0.5]{
		\LARGE  {Ключ}  
	} node[above, midway, yshift=12pt]{\Large $k$ бит};

	\draw[-stealth, thick] (4.0,2.2) -- (4.0,1.0);
	
	\draw[fill=CharCoalDark, draw=white, opacity=0.5] (6.0,0.0) rectangle (9.0,1.0)  node[color=white, opacity=1,align=center, pos=0.5]{
		\LARGE  {Шифр-текст}  
	} node[above, midway, yshift=17pt]{\Large $n$ бит};
	\end{tikzpicture}
	
	\vspace{20pt}
	\LARGE
	Примеры:\\
	\begin{itemize}
		\item AES: $n = 128$, $k = 128, 192, 256$
		\item ГОСТ 34.12-2018: $n=128$, $k=256$ (Кузнечик)
	\end{itemize}
\end{frame}

\begin{frame}{Безопасность блочного шифра $\Pi = (\KeyGen, \Enc = f, \Dec = f^{-1})$}
\Large
 
		$f(k, \cdot): \mathcal{X} \rightarrow \mathcal{X}$ должна быть вычислительно неотличима от случайной перестановки на $\mathcal{X}$. \\[5pt]
		
		$S_{\mathcal{X}}$ -- множество всех перестановок на ${\mathcal{X}}$.
	
	\vspace{15pt}

	
		\begin{columns}
		\begin{column}{0.5\textwidth}
			{\color{Orange}\textbf{Эксперимент 0}} \\[10pt]
			\begin{tabular}{c c c}
				{\color{Orange} $\mathcal{C}$ } & &  {\color{Orange}  $\mathcal{A}$ } \\ [5pt]
				$	k \xleftarrow{\$} \mathcal{K}$ &  &  \\
				$	f \leftarrow \Enc(k, \cdot)$  & & \\
				& $\xleftarrow{\; x_i \in \mathcal{X} \; }$ & \\
				& $\xrightarrow{\; f(k, x_i) \; }$ & \\
				& $\xleftarrow{\; \hat{b} \; }$ & \\
			\end{tabular}
			\begin{tikzpicture}[overlay]
			\draw[fill=none, draw=white, opacity=0.5] (-5.3,-2.0) rectangle (-2.5,2.0); 
			\draw[fill=none, draw=white, opacity=0.5] (-1.2,-2.0) rectangle (-0.2,2.0); 
			\end{tikzpicture}
		\end{column}
		\begin{column}{0.5\textwidth}
			{\color{Orange}\textbf{Эксперимент 1}}\\[10pt]
			\begin{tabular}{c c c}
				{\color{Orange} $\mathcal{C}$ } & &  {\color{Orange}  $\mathcal{A}$ } \\ [5pt]
				&  &  \\
				$	f \xleftarrow{\$}  S_{\mathcal{X}}$  & $\xrightarrow{\; r \; }$ & \\
					& $\xleftarrow{\; x_i \in \mathcal{X} \; }$ & \\
				& $\xrightarrow{\; f(k, x_i) \; }$ & \\
				& $\xleftarrow{\; \hat{b} \; }$ & \\
			\end{tabular}
			\begin{tikzpicture}[overlay]
			\draw[fill=none, draw=white, opacity=0.5] (-4.8,-2.) rectangle (-2.5,2.0); 
			\draw[fill=none, draw=white, opacity=0.5] (-1.2,-2.0) rectangle (-0.2,2.0); 
			\end{tikzpicture}
			
		\end{column}
	\end{columns}
	 
	 \vspace{15pt}
	 \large 
	
	 Выигрыш $\mathcal{A}$ $:	\mathsf{BlockAdv} \left[  \mathcal{A}, \Pi\right ] = |\Pr[b == \hat{b}] - 1/2|$. \\[6pt]
	 Блок-шифр $\Pi$ {\color{Orange}\textbf{безопасный}}, если $	\mathsf{BlockAdv} = \negl(\cdot)$ для всех ppt $\mathcal{A}$.
\end{frame}

\begin{frame}{Немного истории}
\Large
\begin{itemize}
	
	\itemsep1em
	\item {\color{Orange}\textbf{70'е}:} IBM опубликовывает шифр Lucifer. $k=128, n = 128$ 
	\item {\color{Orange}\textbf{'76}:} DES -- стандарт $k=56, n = 64$
	\item {\color{Orange}\textbf{'98}:} 3DES -- стандарт  $k=168, n = 64$
	\item {\color{Orange}\textbf{'00}:} конкурс AES побеждает Rejndael $k=\{128, 192, 256\}, n = 128$
\end{itemize}

\vspace{10pt}
Российские стандарты:\\

\begin{itemize}
	
	\itemsep1em
	\item {\color{Orange}\textbf{'89}:} ГОСТ 28147-89 $k=256, n = 64$ 
	\item {\color{Orange}\textbf{'15 }:} ГОСТ Р 34.12-2015/2018, RFC 7801 $k=256, n = 128$ 
\end{itemize}


\end{frame}

\begin{frame}{Блок-шифры итеративны}


\end{frame}

\begin{frame}{Две основные парадигмы в дизайне блочных шифров}

\begin{itemize}
	\itemsep 2em
	\LARGE
	\item Сеть Фейстеля (создатель: Horst Feistel)
	\\[5pt]
	
	\Large
	Примеры: DES, ГОСТ 28147-89
	\LARGE
	\item Substitution-Permutation Network (SPN) \\ Подстановочно-перестановочная сеть  \\[5pt]
	\Large
	Примеры: AES, ГОСТ 34.12-2018
	
\end{itemize}
\end{frame}

\begin{frame}{Шифр Фейстеля}
\Large
	\vspace{-90pt}
	Сесть Фейстеля предлагает общий метод построения перестановки из \emph{любой} функции\\[10pt]
	Дано \hspace{40pt} $f(k, \cdot): \{0,1\}^n \rightarrow \{0,1\}^n$ \\[7pt]
	построить обратимую $F(k, \cdot):\{0,1\}^{2n} \rightarrow \{0,1\}^{2n}$ \\

%
%	\pause
%	
%	{\color{Orange}\textbf{Security (informal) :}} if $f: \keyS \times \{0,1\}^n \rightarrow \{0,1\}$ ``looks like'' a random function, then 3-Round Feistel $F: \keyS^3 \times \{0,1\}^{2n} \rightarrow \{0,1\}^{2n}$ is a pseudorandom permutation. 
\end{frame}


\begin{frame}{Пример: Раундовая функция $f$ в ГОСТе'89}
\centering
\tikzmark{start}
\includegraphics[height=0.83\textheight]{GostRound}
\tikzmark{end} 

\vfill
\small
{\color{gray} {.tex код avanzi-tikz-defs.tex }} 
\end{frame}

\begin{frame}{Что такое S-бокс?}
	\LARGE
	\[
		S:= \{0,1\}^n \rightarrow \{0,1\}^m - \text{ таблица подстановки}
	\]

	\begin{itemize}
		\item Реализована таблицей поиска (Loopup table)
		\item В одном блочном шифры может быть использовано несколько S-боксов
		\item S-боксов должен быть стойким к  линейным и дифференциальным атакам
		\item S-бокс не должен содержать фиксированных точек: \\ $S(x) \neq x $, $S(x) \neq  \bar{x} \; \forall x$
		\item S-бокс не должен быть линейной или аффинной булевой \\ функцией
	\end{itemize}
\end{frame}

\begin{frame}{Пример: S-боксы в ГОСТе'89}
	\Large
	\[
	S:= \{0,1\}^4 \rightarrow \{0,1\}^4
	\]
	
	\begin{figure}
		\includegraphics[width=0.77\textwidth]{Sbox_GOST}
			\hspace{5em}
	\end{figure}
\vfill
\small
{\color{gray} {Wikipedia, \url{https://ru.wikipedia.org/wiki/\%D0\%93\%D0\%9E\%D0\%A1\%D0\%A2_28147-89}}} 
\end{frame}



\begin{frame}{Подстановочно-перестановочная сеть(SPN)}
\centering
\tikzmark{start}
\includegraphics[height=\textheight, angle=90, origin=c]{SPN}
\tikzmark{end} 

\vfill
\small
{\color{gray} {.tex код crypto.symbols}} 

\end{frame}

\begin{frame}{AES:  SPN шифр}
\Large 
\begin{itemize}
	\itemsep 10pt
	\item стандартизирован в 2001 году (FIPS PUB 197: Advanced Encryption Standard (AES), ISO/IEC 18033-3: Block ciphers)
	\item Длина блока $n = 128 $ бит, длины ключей $k = \{128, 192, 256 \}$
	\item Количество раундов: 10 ($k = 128$), 12 ($k=192$), 14 ($k=256$)
	\item 128 бит организованы в матрицу $4 \times 4$ байт
	\[
	\begin{pmatrix}
	b_0 & b_4 & b_8 & b_{12} \\
	b_1 & b_5 & b_9 & b_{13} \\
	b_2 & b_6 & b_{10} & b_{14} \\
	b_3 & b_7 & b_{11} & b_{15}  \\
	\end{pmatrix}
	\]
\end{itemize}
%	\begin{figure}
%		\includegraphics[width=1.1\textwidth]{AES_128}
%	\end{figure}
	
%	\Large 
%	\[
%		\Pi_{\text{AES}}  = \{0,1\}^{128} \rightarrow \{0,1\}^{128} - \text{ invertible permutation }
%	\]
%	\vfill
%	\small
%	{\color{gray}\textbf{picture is taken from D.Boneh, V.Shoup A Graduate Course in Applied Cryptography}} 
\end{frame}


\begin{frame}{Перестановка $f_{\text{AES}}$}
\Large
Перестановка $f_{\text{AES}}$ состоит из трёх обратимых операций: \\[5pt]
\begin{enumerate}
	\itemsep10pt
	\item $\mathtt{SubBytes}$ (единственная нелинейная операция)
	\[
	S: \{0,1\}^8 \rightarrow \{0,1\}^n - \text{S-бокс }
	\]
	\item  $\mathtt{ShiftRows}$
	\[
	\begin{pmatrix}
	b_0 & b_4 & b_8 & b_{12} \\
	b_1 & b_5 & b_9 & b_{13} \\
	b_2 & b_6 & b_{10} & b_{14} \\
	b_3 & b_7 & b_{11} & b_{15}  \\
	\end{pmatrix}
	\rightarrow
	\begin{pmatrix}
	b_0 & b_4 & b_8 & b_{12} \\
	b_5 & b_9 & b_{13} & b_1 \\
	b_{10} & b_{14} & b_{2} & b_{6} \\
	b_{15} & b_{3} & b_{7} & b_{11}  \\
	\end{pmatrix}
	\]
	
	\item  $\mathtt{MixColumns}$ -- столбцы перемешиваются по определённому \\ правилу (см. \url{en.wikipedia.org/wiki/Advanced_Encryption_Standard})

%	\[
%		\begin{pmatrix}
%		 \\
%		 b^{\text{new}}_j \\
%		\\
%		\end{pmatrix}
%		= 
%		\begin{pmatrix} 
%		2 & 3 & 1 & 1 \\
%		1 & 2 & 3 & 1 \\
%		1 & 1 & 2 & 3 \\
%		3 & 1 & 1 & 2 
%		\end{pmatrix}
%		\cdot
%		\begin{pmatrix}
%		\\
%		b_j \\
%		\\
%		\end{pmatrix}
%	\]
	
\end{enumerate}

\vspace{20pt}

Принцип: минимизировать успешность известных атак.	

\end{frame}

\begin{frame}{Расширение ключа в AES}
	\Large
	Задача: расширить ключ $k  \in \{0,1\}^{128}$ до 10 раундовых ключей $k_i \in \{0,1\}^{128}$
	
	\vspace{20pt}
	
	\LARGE
	\begin{itemize}
		\itemsep 10pt
		\item $k_0 = k = (w_{0,0}, w_{0,1}, w_{0,2}, w_{0,3})$, $w_{0,1} \in \{0,1\}^{32}$
		
		\item $k_i = (w_{i,0}, w_{i,1}, w_{i,2}, w_{i,3})$, где
		\begin{align*}
			w_{i,0} &= w_{i-1, 0} \oplus g_i(w_{i-1, 3}) \\	
			w_{i,1} &= w_{i-1, 1} \oplus w_{i, 0} \\
			w_{i,2} &= w_{i-1, 2} \oplus w_{i, 1} \\
			w_{i,3} &= w_{i-1, 3} \oplus w_{i, 2} \\
		\end{align*}
		$g_i : \{0,1\}^{32} \rightarrow \{0,1\}^{32}  $ -- функция, состоящая из сдвигов, \\ подстановки $\mathtt{SubBytes}$ и $\mathtt{XOR}$ с раундовыми константами $c_i$.
	\end{itemize}
\end{frame}

%\begin{frame}{Attacks on block ciphers}
%
%\Large
%{\color{Orange}\textbf{Exhaustive search}}  for block cipher key. \\[5pt]
%	
%	For DES/AES/GOST: \textbf{two} plaintext/ciphertext pairs $(m_1, c_1 = \Enc(k, m_1)), (m_2, c_2=\Enc(k,m_2))$ \\ determine $k$ with sufficiently high probability \\[10pt]
%	
%	
%{\color{Orange}\textbf{Example}}  : For DES find $k \in \{0,1\}^{56}$	s.t.\ $c_i = \Enc(m_i, k)$. \\[10pt]
%
%Cryptanalytic efforts: \\
%\begin{itemize}
%	\item {\color{Orange}\textbf{In '99}} ~22h on  DeepCrack + distributed.net (a bit expensive hardware)\\
%	\item {\color{Orange}\textbf{In '07}}~13 days COPACOBANA (cheaper)
%\end{itemize}
%\end{frame}
%
%\begin{frame}{Advanced attacks on block ciphers}
%\LARGE
%\begin{itemize}
%	\itemsep 1em
%	\item Design attacks: linear \& differential cryptnalalysis 
%	
%	Target:  find a linear relation in bit positions
%	
%	\begin{align*}
%	& \Pr \left[ m[S_0] \oplus \Enc(k, m)[S_1 ] = k[S_2]  \right]  \geq 1/2 + \varepsilon  \\
%	& S_i \subset \{0,\ldots, n-1\} \quad \forall k \text { and random } m
%	\end{align*}
%	\pause
%	
%	\item Side-channel attacks:
%	\Large{measure {\color{Orange}\textbf{time}} or {\color{Orange}\textbf{power}} needed for  $\Enc, \Dec$ } \pause
%	\LARGE
%	\item Fault-injection attacks: 
%	\Large cause the hardware to introduce errors at runtime (heat, EM interference)
%\end{itemize}
%\end{frame}
%
%\begin{frame}{Take-home message}
%\Huge
%
%\begin{itemize}
%	\itemsep 2em
%	\item {\color{Orange}\textbf{DON'T}} design {\color{Orange}\textbf{YOUR OWN}} block-cipher 
%	\item {\color{Orange}\textbf{TRY NOT TO}}  implement cryptoprimitives yourself if good implementations exist
%	\item Choose key-sizes wisely 
%\end{itemize}
%\end{frame}
%
%
%\begin{frame}{Further reading}
%	\begin{figure}
%		\includegraphics[height=0.7\textheight]{Book_cover}
%	\end{figure}
%\end{frame}

\end{document}