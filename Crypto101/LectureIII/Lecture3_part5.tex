\documentclass[usenames,dvipsnames,8pt,aspectratio=169]{beamer}
\usepackage{amsmath,amsfonts,amssymb}
\usepackage{mathtools}
\usepackage{etex} %for Windows
\usepackage[utf8]{inputenc}
\usepackage[english, russian]{babel} 
%\usepackage{microtype}			% Better interword spacing and additional kerning.
\usepackage{ellipsis}			% Adjusted space with \dots between two words.
\usepackage{graphicx}
\usepackage{pstricks}

\usepackage{xcolor}


\usepackage{changepage}

\usepackage{algorithm}
\usepackage{algpseudocode}
%\usepackage[]{algorithm2e}
%\usepackage{algorithmic}

%\usepackage{tcolorbox}






\usepackage{tikz}
\usetikzlibrary{tikzmark,calc}
\usetikzlibrary{positioning, backgrounds}
\usetikzlibrary{arrows, chains, matrix, scopes, patterns, shapes, fit}
\usetikzlibrary{mindmap,trees,shadows}
\usetikzlibrary{decorations.pathreplacing}
\usetikzlibrary{crypto.symbols}

\usepackage{pgfplots}

\pgfmathdeclarefunction{gauss}{2}{%
	\pgfmathparse{1/(#2*sqrt(2*pi))*exp(-((x-#1)^2)/(2*#2^2))}%
}


\tikzset{
	invisible/.style={opacity=0},
	visible on/.style={alt={#1{}{invisible}}},
	alt/.code args={<#1>#2#3}{%
		\alt<#1>{\pgfkeysalso{#2}}{\pgfkeysalso{#3}} % \pgfkeysalso doesn't change the path
	},
}

\newcommand\strikeout[2][]{%
	\begin{tabular}[b]{@{}c@{}} 
		\makebox(0,0)[cb]{{#1}} \\[-0.2\normalbaselineskip]
		\rlap{\color{Orange}\rule[0.5ex]{\widthof{#2}}{1.5pt}}#2
\end{tabular}}

\newcommand\Fontvi{\fontsize{11}{13.2}\selectfont}

\usepackage{listings} % for C++ code

\usepackage{braket}
%\usepackage[braket, qm]{qcircuit}



\usepackage[T1]{fontenc}
%\usepackage[sfdefault,scaled=.85]{FiraSans}
%\usepackage{newtxsf}
%\usepackage[nomap]{FiraMono}





\usefonttheme[onlymath]{serif}
\renewcommand\sfdefault{cmbr}

\renewcommand{\bfdefault}{sb}

\definecolor{CharCoalDark}{RGB}{13, 16, 19}
\definecolor{Orange}{RGB}{255, 165,0}
\definecolor{DarkOrange}{RGB}{255, 165,0}
\definecolor{LightSalmon}{RGB}{255, 160, 122}
\definecolor{LeafGreen}{RGB}{34, 139,  34}
\definecolor{Coral}{RGB}{255, 127, 80}
\definecolor{DarkTurquoise}{RGB}{0, 206, 209}

%\newtheorem{defRus}{Определение}
%\newtheorem{thmRus}{Теорема}
%s\newtheorem{corRus}{Следствие}


\setbeamercolor{background canvas}{bg=CharCoalDark}

\setbeamerfont{title}{series=\bfseries}
\setbeamercolor{title}{fg=Orange}
\setbeamercolor{section in toc}{fg=white}
\setbeamercolor{frametitle}{fg=Orange}
\setbeamercolor{normal text}{fg=white}
%\setbeamercolor{normal text}{fontsize=12pt}
\setbeamercolor{itemize item}{fg=Orange}
\setbeamercolor{enumerate item}{fg=Orange}
\setbeamercolor{enumerate item item}{fg=Orange}
\setbeamercolor{itemize item item}{fg=Orange}
\setbeamercolor{enumerate item}{fg=Orange}
\setbeamercolor{block title}{bg=DarkOrange,fg=white}
\setbeamerfont{block title}{series=\bfseries}

\setbeamertemplate{itemize item}[circle]
\setbeamertemplate{eumerate subitem}{\color{Orange}[$\checkmark$]}
\setbeamertemplate{itemize subitem}{\color{Orange}\Large$\textbullet$}
\setbeamertemplate{itemize subitem}{\color{Orange} \tiny $\blacksquare$}

% footnote without a marker
\newcommand\blfootnote[1]{%
	\begingroup
	\renewcommand\footnoterule{}
	\renewcommand\thefootnote{}\footnote{#1}%
	\addtocounter{footnote}{-1}%
	\endgroup
}

\newcommand*{\Scale}[2][4]{\scalebox{#1}{\ensuremath{#2}}}%

\newcommand\Item[1][]{%
	\ifx\relax#1\relax  \item \else \item[#1] \fi
	\abovedisplayskip=0pt\abovedisplayshortskip=0pt~\vspace*{-\baselineskip}}

\pgfdeclareradialshading{ring}{\pgfpoint{0cm}{0cm}}%
{rgb(0cm)=(1,1,1);
	rgb(0.7cm)=(1,1,1);
	rgb(0.719cm)=(1,1,1);
	rgb(0.72cm)=(0.975,0,0);
	rgb(0.9cm)=(1,1,1)}

\usepackage[absolute,overlay]{textpos} %to clip to a corner
\newcommand\FrameText[1]{%
	\begin{textblock*}{\paperwidth}(\textwidth-35pt, 10 pt)
		\raggedright #1\hspace{.5em}
\end{textblock*}}

\addtobeamertemplate{footline}{%
	\setlength\unitlength{1ex}%
	\begin{picture}(0,0) 
	% \put{} defines the position of the frame
	\put(155,0){\makebox(0,0)[bl]{
			%\includegraphics[scale=0.65]{white_square}
			%\includegraphics[scale=0.65]{dark_square}
			\includegraphics[scale=0.65]{grey_circle}
	}}%
	\end{picture}%
}{}


\makeatletter
\let\save@measuring@true\measuring@true
\def\measuring@true{%
	\save@measuring@true
	\def\beamer@sortzero##1{\beamer@ifnextcharospec{\beamer@sortzeroread{##1}}{}}%
	\def\beamer@sortzeroread##1<##2>{}%
	\def\beamer@finalnospec{}%
}
\makeatother

\AtBeginSection[]
{
	\begin{frame}<beamer>
		\frametitle{Outline}
		\tableofcontents[currentsection]
	\end{frame}
}


\title{Лекция №3 \\[10pt]
	Часть 5. Атака на выбранный открытый текст. Безопасность режима CRT}

\date{ Елена Киршанова \\  \textbf{Курс ``Основы криптографии''} \\  }


\setbeamertemplate{navigation symbols}{} %removes navigation

% proper highlightling of a code-snippet
\lstset{language=C++,
	keywordstyle=\color{magenta},
	stringstyle=\color{Goldenrod},
	commentstyle=\color{gray},
	breaklines=false,
	%morecomment=[l][\color{magenta}]{\#}
}

\newcommand{\AxisRotator}[1][rotate=0]{%
	\tikz [x=0.45cm,y=1.2cm,line width=.2ex,-stealth,#1] \draw[color=Orange] (0,0) arc (-150:150:2 and 1);%
}

%\setlength{\parskip}{8pt}
\input{header} %all defs
\begin{document}
	
\begin{frame}
	\titlepage
\end{frame}

\begin{frame}{Атака на режим шифрования EBC}
	\begin{figure}
		\includegraphics[scale=0.15]{EBC}
	\end{figure}

\Large
{\color{Orange} Задача атакующего $\adv$:} отличить $C = (c_i)_i = \Enc(k, M)$ от \\[3pt] $C' = (c_i')_i = \Enc(k, M')$ для $M = (m_i)_i$, $M' = (m_i')_i$.  \\[10pt]

{\color{Orange} $\adv$ выбирает:} $M = (m_1, m_1, \ldots)$ и $M' = (m_1', m_2', \ldots)$, $m_1' \neq m_2'$. \\[10pt]

Тогда $C = (c_1, c_1, \ldots ) = \Enc(M)$, $C' = (c_1', c_2', \ldots ) = \Enc(M)$ \\[10pt]

\centering
{\color{Orange} Детерминированный шифр не является безопасным !}

\end{frame}

\begin{frame}{Атака на выбранный открытые текст/ Chosen Plaintext Attack(CPA)}
\large
\begin{center}

			\begin{tabular}{c c c}
				{\color{Orange} Челленджер $\mathcal{C}$ } & & {\color{Orange} Атакующий $\mathcal{A}$ }\\ [5pt]
				$k \leftarrow \KeyGen(1^\lambda)$ & $\xrightarrow{\quad \Huge \lambda \quad}$  &\\[5pt]
				$b \xleftarrow{\$} \{0,1\}  $& &\\ [5pt]
				& $\xleftarrow{\; \Huge m_{0, i}, m_{1, i}  \;}$  &$m_{0, i}, m_{1,i} \leftarrow \mesS $\\ [5pt]
				
				$c_i \leftarrow \Enc(k, m_{b, i})$ &  &\\ [5pt]
				& $\xrightarrow{\quad c_i \quad}$ & \\ [5pt]
				& $\xleftarrow{\quad \hat{b} \quad}$ & \\ [5pt]
			\end{tabular}
			\begin{tikzpicture}[overlay]
			\draw[fill=none, draw=white, opacity=0.5] (-8.3,-2.3) rectangle (-5.0,2.7); 
			\draw[fill=none, draw=white, opacity=0.5] (-3.0,-2.3) rectangle (0.0,2.7); 
			\node at (-3.8, -0.1) {\AxisRotator};
			\end{tikzpicture}
\end{center}

		\vspace{5pt}
		$\mathtt{W_{\Pi, \adv}}$ -- событие $b == \hat{b}$. \\ [4pt]
		$\mathtt{CPAAdv} = \abs{\Pr[\mathtt{W_{\Pi, \adv}}] - \frac{1}{2}}$ --выигрыш  $\adv$. \\ [4pt]
		\color{Orange} Шифр-схема $\Pi$ CPA безопасна, если для любого ppt $\adv:$  $\mathtt{CPAAdv} = \negl(\lambda).$
\end{frame}



\begin{frame}{Режим счётчика или CTR}
\Large
%\centering
$B: \keyS \times \mathcal{X} \rightarrow \mathcal{Y}$ --- псевдослучайная функция, $\mathcal{Y} = \{0,1\}^n = \mesS$.
\begin{columns}
	\begin{column}{0.4\textwidth}
		\begin{figure}
			\hspace{-4em}
			\includegraphics[scale=0.2]{CTR}
		\end{figure}
	\end{column}
	\begin{column}{0.4\textwidth}
	
	\end{column}
	\end{columns}




\end{frame}



\begin{frame}{CPA безопасность режима CTR}
\LARGE

{\color{Orange} Теорема.} Если $B: \keyS \times \mathcal{X} \rightarrow \mathcal{Y} $ -- {\color{Orange} безопасная псевдослучайная функция} и $|\mathcal{X}|$ -- суперполиномиально, то режим шифрования CTR для $B$ --  {\color{Orange} CPA-безопасный} режим.\\[10pt]


\textit{Для любого ppt $\adv$, атакующего CPA-стойкость CTR, найдется ppt $\mathcal{B}$, атакующий стойкость $B$}. \\[10pt]



\end{frame}

\begin{frame}{CPA безопасность режима CTR}
\Large
	\textit{Для любого ppt $\adv$, атакующего CPA-стойкость CTR, найдется ppt $\mathcal{B}$, атакующий стойкость псевдослучайой функции $B$}. \\[20pt]
	
	
	\Large
	\begin{tabular}{c c c c c}
		{\color{Orange}  $\mathcal{C} $ (PRF) } & &  {\color{Orange}  $\mathcal{B}$ (PRF) } & & {\color{Orange}  $ \mathcal{A}$ (CPA)}  \\ [5pt]
		$b \xleftarrow{\$} \{0,1\}  $ &  & $b' \xleftarrow{\$} \{0,1\}  $ & & \\[3pt]
		$f \xleftarrow{\$} \mathcal{F}[\mathcal{X}, \mathcal{X}]$& &	$x_i \xleftarrow{\$} \mathcal{X}$		& $\xleftarrow{\; m_{0,i}, m_{1,i} \;}$   &\\ 
		либо & &	&   &\\ 
		$f = F(k, \cdot)$& &	$m_{b_i} = (m_{b'_i}^{1}, \ldots m_{b_i}^{\ell} )$	&   &\\ 
		& &	$x_{i, j} = x_i \oplus j $	&   &\\
		&$\xleftarrow{\; x_{i,j} \;}$  & & &\\ 
		& $\xrightarrow{f(x_{i,j})}$&  $c_{i,j} = m_{b'_i}^{j}  \oplus f(x_{i,j})$& $\xrightarrow{\; \; c_i \; \;}$ &\\ 
		 & &	&  $\xleftarrow{\; \; \hat{b} \; \; }$ &\\ 
	\end{tabular}
	\begin{tikzpicture}[overlay]
	\draw[fill=none, draw=white, opacity=0.5] (-12.2,-2.5) rectangle (-9.1,3.0); 
	\draw[fill=none, draw=white, opacity=0.5] (-7.9,-2.5) rectangle (-4.0,3.0); 
	\draw[fill=none, draw=white, opacity=0.5] (-2.3,-2.5) rectangle (0.0,3.0); 
	\end{tikzpicture}
\end{frame}



\end{document}