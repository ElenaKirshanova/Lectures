\documentclass[usenames,dvipsnames,8pt,aspectratio=169]{beamer}
\usepackage{amsmath,amsfonts,amssymb}
\usepackage{mathtools}
\usepackage{etex} %for Windows
\usepackage[utf8]{inputenc}
\usepackage[english, russian]{babel} 
%\usepackage{microtype}			% Better interword spacing and additional kerning.
\usepackage{ellipsis}			% Adjusted space with \dots between two words.
\usepackage{graphicx}
\usepackage{pstricks}

\usepackage{xcolor}


\usepackage{changepage}

\usepackage{algorithm}
\usepackage{algpseudocode}
%\usepackage[]{algorithm2e}
%\usepackage{algorithmic}

%\usepackage{tcolorbox}



\addtobeamertemplate{footline}{%
	\setlength\unitlength{1ex}%
	\begin{picture}(0,0) 
	% \put{} defines the position of the frame
	\put(155,0){\makebox(0,0)[bl]{
			%\includegraphics[scale=0.65]{white_square}
			%\includegraphics[scale=0.65]{dark_square}
			\includegraphics[scale=0.65]{grey_circle}
	}}%
	\end{picture}%
}{}



\usepackage{tikz}
\usetikzlibrary{tikzmark,calc}
\usetikzlibrary{positioning, backgrounds}
\usetikzlibrary{arrows, chains, matrix, scopes, patterns, shapes, fit}
\usetikzlibrary{mindmap,trees,shadows}
\usetikzlibrary{decorations.pathreplacing}
\usetikzlibrary{crypto.symbols}

\usepackage{pgfplots}

\pgfmathdeclarefunction{gauss}{2}{%
	\pgfmathparse{1/(#2*sqrt(2*pi))*exp(-((x-#1)^2)/(2*#2^2))}%
}


\tikzset{
	invisible/.style={opacity=0},
	visible on/.style={alt={#1{}{invisible}}},
	alt/.code args={<#1>#2#3}{%
		\alt<#1>{\pgfkeysalso{#2}}{\pgfkeysalso{#3}} % \pgfkeysalso doesn't change the path
	},
}

\newcommand\strikeout[2][]{%
	\begin{tabular}[b]{@{}c@{}} 
		\makebox(0,0)[cb]{{#1}} \\[-0.2\normalbaselineskip]
		\rlap{\color{Orange}\rule[0.5ex]{\widthof{#2}}{1.5pt}}#2
\end{tabular}}

\newcommand\Fontvi{\fontsize{11}{13.2}\selectfont}

\usepackage{listings} % for C++ code

\usepackage{braket}
%\usepackage[braket, qm]{qcircuit}



\usepackage[T1]{fontenc}
%\usepackage[sfdefault,scaled=.85]{FiraSans}
%\usepackage{newtxsf}
%\usepackage[nomap]{FiraMono}





\usefonttheme[onlymath]{serif}
\renewcommand\sfdefault{cmbr}

\renewcommand{\bfdefault}{sb}

\definecolor{CharCoalDark}{RGB}{13, 16, 19}
\definecolor{Orange}{RGB}{255, 165,0}
\definecolor{DarkOrange}{RGB}{255, 165,0}
\definecolor{LightSalmon}{RGB}{255, 160, 122}
\definecolor{LeafGreen}{RGB}{34, 139,  34}
\definecolor{Coral}{RGB}{255, 127, 80}
\definecolor{DarkTurquoise}{RGB}{0, 206, 209}

%\newtheorem{defRus}{Определение}
%\newtheorem{thmRus}{Теорема}
%s\newtheorem{corRus}{Следствие}


\setbeamercolor{background canvas}{bg=CharCoalDark}

\setbeamerfont{title}{series=\bfseries}
\setbeamercolor{title}{fg=Orange}
\setbeamercolor{section in toc}{fg=white}
\setbeamercolor{frametitle}{fg=Orange}
\setbeamercolor{normal text}{fg=white}
%\setbeamercolor{normal text}{fontsize=12pt}
\setbeamercolor{itemize item}{fg=Orange}
\setbeamercolor{enumerate item}{fg=Orange}
\setbeamercolor{enumerate item item}{fg=Orange}
\setbeamercolor{itemize item item}{fg=Orange}
\setbeamercolor{enumerate item}{fg=Orange}
\setbeamercolor{block title}{bg=DarkOrange,fg=white}
\setbeamerfont{block title}{series=\bfseries}

\setbeamertemplate{itemize item}[circle]
\setbeamertemplate{eumerate subitem}{\color{Orange}[$\checkmark$]}
\setbeamertemplate{itemize subitem}{\color{Orange}\Large$\textbullet$}
\setbeamertemplate{itemize subitem}{\color{Orange} \tiny $\blacksquare$}

% footnote without a marker
\newcommand\blfootnote[1]{%
	\begingroup
	\renewcommand\footnoterule{}
	\renewcommand\thefootnote{}\footnote{#1}%
	\addtocounter{footnote}{-1}%
	\endgroup
}

\newcommand*{\Scale}[2][4]{\scalebox{#1}{\ensuremath{#2}}}%

\newcommand\Item[1][]{%
	\ifx\relax#1\relax  \item \else \item[#1] \fi
	\abovedisplayskip=0pt\abovedisplayshortskip=0pt~\vspace*{-\baselineskip}}

\pgfdeclareradialshading{ring}{\pgfpoint{0cm}{0cm}}%
{rgb(0cm)=(1,1,1);
	rgb(0.7cm)=(1,1,1);
	rgb(0.719cm)=(1,1,1);
	rgb(0.72cm)=(0.975,0,0);
	rgb(0.9cm)=(1,1,1)}

\usepackage[absolute,overlay]{textpos} %to clip to a corner
\newcommand\FrameText[1]{%
	\begin{textblock*}{\paperwidth}(\textwidth-35pt, 10 pt)
		\raggedright #1\hspace{.5em}
\end{textblock*}}

\makeatletter
\let\save@measuring@true\measuring@true
\def\measuring@true{%
	\save@measuring@true
	\def\beamer@sortzero##1{\beamer@ifnextcharospec{\beamer@sortzeroread{##1}}{}}%
	\def\beamer@sortzeroread##1<##2>{}%
	\def\beamer@finalnospec{}%
}
\makeatother

\AtBeginSection[]
{
	\begin{frame}<beamer>
		\frametitle{Outline}
		\tableofcontents[currentsection]
	\end{frame}
}


\title{Лекция №3 \\[10pt]
	Часть 4. Псевдослучайные функции.}

\date{ Елена Киршанова \\  \textbf{Курс ``Основы криптографии''} \\  }


\setbeamertemplate{navigation symbols}{} %removes navigation

% proper highlightling of a code-snippet
\lstset{language=C++,
	keywordstyle=\color{magenta},
	stringstyle=\color{Goldenrod},
	commentstyle=\color{gray},
	breaklines=false,
	%morecomment=[l][\color{magenta}]{\#}
}

\newcommand{\AxisRotator}[1][rotate=0]{%
	\tikz [x=0.50cm,y=1.10cm,line width=.3ex,-stealth,#1] \draw[color=Orange] (0,0) arc (-150:150:2 and 1);%
}

%\setlength{\parskip}{8pt}
\input{header} %all defs
\begin{document}
	
\begin{frame}
	\titlepage
\end{frame}

\begin{frame}{Псевдослучайная функция (Pseudorandom function): определение}
\LARGE
	Модель безопасного блок-шифра -- {\color{Orange}псевдослучайная функция (PRF).} \\[20pt]
	
	Псевдослучайная функция $F(k, x)$ -- {\color{Orange} детерминированный} алгоритм
	\begin{align*}
		F : \keyS \times \mathcal{X} &\rightarrow \mathcal{Y} \quad |\keyS|, |\mathcal{X}|, |\mathcal{Y}| < \infty \\ 
		(k, m) & \rightarrow c,
	\end{align*}
	такая что $F(k, \cdot)$ {\color{Orange}``выглядит'' как случайная функция} из $\mathcal{X}$ в $\mathcal{Y}$.\\[10pt]
	

\end{frame}

\begin{frame}{Безопасность псевдослучайной функции}
\Large

	 	$\mathcal{F}[\mathcal{X}, \mathcal{Y}]$ -- множество всех функций $f: \mathcal{X} \rightarrow  \mathcal{Y}$. \\[20pt]
\begin{center}	
		\begin{tabular}{c c c}
		{\color{Orange} Челленджер $\mathcal{C}$ } & & {\color{Orange} Атакующий $\mathcal{A}$ }\\ [5pt]
		$b \xleftarrow{\$} \{0,1\}  $& &\\ [2pt]
		$\mkern-45mu b ==0: f \xleftarrow{\$} \mathcal{F}[\mathcal{X}, \mathcal{Y}]$  & &\\ [2pt]
		$b ==1: k\xleftarrow{\$} \keyS, f = F(k, \cdot )$  & &\\ [2pt]
	 	& $ \xleftarrow{\quad x_i \quad} $  & $x_i \in \mathcal{X}  $\\[5pt]
		& $\xrightarrow{f (x_i)}$  &\\[5pt]
		& $\xleftarrow{\quad \hat{b} \quad}$ & \\ [5pt]
	\end{tabular}
	\begin{tikzpicture}[overlay]
	\node at (-3.8, -0.3) {\AxisRotator};
	\draw[fill=none, draw=white, opacity=0.5] (-9.8,-2.3) rectangle (-4.8,3.0); 
	\draw[fill=none, draw=white, opacity=0.5] (-3.0,-2.3) rectangle (0.0,3.0); 
	\end{tikzpicture}
\end{center}	
	$\mathtt{W_{F, \adv}}$ -- событие $b == \hat{b}$. 
	$\mathtt{PRFAdv} = \abs{\Pr[\mathtt{W_{F, \adv}}] - \frac{1}{2}}$ -- выигрыш  $\adv$ \\ [5pt]
	
	$F$ -- безопасная псевдослучайная функцию, если $\forall$ ppt $\adv:$  \[\mathtt{PRFAdv} = \negl(\lambda).\] 
\end{frame}


\begin{frame}{Блок-шифр $\implies^\star $ PRF}
\Large
Блок-шифр: $\Enc(k, \cdot ): \mathcal{X} \rightarrow \mathcal{X}$ --- перестановка.	 \\[20pt]

\begin{tabular}{c c c}
	{\color{Orange} Челленджер $\mathcal{C}$ } & & {\color{Orange} Атакующий $\mathcal{A}$ }\\ [5pt]
	$b \xleftarrow{\$} \{0,1\}  $& &\\ [2pt]
	$\mkern-45mu b ==0: f \xleftarrow{\$} \color{Orange}\mathcal{F}[\mathcal{X}, \mathcal{X}]$  & &\\ [2pt]
	$b ==1: k\xleftarrow{\$} \keyS, {\color{Orange} f = \Enc(k, \cdot )}$  & &\\ [2pt]
	& $ \xleftarrow{\quad x_i \quad} $  & $x_i \in \mathcal{X}  $\\[5pt]
	& $\xrightarrow{f (x_i)}$  &\\[5pt]
	& $\xleftarrow{\quad \hat{b} \quad}$ & \\ [5pt]
\end{tabular}
\begin{tikzpicture}[overlay]
\node at (-3.8, -0.3) {\AxisRotator};
\draw[fill=none, draw=white, opacity=0.5] (-9.9,-2.3) rectangle (-4.8,3.0); 
\draw[fill=none, draw=white, opacity=0.5] (-3.0,-2.3) rectangle (0.0,3.0); 
\end{tikzpicture}
\vfill
\normalsize
{$\star$ для достаточно большой длины блока открытого/шифр-текста} 
\end{frame}



\begin{frame}{Блок-шифр $\implies^\star $ PRF}
\LARGE
{\color{Orange} Теорема. } Пусть $B = (\KeyGen, \Enc, \Dec)$ --- блок-шифр с 
\[
\Enc: \keyS \times \mathcal{X} \rightarrow \mathcal{X} - \text{ шифр. функцией, такой что } 1/|\mathcal{X}| - \negl(\lambda).
\]
{\color{Orange}$B$ --  безопасный блок-шифр тогда и только тогда, когда $\Enc$ -- безопасная PRF. }\\[10pt]


\end{frame}

\begin{frame}{Блок-шифр $\implies^\star $ PRF}

\Large
\vspace{-30pt}
\textit{Скетч док-ва.} ppt $\adv$ не может отличить $f  \xleftarrow{\$} \mathcal{F}[\mathcal{X}, \mathcal{X}]$ от $f \xleftarrow{\$} S_{|\mathcal{X}|}$. \\[10pt]
Положим, $\adv$ делает $\leq Q$ запросов к $\mathcal{C}$. \\[10pt]

\begin{columns}[T]
	\begin{column}{0.25\textwidth}
		\begin{tabular}{c c}
			{\color{Orange} Челленджер $\mathcal{C} $} &\\
			{\color{Orange} 	$ f  \xleftarrow{\$} \mathcal{F}[\mathcal{X}, \mathcal{X}]$ } & \\[5pt]
			$z_1, \ldots, z_Q \xleftarrow{\$} \mathcal{X}$ & \\
			{\color{Orange} if $x_i == x_j$, {\small$(j<i)$} }& \\
			$\quad y_i := y_j$ & \\
			\hspace{-65pt}{\color{Orange} else} &  \\
			$\mkern17mu  y_i := z_i$   & \\
		\end{tabular}
		\begin{tikzpicture}[overlay]
			\draw[fill=none, draw=white, opacity=0.5] (0,-1.1) rectangle (4.2,3.7); 
		\end{tikzpicture}
	\end{column}%
	\begin{column}{0.65\textwidth}
		\begin{tabular}{c c}
			{\color{Orange} Челленджер $\mathcal{C} $} &\\
			{\color{Orange} 	$ f  \xleftarrow{\$} S_{|\mathcal{X}|}$ } & \\[5pt]
			$z_1, \ldots, z_Q \xleftarrow{\$} \mathcal{X}$ & \\
			{\color{Orange} if $x_i == x_j$, {\small$(j<i)$} }&  \\
			$\quad y_i := y_j$ & \\
			\hspace{-65pt}{\color{Orange} else} &  \\
				$\mkern17mu  y_i := z_i$   &  \\[4pt]
			\hspace{70pt} if $z_i \in \{y_1, \ldots, y_{i-1} \}$ & \\[3pt]
			\hspace{80pt} $y_i \xleftarrow{\$} \mathcal{X} \setminus \{y_1, \ldots, y_{i-1} \}$& \\
		\end{tabular}
		\begin{tikzpicture}[overlay]
		\draw[fill=none, draw=white, opacity=0.5] (-6.0,-2.3) rectangle (-0.7,2.5); 
		\end{tikzpicture}
	\end{column}
\end{columns}
\end{frame}

\begin{frame}[fragile]{Псевдослучайный генератор $\implies$ Псевдослучайная функция}
\vspace{-10pt}
	\Large
	\begin{align*}
		G: & \{0,1\}^n \rightarrow \{0,1\}^{2n} - \text{PRG} \\
		 & \hskip3em \Downarrow \\
		F:  & \{0,1\}^n \times \{0,1\}^{\ell} \rightarrow \{0,1\}^n  - \text{PRF} 
	\end{align*}
	
	Положим $G(s) = (G_0(s), G_1(s))$, $G_i(s) \in \{0,1\}^n$.
	
	\begin{columns}[T]
		\begin{column}{0.3\textwidth}
			\LARGE
			\begin{align*}
				&F(k, x): \\
				&\hspace{10pt} t  = s \\
				&\hspace{10pt}\text{For } i = 1 \text{ to } n: \\
				&\hspace{20pt} t  = G_{x_i}(s) \\
				&\hspace{10pt} \text{return } t\\
			\end{align*}
		\end{column}%
		\begin{column}{0.55\textwidth}
			\begin{tikzpicture}[level/.style={sibling distance=30mm/#1}]
			\node [circle,draw] (z){}
			child {node [circle,draw] (a) {}
				child {node [circle,draw]   {}
						child {node [circle,draw] (d) {} edge from parent node[left,draw=none] {$0$} }
						child {node [circle,draw] (e) {} edge from parent node[right,draw=none] {$1$}}
						edge from parent node[left,draw=none] {$0$}
				}
				child {node [circle,draw] (g) {}
					child {node [circle,draw]  {} edge from parent node[left,draw=none] {$0$} }
					child {node [circle,draw]  {} edge from parent node[right,draw=none] {$1$}}
				 edge from parent node[right,draw=none] {$1$}
				}
			edge from parent node[left,draw=none] {$0$}
			}
			child {node [circle,draw] (j) {}
				child {node [circle,draw]  {}
					child {node [circle,draw]  {} edge from parent node[left,draw=none] {$0$}}
					child {node [circle,draw]  {} edge from parent node[right,draw=none] {$1$}}
				edge from parent node[left,draw=none] {$0$}
				}
				child {node [circle,draw]  {}
					child {node [circle,draw]  {} edge from parent node[left,draw=none] {$0$}}
					child {node [circle,draw]  {} edge from parent node[right,draw=none] {$1$}}
				edge from parent node[right,draw=none] {$1$}
				}
			edge from parent node[right,draw=none] {$1$}
			};
			\end{tikzpicture}
		\end{column}
	\end{columns}
\end{frame}


\end{document}