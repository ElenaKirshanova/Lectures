\documentclass[usenames,dvipsnames,8pt,aspectratio=169]{beamer}
\usepackage{amsmath,amsfonts,amssymb}
\usepackage{mathtools}
\usepackage{etex} %for Windows
\usepackage[utf8]{inputenc}
\usepackage[english, russian]{babel} 
%\usepackage{microtype}			% Better interword spacing and additional kerning.
\usepackage{ellipsis}			% Adjusted space with \dots between two words.
\usepackage{graphicx}
\usepackage{pstricks}

\usepackage{xcolor}


\usepackage{changepage}

\usepackage{algorithm}
\usepackage{algpseudocode}
%\usepackage[]{algorithm2e}
%\usepackage{algorithmic}

%\usepackage{tcolorbox}

\addtobeamertemplate{footline}{%
	\setlength\unitlength{1ex}%
	\begin{picture}(0,0) 
	% \put{} defines the position of the frame
	\put(155,0){\makebox(0,0)[bl]{
			%\includegraphics[scale=0.65]{white_square}
			%\includegraphics[scale=0.65]{dark_square}
			\includegraphics[scale=0.65]{grey_circle}
	}}%
	\end{picture}%
}{}




\usepackage{tikz}
\usetikzlibrary{tikzmark,calc}
\usetikzlibrary{positioning, backgrounds}
\usetikzlibrary{arrows, chains, matrix, scopes, patterns, shapes, fit}
\usetikzlibrary{mindmap,trees,shadows}
\usetikzlibrary{decorations.pathreplacing}
\usetikzlibrary{crypto.symbols}

\usepackage{pgfplots}






%\newcommand\strikeout[2][]{%
%	\begin{tabular}[b]{@{}c@{}} 
%		\makebox(0,0)[cb]{{#1}} \\[-0.2\normalbaselineskip]
%		\rlap{\color{Orange}\rule[0.5ex]{\widthof{#2}}{1.5pt}}#2
%\end{tabular}}



\usepackage{listings} % for C++ code

\usepackage{braket}
%\usepackage[braket, qm]{qcircuit}



\usepackage[T1]{fontenc}
%\usepackage[sfdefault,scaled=.85]{FiraSans}
%\usepackage{newtxsf}
%\usepackage[nomap]{FiraMono}





\usefonttheme[onlymath]{serif}
\renewcommand\sfdefault{cmbr}

\renewcommand{\bfdefault}{sb}

\definecolor{CharCoalDark}{RGB}{13, 16, 19}
\definecolor{Orange}{RGB}{255, 165,0}
\definecolor{DarkOrange}{RGB}{255, 165,0}
\definecolor{LightSalmon}{RGB}{255, 160, 122}
\definecolor{LeafGreen}{RGB}{34, 139,  34}
\definecolor{Coral}{RGB}{255, 127, 80}
\definecolor{DarkTurquoise}{RGB}{0, 206, 209}

%\newtheorem{defRus}{Определение}
%\newtheorem{thmRus}{Теорема}
%s\newtheorem{corRus}{Следствие}


\setbeamercolor{background canvas}{bg=CharCoalDark}

\setbeamerfont{title}{series=\bfseries}
\setbeamercolor{title}{fg=Orange}
\setbeamercolor{section in toc}{fg=white}
\setbeamercolor{frametitle}{fg=Orange}
\setbeamercolor{normal text}{fg=white}
%\setbeamercolor{normal text}{fontsize=12pt}
\setbeamercolor{itemize item}{fg=Orange}
\setbeamercolor{enumerate item}{fg=Orange}
\setbeamercolor{enumerate item item}{fg=Orange}
\setbeamercolor{itemize item item}{fg=Orange}
\setbeamercolor{enumerate item}{fg=Orange}
\setbeamercolor{block title}{bg=DarkOrange,fg=white}
\setbeamerfont{block title}{series=\bfseries}

\setbeamertemplate{itemize item}[circle]
\setbeamertemplate{eumerate subitem}{\color{Orange}[$\checkmark$]}
\setbeamertemplate{itemize subitem}{\color{Orange}\Large$\textbullet$}
\setbeamertemplate{itemize subitem}{\color{Orange} \tiny $\blacksquare$}

% footnote without a marker
\newcommand\blfootnote[1]{%
	\begingroup
	\renewcommand\footnoterule{}
	\renewcommand\thefootnote{}\footnote{#1}%
	\addtocounter{footnote}{-1}%
	\endgroup
}

\newcommand*{\Scale}[2][4]{\scalebox{#1}{\ensuremath{#2}}}%

\newcommand\Item[1][]{%
	\ifx\relax#1\relax  \item \else \item[#1] \fi
	\abovedisplayskip=0pt\abovedisplayshortskip=0pt~\vspace*{-\baselineskip}}

%\pgfdeclareradialshading{ring}{\pgfpoint{0cm}{0cm}}%
%{rgb(0cm)=(1,1,1);
%	rgb(0.7cm)=(1,1,1);
%	rgb(0.719cm)=(1,1,1);
%	rgb(0.72cm)=(0.975,0,0);
%	rgb(0.9cm)=(1,1,1)}
%
%\usepackage[absolute,overlay]{textpos} %to clip to a corner
%\newcommand\FrameText[1]{%
%	\begin{textblock*}{\paperwidth}(\textwidth-35pt, 10 pt)
%		\raggedright #1\hspace{.5em}
%\end{textblock*}}

%\makeatletter
%\let\save@measuring@true\measuring@true
%\def\measuring@true{%
%	\save@measuring@true
%	\def\beamer@sortzero##1{\beamer@ifnextcharospec{\beamer@sortzeroread{##1}}{}}%
%	\def\beamer@sortzeroread##1<##2>{}%
%	\def\beamer@finalnospec{}%
%}
%\makeatother


\title{Лекция №3 \\[10pt]
	Часть 2. Атаки на блочнык шифры.}

\date{ Елена Киршанова \\  \textbf{Курс ``Основы криптографии''} \\  }


\setbeamertemplate{navigation symbols}{} %removes navigation

% proper highlightling of a code-snippet
\lstset{language=C++,
	keywordstyle=\color{magenta},
	stringstyle=\color{Goldenrod},
	commentstyle=\color{gray},
	breaklines=false,
	%morecomment=[l][\color{magenta}]{\#}
}

%\setlength{\parskip}{8pt}
\input{header} %all defs
\begin{document}
	
\begin{frame}
	\titlepage
\end{frame}



\begin{frame}{Блок-шифр}
\begin{figure}
	\includegraphics[width=0.7\textwidth]{BlockCipherGeneral}
\end{figure}

\end{frame}



\begin{frame}{Алгоритм перебора}

\Large
{\color{Orange}\textbf{Идея:}}  перебор ключа $k \in \{0,1\}^{\kappa}$ \\[5pt]
	
	Для DES/AES/GOST:  достаточно двух пар (открытый текст, шифр-текст) $(m_1, c_1 = \Enc(k, m_1)), (m_2, c_2=\Enc(k,m_2))$, чтобы определить $k$ с большой вероятностью. \\
	
{\color{Orange}\textbf{Сложность:}} $\bigO(2^{\kappa})$ \\[110pt]



{\color{Orange}\textbf{Пример:}}  DES $k \in \{0,1\}^{56}$:

\begin{itemize}
	\item {\color{Orange}\textbf{'99-e}} ~22 часа на DeepCrack: дорогое железо+распределенная сеть
	\item {\color{Orange}\textbf{'07-е}}~13 дней COPACOBANА: FPGA, дешевле
\end{itemize}
\end{frame}

\begin{frame}{Встреча посередине}
\LARGE
{\color{Orange}\textbf{Плохая идея:}} шифрование ``матрёшкой'' с двумя ключами (2-DES).

\[
	\Enc(m) = \Enc(k_2 , \Enc(k_1, m)) , \quad k = (k_1, k_2) \in \{0,1\}^{2 \kappa}
\]
{\color{Orange}\textbf{Сложность перебора:}} $\bigO(2^{2\kappa})$ \\
{\color{Orange}\textbf{Сложность Встречи посередине:}} $\bigO(2^{\kappa})$ \\[5pt]

Дано $(m^\star, c^\star = \Enc(k^\star_2 , \Enc(k^\star_1, m^\star))  )$\\[5pt]

\begin{tikzpicture}
\draw[-, color=white] (-2,-2.) -- (-2, 1.5); 
\draw[-, color=white] (2,-2.) -- (2, 1.5); 
\node[] at (-0., 1.4)  {T};

\node[] at (0., 0.5)  {$(k_1, c = \Enc(k_1, m))$};
\node[] at (0., 0.1)  {$\vdots$};
\node[] at (0., -0.6)  {$\forall k_1 \in\{0,1\}^{\kappa} $};

\node[] at (5., 1)  {For all $ k_2 \in\{0,1\}^{\kappa}: $};
\node[] at (5.5, 0.4)  {$c' = \Dec(k_2, c^\star)$};
\node[] at (5.5, -0.2)  {$\exists c \in T == c' $ ?};
\end{tikzpicture}


	
\end{frame}
%
\begin{frame}{Атаки на дизайн: линейный криптанализ }
\LARGE
	Цель {\color{Orange}\textbf{линейного криптанализа:}} получить уравнения для ключа $k$ через биты $m, c$.
	
	{\color{Orange}\textbf{Пример:}} S-бокс $\{0,1\}^2 \rightarrow \{0,1\}^2$
	\begin{columns}
	\begin{column}{0.4\textwidth}
		\begin{tabular}{c | c | c | c}
			0 & 1 & 2 & 3 \\ \hline
			2 & 0 & 3 & 1
		\end{tabular}
	\end{column}
	\begin{column}{0.5\textwidth}
		\begin{tabular}{c | c | c | c}
			$X_0$ & $X_1$ & $Y_0$ & $Y_1$ \\ \hline
			0 & 0 & 1 & 0 \\
			0 & 1 & 0 & 0 \\
			1 & 0 & 1 & 1 \\
			1 & 1 & 0 & 1 
		\end{tabular}
	\end{column}
	\end{columns}
	
	Линейность: \\
	$X_0 = Y_1$ с вероятностью 1\\
	$X_1 = Y_0 \oplus 1$  с вероятностью 1\\[10pt]
	
	{\color{Orange}\textbf{Такой $S$-box не является стойким:}} легко получить линейные \\ уравнения вида $f(m_i, c_i, k_i) = 0$.
	
\end{frame}

\begin{frame}{Линейный криптанализ: таблица линейной аппроксимации }
\large

\begin{center}
\begin{tabular}{c | c | c | c | c | c }
	$X_0$ & $X_1$ & $Y_0$ & $Y_1$ & $X_0 + X_1 $ & $Y_0+Y_1$\\ \hline
	0 & 0 & 1 & 0 & 0 & 1 \\
	0 & 1 & 0 & 0 & 1 & 0\\
	1 & 0 & 1 & 1 & 1 & 0\\
	1 & 1 & 0 & 1 & 0 & 1\\ 
\end{tabular}
\end{center}

\vspace{5pt}


\begin{tabular}{c | c | c | c | c  }
	& \textbf{$0\cdot Y_0 \oplus 0\cdot Y_1 $} & \textbf{$0\cdot Y_0 \oplus 1\cdot Y_1 $} & \textbf{$1\cdot Y_0 \oplus 1\cdot Y_1 $} & \textbf{$1\cdot Y_0 \oplus 1\cdot Y_1 $} \\ \hline
	& & & &\\ 
	\textbf{$0\cdot X_0 \oplus 0\cdot X_1 $} & & & &\\  [8pt] \hline
	& & & &\\ 
	\textbf{$0\cdot X_0 \oplus 1\cdot X_1 $} & & & &\\  [8pt] \hline 
	& & & &\\ 
	\textbf{$1\cdot X_0 \oplus 0\cdot X_1 $} & & & &\\  [8pt] \hline 
	& & & &\\ 
	\textbf{$1\cdot X_0 \oplus 1\cdot X_1 $} & & & &\\  [8pt] 
\end{tabular}

\vspace{55pt}
\end{frame}

\begin{frame}{Атаки на дизайн: дифференциальный криптанализ }
\Large 
Положим $Y = S(m \oplus k), \; Y' = S(m' \oplus k)$ -- одно применение $S$-бокса. \\[10pt]
$\Delta Y = Y' \oplus Y$. $\Delta X = (m \oplus k) \oplus (m' \oplus k) = m \oplus m'$ \\[10pt]

Цель {\color{Orange}\textbf{дифференциального криптанализа:}} использовать частотный анализ для вычисления вероятности появления $\Delta Y$ при каком-либо $\Delta X$. \\[10pt]

	\begin{tabular}{c | c | c | c | c  }
		$X$ & $Y$ & $ \Delta X = 10   $ & $ \Delta X = 01   $ & $ \Delta X = 11   $  \\ \hline
		& & & &\\ [-2pt]
		00 & 10 &  & &  \\  [5pt] \hline
		& & & &\\ [-2pt]
		01 & 00 &  & & \\[5pt] \hline
			& & & &\\ [-2pt]
		10 & 11 &  & & \\ [5pt] \hline
			& & & &\\ [-2pt]
		11 & 01 &  & & 
	\end{tabular}% 
\hspace{5pt}
	\begin{tabular}{c | c | c | c | c  }
		& 00 & 01& 10 & 11 \\  [5pt]  \hline
	00 & & & &  \\   \hline
	01 & & & &  \\   \hline
	10 & & & &  \\  \hline
	11 & & & &  \\   \hline
\end{tabular}%
\end{frame}


\begin{frame}{Атаки на реализацию}
\LARGE
\begin{itemize}
	\itemsep 1em
	\item Атаки по сторонним каналам (side-channel attacks): \\[2pt]
	замер {\color{Orange}\textbf{времени}} или {\color{Orange}\textbf{мощности}}, используемых в процессе $\Enc, \Dec$   \\[5pt]
	Эти величины не должны зависеть от секретного ключа.

	\item Внесение неисправностей (Fault-injection attacks)  \\[5pt]
	 внешние воздействия на устройство, порождения аппаратных ошибок (нагрев, ЭМ волны)
\end{itemize}
\end{frame}


\begin{frame}{Напоследок}
\Huge

\begin{enumerate}
	\itemsep 1em
	\item {\color{Orange}\textbf{Не}} изобретайте {\color{Orange}\textbf{свой собственный}}  блок-шифр
	\item {\color{Orange}\textbf{Используйте}} реализации блок-шифров из проверенных временем библиотек (OpenSSL)

\end{enumerate}
\end{frame}


%\begin{frame}{Что почитать}
%	\begin{figure}
%		\includegraphics[height=0.7\textheight]{Book_cover}
%	\end{figure}
%\end{frame}

\end{document}