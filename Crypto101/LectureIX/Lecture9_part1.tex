\documentclass[usenames,dvipsnames,8pt,aspectratio=169]{beamer}
\usepackage{amsmath,amsfonts,amssymb}
\usepackage{mathtools}
\usepackage{etex} %for Windows
\usepackage[utf8]{inputenc}
\usepackage[english, russian]{babel} 
%\usepackage{microtype}			% Better interword spacing and additional kerning.
\usepackage{ellipsis}			% Adjusted space with \dots between two words.
\usepackage{graphicx}
\usepackage{pstricks}

\usepackage{xcolor}


\usepackage{changepage}

\usepackage{algorithm}
\usepackage{algpseudocode}
%\usepackage[]{algorithm2e}
%\usepackage{algorithmic}

%\usepackage{tcolorbox}


\usepackage{caption}
\usepackage{subcaption}
%\usepackage{stackengine}


\usepackage{tikz}
\usetikzlibrary{tikzmark,calc}
\usetikzlibrary{positioning, backgrounds}
\usetikzlibrary{arrows, chains, matrix, scopes, patterns, shapes, fit}
\usetikzlibrary{mindmap,trees,shadows}
\usetikzlibrary{decorations.pathreplacing}
%\usetikzlibrary{crypto.symbols}

\usepackage{pgfplots}

\pgfmathdeclarefunction{gauss}{2}{%
	\pgfmathparse{1/(#2*sqrt(2*pi))*exp(-((x-#1)^2)/(2*#2^2))}%
}


\tikzset{
	invisible/.style={opacity=0},
	visible on/.style={alt={#1{}{invisible}}},
	alt/.code args={<#1>#2#3}{%
		\alt<#1>{\pgfkeysalso{#2}}{\pgfkeysalso{#3}} % \pgfkeysalso doesn't change the path
	},
}

\newcommand\strikeout[2][]{%
	\begin{tabular}[b]{@{}c@{}} 
		\makebox(0,0)[cb]{{#1}} \\[-0.2\normalbaselineskip]
		\rlap{\color{Orange}\rule[0.5ex]{\widthof{#2}}{1.5pt}}#2
\end{tabular}}

\newcommand\Fontvi{\fontsize{11}{13.2}\selectfont}

\usepackage{listings} % for C++ code

\usepackage{braket}
%\usepackage[braket, qm]{qcircuit}



\usepackage[T1]{fontenc}
%\usepackage[sfdefault,scaled=.85]{FiraSans}
%\usepackage{newtxsf}
%\usepackage[nomap]{FiraMono}





\usefonttheme[onlymath]{serif}
\renewcommand\sfdefault{cmbr}

\renewcommand{\bfdefault}{sb}

\definecolor{CharCoalDark}{RGB}{13, 16, 19}
\definecolor{Orange}{RGB}{255, 165,0}
\definecolor{DarkOrange}{RGB}{255, 165,0}
\definecolor{LightSalmon}{RGB}{255, 160, 122}
\definecolor{LeafGreen}{RGB}{34, 139,  34}
\definecolor{Coral}{RGB}{255, 127, 80}
\definecolor{DarkTurquoise}{RGB}{0, 206, 209}

%\newtheorem{defRus}{Определение}
%\newtheorem{thmRus}{Теорема}
%s\newtheorem{corRus}{Следствие}


\setbeamercolor{background canvas}{bg=CharCoalDark}

\setbeamerfont{title}{series=\bfseries}
\setbeamercolor{title}{fg=Orange}
\setbeamercolor{section in toc}{fg=white}
\setbeamercolor{frametitle}{fg=Orange}
\setbeamercolor{normal text}{fg=white}
%\setbeamercolor{normal text}{fontsize=12pt}
\setbeamercolor{itemize item}{fg=Orange}
\setbeamercolor{enumerate item}{fg=Orange}
\setbeamercolor{enumerate subitem}{fg=Orange}
\setbeamercolor{itemize item item}{fg=Orange}
\setbeamercolor{enumerate item}{fg=Orange}
\setbeamercolor{block title}{bg=DarkOrange,fg=white}
\setbeamerfont{block title}{series=\bfseries}

\setbeamertemplate{itemize item}[circle]
\setbeamertemplate{eumerate subitem}{\color{Orange}[$\checkmark$]}
\setbeamertemplate{itemize subitem}{\color{Orange}\Large$\textbullet$}
\setbeamertemplate{itemize subitem}{\color{Orange} \tiny $\blacksquare$}

% footnote without a marker
\newcommand\blfootnote[1]{%
	\begingroup
	\renewcommand\footnoterule{}
	\renewcommand\thefootnote{}\footnote{#1}%
	\addtocounter{footnote}{-1}%
	\endgroup
}

\newcommand*{\Scale}[2][4]{\scalebox{#1}{\ensuremath{#2}}}%

\newcommand\Item[1][]{%
	\ifx\relax#1\relax  \item \else \item[#1] \fi
	\abovedisplayskip=0pt\abovedisplayshortskip=0pt~\vspace*{-\baselineskip}}

\pgfdeclareradialshading{ring}{\pgfpoint{0cm}{0cm}}%
{rgb(0cm)=(1,1,1);
	rgb(0.7cm)=(1,1,1);
	rgb(0.719cm)=(1,1,1);
	rgb(0.72cm)=(0.975,0,0);
	rgb(0.9cm)=(1,1,1)}

\usepackage[absolute,overlay]{textpos} %to clip to a corner
\newcommand\FrameText[1]{%
	\begin{textblock*}{\paperwidth}(\textwidth-35pt, 10 pt)
		\raggedright #1\hspace{.5em}
\end{textblock*}}

\makeatletter
\let\save@measuring@true\measuring@true
\def\measuring@true{%
	\save@measuring@true
	\def\beamer@sortzero##1{\beamer@ifnextcharospec{\beamer@sortzeroread{##1}}{}}%
	\def\beamer@sortzeroread##1<##2>{}%
	\def\beamer@finalnospec{}%
}
\makeatother

\AtBeginSection[]
{
	\begin{frame}<beamer>
		\frametitle{Outline}
		\tableofcontents[currentsection]
	\end{frame}
}

\addtobeamertemplate{footline}{%
	\setlength\unitlength{1ex}%
	\begin{picture}(0,0) 
	% \put{} defines the position of the frame
	\put(155,0){\makebox(0,0)[bl]{
			%\includegraphics[scale=0.65]{white_square}
			%\includegraphics[scale=0.65]{dark_square}
			\includegraphics[scale=0.65]{grey_circle}
	}}%
	\end{picture}%
}{}


\newcommand{\AxisRotator}[1][rotate=0]{%
	\tikz [x=0.4cm,y=1.3cm,line width=.2ex,-stealth,#1] \draw[color=Orange] (0,0) arc (-150:150:2 and 1);%
}
\newcommand{\AxisRotatorTwo}[1][rotate=0]{%
	\tikz [x=0.5cm,y=1.7cm,line width=.2ex,-stealth,#1] \draw[color=Orange] (0,0) arc (-150:150:2 and 1);%
}

\title{Лекция №9 \\[10pt]
	Шифрование с открытым ключом}

\date{ Елена Киршанова \\  \textbf{Курс ``Основы криптографии''} \\  }



\setbeamertemplate{navigation symbols}{} %removes navigation

% proper highlightling of a code-snippet
\lstset{language=C++,
	keywordstyle=\color{magenta},
	stringstyle=\color{Goldenrod},
	commentstyle=\color{gray},
	breaklines=false,
	%morecomment=[l][\color{magenta}]{\#}
}

%\setlength{\parskip}{8pt}
\input{header} %all defs
\begin{document}
	
\begin{frame}
	\titlepage
\end{frame}

\begin{frame}{Применение}
	\Large 

	\begin{itemize}
		\itemsep 10pt
		\item шифрование имейлов 
		\item доступ к шифрованным файлам\\[5pt]
		\begin{enumerate}
			\itemsep 5pt
			\Large
			\item {\color{Orange} A} шифрует данные $m$ с помощью {\color{Orange} симметрического } алгоритма $\Enc_{\mathtt{sym}}(k_{\text{A}}, m)$. 
			\item Если {\color{Orange} B} нужно получить доступ к $m$, шифрует свой симметрический ключ с помощью ассиметрического алгоритма  $\Enc_{\mathtt{assym}}(\pk_{\text{A}}, k_{\text{A}})$
		\end{enumerate}
		
		\item 
	\end{itemize}

\vspace{15pt}

В этой лекции: алгоритмы $\Enc_{\mathtt{assym}}()$.

\end{frame}

\begin{frame}{Ассиметрическое шифрование: определение}
\Large
{\color{Orange}{Ассиметрическое шифрование}} состоит из трех ppt алгоритмов
\begin{itemize}
	\itemsep 10pt
	\item Генерация ключей: $\KeyGen(1^\lambda) \rightarrow (\sk, \pk) $
	\item Шифрование: $\Enc(\pk, m ) \rightarrow c$
	\item Дешифрование:  $\Dec(\sk, c) \rightarrow m$
\end{itemize}

\vspace{15pt}
{\color{Orange}{Корректность:}} $\forall m, \forall (\pk, \sk) \leftarrow \KeyGen(1^\lambda) : \Dec(\sk, \Enc(\pk, m)) = m$

\vspace{15pt}

{\color{Orange}{Безопасность}} для ассиметрического шифрования определяется \\ аналогично симметрическому: семантическая, CPA, CCA. \\[5pt]

Основное отличие: атакующий $\adv$ получает открытый ключ $\pk$.

\end{frame}

\begin{frame}{Семантическая стойкость: симметрическое шифрование}
\Large

	\begin{center}
			$\Pi = (\KeyGen, \Enc, \Dec)$ \\[15pt]
			
			\begin{tabular}{c c c}
				{\color{Orange} Челленджер $\mathcal{C}$ } & & {\color{Orange} Атакующий $\mathcal{A}$ }\\ [5pt]
				$k \leftarrow \KeyGen(1^\lambda)$ & $\xrightarrow{\quad \Huge \lambda \quad}$  &\\[5pt]
				& $\xleftarrow{\; \Huge m_0, m_1 \in \mesS \;}$  &$m_0, m_1 \leftarrow \mesS $\\ [4pt]
				$b \xleftarrow{\$} \{0,1\}  $& &\\ [4pt]
				$c \leftarrow \Enc(k, m_b)$ & &\\ [4pt]
				& $\xrightarrow{\quad c \quad}$ & \\ [4pt]
				& $\xleftarrow{\quad \hat{b} \quad}$ & \\ [4pt]
			\end{tabular}
			\begin{tikzpicture}[overlay]
			\draw[fill=none, draw=white, opacity=0.5] (-8.5,-2.3) rectangle (-5.3,3.0); 
			\draw[fill=none, draw=white, opacity=0.5] (-3.0,-2.3) rectangle (0.0,3.0); 
			\end{tikzpicture}
	\end{center}
$\mathtt{W_{\Pi, \adv}}$ -- событие $b == \hat{b}$. \\ [4pt]
$\mathtt{SSAdv} = \abs{\Pr[\mathtt{W_{\Pi, \adv}}] - \frac{1}{2}}$ -выигрыш  $\adv$  \\ [4pt]
\color{Orange} Cемантическая безопасность:  для любого ppt $\adv:$  $\mathtt{SSAdv} = \negl(\lambda).$
\end{frame}

\begin{frame}{Семантическая стойкость: асимметрическое шифрование}
\Large

\begin{center}
	$\Pi = (\KeyGen, \Enc, \Dec)$ \\[15pt]
	
	\begin{tabular}{c c c}
		{\color{Orange} Челленджер $\mathcal{C}$ } & & {\color{Orange} Атакующий $\mathcal{A}$ }\\ [5pt]
		{\color{Orange}  $ (\pk, \sk) \leftarrow \KeyGen(1^\lambda)$ } & $\xrightarrow{\quad \Huge \lambda, \; \pk \quad}$  &\\[4pt]
		& $\xleftarrow{\; \Huge m_0, m_1 \in \mesS \;}$  &$m_0, m_1 \leftarrow \mesS $\\ [4pt]
		$b \xleftarrow{\$} \{0,1\}  $& &\\ [4pt]
		$c \leftarrow \Enc(k, m_b)$ & &\\ [4pt]
		& $\xrightarrow{\quad c \quad}$ & \\ [4pt]
		& $\xleftarrow{\quad \hat{b} \quad}$ & \\ [4pt]
	\end{tabular}
	\begin{tikzpicture}[overlay]
	\draw[fill=none, draw=white, opacity=0.5] (-9.6,-2.3) rectangle (-5.3,3.0); 
	\draw[fill=none, draw=white, opacity=0.5] (-3.0,-2.3) rectangle (0.0,3.0); 
	\end{tikzpicture}
\end{center}
$\mathtt{W_{\Pi, \adv}}$ -- событие $b == \hat{b}$. \\ [4pt]
$\mathtt{SSAdv} = \abs{\Pr[\mathtt{W_{\Pi, \adv}}] - \frac{1}{2}}$ -выигрыш  $\adv$  \\ [4pt]
\color{Orange} Cемантическая безопасность:  для любого ppt $\adv:$  $\mathtt{SSAdv} = \negl(\lambda).$
\end{frame}

\begin{frame}{Атака на выбранный открытый текст: симметрическое шифрование}
\large
\begin{center}
	
	\begin{tabular}{c c c}
		{\color{Orange} Челленджер $\mathcal{C}$ } & & {\color{Orange} Атакующий $\mathcal{A}$ }\\ [5pt]
		$k \leftarrow \KeyGen(1^\lambda)$ & $\xrightarrow{\quad \Huge \lambda \quad}$  &\\[5pt]
		$b \xleftarrow{\$} \{0,1\}  $& &\\ [5pt]
		& $\xleftarrow{\; \Huge m_{0, i}, m_{1, i}  \;}$  &$m_{0, i}, m_{1,i} \leftarrow \mesS $\\ [5pt]
		
		$c_i \leftarrow \Enc(k, m_{b, i})$ &  &\\ [5pt]
		& $\xrightarrow{\quad c_i \quad}$ & \\ [5pt]
		& $\xleftarrow{\quad \hat{b} \quad}$ & \\ [5pt]
	\end{tabular}
	\begin{tikzpicture}[overlay]
	\draw[fill=none, draw=white, opacity=0.5] (-8.3,-2.3) rectangle (-5.0,2.7); 
	\draw[fill=none, draw=white, opacity=0.5] (-3.0,-2.3) rectangle (0.0,2.7); 
	\node at (-3.8, -0.1) {\AxisRotator};
	\end{tikzpicture}
\end{center}

\vspace{5pt}
$\mathtt{W_{\Pi, \adv}}$ -- событие $b == \hat{b}$. \\ [4pt]
$\mathtt{CPAAdv} = \abs{\Pr[\mathtt{W_{\Pi, \adv}}] - \frac{1}{2}}$ --выигрыш  $\adv$. \\ [4pt]
\color{Orange} Шифр-схема $\Pi$ CPA безопасна, если для любого ppt $\adv:$  $\mathtt{CPAAdv} = \negl(\lambda).$
\end{frame}

\begin{frame}{Атака на выбранный открытый текст: асимметрическое шифрование}
\large
\begin{center}
	
	\begin{tabular}{c c c}
		{\color{Orange} Челленджер $\mathcal{C}$ } & & {\color{Orange} Атакующий $\mathcal{A}$ }\\ [5pt]
		{\color{Orange}  $ (\pk, \sk) \leftarrow \KeyGen(1^\lambda)$ } & $\xrightarrow{\quad \Huge \lambda, \; \pk \quad}$  &\\[5pt]
		$b \xleftarrow{\$} \{0,1\}  $& &\\ [5pt]
		& $\xleftarrow{\; \Huge m_{0, i}, m_{1, i}  \;}$  &$m_{0, i}, m_{1,i} \leftarrow \mesS $\\ [5pt]
		
		$c_i \leftarrow \Enc(k, m_{b, i})$ &  &\\ [5pt]
		& $\xrightarrow{\quad c_i \quad}$ & \\ [5pt]
		& $\xleftarrow{\quad \hat{b} \quad}$ & \\ [5pt]
	\end{tabular}
	\begin{tikzpicture}[overlay]
	\draw[fill=none, draw=white, opacity=0.5] (-9.3,-2.3) rectangle (-5.0,2.7); 
	\draw[fill=none, draw=white, opacity=0.5] (-3.0,-2.3) rectangle (0.0,2.7); 
	\node at (-3.8, -0.1) {\AxisRotator};
	\end{tikzpicture}
\end{center}

\vspace{2pt}
$\mathtt{W_{\Pi, \adv}}$ -- событие $b == \hat{b}$. \\ [4pt]
$\mathtt{CPAAdv} = \abs{\Pr[\mathtt{W_{\Pi, \adv}}] - \frac{1}{2}}$ --выигрыш  $\adv$. \\ [4pt]
\color{Orange} Шифр-схема $\Pi$ CPA безопасна, если для любого ppt $\adv:$  $\mathtt{CPAAdv} = \negl(\lambda).$  \\ [15pt]

\textbf{Для ассиметрического шифрования: семантическая стойкость $\implies $ CPA стойкость.}
\end{frame}

\begin{frame}{Атака на выбранный открытый текст: симметрическое шифрование}
	\large
	\begin{center}
		
		\begin{tabular}{c c c}
			{\color{Orange} Челленджер $\mathcal{C}$ } & & {\color{Orange} Атакующий $\mathcal{A}$ }\\ [5pt]
			$k \leftarrow \KeyGen(1^\lambda)$ & $\xrightarrow{\quad \Huge \lambda \quad}$  &\\[5pt]
			$b \xleftarrow{\$} \{0,1\}  $& &\\ [5pt]
			& $\xleftarrow{\; \Huge m_{0, i}, m_{1, i}  \;}$  &$m_{0, i}, m_{1,i} \leftarrow \mesS $\\ [2pt]
			
			$c_i \leftarrow \Enc(k, m_{b, i})$ &  &\\ [2pt]
			$C = C \cup \{c_i\} $ & $\xrightarrow{\quad c_i \quad}$  &\\ [8pt]
			
			& $\xleftarrow{\; c_j \notin C   \;}$  &\\ [5pt]
			$\hat{m_j} \leftarrow \Dec(k, c_{j})$  &  $\xrightarrow{\quad c_i \quad}$ & \\[2pt]
			& $\xleftarrow{\quad \hat{b} \quad}$ & \\ [5pt]
		\end{tabular}
		\begin{tikzpicture}[overlay]
		\draw[fill=none, draw=white, opacity=0.5] (-8.3,-2.5) rectangle (-5.0,3.2); 
		\draw[fill=none, draw=white, opacity=0.5] (-3.0,-2.5) rectangle (0.0,3.2); 
		\node at (-3.8, -0.1) {\AxisRotatorTwo};
		\end{tikzpicture}
	\end{center}
	
	\vspace{-5pt}
	$\mathtt{W_{\Pi, \adv}}$ -- событие $b == \hat{b}$. \\ [4pt]
	$\mathtt{CCAAdv} = \abs{\Pr[\mathtt{W_{\Pi, \adv}}] - \frac{1}{2}}$ --выигрыш  $\adv$. \\ [4pt]
	\color{Orange} CCA безопасность: для любого ppt $\adv:$  $\mathtt{CCAAdv} = \negl(\lambda).$
\end{frame}

\begin{frame}{Атака на выбранный открытый текст: асимметрическое шифрование}
\large
\begin{center}
	
	\begin{tabular}{c c c}
		{\color{Orange} Челленджер $\mathcal{C}$ } & & {\color{Orange} Атакующий $\mathcal{A}$ }\\ [5pt]
		{\color{Orange}  $ (\pk, \sk) \leftarrow \KeyGen(1^\lambda)$ } & $\xrightarrow{\quad \Huge \lambda, \; \pk \quad}$  &\\[5pt]
		$b \xleftarrow{\$} \{0,1\}  $& &\\ [5pt]
		& $\xleftarrow{\; \Huge m_{0, i}, m_{1, i}  \;}$  &$m_{0, i}, m_{1,i} \leftarrow \mesS $\\ [2pt]
		
		$c_i \leftarrow \Enc(k, m_{b, i})$ &  &\\ [2pt]
		$C = C \cup \{c_i\} $ & $\xrightarrow{\quad c_i \quad}$  &\\ [8pt]
		
		& $\xleftarrow{\; c_j \notin C   \;}$  &\\ [5pt]
		$\hat{m_j} \leftarrow \Dec(k, c_{j})$  &  $\xrightarrow{\quad c_i \quad}$ & \\[2pt]
		& $\xleftarrow{\quad \hat{b} \quad}$ & \\ [5pt]
	\end{tabular}
	\begin{tikzpicture}[overlay]
	\draw[fill=none, draw=white, opacity=0.5] (-9.5,-2.5) rectangle (-5.0,3.2); 
	\draw[fill=none, draw=white, opacity=0.5] (-3.0,-2.5) rectangle (0.0,3.2); 
	\node at (-3.8, -0.2) {\AxisRotatorTwo};
	\end{tikzpicture}
\end{center}

\vspace{-5pt}
$\mathtt{W_{\Pi, \adv}}$ -- событие $b == \hat{b}$. \\ [4pt]
$\mathtt{CCAAdv} = \abs{\Pr[\mathtt{W_{\Pi, \adv}}] - \frac{1}{2}}$ --выигрыш  $\adv$. \\ [4pt]
\color{Orange} CCA безопасность: для любого ppt $\adv:$  $\mathtt{CCAAdv} = \negl(\lambda).$
\end{frame}

\begin{frame}{CPA безопасная схема шифрования Эль-Гамаля (El-Gamal)}
	\Large 
	%\vspace{-30pt}
	\begin{columns}[T]
		\begin{column}{0.50\textwidth}
			{\color{Orange}Ингредиенты:}
			\begin{itemize}
				\itemsep 5pt
				\item  $g$ -- образующий $\Z_p^\ast$ 
				\item $\Pi = (\KeyGen_s, \Enc_s, \Dec_s )$ -- симметрическая схема с множеством ключей $\keyS$
				\item $\Hash: \Z_p^\ast \times  \Z_p^\ast \rightarrow \keyS$ -- криптографическая хэш-функция
			\end{itemize} 
		\vspace{20pt}
			{\color{Orange} I. $\mathsf{\KeyGen}(1^\lambda):$}
			\begin{enumerate}
				\itemsep4pt
				\item $\sk = a \xleftarrow{\$} \Z_p^\ast$
				\item $\pk = g^a$
			\end{enumerate}
		\end{column}
		\begin{column}{0.45\textwidth}
		{\color{Orange} II. $\Enc(\pk, m):$}
			\begin{enumerate}
				\itemsep3pt
				\item $b \xleftarrow{\$} \Z_p^\ast$
				\item $v = g^b$ 
				\item $w = \pk^b$
				\item $k = \Hash(v, w)$
				\item $z \gets \Enc_s(k, m)$
				\item $\mathtt{return} \; c = (v, z)$
			\end{enumerate}
		\vspace{15pt}
			{\color{Orange} III. $\Dec(\sk, c = (v, z)):$}
			\begin{enumerate}
				\itemsep3pt
				\item $w = v^a$
				\item $k = \Hash(v, w)$
				\item $m \gets \Dec(k, z)$
				\item $\mathtt{return} \; m$ \\
			\end{enumerate}
		\end{column}
	\end{columns}
\end{frame}

\begin{frame}{Семантическая безопасность шифрования Эль-Гамаля}
	\Large
	{\color{Orange} Теорема.} Если $\Hash$ -- {\color{Orange}  случайный оракул}, то шифрование Эль-Гамаля семантически стойко при условии {\color{Orange} сложности задачи CDH} и семантической стойкости $\Pi$.
	%\vspace{10pt}
	\begin{center}
		\large
	\begin{tabular}{c c c}
		{\color{Orange} Челленджер $\mathcal{C}$ } & & {\color{Orange} Атакующий $\mathcal{A}$ }\\ [5pt]
		$ (\pk = g^a, v = g^b)$ & $\xrightarrow{\; \; \Huge g^a \;}$  &\\
		$b \xleftarrow{\$} \{0,1\}  $& &\\ 
		${\color{Orange}k} \xleftarrow{\$} \keyS  $& &\\ 
		& $\xleftarrow{  \Hash_i(v', w')}$ & \\ 
		 If $\mathcal{T}[v', w'] = \emptyset:$& & \\ [-2pt]
		$\mathcal{T}[v', w'] \xleftarrow{\$} \keyS $&$\xrightarrow{\; \; \Huge \mathcal{T}[v', w'] \;}$ &\\ 
		
		& ${\color{Orange}\xleftarrow{  \Hash_i^\star(v, w')}}$ & \\ 
		$\mathcal{T}[v, w'] := {\color{Orange}k} $&$\xrightarrow{\; \; \Huge \mathcal{T}[v, w'] \;}$ &\\[7pt] 
		& $\xleftarrow{\; \Huge m_{0}, m_{1}  \;}$  & \\
		$c = (v,   \Enc(k, m_{b})$ & $\xrightarrow{\quad c \quad}$  &\\ [2pt]
		& $\xleftarrow{\quad \hat{b} \quad}$ & \\ [5pt]
	\end{tabular}
	\begin{tikzpicture}[overlay]
	\draw[fill=none, draw=white, opacity=0.5] (-8.0,-3.2) rectangle (-4.5,3.7); 
	\draw[fill=none, draw=white, opacity=0.5] (-3.0,-3.2) rectangle (0.0,3.7); 
	\end{tikzpicture}
\end{center}
\end{frame}

\begin{frame}{Замечания}
	\Large
	\vspace{-40pt}
	\begin{itemize}
		\item Безопасность схемы Эль-Гамаля можно доказать и в стандартной модели (без случайного оракула) при условии сложности задачи DDH.
		\item Схемы Эль-Гамаля CCA безопасна опять же при условии сложности задачи {\color{Orange}Интерактивный CDH} \\[60pt]
		
		\item CCA безопасное шифрование можно построить на основе RSA (сложность факторизации).
	\end{itemize}
\end{frame}

\begin{frame}{Механизм Инкапсуляции ключа / Key Encapsulation Mechanism (KEM)}
	
	\Large
	
	\begin{center}
		
		{\color{Orange} схема Эль-Гамаля  = Гибрид симметричного шифра и KEM'a} \\[10pt]
	\end{center}
	
	Задача KEM: с помощью ассиметрических ключей сгенерировать ключ $k$ для симметрического шифрования.\\[10pt]
	
	$\mathtt{KEM} = (\KeyGen, \mathtt{Encap}, \mathtt{Decap})$ -- тройка ppt алгоритмов: \\
	\begin{itemize}
		\itemsep 10pt
		\item Генерация ключей: $\KeyGen(1^\lambda) \rightarrow (\sk, \pk) $
		\item Инкапсуляция: $\mathtt{Encap}(\pk ) \rightarrow (c, k)$
		\item Декапсуляция:  $\Dec(\sk, c) \rightarrow k$
	\end{itemize}
	
\end{frame}


\end{document}
