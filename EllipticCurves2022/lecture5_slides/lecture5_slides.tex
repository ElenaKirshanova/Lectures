% !TeX spellcheck = ru_RU-Russian
% !TeX program = xelatex
% !BIB TS-program = biber

\documentclass{beamer}

\usepackage[utf8]{inputenc}
\usepackage[russian]{babel}

%---tikz----
\usepackage{tikz}
\usetikzlibrary{arrows, chains, matrix, positioning, scopes, patterns, shapes}
\usepackage{pgfplots, subfigure}
\usepackage{extarrows}

\usepackage[backend=biber,firstinits=true,hyperref=true,style=numeric-comp]{biblatex}

\usepackage{../beamerthemeec2020}
{\footnotesize\bibliography{../biblio}}

\title{Эллиптические кривые}
\subtitle{Лекция 5. Алгоритмы подсчета $\mathbb{F}_q$-рациональных точек кривой. Часть I}
\author{Семён Новосёлов}
\institute{БФУ им. И. Канта}
\date{2022}

\begin{document}

\frame{\titlepage}

%\begin{frame}{Мотивация}
%    \begin{enumerate}
%        \item Для криптографических целей необходимо уметь считать число точек за время полиномиальное время от $\log{q}$.
%    \end{enumerate}
%\end{frame}
\begin{frame}{Мотивация}
    Криптография на \structure{DLOG} в группе $G$:
    \begin{center}
    (почти) простое $\#G$ \structure{$\implies$} стойкость к атаке Полига-Хелмана
    \end{center}
    
    \structure{Задачи:}
    \begin{enumerate}
        %\item для криптографических целей: простой метод генерации кривой по сравнению CM-методом
        \item подобрать кривую с заданным (простым) числом точек над полем $\mathbb{F}_q$ \structure{$\implies$} $CM$-метод
        \item подобрать кривую с простым числом точек над полем $\mathbb{F}_q$:
        \begin{itemize}
            \item выбирать случайную кривую и считать число точек пока не получится простое число точек
            \item ожидаемое число попыток: $O(\log{|G|})$ (следует из теоремы о распределении простых чисел)
        \end{itemize}
    \end{enumerate}
    $CM$-метод даёт лучшие кривые, но при этом существенно сложнее.
\end{frame}

\begin{frame}{Эндоморфизм Фробениуса}
\begin{equation*}
    \begin{split}
        \varphi_q: \overline{\mathbb{F}}_q &\rightarrow \overline{\mathbb{F}}_q \\
                 x & \mapsto x^q
    \end{split}
\end{equation*}

\begin{enumerate}
    \item $(x_1 + x_2)^q = x_1^q + x_2^q$
    \item $x^q = x, \forall x \in {\mathbb{F}_q}$
\end{enumerate}

\structure{Действие $\varphi_q$} на точки из $E(\overline{\mathbb{F}}_q)$:
\[
\varphi_q(x, y) = (x^q, y^q), \varphi_q(\mathcal{O}) = \mathcal{O}.
\]
\end{frame}

\begin{frame}{}%{Предварительные сведения}
\structure{Свойства $\varphi_q$}:
$E$ -- кривая над ${\mathbb{F}_q}, (x, y) \in E(\overline{\mathbb{F}}_q)$

\begin{enumerate}
    \item ${\varphi _q}(x,y) \in E( \overline{\mathbb{F}}_q )$
    \begin{itemize}
        \item[$\triangleleft$]
        $(y^2)^q = (x^3 + ax + b)^q \Leftrightarrow$
        \item[] $(y^q)^2 = (x^q)^3 + a x^q + b \Leftrightarrow (x^q, y^q) \in E ( \overline{\mathbb{F}}_q )$
        \structure{$\triangleright$}
    \end{itemize}
    
    \item $(x,y) \in E(\mathbb{F}_q) \Leftrightarrow \varphi_q(x,y) = (x,y)$
        \begin{itemize}
            \item[$\triangleleft$] $x \in \mathbb{F}_q \Leftrightarrow \varphi_q(x) = x$ \structure{$\triangleright$}
        \end{itemize}    
    \item $\mathop{\ker}(\varphi^n_q - 1) = E(\mathbb{F}_{q^n})$ \\
    %Отображение $( x, y ) \mapsto ( x^q, y^q ) - ( {x,y} )$, где «$-$» это вычитание на $E$.
    \item $\# E( \mathbb{F}_{q^n} ) = \deg ( \varphi_q^n - 1 )$
\end{enumerate}
\end{frame}

\begin{frame}{Граница Хассе-Вейля} 
\begin{block}{Теорема} Для любой кривой $E/\mathbb{F}_q$ выполняется:
%(Описывает границы $\# E( {{\mathbb{F}_q}} )$)
%Пусть $E$ -- кривая над ${\mathbb{F}_q}$. Тогда справедливо
\[
| {q + 1 - \# E( {{\mathbb{F}_q}} )} | \leqslant 2\sqrt q 
\]
\end{block}
\structure{$\triangleleft$}
Выводится из свойств (1)--(4) для ${\varphi _q}$ (см. [Washington~\S~4.2]). 
\structure{$\triangleright$}

\structure{След эндоморфизма Фробениуса:}
\[t  = q + 1 - \# E( {{\mathbb{F}_q}} ) = q + 1 - \deg ( {\varphi_q - 1} )\]

\begin{itemize}
    \item Асимптотически: $\# E(\mathbb{F}_q) \sim O(q)$
\end{itemize}
\end{frame}

\begin{frame}{Характеристический многочлен эндоморфизма Фробениуса}
\begin{block}{Теорема}
$E$ -- эллиптическая кривая над $\mathbb{F}_q$ и $t = q + 1 - \# E(\mathbb{F}_q)$. Тогда 
\[
\varphi _q^2 - t \varphi _q + q = 0
\]
как эндоморфизм на $E$ и $t$ определено уникально. Другими словами $\forall ( {x,y} ) \in E( \overline{\mathbb{F}}_q ):$
\begin{equation*}
(x^{q^2}, y^{q^2}) - t (x^q, y^q) + q(x,y) = \mathcal{O}.
\end{equation*}
\end{block}
\end{frame}

\begin{frame}{Кривые в подполе}
Кривая $E$ задана над $\mathbb{F}_q$ \structure{$\implies$} можем выразить $\# E(\mathbb{F}_{q^n})$ через $\# E( \mathbb{F}_q )$.

\begin{block}{Теорема}
Пусть $\# E( {{\mathbb{F}_q}} ) = q + 1 - t$. Запишем {\footnotesize $X^2 - t X + q = (X - \alpha)(X - \beta)$}. Тогда: 
$$
\# E(\mathbb{F}_{q^n}) = {q^n} + 1 - ( \alpha^n+ \beta ^n )\quad \forall n \geqslant 1.
$$
\end{block}

\begin{block}{Лемма}
Пусть $t_n = \alpha^n + \beta^n$. Тогда $t_0 = 2, t_1 = t$ и $t_{n+1} = t t_n - q t_{n-1}$ для всех $n \geq 1$.
\end{block}

\begin{itemize}
    \item Т.о., если известно $\#E(\mathbb{F}_q)$, то $\#E(\mathbb{F}_{q^n})$ находится за время $O(n)$ операций в $\mathbb{Z}$.
\end{itemize}
\end{frame}

\begin{frame}{Подсчёт точек на основе символа Лежандра}{Экспоненциальные алгоритмы подсчёта точек}
\begin{block}{Лемма}
Для $E: y^2 = x^3 + ax + b$ над $\mathbb{F}_q$ имеем:
\[
\# E(\mathbb{F}_q) = q + 1 + \sum\limits_{x \in \mathbb{F}_q} {\left(\frac{x^3 + ax + b}{\mathbb{F}_q} \right)}.
\]
\end{block}

\begin{itemize}
    \item Сложность: $O(q \operatorname{polylog}{q})$.
\end{itemize}
\end{frame}

\begin{frame}{Алгоритм Baby Step-Giant Step (BSGS)}{Экспоненциальные алгоритмы подсчёта точек}
\structure{Идея:} Пусть $N = \#E(\mathbb{F}_q)$ -- неизвестно. По теореме Лагранжа: $[N] P = \mathcal{O}, \forall P$.

Т.к. $q + 1 - 2 \sqrt{q} \leq N \leq q + 1 + 2 \sqrt{q}$ \structure{$\implies$} можем перебирать все $N$ проверяя условие $[N] P = \mathcal{O}$.

\begin{itemize}
    \item Наивный метод (brute force): $O(\sqrt{q})$.
    \item Алгоритм поиска коллизий/циклов (BSGS): $O(\sqrt[4]{q})$.
\end{itemize}
\end{frame}


\begin{frame}{Алгоритм (BSGS). Нахождение порядка точки}
\structure{Вход:} $P \in E(\mathbb{F}_q)$.

\structure{Выход:} $ord(P)$.
\begin{enumerate}
    \item $Q = (q+1) P$.
    \item Выбрать $m > \sqrt[4]{q}$, вычислить и сохранить в списке $L$ все пары $(j, [j] P)$ для $j = 0, \ldots, m$. \hfill{\structure{\textit{(baby steps)}}}
    \item \label{bsgs:order:step:giant}
    Вычислять точки $Q + k (2m P)$ для $k = -m, -(m-1), \ldots, m$ пока в $L$ не найдётся точка $\pm [j] P$ т.ч. $Q + k(2m P) = \pm [j] P$ \hfill{\structure{\textit{(giant steps)}}}
    \item Имеем $[q + 1 + 2 m k \mp j] P = \mathcal{O}$.\\$M = q + 1 + 2 m k \mp j$.
    \item \label{bsgs:order:step:factor} Найдём $p_1, \ldots, p_r$ -- различные простые делители $M$.
    \item Если $[M/p_i] P = \mathcal{O}$ для нек. $i$, то $M = M/p_i$ и перейти к шагу \ref{bsgs:order:step:factor}.
    \item Вернуть $M$.
\end{enumerate}
\end{frame}

\begin{frame}{Алгоритм (BSGS). Нахождение порядка точки}{Корректность}
Всегда ли найдётся коллизия на шаге \ref{bsgs:order:step:giant}?
\begin{block}{Лемма}
Пусть $x$ -- целое, т.ч. $|x| \leq 2 m^2$. Тогда $\exists x_0,x_1$ т.ч. $-m < x_0 \leq m$, $-m \leq x_1 \leq m$ и $x = x_0 + 2 m x_1$.
\end{block}
\begin{itemize}
    \item (Разбиение элемента $x$ на $2m$ старших бит и ($\lg{x} - 2m$) младших)
\end{itemize}
\end{frame}

\begin{frame}{Алгоритм (BSGS). Нахождение порядка точки}{Анализ сложности}
\structure{Шаг~1.} (быстрое умножение) $O(\log{q})$ сложений на кривой \structure{$\implies$} %$O(\log\operatorname{polylog}{q})$
$\widetilde{O}(\log^2{q})$
бит. операций

\structure{Шаг~2.} $\widetilde{O}(m) = \widetilde{O}(q^{1/4})$ -- время, $O(q^{1/4})$ -- память.

\structure{Шаг~\ref{bsgs:order:step:giant}.} $\widetilde{O}(2m) = \widetilde{O}(q^{1/4})$ -- ожидаемое количество переборов $k$.

\structure{Шаг~4.} Элементарные операции в $\mathbb{Z}$.

\structure{Шаг~5.} $L_q(1/3, c) = exp(c (\log{q})^{1/3} (\log{\log{q}})^{2/3} )$.

\structure{Шаг~6.} $O(\log{M}) = O(\log{q})$ сложений на кривой.

\structure{Итого}: самый затратный шаг $3$: $\widetilde{O}(q^{1/4})$.

\begin{block}{Замечания}
    \begin{enumerate}
        \item Для оптимизации памяти достаточно хранить только координату $x$.
        \item С помощью $\rho$-метода Полларда можно реализовать алгоритм, использующий только $\operatorname{polylog}{q}$ памяти.
    \end{enumerate}
\end{block}
\end{frame}

\begin{frame}{Алгоритм (BSGS). Нахождение $\#E(\mathbb{F}_q)$}
    \structure{Вход:} $E/\mathbb{F}_q$.
    
    \structure{Выход:} $\#E(\mathbb{F}_q)$.
    \begin{enumerate}
        \item $L = 1$.
        \item Выбрать случайную точку $P \in E(\mathbb{F}_q)$.
        \item $r = ord(P)$.
        \item $L = \operatorname{lcm}(L, r)$.
        \item Если $L$ делит только одно целое $N$ т.ч. $q + 1 - 2\sqrt{q} \leq N \leq q + 1 + 2\sqrt{q}$, то вернуть $N$. В противном случае -- перейти к шагу $2$.
    \end{enumerate}
\end{frame}


%\begin{frame}{}
%TODO: Полиномиальные алгоритмы подсчёта точек
%TODO: оператор Картье
%    content...
%\end{frame}

\begin{frame}{Литература}
    %\nocite{Menezes1993}
    %\nocite{Blake1999}
    \nocite{CohenFrey+2005}
    \nocite{Washington2008}
    \printbibliography
    
    \begin{center}
        \begin{tcolorbox}[enhanced,hbox,colback=block-green-color-bg,colframe=subsection-color!120,title=Контакты,center title]
            \begin{varwidth}{\textwidth}
                \begin{center}
                    \href{mailto:snovoselov@kantiana.ru}{snovoselov@kantiana.ru}
                \end{center}
            \end{varwidth}
        \end{tcolorbox}	
    \end{center}

\structure{Страница курса:}\\
{\footnotesize
    \href{https://crypto-kantiana.com/semyon.novoselov/teaching/elliptic_curves_2021}{crypto-kantiana.com/semyon.novoselov/teaching/elliptic\_curves\_2022}
}
\end{frame}

\end{document}