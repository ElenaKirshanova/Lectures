\documentclass[11pt]{exam}
%%%%%%%%%%%%%%%%%%%%%%%%%%%%%%%%
%\noprintanswers % pour enlever les réponses
%\printanswers

\unframedsolutions
\SolutionEmphasis{\itshape\small}
\renewcommand{\solutiontitle}{\noindent\textbf{A: }}
%%%%%%%%%%%%%%%%%%%%%%%%%%%%%%%%

\usepackage[T2A]{fontenc}
\usepackage[utf8]{inputenc}
\usepackage[english, russian]{babel}

\usepackage{graphicx}
\usepackage{url}
\usepackage{latexsym}
\usepackage{amscd,amsmath,amsthm}
\usepackage{mathtools}
\usepackage{amsfonts}
\usepackage{amssymb}
\usepackage[dvipsnames]{xcolor}
\usepackage{hyperref}


\usepackage[margin=0.73in]{geometry}
%\usepackage[top=1in, bottom=1in, left=1in, right=1in]{geometry}

%\usepackage{fullpage}


\usepackage{hyperref}
\usepackage{appendix}
\usepackage{enumerate}



\usepackage{algorithmicx, enumitem, algpseudocode, algorithm, caption}


%%%%%%%%%%%%%%%%%%%%%%%%%%%
%% THEOREMS
%%%%%%%%%%%%%%%%%%%%%%%%%%%


\newtheorem{theorem}{Theorem}[section]
\newtheorem{axiom}[theorem]{Axiom}
\newtheorem{conclusion}[theorem]{Conclusion}
\newtheorem{condition}[theorem]{Condition}
\newtheorem{conjecture}[theorem]{Conjecture}
\newtheorem{corollary}[theorem]{Corollary}
\newtheorem{criterion}[theorem]{Criterion}
\newtheorem{definition}[theorem]{Definition}
\newtheorem{lemma}[theorem]{Lemma}
\newtheorem{notation}[theorem]{Notation}
\newtheorem{proposition}[theorem]{Proposition}


\theoremstyle{definition}
\newtheorem{problem}{Problem}


\newcommand{\nc}{\newcommand}
\nc{\eps}{\varepsilon}
\nc{\RR}{{{\mathbb R}}}
\nc{\CC}{{{\mathbb C}}}
\nc{\FF}{{{\mathbb F}}}
\nc{\NN}{{{\mathbb N}}}
\nc{\ZZ}{{{\mathbb Z}}}
\nc{\PP}{{{\mathbb P}}}
\nc{\QQ}{{{\mathbb Q}}}
\nc{\UU}{{{\mathbb U}}}
\nc{\OO}{{{\mathbb O}}}
\nc{\EE}{{{\mathbb E}}}

\newcommand{\val}{\operatorname{val}}

\newcommand{\wt}{\ensuremath{\mathit{wt}}}
\newcommand{\Id}{\ensuremath{I}}
\newcommand{\transpose}{\mkern0.7mu^{\mathsf{ t}}}
\newcommand*{\ScProd}[2]{\ensuremath{\langle#1\mathbin{,}#2\rangle}} %Scalar Product

\pretolerance=1000

\newcommand*\abs[1]{\left\lvert#1\right\rvert}

%%%%%%%%%%%%%%%%%%%%%%%%%%%%%%%%
%%%%%%%%%%%%%%%%%%%%%%%%%%%%%%%%
%% DOCUMENT STARTS
%%%%%%%%%%%%%%%%%%%%%%%%%%%%%%%%
%%%%%%%%%%%%%%%%%%%%%%%%%%%%%%%%

\begin{document}
	{\noindent
		\textsc{БФУ им. И. Канта -- Теория кодирования и сжатия информации}
		\hfill {Е. Киршанова // 2023\\}
	\hrule
	\begin{center}
		{\Large\textbf{
				\textsc{Индивидуальная  работа № 3 } \\[5pt] {20.11.23}
		} } 
	\end{center}
	\hrule \vspace{5mm}
	
	\thispagestyle{empty}
	
	\vspace{0.2cm}
	
\section{Списочный алгоритм декодирования для кода Рида-Соломона}
		Рассмотрим код Рида-Соломона размерности $k=2$, заданный над $\FF_7 = <3>$, где в качестве множества $S$ взято $\FF^*$, т.е.\ $S =\{ 1,  3^1 = 3, 3^2 = 2, 3^3 = 6, 3^4 = 4, 3^5 = 5 \}$.
		
		\begin{questions}
 		\question Каковы длина и минимальное расстояние кода? Сколько ошибок может декодировать этот код?
 		\question Покажите на \textbf{своем} примере, что количество допустимых ошибок может увеличено на 1, применяя алгоритм списочного декодирования к своему вектору $y$. 
 		\end{questions}
 		\centering
 		\def\arraystretch{1.5}
 		\setlength{\tabcolsep}{15pt}
 		\begin{tabular}{l  | c }
 			Бакиновский & $y = (1, 0, 0, 1, 4, 1)$\\ \hline
 			Воробьев  & $y = (0, 5, 0, 6, 4, 3)$\\ \hline
 			Уткин  &  $y = (0, 3, 3, 5, 3, 0) $\\  \hline
 			Орлов  &$y = (3, 4, 4, 1, 3, 2) $ \\  \hline
 			Флягин &  $y = (6, 1, 1, 2, 1, 1)$\\  \hline
 			Нецветайлов &  $y = (1, 6, 3, 0, 5, 5)$\\  \hline
 			Гервятович  & $y = (4, 3, 4, 0, 0, 4) $\\  \hline
 			Коршунов  & $y = (3, 1, 3, 5, 4, 3)$\\  \hline
 			Кулигин  &$y = (0, 2, 6, 1, 4, 5) $ \\  \hline
 			Борзенко  & $y = (0, 5, 4, 6, 0, 2)$\\  \hline
 			Затирахин & $y = (3, 4, 2, 2, 4, 2)$ \\  \hline
 			Винников &  $y = (1, 1, 5, 3, 3, 2)$\\  \hline
 			Попков & $y = (3, 6, 2, 1, 4, 5)$ \\  \hline
 			Куртев & $y = (5, 0, 0, 5, 6, 5)$\\  \hline
 		\end{tabular}

	
\end{document}
