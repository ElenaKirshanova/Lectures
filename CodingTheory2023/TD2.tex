\documentclass[11pt]{exam}
%%%%%%%%%%%%%%%%%%%%%%%%%%%%%%%%
%\noprintanswers % pour enlever les réponses
%\printanswers

\unframedsolutions
\SolutionEmphasis{\itshape\small}
\renewcommand{\solutiontitle}{\noindent\textbf{A: }}
%%%%%%%%%%%%%%%%%%%%%%%%%%%%%%%%

\usepackage[T2A]{fontenc}
\usepackage[utf8]{inputenc}
\usepackage[english, russian]{babel}


\usepackage[margin=0.73in]{geometry}
%\usepackage[top=1in, bottom=1in, left=1in, right=1in]{geometry}

%\usepackage{fullpage}


\usepackage{hyperref}
\usepackage{appendix}
\usepackage{enumerate}


\usepackage{times,graphicx,epsfig,amsmath,latexsym,amssymb,verbatim}%,revsymb}
\usepackage{algorithmicx, enumitem, algpseudocode, algorithm, caption}


%%%%%%%%%%%%%%%%%%%%%
% Handling comments and versions %%%
%%%%%%%%%%%%%%%%%%%%%
\newcommand{\extra}[1]{}

\renewcommand{\comment}[1]{\texttt{[#1]}}


%%%%%%%%%%%%%%%%%%%%%%%%%%%
%% THEOREMS
%%%%%%%%%%%%%%%%%%%%%%%%%%%

\usepackage{amsmath,amssymb,amsfonts}
\usepackage{amsthm}

% Landau 
\newcommand{\bigO}{\mathcal{O}}
\newcommand*{\OLandau}{\bigO}
\newcommand*{\WLandau}{\Omega}
\newcommand*{\xOLandau}{\widetilde{\OLandau}}
\newcommand*{\xWLandau}{\widetilde{\WLandau}}
\newcommand*{\TLandau}{\Theta}
\newcommand*{\xTLandau}{\widetilde{\TLandau}}
\newcommand{\smallo}{o} %technically, an omicron
\newcommand{\softO}{\widetilde{\bigO}}
\newcommand{\wLandau}{\omega}
\newcommand{\negl}{\mathrm{negl}} 


\newtheorem{theorem}{Теорема}
\newtheorem{corollary}[theorem]{Следствие}
\newtheorem{lemma}[theorem]{Лемма}
\newtheorem{observation}[theorem]{Observation}
\newtheorem{proposition}[theorem]{Предложение}

\theoremstyle{definition}
\newtheorem{definition}[theorem]{Определение}


\newcommand{\nc}{\newcommand}
\nc{\eps}{\varepsilon}
\nc{\RR}{{{\mathbb R}}}
\nc{\CC}{{{\mathbb C}}}
\nc{\FF}{{{\mathbb F}}}
\nc{\NN}{{{\mathbb N}}}
\nc{\ZZ}{{{\mathbb Z}}}
\nc{\PP}{{{\mathbb P}}}
\nc{\QQ}{{{\mathbb Q}}}
\nc{\UU}{{{\mathbb U}}}
\nc{\OO}{{{\mathbb O}}}
\nc{\EE}{{{\mathbb E}}}

\newcommand{\val}{\operatorname{val}}

\newcommand{\wt}{\ensuremath{\mathit{wt}}}
\newcommand{\Id}{\ensuremath{I}}
\newcommand{\transpose}{\mkern0.7mu^{\mathsf{ t}}}
\newcommand*{\ScProd}[2]{\ensuremath{\langle#1\mathbin{,}#2\rangle}} %Scalar Product

\pretolerance=1000

%%%%%%%%%%%%%%%%%%%%%%%%%%%%%%%%
%%%%%%%%%%%%%%%%%%%%%%%%%%%%%%%%
%% DOCUMENT STARTS
%%%%%%%%%%%%%%%%%%%%%%%%%%%%%%%%
%%%%%%%%%%%%%%%%%%%%%%%%%%%%%%%%
\usepackage{tikz}
\usetikzlibrary{automata}
\DeclareMathOperator{\Vol}{Vol}

\begin{document}
	{\noindent
		\textsc{БФУ им. И. Канта -- Теория кодирования и сжатия информации}
		\hfill {Е. Киршанова // 2023\\}
	\hrule
	\begin{center}
		{\Large\textbf{
				\textsc{Практика № 2} \\[5pt] {11.09.23}
		} } 
	\end{center}
	\hrule \vspace{5mm}
	
	\thispagestyle{empty}
	
	\vspace{0.2cm}
	
\section{Код Адамара}
Бинарный код Адамара, $\mathsf{Had_r}$ -- это  $[2^r, r]_2$- код с порождающей матрицей $r \times 2^r$, столбцы которой представляют всевозможные битовые строки длины $r$. Докажите, что минимальное расстояние кода $\mathsf{Had_r}$  равно $2^{r-1}$.
	
\section{Классы смежности}
Докажите пункты 1--5 теоремы:
\begin{theorem}
	Для $[n,k]_q-$ линейного кода $C$ справедливо
	\begin{enumerate}
		\item $\forall u \in \FF_q^n$ существует класс смежности $C$, содержащий $u$
		\item $\forall u \in \FF_q^n$ имеем $|C+u| = |C| = q^k$
		\item $\forall u, v\in \FF_q^n$ из $u \in C+v$ следует $C+u  = C + v$
		\item $\forall u, v\in \FF_q^n$ либо $C+u = C+v$, либо $(C+u) \cap (C+v) = \emptyset$
		\item  Существует $q^{n-k}$ различных классов смежности $C$.
	\end{enumerate}
\end{theorem} 

\section{Линейные коды}
Пусть $C_1, C_2$ -- линейные коды длины $n$, заданные над $\FF_q$ порождающими матрицами $G_1, G_2$. Определим следующие коды
\begin{itemize}
	\item $C_3 = C_1 \cup C_2$
	\item $C_4 = C_1 \cap C_2$
	\item $C_5 = C_1 + C_2 = \{  c_1 + c_2 \, : \, c_1 \in C_1, c_2 \in C_2 \}$
	\item $C_6 = \{  (c_1 \, |\, c_2 ) \, : \, c_1 \in C_1, c_2 \in C_2 \}$, где $(\cdot|\cdot)$ обозначает конкатинацию слов. 
\end{itemize}
Для $i = 1, \ldots 6$ обозначим за $k_i$ -- размерность кода $\log_q |C_i|$, а за $d_i$-- минимальное расстояние кода $C_i$. Положим $k_1, k_2 >0$.
\begin{questions}
	\question Докажите, что $C_3$ -- линейный тогда и только тогда, когда либо $C_1 \subseteq C_2$, либо $C_2 \subseteq C_1$.
	\question Докажите, что коды $C_4, C_5, C_6$-- линейные
	\question Докажите, что если $k_4>0$, то $d_4 \geq \max\{d_1, d_2\}$
	\question Докажите, что $k_5 \leq k_1 + k_2$, и что равенство достигается тогда и только тогда, когда $k_4 = 0$
	\question Докажите, что $d_5 \leq \min\{ d_1, d_2\}$
	\question Докажите, что 
	\[
	\begin{pmatrix}
		G_1 & 0 \\
		0 & G_2 \\
	\end{pmatrix}
	\]
	является порождающей матрицей для $C_6$, а следовательно, $k_6 = k_1 + k_2$
	\question Докажите, что $d_6 = \min\{d_1, d_2\}$.
\end{questions}

%\section{Декодирование по таблице классов смежности}
%	Для линейного кода $C$, заданного над $\FF_3$ с порождающей матрицей
%	\[
%		G = 
%		\begin{pmatrix}
%		2& 1&2&1 \\
%		1&1&1&0
%		\end{pmatrix}
%	\]
%	\begin{enumerate}
%		\item привести $G$ к систематической форме
%		\item определить мин.\ расстояние кода
%		\item построить таблицу классов смежности и декодировать $y = (1\, 1\, 1\, 1\,)$
%		\item построить таблицу синдромов
%	\end{enumerate}
%\section{Декодирование бинарного кода Хэмминга}
%Бинарный код Хэмминга $[7,4,3]_2$ задаётся проверочной матрицей
%\[
%	H = \begin{pmatrix}
%	0& 0&0&1&1&1&1 \\
%	0& 1&1&0&0&1&1 \\
%	1&0&1&0&1&0&1
%	\end{pmatrix}.
%\]
%\begin{enumerate}
%	\item Сколько ошибок может исправить этот код?
%	\item Построить таблицу синдромов для бинарного кода Хэмминга $[7,4,3]_2$. По полученному слову $y = (1\, 1\, 1\, 0\, 1\,1\,1)$, определить соответствующее кодовое слово и вектор ошибок
%	\item Для обобщенного кода Хэмминга $[n = 2^r - 1, k=2^r-r-1, 3]_2$, предложить алгоритм декодирования без построения таблицы синдромов, работающий за время $\bigO(n \lg n)$.
%\end{enumerate}
%	\section{Минимальное расстояние кода}
%		Покажите, что минимальное расстояние любого линейного кода $C$ равно минимальному весу Хэмминга ненулевого слова в $C$, т.е.,
%		\[
%			\Delta(C) = \min_{c \in C, c \neq 0} \wt(c).
%		\]
%		
%	\section{Систематическая форма}
%		Пусть $G = [\Id_k | A] \in \FF_q^{k \times n}  $ -  порождающая матрица $[n,k]_q-$кода $C$ в систематической форме, где $\Id_k-$ единичная матрица $k \times k$, $A \in \FF_q^{k \times n-k}$. Опишите проверочную матрица для $C$.
%		
%	\section{Дуальный код}
%	На лекции дуальный код для кода $C$ был определен как
%	\[
%	C^\perp = \{ x \in \FF_q \; : \; \ScProd{x}{c} = 0  \,\forall c \in C \}.
%	\]
%	\begin{questions}
%		
%		\question Пусть $G$ - порождающая матрица $C$. Докажите эквивалентность второго определения:
%		\[
%		 C^\perp = \{ x \in \FF_q \; : \; xG\transpose = 0 \}.
%		\]
%		Вывод: $G$ -- проверочная матрица $C^\perp$, $C^\perp$-- линейный $[n, n-k]_q$--код для $C$-- линейного $[n, k]_q$ -- кода.
%		\question Эквивалентное утверждение: любая образующая матрица $H$ дуального кода $C^\perp$ является проверочной матрицей кода $C$. Вывод: $(C^\perp)^\perp = C$.
%	\question Постройте код, дуальный к $[n, 1,  n]_2$ коду с повторением.
%	\end{questions}
%
%	\section{Количество порождающих матриц}
%	Покажите, что для $[n,k]_q$-- линейного кода $C$ ($q$ -- простое), количество различных порождающих матриц равно
%	\[
%		\prod_{i=0}^{k-1}(q^k - q^i).
%	\]
	 
\end{document}