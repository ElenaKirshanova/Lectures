\documentclass[11pt]{exam}
%%%%%%%%%%%%%%%%%%%%%%%%%%%%%%%%
%\noprintanswers % pour enlever les réponses
%\printanswers

\unframedsolutions
\SolutionEmphasis{\itshape\small}
\renewcommand{\solutiontitle}{\noindent\textbf{A: }}
%%%%%%%%%%%%%%%%%%%%%%%%%%%%%%%%

\usepackage[T2A]{fontenc}
\usepackage[utf8]{inputenc}
\usepackage[english, russian]{babel}


\usepackage[margin=0.73in]{geometry}
%\usepackage[top=1in, bottom=1in, left=1in, right=1in]{geometry}

%\usepackage{fullpage}


\usepackage{hyperref}
\usepackage{appendix}
\usepackage{enumerate}


\usepackage{times,graphicx,epsfig,amsmath,latexsym,amssymb,verbatim}%,revsymb}
\usepackage{algorithmicx, enumitem, algpseudocode, algorithm, caption}


%%%%%%%%%%%%%%%%%%%%%
% Handling comments and versions %%%
%%%%%%%%%%%%%%%%%%%%%
\newcommand{\extra}[1]{}

\renewcommand{\comment}[1]{\texttt{[#1]}}


%%%%%%%%%%%%%%%%%%%%%%%%%%%
%% THEOREMS
%%%%%%%%%%%%%%%%%%%%%%%%%%%

\usepackage{amsmath,amssymb,amsfonts}
\usepackage{amsthm}

% Landau 
\newcommand{\bigO}{\mathcal{O}}
\newcommand*{\OLandau}{\bigO}
\newcommand*{\WLandau}{\Omega}
\newcommand*{\xOLandau}{\widetilde{\OLandau}}
\newcommand*{\xWLandau}{\widetilde{\WLandau}}
\newcommand*{\TLandau}{\Theta}
\newcommand*{\xTLandau}{\widetilde{\TLandau}}
\newcommand{\smallo}{o} %technically, an omicron
\newcommand{\softO}{\widetilde{\bigO}}
\newcommand{\wLandau}{\omega}
\newcommand{\negl}{\mathrm{negl}} 


\newtheorem{theorem}{Теорема}
\newtheorem{corollary}[theorem]{Следствие}
\newtheorem{lemma}[theorem]{Лемма}
\newtheorem{observation}[theorem]{Observation}
\newtheorem{proposition}[theorem]{Предложение}

\theoremstyle{definition}
\newtheorem{definition}[theorem]{Определение}


\newcommand{\nc}{\newcommand}
\nc{\eps}{\varepsilon}
\nc{\RR}{{{\mathbb R}}}
\nc{\CC}{{{\mathbb C}}}
\nc{\FF}{{{\mathbb F}}}
\nc{\NN}{{{\mathbb N}}}
\nc{\ZZ}{{{\mathbb Z}}}
\nc{\PP}{{{\mathbb P}}}
\nc{\QQ}{{{\mathbb Q}}}
\nc{\UU}{{{\mathbb U}}}
\nc{\OO}{{{\mathbb O}}}
\nc{\EE}{{{\mathbb E}}}

\newcommand{\val}{\operatorname{val}}

\newcommand{\wt}{\ensuremath{\mathit{wt}}}
\newcommand{\Id}{\ensuremath{I}}
\newcommand{\transpose}{\mkern0.7mu^{\mathsf{ t}}}
\newcommand*{\ScProd}[2]{\ensuremath{\langle#1\mathbin{,}#2\rangle}} %Scalar Product
%\newcommand*{\eps}{\ensuremath{\varepsilon}}
\newcommand*{\Sphere}[1]{\ensuremath{\mathsf{S}^{#1}}}
\newcommand{\vol}{\operatorname{Vol}}

\pretolerance=1000

%%%%%%%%%%%%%%%%%%%%%%%%%%%%%%%%
%%%%%%%%%%%%%%%%%%%%%%%%%%%%%%%%
%% DOCUMENT STARTS
%%%%%%%%%%%%%%%%%%%%%%%%%%%%%%%%
%%%%%%%%%%%%%%%%%%%%%%%%%%%%%%%%
\usepackage{tikz}
\usetikzlibrary{automata}
\DeclareMathOperator{\Vol}{Vol}

\begin{document}
	{\noindent
		\textsc{БФУ им. И. Канта -- Теория кодирования и сжатия информации}
		\hfill {Е. Киршанова // 2023\\}
	\hrule
	\begin{center}
		{\Large\textbf{
				\textsc{Практика № 4} \\[5pt] {02.10.23}
		} } 
	\end{center}
	\hrule \vspace{5mm}
	
	\thispagestyle{empty}
	
	\vspace{0.2cm}
	

\section{Лемма из лекции}
	Пусть $v_1, \ldots, v_m \in \Sphere{n-1} \subset \RR^n$. Докажите, что если $\ScProd{v_i}{v_j} \leq -\eps$, $\forall i \neq j, \eps >0$, то $m \leq 1 + \frac{1}{\eps}.$
	
	
%\section{Уточнение границы ГВ для линейных кодов}
%
%Покажите, что для $n, d \in \mathbb{N}, 2 \leq d \leq n$ и $q $ -- степени простого, справедливо
%\[
%	A_q(n,d) \geq \frac{q^{n-1}}{\vol_q^{n-1}(d-2)}.
%\]	
%	
\section{Обобщенный код Хэмминга}
	Покажите, что обобщенный код Хэмминга с параметрами $[2^r-1, 2^r - 1 -r, 3]_2$ удовлетворяет границе Гильберта-Варшамова точно (граница ГВ: $\Vol_q^{n-1}(d-2) < q^{n-k}$)
	%\paragraph{Ответ.} Имеем $\Vol_2^{n-1}(1) = 1 + (n-1) = n$ с одной стороны, и $2^{n-k} = 2^{2^r - 1 - 2^r + r +1}= 2^r = n+1$, так как $2^r- 1= n$.

	
%\section{Код Уолша-Адамара}
%	Код  Уолша-Адамара (Walsh-Hadamard code)  -- бинарный код размерности $k$, в котором функция кодирования $x \in \FF_2^k$ задаётся
%	\[
%		C(x) = (\ScProd{x}{z_1}, \ldots \ScProd{x}{z_{2^k-1}}),
%	\]
%	где $z_1, \ldots, z_{2^k-1}$ -- все ненулевые вектора в $\FF_2^k$.
%	Покажите, что код Уолша-Адамара достигает границу Плоткина.
%	
%	\paragraph{Ответ:} для того, чтобы код Уолша-Адамара достигал границу Плоткина, необходимо для каждого $c_i \in C$ добавить $\bar{c_i}$. См. \url{https://cse.buffalo.edu/faculty/atri/courses/coding-theory/lectures/lect16.pdf}. 
\end{document}