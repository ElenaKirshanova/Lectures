% !TeX encoding = UTF-8
\documentclass[11pt]{exam}

%---enable russian----

\usepackage[utf8]{inputenc}
\usepackage[russian]{babel}



\usepackage[margin=0.73in]{geometry}
%\usepackage[top=1in, bottom=1in, left=1in, right=1in]{geometry}

\usepackage{graphicx}
\usepackage{url}
\usepackage{latexsym}
\usepackage{amscd,amsmath,amsthm}
\usepackage{mathtools}
\usepackage{amsfonts}
\usepackage{amssymb}
\usepackage[dvipsnames]{xcolor}
\usepackage{hyperref}

\usepackage{algorithmicx, enumitem, algpseudocode, algorithm, caption}
\usepackage{tikz}
\usetikzlibrary{automata}

\usepackage{textcomp}

\usepackage{kbordermatrix} % to label matrix rows and columns

%%%%%%%%%%%%%%%%%%%%%%%%%%%
%% THEOREMS
%%%%%%%%%%%%%%%%%%%%%%%%%%%

\usepackage{amsmath,amssymb,amsfonts}
\usepackage{amsthm}

\newtheorem{theorem}{Теорерма}
\newtheorem{corollary}[theorem]{Следствие}
\newtheorem{lemma}[theorem]{Лемма}
\newtheorem{observation}[theorem]{Observation}
\newtheorem{proposition}[theorem]{Предложение}
\newtheorem{definition}[theorem]{Определение}
\newtheorem{claim}[theorem]{Утверждение}
\newtheorem{fact}[theorem]{Факт}
\newtheorem{assumption}[theorem]{Предположение}

\theoremstyle{definition}
\newtheorem{problem}{Problem}


\newcommand{\nc}{\newcommand}
\nc{\eps}{\varepsilon}
\nc{\RR}{{{\mathbb R}}}
\nc{\CC}{{{\mathbb C}}}
\nc{\FF}{{{\mathbb F}}}
\nc{\NN}{{{\mathbb N}}}
\nc{\ZZ}{{{\mathbb Z}}}
\nc{\PP}{{{\mathbb P}}}
\nc{\QQ}{{{\mathbb Q}}}
\nc{\UU}{{{\mathbb U}}}
\nc{\OO}{{{\mathbb O}}}
\nc{\EE}{{{\mathbb E}}}

\newcommand{\bigO}{\mathcal{O}}

\newcommand{\val}{\operatorname{val}}

\newcommand{\wt}{\ensuremath{\mathit{wt}}}
\newcommand{\Id}{\ensuremath{I}}
\newcommand{\transpose}{\mkern0.7mu^{\mathsf{ t}}}
\newcommand*{\ScProd}[2]{\ensuremath{\langle#1\mathbin{,}#2\rangle}} %Scalar Product
\newcommand*\abs[1]{\left\lvert#1\right\rvert}
\newcommand*\norm[1]{\left\lVert#1\right\rVert}

\pretolerance=1000

%%%%%%%%%%%%%%%%%%%%%%%%%%%%%%%%
%%%%%%%%%%%%%%%%%%%%%%%%%%%%%%%%
%% DOCUMENT STARTS
%%%%%%%%%%%%%%%%%%%%%%%%%%%%%%%%
%%%%%%%%%%%%%%%%%%%%%%%%%%%%%%%%
\usepackage{tikz}
\usetikzlibrary{automata}
\DeclareMathOperator{\Vol}{Vol}

\begin{document}
	{\noindent
		\textsc{БФУ им. И. Канта -- Криптография на решётках}
		\hfill {Е. Киршанова // 2023}
\hrule
\begin{center}
	{\LARGE
			Лабораторная работа № 2 \\[5pt]
			\textbf{
				Задача спрятанного числа 
			} \\[10pt]
	 	{Дедлайн: 13.03.2023} 
 	} 
\end{center}
\hrule \vspace{5mm}
	
	\thispagestyle{empty}
	
	\vspace{0.2cm}
	\section{Формулировка задачи}
	Задача ``спрятанного числа'' (англ.\ the Hidden Number Problem, HNP) была поставлена в статье~\cite{BV} в связи со сторонними атаками на задачу Диффи-Хэллмана (ДХ). На сложности задачи ДХ основан популярный алгоритм генерации общего ключа: Алиса и Боб, обладая секретными $a$, $b$ соответственно, обмениваются значениями $g^a$, $g^b$, формируя тем самым общий ключ $g^{ab}$. Полагаем, что $g$–образующий мультипликативной группы $\FF_p^\star$ для большого простого $p$. Задача ДХ говорит о том, что любой эффективный атакующий, зная $g,g^a,g^b$, не может вычислить $g^{ab}$. Что, если атакующий имеет доступ к некоторым битам $g^{ab}$? С учетом корявых реализаций этого важного протокола, вопрос не праздный~\cite{MB+}. Этот вопрос мотивирует задачу HNP.
	\begin{definition}
	 Задача спрятанного числа. Зафиксируем простое $p$ и целое положительное $\delta$. Обозначим за $\bigO_\alpha(t)$–оракул, который выдает $\delta$ значимых бит числа $\alpha t~\bmod~p$:
	 \[
	 	\bigO_\alpha(t) = \mathrm{MSB}_{\delta}(\alpha \cdot t~\bmod~p).
	 \]
	Задача состоит в вычислении $\alpha$, имея доступ к $\bigO_\alpha(t)$.
	\end{definition}

	Суть этой лабораторной состоит в реализации алгоритма, решающую задачу HNP. Отметим, что чем меньше $\delta$, тем сильнее считается атака ($\delta = \lceil \log(p)\rceil$ делает задачу тривиальной).
	
	Уточним определение $\mathrm{MSB}$. Для атаки удобно думать о функции $\mathrm{MSB}_{\delta}(x)$ как о функции, возвращающей любое целое $z$, удовлетворяющее $|x-z| < p/2^\delta$. Такое определение будет удобно для анализа алгоритма, который мы рассмотрим ниже.
	
	\section{Решение задачи HNP}
	
	Основной результат Боне-Венкатесана~\cite{BV} формулируется так: для $\delta = \lceil \sqrt{n} \rceil + \lceil \log n\rceil$ и $d = 2\sqrt{n}$ вызовов оракула $\bigO_\alpha(\cdot)$ задача HNP решается за детерминированное полиномиальное (от $\log p$) время.
	
	Положим, мы выбрали $d$ случайных значений $t_1, \ldots, t_d$ из $\FF_p$ и значений $a_1, \ldots, a_d$, каждый из которых удовлетворяет
	\[
		\abs{\alpha t_i~\bmod~p - a_i} < p/2^{\delta}.
	\]
	
	Построим решетку ранга $d+1$, порожденную \emph{столбцами}
	\[
	B = \begin{pmatrix}
	p & 0 & 0 & \ldots & 0 & t_1 \\
	0 & p & 0 & \ldots & 0 & t_2 \\
	0 & 0 & p & \ldots & 0 & t_3 \\
	\vdots & \vdots & \vdots & \ddots & \vdots & \vdots \\
	0 & 0 & 0 & \ldots & p & t_d \\
	0 & 0 & 0 & \ldots & 0 & 1/p \\
	\end{pmatrix}.
	\]
	
	Заметьте, во-первых, что матрица не целочисленная. Эту проблему мы далее решим. Во-вторых, умножив последний столбец матрицы $B$ на $\alpha$ (который нам неизвестен) и отняв необходимые кратные $p$ с помощью остальных столбцов, мы получим вектор решетки
	\[
	v = (r_1,\ldots,r_d,\alpha/p),
	\]
	где $|r_i - a_i| < p/2^\delta$. Обозначим $u = (a_1,\ldots,a_d,0)$. Этот вектор нам известен и, кроме этого, $\| u - v\| \leq \sqrt{d+1}\cdot p/2^\delta$. 
	
	Таким образом, пара $(L(B),u)$ формирует задачу близкого вектора. Однако, с базисом, описанным выше, трудно оперировать на практике, так как он не является целочисленным. Для решения этой проблемы воспользуемся трюком, предложенным в~\cite{AH}. Напомним, что мы имеем сравнения
	\begin{equation} \label{eq:equiv_relation}
		\alpha t_i \equiv a_i + b_i \bmod~p,
	\end{equation}
	из которых нам известны $t_i, b_i$, неизвестны $\alpha$ и $b_i$, т.ч.\ $|b_i|< p/2^\delta$. Из сравнений~\eqref{eq:equiv_relation}, выполняется $\alpha \equiv t_1^{-1}(a_1 +b_i)\bmod~p \equiv t_i^{-1}(a_i +b_i)\bmod~p$, $\forall i$. Значит,
	\begin{align*}
		t_i \cdot t_1^{-1}(a_1 +b_1) &\equiv (a_i +b_i) \bmod~p \iff  \\
		t_i \cdot t_1^{-1} b_1 & \equiv a_i -  t_i \cdot t_1^{-1}a_1 + b_i \bmod~p 
	\end{align*}
	Полагая $\hat{t}_{i-1}:= t_i\cdot t_1^{-1}$ для $i>1$, а $\hat{a}_{i-1}:= a_i -  t_i \cdot t_1^{-1}a_1$, мы получаем новую инстанцию задачи HNP с неизвестным $b_1$ (вместо $\alpha$). Рассмотрим теперь решетку, порожденную столбцами $\hat{B}$:
	\[
	\hat{B} = \begin{pmatrix}
	p & 0 & 0 & \ldots & 0 & \hat{t}_1 \\
	0 & p & 0 & \ldots & 0 & \hat{t}_1 \\
	0 & 0 & p & \ldots & 0 & \hat{t}_1 \\
	\vdots & \vdots & \vdots & \ddots & \vdots & \vdots \\
	0 & 0 & 0 & \ldots & p & \hat{t}_{d-1} \\
	0 & 0 & 0 & \ldots & 0 & 1 \\
	\end{pmatrix}.
	\]
	Заметьте, что (известным нам) вектор $\hat{u} = (\hat{a}_1, \ldots, \hat{a}_{d-1},  \lceil p/2^\delta \rceil)$, отстает от вектора решетки $L(\hat{B})$ на вектор $(b_2, \ldots, b_d, \star)$, где $|\star| = p/2^\delta - b_1 <p/2^\delta$, а значит $(L(\hat{B}), \hat{u})$ формируют задачу нахождения близкого вектора.
	
		
	 %Боне-Венкатесан~\cite{BV} доказывают, что это не просто задача ближайшего вектора, а уникального ближайшего вектора, которую при указанных параметрах можно решить с помощью алгоритма Бабая (LLL).
	
	\section{Задание}
	
	Реализовать алгоритм, решающий задачу HNP для $p$ порядка 512 бит. Для реализации оракула можете использовать скрипт~\cite{script} а для нахождения ближайшего вектора можно использовать функцию babai (реализована в fpylll). Вам нужно восстановить как значение $b_i$, так и $\alpha$.
	
	
	
	\begin{thebibliography}{9}
		\bibitem{BV} 
		D. Boneh, R. Venkatesan.  
		\textit{Hardness of computing the most significant bits of secret keys in Diffie-Hellman and related schemes.} 
		\url{https://link.springer.com/content/pdf/10.1007\%2F3-540-68697-5_11.pdf}
		
		\bibitem{MB+} 
		Robert Merget, Marcus Brinkmann, Nimrod Aviram, Juraj Somorovsky and Johannes Mittmann, J{\"o}rg Schwenk.
		\textit{Raccoon Attack: Finding and Exploiting Most-Significant-Bit-Oracles in TLS-DH(E)} \url{https://eprint.iacr.org/2020/1151}
		
		\bibitem{AH}
		M. R. Albrecht, Heninger. \textit{On Bounded Distance Decoding with Predicate: Breaking the ``Lattice Barrier'' for the Hidden Number Problem}
		\url{https://eprint.iacr.org/2020/1540.pdf}
		
		\bibitem{script}
		\url{https://crypto-kantiana.com/elena.kirshanova/teaching/lattices_2023/tp2_oracle.sage}
		
		
	\end{thebibliography}	
		
\end{document}