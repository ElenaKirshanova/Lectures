\documentclass[11pt]{exam}
%%%%%%%%%%%%%%%%%%%%%%%%%%%%%%%%
\noprintanswers % pour enlever les réponses
%\printanswers

\unframedsolutions
\SolutionEmphasis{\small\color{blue}}
\renewcommand{\solutiontitle}{\noindent\textbf{A: }}
%%%%%%%%%%%%%%%%%%%%%%%%%%%%%%%%

\usepackage[T2A]{fontenc}
\usepackage[utf8]{inputenc}
\usepackage[english, russian]{babel}

\usepackage[margin=0.73in]{geometry}
%\usepackage[top=1in, bottom=1in, left=1in, right=1in]{geometry}

%\usepackage{fullpage}


\usepackage{hyperref}
\usepackage{appendix}
\usepackage{enumerate}


\usepackage{times,graphicx,epsfig,amsmath,latexsym,amssymb,verbatim}%,revsymb}
\usepackage{algorithmicx, enumitem, algpseudocode, algorithm, caption}


%%%%%%%%%%%%%%%%%%%%%
% Handling comments and versions %%%
%%%%%%%%%%%%%%%%%%%%%
\newcommand{\extra}[1]{}

\renewcommand{\comment}[1]{\texttt{[#1]}}


%%%%%%%%%%%%%%%%%%%%%%%%%%%
%% THEOREMS
%%%%%%%%%%%%%%%%%%%%%%%%%%%

\usepackage{amsmath,amssymb,amsfonts}
\usepackage{amsthm}

\newtheorem{theorem}{Theorem}[section]
\newtheorem{axiom}[theorem]{Axiom}
\newtheorem{conclusion}[theorem]{Conclusion}
\newtheorem{condition}[theorem]{Condition}
\newtheorem{conjecture}[theorem]{Conjecture}
\newtheorem{corollary}[theorem]{Corollary}
\newtheorem{criterion}[theorem]{Criterion}
\newtheorem{definition}[theorem]{Definition}
\newtheorem{lemma}[theorem]{Lemma}
\newtheorem{notation}[theorem]{Notation}
\newtheorem{proposition}[theorem]{Proposition}


\theoremstyle{definition}
\newtheorem{problem}{Problem}


\newcommand{\nc}{\newcommand}
\nc{\eps}{\varepsilon}
\nc{\RR}{{{\mathbb R}}}
\nc{\CC}{{{\mathbb C}}}
\nc{\FF}{{{\mathbb F}}}
\nc{\NN}{{{\mathbb N}}}
\nc{\ZZ}{{{\mathbb Z}}}
\nc{\PP}{{{\mathbb P}}}
\nc{\QQ}{{{\mathbb Q}}}
\nc{\UU}{{{\mathbb U}}}
\nc{\OO}{{{\mathbb O}}}
\nc{\EE}{{{\mathbb E}}}
\nc{\lat}{{{\mathcal L}}}
\nc{\GL}{{{\mathrm{GL}}}}

\renewcommand{\vec}{\mathbf}


\pretolerance=1000

%%%%%%%%%%%%%%%%%%%%%%%%%%%%%%%%
%%%%%%%%%%%%%%%%%%%%%%%%%%%%%%%%
%% DOCUMENT STARTS
%%%%%%%%%%%%%%%%%%%%%%%%%%%%%%%%
%%%%%%%%%%%%%%%%%%%%%%%%%%%%%%%%
\usepackage{tikz}
\usetikzlibrary{automata}
\DeclareMathOperator{\Vol}{Vol}

\begin{document}
	{\noindent
		\textsc{БФУ им. И. Канта -- Криптография на решётках}
		\hfill {Е. Киршанова // 2024\\}
		\hrule
		\begin{center}
			{\Large\textbf{
					\textsc{Практика № 3}  \\[5pt]  К сдаче: 19.02
			} }
		\end{center}
		\hrule \vspace{5mm}
		
		\thispagestyle{empty}
		
		\vspace{0.2cm}
		\section{Свойства LLL редуцированного базиса}
		
		Докажите Лемму из Лекции: пусть $\delta \in (1/2, 1)$ -- параметр LLL редукции, и $\alpha = \frac{1}{\sqrt{\delta - 1/4}}$. Тогда для $B\in \ZZ^{n\times n}$ -- LLL редуцированного базиса справедливо:
		\begin{enumerate}
			\item  $\| \vec b_1 \| \leq \alpha^{n-1} \lambda_1(L(B))$
			\item  $\| \vec b_1 \| \leq \alpha^{\frac{n-1}{2}} \det(L(B))^{1/n}$
			\item $\frac{r_{i,i}}{r_{i+1, i+1}} \leq \alpha \quad \forall i \leq n$.
		\end{enumerate}


		\section{Свойство BKZ редуцированного базиса}
		
		Пусть $B$ -- базис, полученный на вход BKZ алгоритма, а $B_{[i,j]}$ для $i<j$ базис проективной подрешетки, полученный из проекций векторов ($\vec b_i, \ldots \vec b_j$),ортогонально векторам $\vec b_i, \ldots \vec b_j$ (иными словами, вырезанный из $R$-фактора квадрат на позициях с $i$ по $j$). Первый вектор любого такого проективного базиса соответствует $\vec b^{\star}$ -- $i$-ому вектору базиса Грамма-Шмидта (остальные вектора из этого вырезанного проективного базиса, в общем, не обязаны соответствовать векторам базиса Грама-Шмидта).
		\begin{questions}
		
		\question Примените теорему Минковского к $B_{[i, i+\beta-1]}$ и получите верхнюю границу на  $	\| \tilde{\vec{b}}_i \|$. Конкретно, покажите, что
		\begin{align} \label{ineq:q2}
			\| {\vec b}_i^\star \|^\beta \leq \beta^{\beta/2} \prod_{j=i}^{i+\beta - 1} \| {\vec b}^\star_j \|
		\end{align}
		
		\question Используя неравенство выше, покажите, что для $1 \leq i \leq n-\beta +1$'s, справедливо
		\begin{align}\label{ineq:q3}
		\| {\vec{b}}_1^\star \|^{\beta-1} \cdot \| {\vec{b}}_2^\star \|^{\beta-2} \cdot \ldots \cdot \| {\vec{b}}^\star_{\beta-1} \| \leq 
		\beta^{\frac{\beta(n-\beta+1)}{2}} \| {\vec{b}}^\star_{n-\beta+2} \|^{\beta-1} \| {\vec{b}}^\star_{n-\beta+3} \|^{\beta-2} \cdot \ldots \cdot  \|  {\vec{b}}^\star_n\|.
		\end{align}
		
		Для этого примените Неравенство~(\ref{ineq:q2}) к  $\prod_{i=1}^{n-\beta+1}\|{\vec b}^\star_i \|^\beta$.
		
		\question Используя тот факт, что не только базис $B_{[1,\beta]}$  SVP-редуцирован (то есть его первый вектор есть кратчайший ), но также и базисы $B_{[1, i]}$  при $i \leq \beta$ SVP редуцированы (подумайте, почему это верно), сделайте вывод, что:
		\begin{align}\label{ineq:q4}
			\| {\vec b}_1^\star \|^i \leq i^{i/2} \prod_{j=1}^{i} \| {\vec b}_j^\star \| \quad \forall i \leq \beta
		\end{align}
	
		Сравните результат с Неравенством~(\ref{ineq:q2})):
		
		\question Перемножьте  Неравенства~(\ref{ineq:q4}) для $1 \leq i \leq \beta -1$ и используйте Неравенство~(\ref{ineq:q3}), чтобы получить
		\begin{align}\label{ineq:q5}
		\| {\vec b}^\star_1 \|^{\frac{\beta(\beta-1)}{2}} \leq \beta^{\frac{\beta(n-1)}{2}} \cdot \| {\vec{b}}^\star_{n-\beta+2} \|^{\beta-1} \| {\vec{b}}^\star_{n-\beta+3} \|^{\beta-2} \cdot \ldots \cdot  \|  {\vec{b}}^\star_n\|
		\end{align}
		\question 
		Положим, что в нашей решетке существует кратчайший вектора  $\vec{v}_{\texttt{shortest}}$, чья проекция ортогонально первым $n - 1$ базисным векторам ненулевая (так как иначе, если \emph{все} кратчайшие вектора ортогональны ${\vec b}_{n}^\star$, то мы знаем, что все они лежат в подрешетке размерности не больше $n-1$, и в таком случае мы можем убрать из базиса $\vec b_n$).
		
		Из этого следует, что $\lambda_1 = \|\vec{v}_{\texttt{shortest}} \| \geq \| {\vec b}^\star_i \|$ для $n-\beta+2 \leq i \leq n  $ (подумайте, почему). Подставив это неравенство в правую часть Неравенства~(\ref{ineq:q5}), сделайте следующий вывод:
		\[
			\| \vec b_1 \| \leq \beta^{\frac{n-1}{\beta-1}} \lambda_1.
		\]
		
		\end{questions}

\end{document}


