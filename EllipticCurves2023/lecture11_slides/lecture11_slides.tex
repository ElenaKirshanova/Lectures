% !TeX program = xelatex
% !BIB TS-program = biber

\documentclass{beamer}

\usepackage[utf8]{inputenc}
\usepackage[russian]{babel}

%---tikz----
\usepackage{tikz}
\usetikzlibrary{arrows, chains, matrix, positioning, scopes, patterns, shapes}
\usepackage{pgfplots, subfigure}
\usepackage{extarrows}
\usepackage{tikz-cd}
\usepackage{makecell}
\usepackage{quiver}
\usetikzlibrary{babel}

\usepackage[backend=biber,firstinits=true,hyperref=true,style=numeric-comp]{biblatex}

\usepackage{../beamerthemeec2020}
{\footnotesize\bibliography{../biblio}}

\title{Эллиптические кривые}
\subtitle{Лекция 11. Изогении II}
\author{Семён Новосёлов}
\institute{БФУ им. И. Канта}
\date{2023}

\begin{document}

\frame{\titlepage}

\newcommand{\UserA}{{\structure{{\Large\faUserSecret}}}}
\newcommand{\UserB}{{\structure{{\Large\faCat}}}}

\begin{frame}{Схемы стойкие к атаке Кастрика-Декру}

Обмен ключами:
\begin{itemize}
	\item CSIDH, \sout{SIKE}, CRS, OSIDH
\end{itemize}
\vspace{0.5em}

Подписи:
\begin{itemize}
	\item SQISign, weakSIDH
\end{itemize}

\vspace{0.5em}
\begin{itemize}
	\item[\structure{{\faGlobe}}] \href{https://issikebrokenyet.github.io/}{issikebrokenyet.github.io}
\end{itemize}
\end{frame}

\begin{frame}{Соответствие Дойринга}
Эквивалентность изогениями эллиптических кривых и идеалами кольца эндоморфизмов.
\end{frame}

\begin{frame}{CSIDH}
	\structure{Публичные параметры схемы:}
	\begin{itemize}
		\item простое $p=4\cdot \ell_1 \cdots \ell_n - 1$, где $\ell_1, \ldots, \ell_n$ -- малые простые.
		\item Суперсингулярная эллиптическая кривая $E_0: y^2=x^3+x$ над полем $\mathbb{F}_p$.
		\item $\mathfrak{l}_i=(\ell_i, \pi_p-1)$, $\mathfrak{l}_i^{-1}=(\ell_i, \pi_p+1)$ -- идеалы $\mathbb{Z}[\pi_p]$
		\item $m$ -- наименьшее положительное целое, такое, что $2m+1 \geq \sqrt[n]{\# \operatorname{Cl}(\mathbb{Z}[\pi_p])}$.
	\end{itemize}
\end{frame}

\begin{frame}
	\structure{Схема обмена ключами}
	
	\vspace{1em}
	Пользователь \UserA:
	%\begin{itemize}
	%    \item  
	\begin{enumerate}
		\item выбирает секретный вектор $(e_1, \ldots, e_n) \in \lbrace -m,\ldots ,m \rbrace ^n$
		\item определяет класс идеала $[\mathfrak{a}]=[\mathfrak{l}_1^{e_1}\cdots \mathfrak{l}_n^{e_n}]\in \operatorname{Cl}(\mathbb{Z}[\pi_p])$%, где
		\item вычисляет свой открытый ключ $E_A = [\mathfrak{a}]*E_0$
	\end{enumerate}
	
	\vspace{1em}
	Пользователь \UserB:
	
	\begin{enumerate}
		\item выбирает секретный вектор $(f_1,\ldots ,f_n) \in \lbrace -m,\ldots ,m \rbrace ^n$
		\item определяет класс идеала $[\mathfrak{b}]=[\mathfrak{l}_1^{f_1}\cdots \mathfrak{l}_n^{f_n}]\in \operatorname{Cl}(\mathbb{Z}[\pi_p])$
		\item вычисляет свой открытый ключ $E_B = [\mathfrak{b}]*E_0$
	\end{enumerate}
	
	\vspace{1em}
	\structure{Общий ключ:} $E_{AB} = [\mathfrak{a}] * E_B = [\mathfrak{b}] * E_A = [\mathfrak{a}\mathfrak{b}] * E_0$
\end{frame}

\begin{frame}[fragile]{CSIDH}
	\[
	\begin{tikzcd}
		\text{\UserA} &\arrow{ld}[swap]{\phi_{A}} E  \arrow{rd}{\phi_{B}} & \text{\UserB} \\
		\left[\mathfrak{a}\right] * E  = E_A&
		\arrow[yshift=-0.7ex]{r}{E_A}
		\arrow[yshift=0.7ex]{l}{E_B}
		& E_B = \left[\mathfrak{b}\right] * E\\
	\end{tikzcd}
	\]
	\vspace{-0.5em}
	\structure{Общий ключ:} $E_{AB} = [\mathfrak{a}] * E_B = [\mathfrak{b}] * E_A = [\mathfrak{a}\mathfrak{b}] * E_0$
	\vspace{0.5em}
	\begin{itemize}
		\item для формирования ключа требуется коммутативность
		\item из-за этого доступны квантовые субэксп. атаки
	\end{itemize}

\end{frame}

\begin{frame}{Размеры ключей}
	%65 ms, субэкспоненциальная квантовая атака, сложность $L_p(1/2)$ https://www.maths.ox.ac.uk/system/files/attachments/CSIDH.pdf
	
	\begin{table}[h]
		\begin{center}
			\begin{tabular}{|c|c|c|c|c|}
				\hline
				Схема & \thead{Уровень\\ стойкости} &
				\thead{Открытый\\ключ}
				& \thead{Закрытый\\ключ} & \thead{Общий\\ключ} \\
				\hline
				CRS%~\cite{DeFeoKiefferSmith2018}
				& 128/56 & 64 & 8 & 64  \\
				\hline
				OSIDH% \cite{ColoKohel2020}
				& 128/128 & 36 & 31 & 36 \\
				\hline
				CSIDH-512 & 128/62 & 64 & 32 & 64 \\
				\hline
			\end{tabular}
		\end{center}
		\caption{Размеры ключей (в байтах) для актуальных схем обмена ключами на изогениях.}
		\label{Table1}
	\end{table}
	
	\begin{itemize}
		\item CRS/CSIDH: субэкспоненциальные квантовые атаки
		\item OSIDH: экспоненциальные квантовые атаки
	\end{itemize}
\end{frame}

\begin{frame}{Литература}
	%\nocite{Menezes1993}
	%\nocite{Lenstra1987}
	%\nocite{Blake1999}
	%\nocite{CohenFrey+2005}
	%\nocite{Washington2008}
	%\nocite{GoldwasserKilian1999}
	%\nocite{CohenLenstra1984}
	%\nocite{JaoDeFeo2011}
	%\nocite{Galbraith2012}
	%\nocite{DeFeo2018}
	%\nocite{Costello2019}
	%\nocite{SIKE}
	%\nocite{SafeCurves}
    %\nocite{CastryckDecru2022}
	%\printbibliography
	\begin{itemize}
		\item[\structure{{\faScroll}}] Castryck W., Decru T. An efficient key recovery attack
		on SIDH. 2022.
		\vspace{0.5em}
		
		\item[\structure{{\faGlobe}}] SIKE – Supersingular Isogeny Key Encapsulation. 2020. \url{https://sike.org/}
		\vspace{0.5em}
		
		\item[\structure{{\faYoutube}}]
		Выступление Castryck на ANTS XV:
		\url{https://www.youtube.com/watch?v=_eNv7An3Qj0}
	\end{itemize}

    \begin{center}
        \begin{tcolorbox}[enhanced,hbox,colback=block-green-color-bg,colframe=subsection-color!120,title=Контакты,center title]
            \begin{varwidth}{\textwidth}
                \begin{center}
                    \href{mailto:snovoselov@kantiana.ru}{snovoselov@kantiana.ru}
                \end{center}
            \end{varwidth}
        \end{tcolorbox}
    \end{center}\end{frame}

\end{document}
