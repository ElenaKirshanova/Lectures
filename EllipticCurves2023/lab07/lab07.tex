% !TeX spellcheck = ru_RU-Russian
\documentclass[11pt]{exam}

%---enable russian----

\usepackage[utf8]{inputenc}
\usepackage[russian]{babel}

\usepackage[margin=0.73in]{geometry}
%\usepackage[top=1in, bottom=1in, left=1in, right=1in]{geometry}

\usepackage{graphicx}
\usepackage{url}
\usepackage{latexsym}
\usepackage{amscd,amsmath,amsthm}
\usepackage{mathtools}
\usepackage{amsfonts}
\usepackage{amssymb}
\usepackage[dvipsnames]{xcolor}
\usepackage{hyperref}
\usepackage{cmap}

\usepackage{algorithmicx, enumitem, algpseudocode, algorithm, caption}
\usepackage{tikz}
\usetikzlibrary{automata}

\usepackage{textcomp}


%%%%%%%%%%%%%%%%%%%%%
% Handling comments and versions %%%
%%%%%%%%%%%%%%%%%%%%%

%\renewcommand{\comment}[1]{\texttt{[#1]}}


%%%%%%%%%%%%%%%%%%%%%%%%%%%
%% THEOREMS
%%%%%%%%%%%%%%%%%%%%%%%%%%%

\newtheorem{theorem}{Theorem}[section]
\newtheorem{axiom}[theorem]{Axiom}
\newtheorem{conclusion}[theorem]{Conclusion}
\newtheorem{condition}[theorem]{Condition}
\newtheorem{conjecture}[theorem]{Conjecture}
\newtheorem{corollary}[theorem]{Corollary}
\newtheorem{criterion}[theorem]{Criterion}
\newtheorem{definition}[theorem]{Definition}
\newtheorem{lemma}[theorem]{Lemma}
\newtheorem{notation}[theorem]{Notation}
\newtheorem{proposition}[theorem]{Proposition}


\theoremstyle{definition}
\newtheorem{problem}{Problem}


\newcommand{\nc}{\newcommand}
\nc{\eps}{\varepsilon}
\nc{\RR}{{{\mathbb R}}}
\nc{\CC}{{{\mathbb C}}}
\nc{\FF}{{{\mathbb F}}}
\nc{\NN}{{{\mathbb N}}}
\nc{\ZZ}{{{\mathbb Z}}}
\nc{\PP}{{{\mathbb P}}}
\nc{\QQ}{{{\mathbb Q}}}
\nc{\UU}{{{\mathbb U}}}
\nc{\OO}{{{\mathbb O}}}
\nc{\EE}{{{\mathbb E}}}

\newcommand{\val}{\operatorname{val}}
\newcommand{\wt}{\ensuremath{\mathit{wt}}}
\newcommand{\Id}{\ensuremath{I}}
\newcommand{\transpose}{\mkern0.7mu^{\mathsf{ t}}}
\newcommand*{\ScProd}[2]{\ensuremath{\langle#1\mathbin{,}#2\rangle}} %Scalar Product
\renewcommand{\char}{\ensuremath{\mathsf{char}}}

\DeclareMathOperator{\Vol}{Vol}

%\pretolerance=1000

%%%%%%%%%%%%%%%%%%%%%%%%%%%%%%%%
%%%%%%%%%%%%%%%%%%%%%%%%%%%%%%%%
%% DOCUMENT STARTS
%%%%%%%%%%%%%%%%%%%%%%%%%%%%%%%%
%%%%%%%%%%%%%%%%%%%%%%%%%%%%%%%%


\begin{document}	
	{\noindent
		\textsc{БФУ им. И. Канта -- Компьютерный практикум по криптографии на эллиптических кривых }\\[5pt]
		Преподаватель: {С. Новоселов}   \hfill{2023\\}
	\hrule
	\begin{center}
		{\LARGE\textbf{
				Лабораторная работа № 7 \\[5pt]
		}} 
			Опубликована \textbf{09.11.2023} \\[5pt] 
			Дэдлайн \textbf{23.11.2023}
		
	\end{center}
	\hrule \vspace{5mm}
	
	\thispagestyle{empty}
	
	Разработать программу в системе компьютерной алгебры Sage, реализующую функцию:
    
    %\begin{enumerate}
        %\item
        \texttt{Check\_curve($p, a, b, \lambda$)}, где $p$ -- простое, размер конечного поля; $a, b$ -- коэффициенты эллиптической кривой $E/\mathbb{F}_p: y^2 = x^3 + a x + b$ и  $\lambda$ -- уровень безопасности в битах. Возвращает $True$, если кривая проходит минимум следующие проверки:
        \begin{enumerate}
            \item на наличие подгруппы большого порядка (стойкость к атаке Полига-Хеллмана);
            \item на неравенство характеристики порядку подгруппы;
            \item на стойкость задачи дискретного логарифма в данной группе относительно $\rho$-метода Полларда;
            \item на стойкость к атакам на спариваниях (подсчёт степени вложения);
            \item кривая должна соответствовать минимальному уровню безопасности $\lambda$ относительно всех предыдущих пунктов.
        \end{enumerate}
    %\end{enumerate}
    
    Проверить отдельно кривые, рекомендованные КриптоПро\textsuperscript{TM}, \url{https://www.cryptopro.ru/sites/default/files/blog/cpecc12-tc26.pdf}. А именно, два набора параметров: \textsc{ id-tc26-gost-3410-12-512-paramSetA} и \textsc{id-tc26-gost-3410-12-512-paramSetB}.
    
    
\normalsize
\section*{Требования к сдаче}
\begin{itemize}
    \item Исходный код должен содержать комментарии к каждой из функций с описанием входных и выходных параметров
    \item Лабораторную следует выполнять модификацией файла с тестами, заменяя строку
    
    "\texttt{\# your code here.}"
    
    на код, реализующий функцию.
    \item Функции должны работать на всех примерах, что проверяется запуском команды:
    \\\texttt{sage -t file\_with\_tests.sage}
    \item Студент должен понимать, что он написал, зачем, а также ответить на теоретические вопросы.
\end{itemize}
\end{document}