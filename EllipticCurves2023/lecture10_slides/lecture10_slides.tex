% !TeX program = xelatex
% !BIB TS-program = biber

\documentclass{beamer}

\usepackage[utf8]{inputenc}
\usepackage[russian]{babel}

%---tikz----
\usepackage{tikz}
\usetikzlibrary{arrows, chains, matrix, positioning, scopes, patterns, shapes}
\usepackage{pgfplots, subfigure}
\usepackage{extarrows}
\usepackage{tikz-cd}

\usepackage[backend=biber,firstinits=true,hyperref=true,style=numeric-comp]{biblatex}

\usepackage{../beamerthemeec2020}
{\footnotesize\bibliography{../biblio}}

\title{Эллиптические кривые}
\subtitle{Лекция 10. Изогении}
\author{Семён Новосёлов}
\institute{БФУ им. И. Канта}
\date{2023}

\begin{document}

\frame{\titlepage}

\begin{frame}{Мотивация}
	
Постквантовая криптография на изогениях.
\vspace{0.5em}
	
\begin{itemize}
	\item схемы CSIDH, \sout{SIKE}, weakSIDH, SeaSign
	\item в 2022 году появилась полиномиальная атака Кастрика-Декру на схему SIDH/SIKE, перевернувшая данную область
	\item многие схемы стали неактуальными
	\item однако базовые задачи остались трудными
\end{itemize}
\end{frame}

\begin{frame}{План}
\begin{enumerate}
	\item Определение и примеры изогений
	\item SIKE/SIDH
	\item Атака Castryck-Decru (общая схема)
	\item ``Оставшиеся в живых'' схемы (след. лекция)
\end{enumerate}
\end{frame}

\begin{frame}{I. Изогении}
Пусть $E_1, E_2$ -- эллиптические кривые.

\begin{itemize}
    \item В общем случае абелевых многообразий, \structure{изогения} -- гомоморфизм с конечным ядром, сюръективный над замыканием поля.
    \item Для эллиптических кривых определение упрощается: \structure{изогения} -- ненулевой гомоморфизм.
\end{itemize}
\end{frame}

\begin{frame}
    В явном виде:
	\[
	\phi(x, y) = \left( \frac{f_1(x,y)}{f_2(x,y)}, \frac{g_1(x,y)}{g_2(x,y)} \right)
	=
	\left(
	\frac{p(x)}{q(x)}, y \frac{s(x)}{t(x)}
	\right)
	\]
	%Такая форма называется стандартной.
	
	\vspace*{1em}
	\structure{Степень изогении:} $\deg\phi = \max\{\deg{p(x)}, \deg{q(x)}\}$.
	
	\vspace{1em}
	Изогения называется \structure{сепарабельной}, если производная $\frac{u}{v}$ по $x$ не равна $0$, и \structure{несепарабельной} в противном случае.
	
	\vspace{1em}
	Для сепарабельных изогений $\deg\phi = \#ker\phi$.
	
	\vspace{1em}
	Если $E_1 = E_2$, то $\phi$ -- эндоморфизм.
	
	%как приводить в стандартную форму как описано здесь:
	%https://math.mit.edu/classes/18.783/2021/LectureNotes4.pdf
\end{frame}

\begin{frame}{Пример 1: Умножение на $m$}
    \[[m]: E \rightarrow E,\]
    \[P \mapsto m \cdot P.\]
Задаётся многочленами деления.
\[E/\mathbb{Q}: y^2 = x^3 + x\]
\[
[2]P = \left( \frac{(x^2-1)^2}{4 (x^3 + x)}, y \frac{x^6 + 5 x^4 - 5 x - 1}{8 (x^3 + x)^2} \right)
\]
\[
\ker[2] = \{ \mathcal{O}; (x_P,0): x_P^3 + x = 0\}
\]
\[\#ker[2] = 4 = \deg[2],\]
Для сепарабельных изогений степень совпадает с $\#ker$.
\end{frame}

\begin{frame}{Пример 2: Эндоморфизм Фробениуса}
\[\phi: E \rightarrow E,\]
\[(x,y) \mapsto (x^q, y^q),\]
\[
\phi = (x^q, y (x^3 + a x + b)^{\frac{q-1}{2}})
\]
\[
\ker{\phi} = \mathcal{O}_E, \deg{\phi} = q
\]
\begin{center}
(изогения не сепарабельная)
\end{center}
\end{frame}

\begin{frame}{Теорема Тейта о изогениях эллиптических кривых}
	\begin{center}
		Эллиптические кривые $E_1, E_2$ изогенны над $\mathbb{F}_q$ \structure{$\iff$} $\#E_1(\mathbb{F}_q) = \#E_2(\mathbb{F}_q)$
	\end{center}
	%\begin{itemize}
		%\item 
		\structure{Следствие:} проверка кривых на изогенность имеет сложность $O(\log^4{q})$ при использовании SEA.
	%\end{itemize}
\end{frame}

\begin{frame}{Формулы V\'{e}lu}
Пусть $E/\mathbb{F}_q$ -- эллиптическая кривая, $G$ -- подгруппа $E(\overline{\mathbb{F}}_q)$.
\vspace{0.3em}

Тогда:
\vspace{0.5em}
\begin{enumerate}
    \item $\exists E'/\mathbb{F}_q$ и сепарабельная изогения $\phi: E \rightarrow E'$ определённая над $\mathbb{F}_q$ степени $\#G$ т.ч. $\ker\phi = G$.
    \vspace{0.2em}
    \item если $\psi: E \rightarrow E''$ -- другая сепарабельная изогения степени $\#G$ т.ч. $G = \ker\psi$, то $j(E') = j(E'')$.
\end{enumerate}
\vspace{0.5em}
Обозначение: $E/G := E'$ -- фактор-кривая.

\vspace{0.5em}
\structure{Важно!} Не путать с фактор-группой.
\end{frame}

\begin{frame}
V\'{e}lu описал явные формулы для $E'$, $\phi$.

\[E: y^2 = x^3 + a x + b\]
\[
\phi(P)
=
\left(
  x_P + \sum_{Q \in G \setminus \{\mathcal{O}\}} \left( x_{P+Q} - x_Q \right),
  y_P + \sum_{Q \in G \setminus \{ \mathcal{O} \}} \left( y_{P+Q} - y_Q \right)
\right).
\]

А изогенная кривая определяется как:
\[E/G: y^2 = x^3 + a' x + b',\] где
\[
a' = a - 5 \sum_{Q \in G \setminus \{ \mathcal{O} \}} \left( 3 x_Q^2 + a \right),
\]
\[
b' = b - 7 \sum_{Q \in G \setminus \{\mathcal{O}\}} \left(
5 x_Q^3 + 3 a x_Q + b
\right).
\]
\end{frame}

\begin{frame}{Пример 3: Сепарабельная изогения}
	\[E/\mathbb{F}_{7}: y^2 = x^3 + 2 x + 4\]
	\[P  = (3,3), G = \left< P \right>, \#G = 5\]
	
	\begin{footnotesize}
	\[\phi: (x,y) \mapsto \left(
	%
	\frac{x^5 + 4 x^4 + 4 x^3 + 5 x^2 + 2 x + 3}{x^4 + 4 x^3 + 2 x^2 + 3 x + 1},
	%
	y\frac{x^6 - x^5 + 3 x^3 + 3 x^2 + 2 x}{x^6 - x^5 + 2 x^4 + 3 x^3 - 2 x^2 - x - 1} \right)
	\]
	\end{footnotesize}
	
	\[E/G: y^2 = x^3 + 6 x + 4\]
	\begin{center}
	Степень~$\phi$ равна~$5$.
	\end{center}
\end{frame}

\begin{frame}{Ядра изогений}
	\[
	[\ell] P = P + \ldots + P~\text{($\ell$-раз)}
	\]
	
	\structure{Группа-кручения}
	\[
	E[\ell] = \{ P \in E(\overline{\mathbb{F}}) \mid [\ell] P = \mathcal{O} \}
	\]
	
	\begin{itemize}
		\item все ядра изогений степени $\ell$ -- подгруппы $E[\ell]$
		\item перебирая все подгруппы $G  \subseteq E[\ell]$ можно построить с помощью формул Велу все изогении степени $\ell$
	\end{itemize}
	
	\vspace{1em}
	\structure{Важно:} ядра изогений не принадлежат базовому полю в общем случае.
\end{frame}

\begin{frame}{Пример 4: Изогения с ядром над расширением}
	\[E/\mathbb{F}_{7}: y^2 = x^3 + 2 x + 4\]
	\[\mathbb{F}_{7^4} = \mathbb{F}_7 / \left< \alpha^4 + 5 \alpha^2 + 4 \alpha + 3 \right>\]
	\[P  = (5\alpha^3 + \alpha^2 + 5 \alpha + 2, 5 \alpha^3 + 6 \alpha^2 + 4 \alpha + 2)\]
	\[
	G = \left< P \right> \subset E[5],
	\#G = 5
	\]
	
	\begin{footnotesize}
	\[\phi: (x,y) \mapsto \left(\frac{x^5 - x^4 - 3 x^3 - 3 x^2 - x - 2}{x^4 - x^3 + x + 1}, y \frac{x^6 + 2 x^5 - x^4 + x^3 - 2 x^2 + 3 x - 1}{x^6 + 2 x^5 + 3 x^4 + 2 x^3 - 3 x^2 + 2 x - 1}\right)\]
	\end{footnotesize}

	
	\[E/G: y^2 = x^3 + 3 x + 4 \]
	\begin{center}
		Степень~$\phi$ равна~$5$. Изогения определена над~$\mathbb{F}_7$ несмотря на то, что её ядро~$G$ определено над~$\mathbb{F}_{7^4}$.
	\end{center}
\end{frame}

\begin{frame}
    Сложность вычисления $\phi$ и $E/G$: $O(|G|)$.

	\vspace{1em}
	Оптимизации:
	\begin{itemize}
		\item Castryck-Decru-Vercauteren, ''Radical isogenies''
		\item Bernstein-De Feo-Leroux-Smith: $O(\sqrt{|G|})$, \url{velusqrt.isogeny.org}
	\end{itemize}
	\vspace{1em}
	
	$G$ -- подгруппа большого порядка \structure{$\implies$} вычисление $E/G$ является трудной задачей.
	
	\vspace{1em}
	Это делает невозможными вычисления с секретными изогениями ``в лоб'' в криптосистемах.
	
	\vspace{1em} 
	\structure{Выход}: брать $|G| = \ell_1^{e_1} \cdot \ldots \cdot \ell_r^{e_r}$ для малых $\ell_i$ и вычислять изогению как композицию изогений малых степеней.
\end{frame}

\begin{frame}{Проблема нахождения изогении}
	\begin{center}
		\begin{tcolorbox}[enhanced,hbox,colback=title-and-section-color!5,colframe=title-and-section-color!120,title=Общая задача нахождения изогении,center title]
			\begin{varwidth}{\textwidth}
				\begin{center}
					Даны две изогенные кривые~$E_1$ и $E_2$. 
					
					Известно, что степень изогении равна~$\ell$. 
					
					Вычислить изогению между ними.
				\end{center}
			\end{varwidth}
		\end{tcolorbox}	
	\end{center}
\begin{itemize}
	\item При известном ядре~$G$ задача решается за полиномиальное время (если~$\#G$ -- гладкое).
	\item Для суперсингулярных кривых наилучший алгоритм поиска -- на основе парадокса дней рождений.
	\item Сложность $\mathcal{O}(p^6)$ в квантовом случае и классическом -- $\mathcal{O}(p^4)$
	\item Для обычных кривых есть квантовый субэкспоненциальный алгоритм.
\end{itemize}
\end{frame}

\begin{frame}{SIKE}
	\begin{itemize}
		\item В криптосистеме SIKE для оптимизации добавили дополнительную информацию об изогениях -- значения секретной изогении в точках кручения.
		\item Что и привело в итоге к взлому данной системы.
	\end{itemize}
\end{frame}

\begin{frame}[fragile]{``Стандартный'' протокол $DH$ в абстрактной группе}
	$G$ -- группа, $\left<g\right> = G$, $\phi_A(x) = [A] \cdot x$ -- гомоморфизм групп.
	\[
	\begin{tikzcd}
		\text{\structure{{\Large\faUserSecret}}} & & \text{\structure{{\Large\faCat}}} \\
		& \arrow{ld}[swap]{\phi_A} g \arrow{rd}{\phi_B}  \\
		\phi_A(g) & \arrow[yshift=-0.7ex]{r}{\phi_A(g)} \arrow[yshift=0.7ex]{l}{\phi_B(g)}  & \phi_B(g)\\
	\end{tikzcd}
	\]
	\[
	\phi_A(\phi_B(g)) = \phi_B(\phi_A(g)) = [A B]\cdot g
	\]
	\begin{itemize}
		\item изогении суперсингулярных кривых в качестве гомоморфизмов \structure{$\Rightarrow$} протокол SIDH (de Feo \& Jao 2011)
	\end{itemize}
\end{frame}

\begin{frame}[fragile]{2. SIDH (Supersingular Isogeny Diffie-Hellman)}
	\[
	\begin{tikzcd}
		&\arrow{ld}[swap]{\phi_{A}} E  \arrow{rd}{\phi_{B}} & \\
		E/\left<P\right>\arrow{rd}[swap]{\phi'_{A}}  &  & \arrow{ld}{\phi'_{B}}  E/\left< Q \right> \\
		& E/\left< P, Q \right> &\\
	\end{tikzcd}
	\]
	Краткое описание:
	\begin{enumerate}
		\item Публичные параметры: $E$ -- суперсингулярная кривая.
		\item \structure{{\Large\faUserSecret}} выбирает секретное ядро $\left<P\right>$, строит изогению и отсылает \structure{{\Large\faCat}} кривую
		$E/\left<P\right>$
		\item \structure{{\Large\faCat}} выбирает своё секретное ядро $\left<Q\right>$, строит изогению и отсылает \structure{{\Large\faUserSecret}} кривую
		$E/\left<Q\right>$
		\item Общий секретный ключ: $E/\left<P,Q\right> = (E/\left<P\right>)/\phi_A(Q) = (E/\left<Q\right>)/\phi_B(P)$
	\end{enumerate}
\end{frame}

\begin{frame}{Детальное описание}
	\structure{Публичные параметры:}
	\begin{enumerate}
		\item простое
		$p = \ell_A^{e_A} \ell_B^{e_B} \cdot c \pm 1$, где $\ell_A, \ell_B$ -- малые простые
		\item $E$ -- суперсингулярная кривая над $\mathbb{F}_{p^2}$ т.ч. $\#E(\mathbb{F}_{p^2}) = (\ell_A^{e_A} \ell_{B}^{e_B} c)^2$
		\item $\left<P_A, Q_A\right>$ -- базис $E[\ell_A^{e_A}]$, $\left<P_B, Q_B\right>$ -- базис $E[\ell_B^{e_B}]$
	\end{enumerate}
	\vspace*{1em}
	\structure{Секретные параметры:}
	\begin{enumerate}
		\item[\structure{{\Large\faUserSecret}}] $m_A, n_A \in \mathbb{Z}/\ell_A^{e_A} \mathbb{Z}$, изогения $\phi_A$ с ядром $\left< [m_A]P_A + [n_A]Q_A \right>$
		\item[\structure{{\Large\faCat}}] $m_B, n_B \in \mathbb{Z}/\ell_B^{e_B} \mathbb{Z}$, изогения $\phi_B$ с ядром $\left< [m_B]P_B + [n_B]Q_B \right>$
	\end{enumerate}
\end{frame}

\begin{frame}
	\structure{Выработка общего ключа:}
	\begin{enumerate}
		\item
		\structure{{\Large\faUserSecret}} \structure{$\implies$} \structure{{\Large\faCat}:} $(E_A, \phi_A(P_B), \phi_A(Q_B))$
		\item
		\structure{{\Large\faCat}} \structure{$\implies$} \structure{{\Large\faUserSecret}:} $(E_B, \phi_B(P_A), \phi_B(Q_A))$
		\item \structure{{\Large\faUserSecret}:} $E_{AB} := E_B/\left< [m_A]\phi_B(P_A) + [n_A]\phi_B(Q_A) \right>$
		\item \structure{{\Large\faCat}:} $E_{BA} := E_A/\left< [m_B]\phi_A(P_B) + [n_B]\phi_A(Q_B) \right>$
		\item Общий секретный ключ: $j(E_{AB}) = j(E_{BA})$
	\end{enumerate}
\end{frame}

\begin{frame}{Замечания}
	\begin{itemize}
		\item сложность атаки (MITM): $O(\sqrt[4]{p})$ на классическом компьютере и $O(\sqrt[6]{p})$ для квантового компьютера
		\item гладкое число точек необходимо для быстрого вычисления изогений в точке
		\item можно выбрать~$E$ -- обычную кривую с гладким числом точек \structure{$\implies$} сложность атаки на квантовом компьютере становится субэкспоненциальной, т.к. кольцо эндоморфизмов -- коммутативное.
	\end{itemize}
\end{frame}

\begin{frame}{SIKE. Параметры}
\begin{enumerate}
	\item $E: y^2 = x^3 + 6 x^2 + x$
	\item $p = 2^{e_A} 3^{e_B} + 1$
	\item $\#E(\mathbb{F}_{p^2}) = 2^{e_A} 3^{e_B}$
	\item $2^{e_A} \approx 3^{e_B}$
	%\item \structure{{\Large\faUserSecret}} выбирает изогении степени $(2^e)^2$
	%\item \structure{{\Large\faCat}}
\end{enumerate}
\end{frame}

\begin{frame}{Атака Castryck-Decru}
\begin{enumerate}
	\item Castryck, Decru - An efficient key recovery attack on SIDH
	\item Maino, Martindale - An attack on SIDH with arbitrary starting curve
	\item Robert - Breaking SIDH in polynomial time
\end{enumerate}
Данные статьи -- предварительные версии статей и нечитаемы.
\vspace*{1em}

Выступление Castryck на ANTS XV:
\url{https://www.youtube.com/watch?v=_eNv7An3Qj0}
\end{frame}

\begin{frame}{Восстановление ключа}
Пусть~$B = \left< [m_b] P_B + [n_B] Q_B \right>$.

\vspace*{1em}

\structure{Задача восстановления ключа:}
\begin{center}
$E, E/B$, $\phi_B(P_A)$, $\phi_B(Q_A)$ \structure{$\Longrightarrow$} $\phi_B$
\end{center}
\begin{itemize}
	\item
	Атака Decru-Castryck использует группу
	\[
	\left< (P_A, \phi_B(P_A)), (Q_A, \phi_B(Q_A)) \right> \subseteq E \times E/B
	\]
	и соответствующую этой группе фактор-изогению из $E \times E/B$ (используя аналогии формул Велу для размерности~$2$).
	\item В большинстве случаев фактор-изогения будет вести в якобиан кривой рода~$2$.
	\item В редких случаях -- в произведение эллиптических кривых.
	\item Детектирование последнего редкого случая позволяет выявить правильное направление движения при поиске пути в графе изогений.
\end{itemize}
\end{frame}

\begin{frame}{Детектирование разложимого случая}
Тройка~$(\phi, G_1, G_2)$ -- ``алмазная'' изогенная конфигурация степени~$N$, если:
\begin{enumerate}
	\item $\phi: E \rightarrow E'$ -- изогения
	\item $G_1, G_2 \subseteq \ker{\phi}$
	\item $\deg{\phi} = \#G_1 \cdot \#G_2$, $N = \#G_1  + \#G_2$, $G_1 \cap G_2 = \{ 0 \}$
\end{enumerate}
\vspace*{1em}

\structure{Теорема Кани (неформально).}
$(N,N)$-подгруппа $E \times E'$ разложимая \structure{$\iff$} она получена из некоторой ``алмазной'' изогенной конфигурации степени~$N$.
\end{frame}

\begin{frame}{Атака}
	Изогения $\phi_B: E \rightarrow E/B$ степени~$3^{e_B}$ восстанавливается методом поиска алмазной конфигурации степени~$2^{{e_A}}$.
	
	\begin{itemize}
		\item строится изогения~$\gamma: E \rightarrow C$ степени~$2^{{e_A}} - 3^{{e_B}}$, т.ч. $(\phi_B \circ \widehat{\gamma}, \ker{\widehat{\gamma}}, \gamma(B))$ -- алмазная конфигурация степени~$2^{e_A}$
		\item $P_C = \gamma(P_A), Q_C = \gamma(Q_A)$
		\item по теореме Кани подгруппа $\left< (P_C, \phi_B(P_A)), (Q_C, \phi_B(Q_A)) \right> \subseteq C \times E/B$ -- разложимая
	\end{itemize}
\vspace{1em}
	\structure{Основная идея:} Если бы точки~$\phi_B(P_A)$ и~$\phi_B(Q_A)$ не были бы образами точек относительно (какой-либо) изогении степени~$3^{e_B}$, то с большой вероятностью группа не была бы разложимой
\end{frame}

%\begin{frame}
%\[
%E \times E/B \rightarrow J_{C_1} \rightarrow \ldots \rightarrow J_{C_r} \rightarrow E' \times E''
%\]
%\begin{itemize}
%	\item $\rightarrow$ -- изогении с ядром $\mathbb{Z}/2\mathbb{Z} \times \mathbb{Z}/2\mathbb{Z}$
%\end{itemize}
%\end{frame}

\begin{frame}{}
Имеем~$\phi_B = \phi_1 \circ \phi_2 \circ \ldots \circ \phi_{e_B}$, где~$\phi_i$ изогении степени~$3$.
\[
E \xrightarrow{\phi_1} E_1  \xrightarrow{\phi_2} E_2 \xrightarrow{\phi_3} \ldots \xrightarrow{\phi_{e_B}} E/B
\]

Будем восстанавливать изогению~$\phi_B$ по шагам.

\begin{enumerate}
	\item Выбираем изогению~$\phi_1': E \rightarrow E_1'$ 
	
	(т.к. $\deg \phi_1' = 3$, таких изогений немного)
	\item Строим~$\gamma: E_1' \rightarrow C$ степени~$2^{e_A} - 3^{e_B-1}$
	
	($\gamma$ можно построить используя эндоморфизмы кривой малой степени)
	\item $P_1' = \phi_1'(P_A)$, $Q_1' = \phi_1'(Q_A)$
	\item $P_C = \gamma(P_1')$, $Q_C = \gamma(Q_1')$
	\item Проверяем, что подгруппа \[(P_C, \phi_B(P_A)), (Q_C, \phi_B(Q_A)) \subseteq C \times E/B\] разложимая
	\item Если разложимая, то~$\phi_1 = \phi_1'$ и~$E_1' = E_1$
\end{enumerate}
\end{frame}

\begin{frame}{Схемы стойкие к атаке}
	Схемы не использующие точки кручения.
	\vspace*{1em}
	
	CSIDH, SeaSign, OSIDH, weakSIDH PoK, CSI-FiSh
	\vspace*{1em}
	\begin{itemize}
		\item \href{https://issikebrokenyet.github.io/}{issikebrokenyet.github.io}
	\end{itemize}
\end{frame}

\begin{frame}{Литература}
	%\nocite{Menezes1993}
	%\nocite{Lenstra1987}
	%\nocite{Blake1999}
	%\nocite{CohenFrey+2005}
	%\nocite{Washington2008}
	%\nocite{GoldwasserKilian1999}
	%\nocite{CohenLenstra1984}
	%\nocite{JaoDeFeo2011}
	%\nocite{Galbraith2012}
	%\nocite{DeFeo2018}
	%\nocite{Costello2019}
	\nocite{SIKE}
	%\nocite{SafeCurves}
    \nocite{CastryckDecru2022}
	\printbibliography

    \begin{center}
        \begin{tcolorbox}[enhanced,hbox,colback=block-green-color-bg,colframe=subsection-color!120,title=Контакты,center title]
            \begin{varwidth}{\textwidth}
                \begin{center}
                    \href{mailto:snovoselov@kantiana.ru}{snovoselov@kantiana.ru}
                \end{center}
            \end{varwidth}
        \end{tcolorbox}
    \end{center}\end{frame}

\end{document}
