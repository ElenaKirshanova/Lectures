\documentclass[11pt]{exam}
%%%%%%%%%%%%%%%%%%%%%%%%%%%%%%%%
%\noprintanswers % pour enlever les réponses
%\printanswers

\unframedsolutions
\SolutionEmphasis{\itshape\small}
\renewcommand{\solutiontitle}{\noindent\textbf{A: }}
%%%%%%%%%%%%%%%%%%%%%%%%%%%%%%%%

\usepackage[T2A]{fontenc}
\usepackage[utf8]{inputenc}
\usepackage[english, russian]{babel}


\usepackage[margin=0.73in]{geometry}
%\usepackage[top=1in, bottom=1in, left=1in, right=1in]{geometry}

%\usepackage{fullpage}


\usepackage{hyperref}
\usepackage{appendix}
\usepackage{enumerate}


\usepackage{times,graphicx,epsfig,amsmath,latexsym,amssymb,verbatim}%,revsymb}
\usepackage{algorithmicx, enumitem, algpseudocode, algorithm, caption}


%%%%%%%%%%%%%%%%%%%%%
% Handling comments and versions %%%
%%%%%%%%%%%%%%%%%%%%%
\newcommand{\extra}[1]{}

\renewcommand{\comment}[1]{\texttt{[#1]}}


%%%%%%%%%%%%%%%%%%%%%%%%%%%
%% THEOREMS
%%%%%%%%%%%%%%%%%%%%%%%%%%%

\usepackage{amsmath,amssymb,amsfonts}
\usepackage{amsthm}

% Landau 
\newcommand{\bigO}{\mathcal{O}}
\newcommand*{\OLandau}{\bigO}
\newcommand*{\WLandau}{\Omega}
\newcommand*{\xOLandau}{\widetilde{\OLandau}}
\newcommand*{\xWLandau}{\widetilde{\WLandau}}
\newcommand*{\TLandau}{\Theta}
\newcommand*{\xTLandau}{\widetilde{\TLandau}}
\newcommand{\smallo}{o} %technically, an omicron
\newcommand{\softO}{\widetilde{\bigO}}
\newcommand{\wLandau}{\omega}
\newcommand{\negl}{\mathrm{negl}} 


\newtheorem{theorem}{Теорема}
\newtheorem{corollary}[theorem]{Следствие}
\newtheorem{lemma}[theorem]{Лемма}
\newtheorem{observation}[theorem]{Observation}
\newtheorem{proposition}[theorem]{Предложение}

\theoremstyle{definition}
\newtheorem{definition}[theorem]{Определение}


\newcommand{\nc}{\newcommand}
\nc{\eps}{\varepsilon}
\nc{\RR}{{{\mathbb R}}}
\nc{\CC}{{{\mathbb C}}}
\nc{\FF}{{{\mathbb F}}}
\nc{\NN}{{{\mathbb N}}}
\nc{\ZZ}{{{\mathbb Z}}}
\nc{\PP}{{{\mathbb P}}}
\nc{\QQ}{{{\mathbb Q}}}
\nc{\UU}{{{\mathbb U}}}
\nc{\OO}{{{\mathbb O}}}
\nc{\EE}{{{\mathbb E}}}

\newcommand{\val}{\operatorname{val}}

\newcommand{\wt}{\ensuremath{\mathit{wt}}}
\newcommand{\Id}{\ensuremath{I}}
\newcommand{\transpose}{\mkern0.7mu^{\mathsf{ t}}}
\newcommand*{\ScProd}[2]{\ensuremath{\langle#1\mathbin{,}#2\rangle}} %Scalar Product
%\newcommand*{\eps}{\ensuremath{\varepsilon}}
\newcommand*{\Sphere}[1]{\ensuremath{\mathsf{S}^{#1}}}

\pretolerance=1000

%%%%%%%%%%%%%%%%%%%%%%%%%%%%%%%%
%%%%%%%%%%%%%%%%%%%%%%%%%%%%%%%%
%% DOCUMENT STARTS
%%%%%%%%%%%%%%%%%%%%%%%%%%%%%%%%
%%%%%%%%%%%%%%%%%%%%%%%%%%%%%%%%
\usepackage{tikz}
\usetikzlibrary{automata}
\DeclareMathOperator{\Vol}{Vol}

\begin{document}
	{\noindent
		\textsc{БФУ им. И. Канта -- Теория кодирования и сжатия информации}
		\hfill {Е. Киршанова // 2021\\}
	\hrule
	\begin{center}
		{\Large\textbf{
				\textsc{Практика № 10} \\[5pt] {12.11.21}
		} } 
	\end{center}
	\hrule \vspace{5mm}
	
	\thispagestyle{empty}
	
	\vspace{0.2cm}
	

	
%
%\section{BCH (БЧХ) код}
%		Для $n = 2^m - 1$, покажите, что $\mathrm{BCH}(n,3)$--код совпадает (до, быть может, перестановки координат) с расширенным кодом Хэмиинга $\mathrm{Ham}(2^m - 1, 2^m -m-1, 3)$. Напомним, что проверочная матрица $H\in \FF_2^{m \times 2^m - 1}$ кода Хэмиинга в $i$-ом столбце содержит бинарное представление числа $i$.
		
	
%\section{Минимальное расстояние кода Рида-Маллера}
%		\begin{questions}
%			
% 		\question Пусть $C_1$-- линейный $[n, k_1, d_1]$-код, $C_2$-- линейный $[n, k_2, d_2]$-код. Докажите, что \[C_3 = \{ [u | u+v], u \in C_1, v \in C_2 \}-\]--линейный $[2n, k_1 + k_2, \min\{ 2d_1, d_2 \}]$--код.
% 		\question Докажите, что $d(\mathrm{RM}(r,m)) = 2^{m-r}$. Можете использовать индукцию по $m$.
% 		
% 	\end{questions}
 		
 
\section{Пример кода Рида-Маллера}

\begin{questions}
	\question Напишите порождающую матрицу кода  $\mathrm{RM}(2,4)$. Каковы параметры этого кода (длина, размерность, минимальное расстояние)?
	\question Декодируйте
	
	\def\arraystretch{1.5}
	\setlength{\tabcolsep}{15pt}
	\begin{tabular}{l  | c }
		Нейман Даниил & $y = (1, 0, 0, 0, 1, 0, 0, 1, 1, 1, 0, 0, 1, 0, 0, 1)$\\ \hline
		Раточка Вячеслав   & $y=(0, 0, 1, 1, 0, 0, 1, 0, 1, 1, 0, 0, 1, 1, 0, 0)$\\ \hline
		Овсянников Никита  &  $y=(0, 1, 1, 0, 0, 1, 1, 0, 0, 0, 0, 1, 0, 1, 1, 0)$\\  \hline
		Гринчуков Владимир  &$y = (0, 0, 1, 0, 1, 0, 1, 0, 1, 1, 0, 0, 0, 1, 1, 0)$ \\  \hline
		Тогунова Валерия &  $y=(1, 1, 1, 0, 0, 0, 0, 0, 1, 0, 1, 0, 1, 0, 1, 0)$\\  \hline
		Мацуль Илья &  $y=(1, 1, 0, 0, 0, 1, 0, 0, 0, 0, 0, 0, 0, 0, 0, 0)$\\  \hline
		Ишметов Павел  & $y = (0, 1, 0, 0, 1, 1, 0, 0, 0, 1, 0, 1, 0, 1, 0, 1)$\\  \hline
		Инна Торсукова  & $y=(0, 0, 1, 1, 1, 1, 1, 1, 1, 0, 1, 1, 0, 0, 1, 1)$\\  \hline
		Соколенко Анастасия  &$y=(1, 1, 1, 1, 0, 1, 0, 0, 0, 1, 0, 1, 1, 0, 1, 0)$ \\  \hline
		Дупленко Александр  & $y = (1, 1, 1, 0, 1, 1, 1, 0, 0, 0, 0, 1, 1, 0, 1, 0)$\\  \hline
		Камбаров Игорь & $y=(0, 1, 0, 1, 1, 0, 0, 1, 1, 0, 1, 0, 0, 0, 0, 1)$ \\  \hline
		Шерстобитов Глеб &  $y=(1, 0, 0, 0, 1, 0, 0, 1, 1, 1, 0, 0, 1, 0, 0, 1)$\\  \hline
		Гудасов Александр & $y = (0, 0, 1, 1, 0, 0, 1, 0, 1, 1, 0, 0, 1, 1, 0, 0)$ \\  \hline
		Мартынюк Жанна & $y=(0, 1, 1, 0, 0, 1, 1, 0, 0, 0, 0, 1, 0, 1, 1, 0)$\\  \hline
	\end{tabular}

\end{questions}
\end{document}
