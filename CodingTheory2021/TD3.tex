\documentclass[11pt]{exam}
%%%%%%%%%%%%%%%%%%%%%%%%%%%%%%%%
%\noprintanswers % pour enlever les réponses
%\printanswers

\unframedsolutions
\SolutionEmphasis{\itshape\small}
\renewcommand{\solutiontitle}{\noindent\textbf{A: }}
%%%%%%%%%%%%%%%%%%%%%%%%%%%%%%%%

\usepackage[T2A]{fontenc}
\usepackage[utf8]{inputenc}
\usepackage[english, russian]{babel}


\usepackage[margin=0.73in]{geometry}
%\usepackage[top=1in, bottom=1in, left=1in, right=1in]{geometry}

%\usepackage{fullpage}


\usepackage{hyperref}
\usepackage{appendix}
\usepackage{enumerate}


\usepackage{times,graphicx,epsfig,amsmath,latexsym,amssymb,verbatim}%,revsymb}
\usepackage{algorithmicx, enumitem, algpseudocode, algorithm, caption}


%%%%%%%%%%%%%%%%%%%%%
% Handling comments and versions %%%
%%%%%%%%%%%%%%%%%%%%%
\newcommand{\extra}[1]{}

\renewcommand{\comment}[1]{\texttt{[#1]}}


%%%%%%%%%%%%%%%%%%%%%%%%%%%
%% THEOREMS
%%%%%%%%%%%%%%%%%%%%%%%%%%%

\usepackage{amsmath,amssymb,amsfonts}
\usepackage{amsthm}

% Landau 
\newcommand{\bigO}{\mathcal{O}}
\newcommand*{\OLandau}{\bigO}
\newcommand*{\WLandau}{\Omega}
\newcommand*{\xOLandau}{\widetilde{\OLandau}}
\newcommand*{\xWLandau}{\widetilde{\WLandau}}
\newcommand*{\TLandau}{\Theta}
\newcommand*{\xTLandau}{\widetilde{\TLandau}}
\newcommand{\smallo}{o} %technically, an omicron
\newcommand{\softO}{\widetilde{\bigO}}
\newcommand{\wLandau}{\omega}
\newcommand{\negl}{\mathrm{negl}} 


\newtheorem{theorem}{Теорема}
\newtheorem{corollary}[theorem]{Следствие}
\newtheorem{lemma}[theorem]{Лемма}
\newtheorem{observation}[theorem]{Observation}
\newtheorem{proposition}[theorem]{Предложение}

\theoremstyle{definition}
\newtheorem{definition}[theorem]{Определение}


\newcommand{\nc}{\newcommand}
\nc{\eps}{\varepsilon}
\nc{\RR}{{{\mathbb R}}}
\nc{\CC}{{{\mathbb C}}}
\nc{\FF}{{{\mathbb F}}}
\nc{\NN}{{{\mathbb N}}}
\nc{\ZZ}{{{\mathbb Z}}}
\nc{\PP}{{{\mathbb P}}}
\nc{\QQ}{{{\mathbb Q}}}
\nc{\UU}{{{\mathbb U}}}
\nc{\OO}{{{\mathbb O}}}
\nc{\EE}{{{\mathbb E}}}

\newcommand{\val}{\operatorname{val}}

\newcommand{\wt}{\ensuremath{\mathit{wt}}}
\newcommand{\Id}{\ensuremath{I}}
\newcommand{\transpose}{\mkern0.7mu^{\mathsf{ t}}}
\newcommand*{\ScProd}[2]{\ensuremath{\langle#1\mathbin{,}#2\rangle}} %Scalar Product

\pretolerance=1000

%%%%%%%%%%%%%%%%%%%%%%%%%%%%%%%%
%%%%%%%%%%%%%%%%%%%%%%%%%%%%%%%%
%% DOCUMENT STARTS
%%%%%%%%%%%%%%%%%%%%%%%%%%%%%%%%
%%%%%%%%%%%%%%%%%%%%%%%%%%%%%%%%
\usepackage{tikz}
\usetikzlibrary{automata}
\DeclareMathOperator{\Vol}{Vol}

\begin{document}
	{\noindent
		\textsc{БФУ им. И. Канта -- Теория кодирования и сжатия информации}
		\hfill {Е. Киршанова // 2021\\}
	\hrule
	\begin{center}
		{\Large\textbf{
				\textsc{Практика № 3} \\[5pt] {24.09.20}
		} } 
	\end{center}
	\hrule \vspace{5mm}
	
	\thispagestyle{empty}
	
	\vspace{0.2cm}

	
\section{Граница Синглтона}
	Докажите, что для $q>1$, $n, d \in \NN$, таких что $1 \leq d \leq n$, выполняется
	\[
		A_q(n,d) \leq q^{n-d+1}.
	\]
	

\section{Коды с максимальным расстоянием}
Пусть $C$-линейный $[n,k,d]$-- код с проверочной матрицей $H$. Докажите, что $C-$ код с максимальным расстоянием тогда и только тогда, когда $C^\perp$-- код с максимальным расстоянием.

	
	
\section{Количество порождающих матриц}
Покажите, что для $[n,k]_q$-- линейного кода $C$ ($q$ -- простое), количество различных порождающих матриц равно
\[
\prod_{i=0}^{k-1}(q^k - q^i).
\]

\section{Альтернативное доказательство Теоремы 4}
	Пусть $q$ - простое, $n,k,d \in \NN$, такие что $k \leq n-d+1$.	 Рассмотрим множество матриц над $\FF_q$ вида
	\[
		H = [A | \Id_{n-k}] \in \FF_q^{n-k \times n}
	\]
	и зададим распределение на этом множестве через равномерное распределение над $(n-k \times k)$ матрицами $A$ над $\FF_q$.
	\begin{enumerate}
		\item Докажите, что для любого ненулевого вектора $y \in \FF_q^n$
		\[
			\Pr_{H \xleftarrow{\$} \FF_q^{n-k \times n}}  \left[Hy = 0\right]  = 
			\begin{cases}
				0, & \text{первые $k$ позиций $y$ нулевые} \\
				q^{k-n}, & \text{иначе}.
			\end{cases}
		\]
		\item  Докажите, что \[\Pr_{H \xleftarrow{\$} \FF_q^{n-k \times n}}  \left[H \text{ содержит $d-1$ лин.\ завис.\ столбцов }\right] \leq \rho, \] где
		\[
			\rho = q^{k-n} \cdot \frac{\Vol_q^n(d-1) - \Vol_q^{n-k}(d-1)}{q-1}.
		\]
	\end{enumerate}

\end{document}