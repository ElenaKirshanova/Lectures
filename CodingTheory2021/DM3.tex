\documentclass[11pt]{exam}
%%%%%%%%%%%%%%%%%%%%%%%%%%%%%%%%
%\noprintanswers % pour enlever les réponses
%\printanswers

\unframedsolutions
\SolutionEmphasis{\itshape\small}
\renewcommand{\solutiontitle}{\noindent\textbf{A: }}
%%%%%%%%%%%%%%%%%%%%%%%%%%%%%%%%

\usepackage[T2A]{fontenc}
\usepackage[utf8]{inputenc}
\usepackage[english, russian]{babel}

\usepackage{graphicx}
\usepackage{url}
\usepackage{latexsym}
\usepackage{amscd,amsmath,amsthm}
\usepackage{mathtools}
\usepackage{amsfonts}
\usepackage{amssymb}
\usepackage[dvipsnames]{xcolor}
\usepackage{hyperref}


\usepackage[margin=0.73in]{geometry}
%\usepackage[top=1in, bottom=1in, left=1in, right=1in]{geometry}

%\usepackage{fullpage}


\usepackage{hyperref}
\usepackage{appendix}
\usepackage{enumerate}



\usepackage{algorithmicx, enumitem, algpseudocode, algorithm, caption}


%%%%%%%%%%%%%%%%%%%%%%%%%%%
%% THEOREMS
%%%%%%%%%%%%%%%%%%%%%%%%%%%


\newtheorem{theorem}{Theorem}[section]
\newtheorem{axiom}[theorem]{Axiom}
\newtheorem{conclusion}[theorem]{Conclusion}
\newtheorem{condition}[theorem]{Condition}
\newtheorem{conjecture}[theorem]{Conjecture}
\newtheorem{corollary}[theorem]{Corollary}
\newtheorem{criterion}[theorem]{Criterion}
\newtheorem{definition}[theorem]{Definition}
\newtheorem{lemma}[theorem]{Lemma}
\newtheorem{notation}[theorem]{Notation}
\newtheorem{proposition}[theorem]{Proposition}


\theoremstyle{definition}
\newtheorem{problem}{Problem}


\newcommand{\nc}{\newcommand}
\nc{\eps}{\varepsilon}
\nc{\RR}{{{\mathbb R}}}
\nc{\CC}{{{\mathbb C}}}
\nc{\FF}{{{\mathbb F}}}
\nc{\NN}{{{\mathbb N}}}
\nc{\ZZ}{{{\mathbb Z}}}
\nc{\PP}{{{\mathbb P}}}
\nc{\QQ}{{{\mathbb Q}}}
\nc{\UU}{{{\mathbb U}}}
\nc{\OO}{{{\mathbb O}}}
\nc{\EE}{{{\mathbb E}}}

\newcommand{\val}{\operatorname{val}}

\newcommand{\wt}{\ensuremath{\mathit{wt}}}
\newcommand{\Id}{\ensuremath{I}}
\newcommand{\transpose}{\mkern0.7mu^{\mathsf{ t}}}
\newcommand*{\ScProd}[2]{\ensuremath{\langle#1\mathbin{,}#2\rangle}} %Scalar Product

\pretolerance=1000

%%%%%%%%%%%%%%%%%%%%%%%%%%%%%%%%
%%%%%%%%%%%%%%%%%%%%%%%%%%%%%%%%
%% DOCUMENT STARTS
%%%%%%%%%%%%%%%%%%%%%%%%%%%%%%%%
%%%%%%%%%%%%%%%%%%%%%%%%%%%%%%%%
\usepackage{tikz}
\usetikzlibrary{automata}
\DeclareMathOperator{\Vol}{Vol}

\begin{document}
	{\noindent
		\textsc{БФУ им. И. Канта -- Теория кодирования и сжатия информации}
		\hfill {Е. Киршанова // 2019--2020\\}
	\hrule
	\begin{center}
		{\Large\textbf{
				\textsc{Домашнее задание № 3} \\[5pt] {Опубликовано 12.11.21, Срок сдачи: 26.11.21}
		} } 
	\end{center}
	\hrule \vspace{5mm}
	
	\thispagestyle{empty}
	
	\vspace{0.2cm}
	\section{Циклические коды}
	
	Линейный $[n,k]$-код над полем $\FF$ называется \emph{циклическим}, если любой сдвиг кодового слова слова также принадлежит коду, то есть 
	\[
			c = (c_0, \ldots, c_{n-1})\in C \implies (c_{n-1}, c_0, \ldots , c_{n-1})  \in C. 
	\]
	Векторам из $\FF^n$ будем сопоставлять многочлены из $\FF[x]$ степени $<n$: $c_0, \ldots c_{n-1} \mapsto c_0 + c_1 x + \ldots + c_{n-1} x^{n-1}$.
	\begin{questions}
		\question Покажите, что циклический $[n,k]$-код $C$ образует идеал в $F[x]/(x^n - 1)$.
		\question Докажите, что код Рида-Соломона $\mathrm{RS}_{\FF_q, \FF_q^*}(n,d)$-- циклический. Напомним, $\mathrm{RS}_{\FF_q, \FF_q^*}(n,d) = \{ c \in \FF_q^n \; | \; c(\alpha) = c(\alpha^2) = \ldots = c(\alpha^{n-k}) = 0  \}$
		\question Докажите, что для циклического  $[n,k]$-кода $C$, где $k>0$, существует единственный унитарный многочлен $g(x)$, такой что \[c(x) \in C \iff g(x) | c(x).\]
		\noindent Такой многочлен называется \emph{порождающим}. Для него справедливо $C = \{ u(x) g(x) \; | \; u(x) \in \FF[x], \deg u < n-\deg g \}$.
		\question Постройте порождающий многочлен для кода Рида-Соломона $\mathrm{RS}_{\FF_q, \FF_q^*}(n,d)$.
		\question Многочлен $h(x) = \frac{x^n-1}{g(x)}$ называется \emph{проверочным}. Докажите, что $h(x) \in \FF[x]/(x^n-1)$, т.е.\ $g(x) | x^n - 1$.
		\question Докажите обратное: если $g(x) | x^n - 1$, то множество $C = \{ u(x) g(x) \; | \; u(x) \in \FF[x], \deg u < n-\deg g \}$ является циклическим кодом.
		\question Из многочленов $g, h$ можно построить порождающую и проверочную матрицы:
		\begin{equation*}
		\begin{aligned}[c]
		G = 
		\begin{pmatrix}
		   g_0 & g_1 & \ldots & g_{n-k} & 0 & 0 \\
			0 & g_0 & g_1 & \ldots & g_{n-k}  & 0 \\
			\vdots & \vdots  & \vdots & \ddots & \vdots & \vdots \\
			0 & 0 & 0 & \ldots &g_{n-k-1} & g_{n-k}  \\
		\end{pmatrix}
		\end{aligned} \quad \quad 
		\begin{aligned}[c]
		H = 
		\begin{pmatrix}
		h_k & h_{k-1} & \ldots & h_{0} & 0 & 0 \\
		0 & h_k & h_{k-1} & \ldots & h_{0}  & 0 \\
		\vdots & \vdots  & \vdots & \ddots & \vdots & \vdots \\
		0 & 0 & 0 & \ldots &h_{1} & h_{0}  \\
		\end{pmatrix}
		\end{aligned}.
		\end{equation*}
		
		Для циклического кода длины $7$ над $\FF_2$ c порождающим многочленом $g(x) =x^3+x+1 $, постройте $h(x), G, H$. Каково минимальное расстояние этого кода?
	
	\end{questions}
 	
\end{document}