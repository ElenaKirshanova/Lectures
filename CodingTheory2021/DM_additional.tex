\documentclass[11pt]{exam}
%%%%%%%%%%%%%%%%%%%%%%%%%%%%%%%%
%\noprintanswers % pour enlever les réponses
%\printanswers

\unframedsolutions
\SolutionEmphasis{\itshape\small}
\renewcommand{\solutiontitle}{\noindent\textbf{A: }}
%%%%%%%%%%%%%%%%%%%%%%%%%%%%%%%%

\usepackage[T2A]{fontenc}
\usepackage[utf8]{inputenc}
\usepackage[english, russian]{babel}

\usepackage{graphicx}
\usepackage{url}
\usepackage{latexsym}
\usepackage{amscd,amsmath,amsthm}
\usepackage{mathtools}
\usepackage{amsfonts}
\usepackage{amssymb}
\usepackage[dvipsnames]{xcolor}
\usepackage{hyperref}


\usepackage[margin=0.73in]{geometry}
%\usepackage[top=1in, bottom=1in, left=1in, right=1in]{geometry}

%\usepackage{fullpage}


\usepackage{hyperref}
\usepackage{appendix}
\usepackage{enumerate}



\usepackage{algorithmicx, enumitem, algpseudocode, algorithm, caption}


%%%%%%%%%%%%%%%%%%%%%%%%%%%
%% THEOREMS
%%%%%%%%%%%%%%%%%%%%%%%%%%%


\newtheorem{theorem}{Theorem}[section]
\newtheorem{axiom}[theorem]{Axiom}
\newtheorem{conclusion}[theorem]{Conclusion}
\newtheorem{condition}[theorem]{Condition}
\newtheorem{conjecture}[theorem]{Conjecture}
\newtheorem{corollary}[theorem]{Corollary}
\newtheorem{criterion}[theorem]{Criterion}
\newtheorem{definition}[theorem]{Definition}
\newtheorem{lemma}[theorem]{Lemma}
\newtheorem{notation}[theorem]{Notation}
\newtheorem{proposition}[theorem]{Proposition}


\theoremstyle{definition}
\newtheorem{problem}{Problem}


\newcommand{\nc}{\newcommand}
\nc{\eps}{\varepsilon}
\nc{\RR}{{{\mathbb R}}}
\nc{\CC}{{{\mathbb C}}}
\nc{\FF}{{{\mathbb F}}}
\nc{\NN}{{{\mathbb N}}}
\nc{\ZZ}{{{\mathbb Z}}}
\nc{\PP}{{{\mathbb P}}}
\nc{\QQ}{{{\mathbb Q}}}
\nc{\UU}{{{\mathbb U}}}
\nc{\OO}{{{\mathbb O}}}
\nc{\EE}{{{\mathbb E}}}

\newcommand{\val}{\operatorname{val}}

\newcommand{\wt}{\ensuremath{\mathit{wt}}}
\newcommand{\Id}{\ensuremath{I}}
\newcommand{\transpose}{\mkern0.7mu^{\mathsf{ t}}}
\newcommand*{\ScProd}[2]{\ensuremath{\langle#1\mathbin{,}#2\rangle}} %Scalar Product

\pretolerance=1000

%%%%%%%%%%%%%%%%%%%%%%%%%%%%%%%%
%%%%%%%%%%%%%%%%%%%%%%%%%%%%%%%%
%% DOCUMENT STARTS
%%%%%%%%%%%%%%%%%%%%%%%%%%%%%%%%
%%%%%%%%%%%%%%%%%%%%%%%%%%%%%%%%
\usepackage{tikz}
\usetikzlibrary{automata}
\DeclareMathOperator{\Vol}{Vol}

\begin{document}
	{\noindent
		\textsc{БФУ им. И. Канта -- Теория кодирования и сжатия информации}
		\hfill {Е. Киршанова // 2019--2020\\}
	\hrule
	\begin{center}
		{\Large\textbf{
				\textsc{Дополнительное домашнее задание} \\[5pt] {Срок сдачи: 13.12.21}
		} } 
	\end{center}
	\hrule \vspace{5mm}
	
	\thispagestyle{empty}
	
	\vspace{0.2cm}
	\section{Разделимые коды с максимальным расстоянием}
	Коды, для которых выполняется равенство $d = n-k+1$, называются разделимыми кодами с максимальным расстоянием (краткое МДР) (вводились в лекции № 3). Пусть $C$ -- линейный $[n, k, d]$--код. 
	\begin{questions}
		\question Докажите, что $C$ -- код МДР тогда и только тогда, когда любые $n-k$ столбцов проверочной матрицы $H$ линейно независимы.
		\question Докажите, что если $C$ -- код МДР, то и его дуальный код, $C^\perp$, также код МДР.
		\question Пусть линейный код $C$  задан над $GF(4) = \{0, 1, \alpha, \alpha^2  \}$ порождающей матрицей
		\[
			G = 
			\begin{pmatrix}
			1 & 0 & 1 & 1 \\
			0 & 1 & \alpha & \alpha^2
			\end{pmatrix}
		\]
		\begin{enumerate}
			\item Какова длина, размерность и минимальное расстояние кода? Сколько ошибок может исправить этот код? 
			\item Сколько кодовых слов содержит $C$?
			\item Напишите порождающую матрицу $C^\perp$. Является ли $C^\perp$ МДР кодом?
			\item Принадлежит ли вектор $[\alpha \ 1 \ 0 \ \alpha+\alpha^2]$ коду $C$? Ответ поясните.
		\end{enumerate}
		

	\end{questions}

 	
\end{document}