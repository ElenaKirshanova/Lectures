\documentclass[11pt]{exam}
%%%%%%%%%%%%%%%%%%%%%%%%%%%%%%%%
%\noprintanswers % pour enlever les réponses
%\printanswers

\unframedsolutions
\SolutionEmphasis{\itshape\small}
\renewcommand{\solutiontitle}{\noindent\textbf{A: }}
%%%%%%%%%%%%%%%%%%%%%%%%%%%%%%%%

\usepackage[T2A]{fontenc}
\usepackage[utf8]{inputenc}
\usepackage[english, russian]{babel}

\usepackage{graphicx}
\usepackage{url}
\usepackage{latexsym}
\usepackage{amscd,amsmath,amsthm}
\usepackage{mathtools}
\usepackage{amsfonts}
\usepackage{amssymb}
\usepackage[dvipsnames]{xcolor}
\usepackage{hyperref}


\usepackage[margin=0.73in]{geometry}
%\usepackage[top=1in, bottom=1in, left=1in, right=1in]{geometry}

%\usepackage{fullpage}


\usepackage{hyperref}
\usepackage{appendix}
\usepackage{enumerate}



\usepackage{algorithmicx, enumitem, algpseudocode, algorithm, caption}


%%%%%%%%%%%%%%%%%%%%%%%%%%%
%% THEOREMS
%%%%%%%%%%%%%%%%%%%%%%%%%%%


\newtheorem{theorem}{Theorem}[section]
\newtheorem{axiom}[theorem]{Axiom}
\newtheorem{conclusion}[theorem]{Conclusion}
\newtheorem{condition}[theorem]{Condition}
\newtheorem{conjecture}[theorem]{Conjecture}
\newtheorem{corollary}[theorem]{Corollary}
\newtheorem{criterion}[theorem]{Criterion}
\newtheorem{definition}[theorem]{Definition}
\newtheorem{lemma}[theorem]{Lemma}
\newtheorem{notation}[theorem]{Notation}
\newtheorem{proposition}[theorem]{Proposition}


\theoremstyle{definition}
\newtheorem{problem}{Problem}


\newcommand{\nc}{\newcommand}
\nc{\eps}{\varepsilon}
\nc{\RR}{{{\mathbb R}}}
\nc{\CC}{{{\mathbb C}}}
\nc{\FF}{{{\mathbb F}}}
\nc{\NN}{{{\mathbb N}}}
\nc{\ZZ}{{{\mathbb Z}}}
\nc{\PP}{{{\mathbb P}}}
\nc{\QQ}{{{\mathbb Q}}}
\nc{\UU}{{{\mathbb U}}}
\nc{\OO}{{{\mathbb O}}}
\nc{\EE}{{{\mathbb E}}}

\newcommand{\val}{\operatorname{val}}

\newcommand{\wt}{\ensuremath{\mathit{wt}}}
\newcommand{\Id}{\ensuremath{I}}
\newcommand{\transpose}{\mkern0.7mu^{\mathsf{ t}}}
\newcommand*{\ScProd}[2]{\ensuremath{\langle#1\mathbin{,}#2\rangle}} %Scalar Product

\pretolerance=1000

%%%%%%%%%%%%%%%%%%%%%%%%%%%%%%%%
%%%%%%%%%%%%%%%%%%%%%%%%%%%%%%%%
%% DOCUMENT STARTS
%%%%%%%%%%%%%%%%%%%%%%%%%%%%%%%%
%%%%%%%%%%%%%%%%%%%%%%%%%%%%%%%%
\usepackage{tikz}
\usetikzlibrary{automata}
\DeclareMathOperator{\Vol}{Vol}

\begin{document}
	{\noindent
		\textsc{БФУ им. И. Канта -- Теория кодирования и сжатия информации}
		\hfill {Е. Киршанова //2021\\}
	\hrule
	\begin{center}
		{\Large\textbf{
				\textsc{Домашнее задание № 2} \\[5pt] {Опубликовано 22.10.21, Срок сдачи: 05.11.21}
		} } 
	\end{center}
	\hrule \vspace{5mm}
	
	\thispagestyle{empty}
	
	\noindent \textsf{Инструкция}
	
	\begin{enumerate}
		\itemsep 2pt
		\item Решения принимаются \emph{исключительно} до дэдлайна, указанного в шапке ДЗ. Время сдачи указано в соответствии с Калининградским часовым поясом.
		\item Решения должны быть выполнены \textbf{индивидуально}.
		\item Решения должны быть написано разборчиво,  все утверждения должны быть аргументированы.
		\item Решения принимаются по электронной почте \textsf{elenakirshanova@gmail.com} с четкой пометкой, кто автор присланных решений.
		\item Все решения должны быть оформлены в \textbf{один} файл формата {pdf}, которые должен иметь адекватное соотношение размер-качество (весь файл должен занимать не более 15Мb).
	\end{enumerate}	
	
	\vspace{0.2cm}
	\section{Код Рида-Соломона}
	
	Найдите в файле \url{https://crypto-kantiana.com/elena.kirshanova/teaching/coding_2021/HW2.txt} параметры своего кода и значение вектора $y$. Декодируйте $y$ (то есть найдите вектор ошибок и исходное сообщение) любым из изученных алгоритмов декодирования. Опишите промежуточные вычисления. Для решений можете использовать Sage. Исходный код присылать необязательно.
		
	\section{Совершенный код}
	Формуляр заполнения ставок на футбольные матчи содержит в себе таблицу, составленную из списка 13 игр, где каждой из них сопоставлены три выбора, описывающие  все возможные исходы игры: \textit{Выигрыш команды №1}, \textit{Выигрыш команды № 2}, \textit{Ничья}. Для заполнения формуляра нужно выбрать один из исходов напротив каждого матча.
	
	Опишите стратегию заполнения минимального количества формуляров так, чтобы хотя бы один из этих формуляров содержал как минимум 12 верных ставок. Какое число формуляров необходимо заполнить при этой стратегии?
	
	\textit{Подсказка:} существует совершенный код длины 13 с минимальным расстоянием 3 над $\FF_3$.
 	
\end{document}