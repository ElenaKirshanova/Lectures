\documentclass[12pt,a4paper]{scrartcl}

%\usepackage[T1]{fontenc}
%\usepackage[utf8x]{inputenc}
\usepackage{array}

%\usepackage[ngerman]{babel}
%\usepackage{ngerman}

%---enable russian----

\usepackage[T2A]{fontenc}
\usepackage[utf8]{inputenc}
\usepackage[russian,ngerman]{babel}


\usepackage{amsmath, amsfonts}
\newcounter{blatt}
\setcounter{blatt}{99}
\parindent = 0mm
\pagestyle{empty}
\setlength{\textheight}{26cm}

\usepackage{amssymb,epsfig,hyperref}
\usepackage{fancyhdr,pdfpages,graphicx}
\newcommand{\N}{\ensuremath{\mathbb{N}}\xspace}
\newcommand{\rem}[1]{}

\usepackage{enumerate}


%complexity classes
\newcommand*{\POLY}{{\mathcal{P}}}
\newcommand*{\NPOLY}{{\mathcal{NP}}}

\newcommand{\marrow}{\marginpar[\hfill$\longrightarrow$]{$\longleftarrow$}}
\newcommand{\todo}[1][]{\textbf{TODO} \marrow\textsf{#1}}

%
% Aufruf: \auf{Aufgabentext}
%
\newcounter{auf}
\newcommand{\auf}[2]
{
\stepcounter{auf}{\textbf{Задание} \textbf{\arabic{auf}} } ({#1} баллов) \vspace{3pt}

#2

% TODO: Uncomment page break here
%\bigskip
\newpage %\phantom{X} \newpage
}
%
% kleine Buchstaben zum durchnummerieren mit enumerate auf Level 1
%
%\renewcommand{\theenumi}{\alph{enumi}}
%\renewcommand{\labelenumi}{\theenumi)}

\newcommand{\nc}{\newcommand}
\nc{\eps}{\varepsilon}
\nc{\RR}{{{\mathbb R}}}
\nc{\CC}{{{\mathbb C}}}
\nc{\FF}{{{\mathbb F}}}
\nc{\NN}{{{\mathbb N}}}
\nc{\ZZ}{{{\mathbb Z}}}
\nc{\PP}{{{\mathbb P}}}
\nc{\QQ}{{{\mathbb Q}}}
\nc{\UU}{{{\mathbb U}}}
\nc{\OO}{{{\mathbb O}}}
\nc{\EE}{{{\mathbb E}}}



\newcommand{\set}[1]{\left\{#1\right\}}
\newcommand{\cset}[2]{\left\{#1 \middle\arrowvert #2\right\}}
\newcommand{\gor}{\; | \;}

\newcolumntype{L}[1]{>{\raggedright\arraybackslash}p{#1}} % linksbündig mit Breitenangabe
\newcolumntype{C}[1]{>{\centering\arraybackslash}p{#1}} % zentriert mit Breitenangabe
\newcolumntype{R}[1]{>{\raggedleft\arraybackslash}p{#1}} % rechtsbündig mit Breitenangabe


\begin{document}
Е.\ A.\ Киршанова \hfill БФУ им. И.Канта \\





\begin{center}
  \LARGE
  \underbar{Контрольная работа}\\[1ex]
  \Large
  по дисциплине \\[1ex]
  \textsc{Теория Кодироваия и Сжатия Информации}\\[2ex]
  \large
  Время: 180 минут \\
 17.12.2021
\end{center}
\vspace*{1.5cm}


Имя   :\\[-2.2ex] \rule{0.9\textwidth}{.2pt}\\[2ex]
Фамилия : \\ [-2.2ex] \rule{0.9\textwidth}{.2pt}\\[1ex]
%Studiengang:\\[-2.2ex] \rule{0.7\textwidth}{.2pt}\\[1ex]
%Geburtsdatum:\\[-2.2ex] \rule{0.7\textwidth}{.2pt}\\[1ex]

\vspace*{1.5cm}
\textbf{Требования:}
\begin{itemize}
  \item Не следует разделять скреплённые листы.  Если вам не хватает места для ответа, попросите у экзаменатора дополнительные листы. Подпишите их и приложите их к контрольной, чётко указав, решения каких заданий содержатся на каком листе.
%Schreiben Sie die Lösung jeder Aufgabe direkt auf das Blatt mit der Aufgabenstellung. Es dürfen Vorder- und R"uckseite verwendet werden. Wenn der Platz nicht ausreicht, k"onnen die leeren Seiten am Ende der Klausur benutzt werden.
 \item Для записи ответов вы можете использовать обе стороны листов.
  \item Пишите \textbf{разборчиво}.
  \item Поясняйте свои ответы.
    
    
  
\end{itemize}

%\vspace*{0.5cm}
\rule{\textwidth}{.2pt}
\vfill

\begin{center}
\renewcommand{\arraystretch}{1.3}
  \begin{tabular}{|p{5cm}||c|c|c|}
    \hline
    Задание & 1 & 2 & 3  \\
    \hline
    Баллы & \phantom{XXXXXX}/ 6 & \phantom{XXXXXX}/ 6 & \phantom{XXXXXX}/ 9  \\
    \hline
  \end{tabular}\\[5ex]

  \begin{tabular}{|p{4.2cm}|p{4.2cm}|p{4.2cm}|}
    \hline
    Контрольная & Бонусы & Общая  \\
    \hline
     &   &  \\[1ex]
    \hline
  \end{tabular}
  
\end{center}

\newpage 
 
\auf{$6 \times 1 $}{%

\begin{enumerate}[1]
	\setlength\itemsep{13em}
	\item Является множество $C$ линейным кодом? Ответ поясните.
	\[
		C = \{ 0000, 1100, 0011, 1111 \}.
	\]
	\item Линейный код $C$ над $\FF_5$  задан проверочной матрицей
	\[
		H = \begin{bmatrix}
		1&2&0&1&4\\ 
		3&1&1&2&1
		\end{bmatrix}.
	\]
	Есть ли среди ниже перечисленных векторов $x$ кодовые слова из $C$? Если да, то какие? Ответ поясните. \\
	
	\begin{tabular}{ p{7cm} p{7cm}  }
	$x = \left[ \begin {array}{ccccc} 0&0&1&2&2\end {array} \right]  $ 	 & $x = \left[ \begin {array}{ccccc} 1&0&1&0&2\end {array} \right]  $ \\[3ex]
		$x = \left[ \begin {array}{ccccc} 4&4&3&1&0\end {array} \right] $ & $x = \left[ \begin {array}{ccccc} 2&3&0&1&1\end {array} \right] $ 
	\end{tabular}

	\item Каково максимальное число ошибок, исправляемое следующими линейными кодами?
	\begin{enumerate}
		\setlength\itemsep{1em}
		\item $[7,4,3]_2$ -- код Хэмминга,
		\item $[2^r, r, 2^{r-1}]$ -- код Адамара,
		\item Код Рида-Соломона над $\FF_{7}$ длины $n = |\FF^*_{7}|$, размерности $k = 4$.
	\end{enumerate}	

	\item Линейный код $C$ задан параметрами $[10, 5, 4]$. Каковы длина и размерность $C^\perp$ -- дуального к $C$ кода?

	\item Линейный код над $\FF_2$ задан проверочной матрицей 	
	\[
	H = \begin{bmatrix}
		0&1&1&1&0\\ 
		1&0&1&0&1
	\end{bmatrix}.
	\]
	Опишите порождающую матрицу для этого кода.
	
	\item Покажите, что нельзя найти 32 бинарных вектора длины 8, таких, что минимальное расстояние между ними как минимум равно 3.
	
	
\end{enumerate}

}
\newpage


\auf{$6$}{%
	Пусть $C-[5,2]$--линейный код над $\FF_2$, заданный проверочной матрицей
	\[
	H = 
	\begin{pmatrix}
	1&0&1&1&1\\
	0&0&1&0&1\\
	0&1&1&1&0\\
	\end{pmatrix}
	\]  
	\begin{enumerate}
		\item Найдите порождающую матрицу кода.
		\item Определите минимальное расстояние кода с помощью проверочной матрицы.
		\item Какое количество ошибок может исправить этот код?
		\item С помощью декодирования по синдрому декодируйте
		\[
		y = [0\; 1\; 0\; 0\; 1]
		\]
		и восстановите исходное сообщение.
		\item Какие вектора $y \in \FF_2^5$ не получится однозначно декодировать этим кодом?
		\item Опишите проверочную и порождающую матрицы кода, дуального к $C$.
	\end{enumerate}
	
}

\newpage\ \newpage
 


\auf{$9$}{%
	Для кода Гоппы, заданного над $\FF_{3^2} \cong \FF_3[x]/(x^2+2x+2)\cong \FF_3(\alpha)$ , c параметрами  $g(x) = x^2+\alpha  x+1$, $L = \{ 1, 2, 2\alpha+2, \alpha, 2\alpha, \alpha+1, 2\alpha+1 \}$, декодируйте \[y = [1 \; 0\; 1\; 0\; 2\; 0\; 1].\] \\
Можете использовать следующие равенства
\begin{equation*}
\begin{aligned}[c]
\frac{1}{x-1} &\equiv (2\alpha+1)x+2\alpha \bmod g(x) \\ 
\frac{1}{x-2} &\equiv (\alpha+2)x+(2\alpha+2) \bmod g(x)   \\
\frac{1}{x-(2\alpha+2)} &\equiv (2\alpha+1)x+(\alpha+2) \bmod g(x)  \\
\frac{1}{x-\alpha} &\equiv \alpha x+2\alpha+2 \bmod g(x)  \\
\frac{1}{x-2\alpha} &\equiv x \bmod g(x) \\
\frac{1}{x-(\alpha+1)} &\equiv (2\alpha+2)x +\alpha \bmod g(x) \\
\frac{1}{x-(2\alpha+1)}& \equiv (\alpha+2) x +\alpha+2 \bmod g(x) \\
\end{aligned}
\end{equation*}
		
Можете также использовать \[\FF_{3^2}^\star = \{ 1, \alpha, \alpha^2 = \alpha+1, \alpha^3 = 2\alpha+1, \alpha^4 = 2, \alpha^5 = 2\alpha, \alpha^6 =2\alpha+2, \alpha^7= \alpha+2  \}.\]
}


\newpage\ 

\end{document}



