\documentclass[11pt]{exam}
%%%%%%%%%%%%%%%%%%%%%%%%%%%%%%%%
%\noprintanswers % pour enlever les réponses
%\printanswers

\unframedsolutions
\SolutionEmphasis{\itshape\small}
\renewcommand{\solutiontitle}{\noindent\textbf{A: }}
%%%%%%%%%%%%%%%%%%%%%%%%%%%%%%%%

\usepackage[T2A]{fontenc}
\usepackage[utf8]{inputenc}
\usepackage[english, russian]{babel}


\usepackage[margin=0.73in]{geometry}
%\usepackage[top=1in, bottom=1in, left=1in, right=1in]{geometry}

%\usepackage{fullpage}


\usepackage{hyperref}
\usepackage{appendix}
\usepackage{enumerate}


\usepackage{times,graphicx,epsfig,amsmath,latexsym,amssymb,verbatim}%,revsymb}
\usepackage{algorithmicx, enumitem, algpseudocode, algorithm, caption}


%%%%%%%%%%%%%%%%%%%%%
% Handling comments and versions %%%
%%%%%%%%%%%%%%%%%%%%%
\newcommand{\extra}[1]{}

\renewcommand{\comment}[1]{\texttt{[#1]}}


%%%%%%%%%%%%%%%%%%%%%%%%%%%
%% THEOREMS
%%%%%%%%%%%%%%%%%%%%%%%%%%%

\usepackage{amsmath,amssymb,amsfonts}
\usepackage{amsthm}

% Landau 
\newcommand{\bigO}{\mathcal{O}}
\newcommand*{\OLandau}{\bigO}
\newcommand*{\WLandau}{\Omega}
\newcommand*{\xOLandau}{\widetilde{\OLandau}}
\newcommand*{\xWLandau}{\widetilde{\WLandau}}
\newcommand*{\TLandau}{\Theta}
\newcommand*{\xTLandau}{\widetilde{\TLandau}}
\newcommand{\smallo}{o} %technically, an omicron
\newcommand{\softO}{\widetilde{\bigO}}
\newcommand{\wLandau}{\omega}
\newcommand{\negl}{\mathrm{negl}} 


\newtheorem{theorem}{Теорема}
\newtheorem{corollary}[theorem]{Следствие}
\newtheorem{lemma}[theorem]{Лемма}
\newtheorem{observation}[theorem]{Observation}
\newtheorem{proposition}[theorem]{Предложение}

\theoremstyle{definition}
\newtheorem{definition}[theorem]{Определение}


\newcommand{\nc}{\newcommand}
\nc{\eps}{\varepsilon}
\nc{\RR}{{{\mathbb R}}}
\nc{\CC}{{{\mathbb C}}}
\nc{\FF}{{{\mathbb F}}}
\nc{\NN}{{{\mathbb N}}}
\nc{\ZZ}{{{\mathbb Z}}}
\nc{\PP}{{{\mathbb P}}}
\nc{\QQ}{{{\mathbb Q}}}
\nc{\UU}{{{\mathbb U}}}
\nc{\OO}{{{\mathbb O}}}
\nc{\EE}{{{\mathbb E}}}

\newcommand{\val}{\operatorname{val}}

\newcommand{\wt}{\ensuremath{\mathit{wt}}}
\newcommand{\Id}{\ensuremath{I}}
\newcommand{\transpose}{\mkern0.7mu^{\mathsf{ t}}}
\newcommand*{\ScProd}[2]{\ensuremath{\langle#1\mathbin{,}#2\rangle}} %Scalar Product
%\newcommand*{\eps}{\ensuremath{\varepsilon}}
\newcommand*{\Sphere}[1]{\ensuremath{\mathsf{S}^{#1}}}

\pretolerance=1000

%%%%%%%%%%%%%%%%%%%%%%%%%%%%%%%%
%%%%%%%%%%%%%%%%%%%%%%%%%%%%%%%%
%% DOCUMENT STARTS
%%%%%%%%%%%%%%%%%%%%%%%%%%%%%%%%
%%%%%%%%%%%%%%%%%%%%%%%%%%%%%%%%
\usepackage{tikz}
\usetikzlibrary{automata}
\DeclareMathOperator{\Vol}{Vol}

\begin{document}
	{\noindent
		\textsc{БФУ им. И. Канта -- Теория кодирования и сжатия информации}
		\hfill {Е. Киршанова //2021\\}
	\hrule
	\begin{center}
		{\Large\textbf{
				\textsc{Практика № 6} \\[5pt] {15.10.20}
		} } 
	\end{center}
	\hrule \vspace{5mm}
	
	\thispagestyle{empty}
	
	\vspace{0.2cm}
	

	
%
%\section{Альтернативное доказательство минимального расстояние кода Рида-Соломона}
%
%		В этом упражнении мы докажем, что $d(\mathsf{RS}_{\FF, S}(n,k)) = n-k+1$.
%		\begin{questions}
%				\question Покажите, что умножение проверочной матрицы $H$ любого линейного кода не меняет минимальное расстояние кода
%				\question Для ненулевых попарно различных $(x_1, \ldots, x_n)$ матрица Вандермонда -- квадратная матрица $n \times n$, определённая
%				\[
%					V = \mathsf{Vand}(x_1, \ldots, x_n) = 
%					\begin{pmatrix}
%					1 & x_1 & x_1^2 & \ldots & x_1^{n01} \\
%					1 & x_2 & x_2^2 & \ldots & x_2^{n-1} \\
%					\vdots & \vdots & \vdots & \ddots & \vdots \\
%					1 & x_n & x_n^2 & \ldots & x_n^{n-1} \\
%					\end{pmatrix}.
%				\]
%				Покажите, что $\det(V) = \prod_{1 \leq i \le j \leq n} (x_j - x_i)$.
%				
%				\textit{Для доказательства можете использовать формулу Лейбница}: $\det(A) = \sum_{\sigma \in S_n} \mathsf{sgn}(\sigma) \prod_{i=1}^n a_{\sigma(i), i}$.
%				
%				\question Проверочная матрица кода Рида-Соломона имеет вид
%				\[
%				H = 
%				\begin{pmatrix}
%				1 & \alpha_1 & \alpha_2 & \ldots & \alpha_{n-1} \\
%				1 & \alpha_1^2 & \alpha_2^2 & \ldots & \alpha_{n-1}^{n-1} \\
%				\vdots & \vdots & \vdots & \ddots & \vdots \\
%				1 & \alpha_1^{n-k} &\alpha_2^{n-k} & \ldots & \alpha_{n-1}^{n-k} \\
%				\end{pmatrix}.
%				\]
%				Используя тот факт, что минимальное расстояние кода есть наибольшее целое $d$, такое что, любые $(d-1)$ столбцы $H$ линейно независимы (см.\ лекцию №2), докажите справедливость равенства $d(\mathsf{RS}_{\FF, S}(n,k)) = n-k+1$.
%		\end{questions}
	
\section{Пример кода Рида-Соломона}
 	Код Рида-Соломона $R_{\FF, S}(n,k)$ размерности $k=4$ определён над $F = GF(3^2) = \FF_3[x] / (x^2+x+2)$. Обозначим $\alpha-$ корень $f(x) = x^2+x+2$ и положим $S = \{1, \alpha, \alpha^2, \ldots, \alpha^7\}$.
 	\begin{questions}
 		\question Каково минимальное расстояние $R_{\FF, S}(n,k)$?
 		\question Закодируйте сообщение $m = [2, 0, \alpha+1, 1]$
 		\question Докажите, что $c =[2, 1, 2\alpha+2, 0, \alpha, \alpha+1, 2\alpha, \alpha+2]$ принадлежит коду
 		\question Восстановите исходное сообщение по полученному слову $c = [\star, 1, \star, 0, \alpha, \star, 2\alpha, \star]$, где $\star$ обозначает, что символ кодового слова был стёрт.
 	\end{questions}
\end{document}