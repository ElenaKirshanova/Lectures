\documentclass[11pt]{exam}
%%%%%%%%%%%%%%%%%%%%%%%%%%%%%%%%
%\noprintanswers % pour enlever les réponses
%\printanswers

\unframedsolutions
\SolutionEmphasis{\itshape\small}
\renewcommand{\solutiontitle}{\noindent\textbf{A: }}
%%%%%%%%%%%%%%%%%%%%%%%%%%%%%%%%

\usepackage[T2A]{fontenc}
\usepackage[utf8]{inputenc}
\usepackage[english, russian]{babel}


%\usepackage[margin=0.73in]{geometry}
%\usepackage[top=1in, bottom=1in, left=1in, right=1in]{geometry}

%\usepackage{fullpage}


\usepackage{hyperref}
\usepackage{appendix}
\usepackage{enumerate}


\usepackage{graphicx,epsfig,amsmath,latexsym,amssymb,verbatim}%,revsymb}
\usepackage{algorithmicx, enumitem, algpseudocode, algorithm, caption}


%%%%%%%%%%%%%%%%%%%%%
% Handling comments and versions %%%
%%%%%%%%%%%%%%%%%%%%%
\newcommand{\extra}[1]{}

\renewcommand{\comment}[1]{\texttt{[#1]}}


%%%%%%%%%%%%%%%%%%%%%%%%%%%
%% THEOREMS
%%%%%%%%%%%%%%%%%%%%%%%%%%%

\usepackage{amsmath,amssymb,amsfonts}
\usepackage{amsthm}

\newtheorem{theorem}{Theorem}[section]
\newtheorem{axiom}[theorem]{Axiom}
\newtheorem{conclusion}[theorem]{Conclusion}
\newtheorem{condition}[theorem]{Condition}
\newtheorem{conjecture}[theorem]{Conjecture}
\newtheorem{corollary}[theorem]{Corollary}
\newtheorem{criterion}[theorem]{Criterion}
\newtheorem{definition}[theorem]{Definition}
\newtheorem{lemma}[theorem]{Lemma}
\newtheorem{notation}[theorem]{Notation}
\newtheorem{proposition}[theorem]{Proposition}


\theoremstyle{definition}
\newtheorem{problem}{Problem}


\newcommand{\nc}{\newcommand}
\nc{\eps}{\varepsilon}
\nc{\RR}{{{\mathbb R}}}
\nc{\CC}{{{\mathbb C}}}
\nc{\FF}{{{\mathbb F}}}
\nc{\NN}{{{\mathbb N}}}
\nc{\ZZ}{{{\mathbb Z}}}
\nc{\PP}{{{\mathbb P}}}
\nc{\QQ}{{{\mathbb Q}}}
\nc{\UU}{{{\mathbb U}}}
\nc{\OO}{{{\mathbb O}}}
\nc{\EE}{{{\mathbb E}}}

\newcommand{\val}{\operatorname{val}}

\newcommand{\wt}{\ensuremath{\mathit{wt}}}
\newcommand{\Id}{\ensuremath{I}}
\newcommand{\transpose}{\mkern0.7mu^{\mathsf{ t}}}
\newcommand*{\ScProd}[2]{\ensuremath{\langle#1\mathbin{,}#2\rangle}} %Scalar Product

\pretolerance=1000

%%%%%%%%%%%%%%%%%%%%%%%%%%%%%%%%
%%%%%%%%%%%%%%%%%%%%%%%%%%%%%%%%
%% DOCUMENT STARTS
%%%%%%%%%%%%%%%%%%%%%%%%%%%%%%%%
%%%%%%%%%%%%%%%%%%%%%%%%%%%%%%%%
\usepackage{tikz}
\usetikzlibrary{automata}
\DeclareMathOperator{\Vol}{Vol}

\begin{document}
	{\noindent
		\textsc{БФУ им. И. Канта -- Теория кодирования и сжатия информации'21}
		\hfill {Е. Киршанова \\}
	\hrule
	\begin{center}
		{\Large\textbf{
				\textsc{Домашнее задание № 1} \\[5pt] {Опубликовано 10.09.21, Срок сдачи: 24.09.21 (23:59)}
		} } 
	\end{center}
	\hrule \vspace{5mm}
	
	\thispagestyle{empty}
	
	\vspace{0.2cm}
	
	\noindent \textsf{Инструкция}
	
	\begin{enumerate}
		\itemsep 2pt
		\item Решения принимаются \emph{исключительно} до дэдлайна, указанного в шапке ДЗ. Время сдачи указано в соответствии с Калининградским часовым поясом.
		\item Решения должны быть выполнены индивидуально.
		\item Решения должны быть написано разборчиво,  все утверждения должны быть аргументированы.
		\item Решения принимаются по электронной почте \textsf{elenakirshanova@gmail.com} с четкой пометкой, кто автор присланных решений.
		\item Все решения должны быть оформлены в \textbf{один} файл формата {pdf}, которые должен иметь адекватное соотношение размер-качество (весь файл должен занимать не более 15Мb).
	\end{enumerate}	
	\section{Код Адамара}
	Бинарный код Адамара, $\mathsf{Had_r}$ -- это  $[2^r, r]_2$- код с порождающей матрицей $r \times 2^r$, столбцы которой представляют всевозможные битовые строки длины $r$. Докажите, что минимальное расстояние кода $\mathsf{Had_r}$  равно $2^{r-1}$.

		
	\section{Линейные коды}
		Пусть $C_1, C_2$ -- линейные коды длины $n$, заданные над $\FF_q$ порождающими матрицами $G_1, G_2$. Определим следующие коды
		\begin{itemize}
			\item $C_3 = C_1 \cup C_2$
			\item $C_4 = C_1 \cap C_2$
			\item $C_5 = C_1 + C_2 = \{  c_1 + c_2 \, : \, c_1 \in C_1, c_2 \in C_2 \}$
			\item $C_6 = \{  (c_1 \, |\, c_2 ) \, : \, c_1 \in C_1, c_2 \in C_2 \}$, где $(\cdot|\cdot)$ обозначает конкатинацию слов. 
		\end{itemize}
		Для $i = 1, \ldots 6$ обозначим за $k_i$ -- размерность кода $\log_q |C_i|$, а за $d_i$-- минимальное расстояние кода $C_i$. Положим $k_1, k_2 >0$.
	 	\begin{questions}
	 		\question Докажите, что $C_3$ -- линейный тогда и только тогда, когда либо $C_1 \subseteq C_2$, либо $C_2 \subseteq C_1$.
	 		\question Докажите, что коды $C_4, C_5, C_6$-- линейные
	 		\question Докажите, что если $k_4>0$, то $d_4 \geq \max\{d_1, d_2\}$
	 		\question Докажите, что $k_5 \leq k_1 + k_2$, и что равенство достигается тогда и только тогда, когда $k_4 = 0$
	 		\question Докажите, что $d_5 \leq \min\{ d_1, d_2\}$
	 		\question Докажите, что 
	 		\[
	 			\begin{pmatrix}
	 			G_1 & 0 \\
	 			0 & G_2 \\
	 			\end{pmatrix}
	 		\]
	 		является порождающей матрицей для $C_6$, а следовательно, $k_6 = k_1 + k_2$
	 		\question Докажите, что $d_6 = \min\{d_1, d_2\}$.
	 	\end{questions}
 	
\section{Задача о заключенных}

В тюрьме сидят семеро заключенных (в оригинальной версии задачи была темница и Минотавр, но мы цивилизованные люди). На всех заключенных надевают по одной шляпе либо красного, либо синего цвета так, что каждый заключенный видит цвета шляп других заключенных, но не свой.

В фиксированное время все заключенные декларируют свой выбор в надежде угадать цвет своей шляпы: либо `красный', либо `черный', либо `пасс'.  Заключенные не имеют права передавать друг другу какую-либо информацию ни в какое время.

Если все заключенные сказали `пасс', они все остаются в тюрьме. Если хотя бы один из них не угадал свой цвет, они также все остаются в тюрьме. Если все заключенные, кто угадывал свой цвет, корректно его угадал (таковых должно быть как минимум 1), они все выходят на свободу.

Предложите и обоснуйте стратегию, в которой заключенные имеют вероятность $7/8$ выйти из тюрьмы.
Объясните, как ваша стратегия связано с кодом Хэмминга $[7,4,3]$. Если бы заключенных было не 7, а 15, имели бы они бОльший успех на положительный для них исход?
 	
% название задачи ``The Hat problem'' (solution and description \url{https://web.njit.edu/~wguo/Hat%20Probelm.pdf}, see also \url{https://www.cs.cmu.edu/puzzle/solution15.pdf}) 	
 	
% \section{Задание на программирование: расширенный код Хэмминга [15,11,3]}
% 
% \begin{enumerate}
% 	\item Кода Хэмминга  [15,11,3] -- обобщение изученного на лекции кода Хэмминга [7,4,3]. Отличное описание этого кода можно найти на \\
% 	\url{https://www.youtube.com/watch?v=X8jsijhllIA} \\
% 	\url{https://www.youtube.com/watch?v=b3NxrZOu_CE\&t=472s}
% 	
% 	\item Задача: реализовать алгоритм кодировать и декодирования для кода Хэмминга [15,11,3]. Названия функций и doctests можно найти по ссылке \\
% 	https://crypto-kantiana.com/elena.kirshanova/teaching/coding\_2020/DM1\_template.py
% \end{enumerate}
 	
\end{document}