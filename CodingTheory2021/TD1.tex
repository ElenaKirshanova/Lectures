\documentclass[11pt]{exam}
%%%%%%%%%%%%%%%%%%%%%%%%%%%%%%%%
%\noprintanswers % pour enlever les réponses
%\printanswers

\unframedsolutions
\SolutionEmphasis{\itshape\small}
\renewcommand{\solutiontitle}{\noindent\textbf{A: }}
%%%%%%%%%%%%%%%%%%%%%%%%%%%%%%%%

\usepackage[T2A]{fontenc}
\usepackage[utf8]{inputenc}
\usepackage[english, russian]{babel}


\usepackage[margin=0.73in]{geometry}
%\usepackage[top=1in, bottom=1in, left=1in, right=1in]{geometry}

%\usepackage{fullpage}


\usepackage{hyperref}
\usepackage{appendix}
\usepackage{enumerate}


\usepackage{times,graphicx,epsfig,amsmath,latexsym,amssymb,verbatim}%,revsymb}
\usepackage{algorithmicx, enumitem, algpseudocode, algorithm, caption}


%%%%%%%%%%%%%%%%%%%%%
% Handling comments and versions %%%
%%%%%%%%%%%%%%%%%%%%%
\newcommand{\extra}[1]{}

\renewcommand{\comment}[1]{\texttt{[#1]}}


%%%%%%%%%%%%%%%%%%%%%%%%%%%
%% THEOREMS
%%%%%%%%%%%%%%%%%%%%%%%%%%%

\usepackage{amsmath,amssymb,amsfonts}
\usepackage{amsthm}

\newtheorem{theorem}{Theorem}[section]
\newtheorem{axiom}[theorem]{Axiom}
\newtheorem{conclusion}[theorem]{Conclusion}
\newtheorem{condition}[theorem]{Condition}
\newtheorem{conjecture}[theorem]{Conjecture}
\newtheorem{corollary}[theorem]{Corollary}
\newtheorem{criterion}[theorem]{Criterion}
\newtheorem{definition}[theorem]{Definition}
\newtheorem{lemma}[theorem]{Lemma}
\newtheorem{notation}[theorem]{Notation}
\newtheorem{proposition}[theorem]{Proposition}


\theoremstyle{definition}
\newtheorem{problem}{Problem}


\newcommand{\nc}{\newcommand}
\nc{\eps}{\varepsilon}
\nc{\RR}{{{\mathbb R}}}
\nc{\CC}{{{\mathbb C}}}
\nc{\FF}{{{\mathbb F}}}
\nc{\NN}{{{\mathbb N}}}
\nc{\ZZ}{{{\mathbb Z}}}
\nc{\PP}{{{\mathbb P}}}
\nc{\QQ}{{{\mathbb Q}}}
\nc{\UU}{{{\mathbb U}}}
\nc{\OO}{{{\mathbb O}}}
\nc{\EE}{{{\mathbb E}}}

\newcommand{\val}{\operatorname{val}}

\newcommand{\wt}{\ensuremath{\mathit{wt}}}
\newcommand{\Id}{\ensuremath{I}}
\newcommand{\transpose}{\mkern0.7mu^{\mathsf{ t}}}
\newcommand*{\ScProd}[2]{\ensuremath{\langle#1\mathbin{,}#2\rangle}} %Scalar Product

\pretolerance=1000

%%%%%%%%%%%%%%%%%%%%%%%%%%%%%%%%
%%%%%%%%%%%%%%%%%%%%%%%%%%%%%%%%
%% DOCUMENT STARTS
%%%%%%%%%%%%%%%%%%%%%%%%%%%%%%%%
%%%%%%%%%%%%%%%%%%%%%%%%%%%%%%%%
\usepackage{tikz}
\usetikzlibrary{automata}
\DeclareMathOperator{\Vol}{Vol}

\begin{document}
	{\noindent
		\textsc{БФУ им. И. Канта -- Теория кодирования и сжатия информации}
		\hfill {Е. Киршанова // 2021\\}
	\hrule
	\begin{center}
		{\Large\textbf{
				\textsc{Практика № 1} \\[5pt] {03.09.20}
		} } 
	\end{center}
	\hrule \vspace{5mm}
	
	\thispagestyle{empty}
	
	\vspace{0.2cm}
	\section{Порождающие матрицы кодов}
		Напишите порождающие и проверочные матрицы для $[n, n-1, 2]_2$  -- кода проверки на четность и для $[n, 1,  n]_2$ кода с повторением.
		
	\section{Минимальное расстояние кода}
		Покажите, что минимальное расстояние любого линейного кода $C$ равно минимальному весу Хэмминга ненулевого слова в $C$, т.е.,
		\[
			\Delta(C) = \min_{c \in C, c \neq 0} \wt(c).
		\]
		
	\section{Систематическая форма}
		Пусть $G = [\Id_k | A] \in \FF_q^{k \times n}  $ -  порождающая матрица $[n,k]_q-$кода $C$ в систематической форме, где $\Id_k-$ единичная матрица $k \times k$, $A \in \FF_q^{k \times n-k}$. Опишите проверочную матрицу для $C$.
		
	\section{Дуальный код}
	На лекции дуальный код для кода $C$ был определен как
	\[
	C^\perp = \{ x \in \FF_q \; : \; \ScProd{x}{c} = 0  \,\forall c \in C \}.
	\]
	\begin{questions}
		
		\question Пусть $G$ - порождающая матрица $C$. Докажите эквивалентность второго определения:
		\[
		 C^\perp = \{ x \in \FF_q \; : \; xG\transpose = 0 \}.
		\]
		Вывод: $G$ -- проверочная матрица $C^\perp$, $C^\perp$-- линейный $[n, n-k]_q$--код для $C$-- линейного $[n, k]_q$ -- кода.
		\question Эквивалентное утверждение: любая образующая матрица $H$ дуального кода $C^\perp$ является проверочной матрицей кода $C$. Вывод: $(C^\perp)^\perp = C$.
	\question Постройте код, дуальный к $[n, 1,  n]_2$ коду с повторением.
	\end{questions}

	\section{Количество порождающих матриц}
	Покажите, что для $[n,k]_q$-- линейного кода $C$ ($q$ -- простое), количество различных порождающих матриц равно
	\[
		\prod_{i=0}^{k-1}(q^k - q^i).
	\]
	 
\end{document}