\documentclass[11pt]{exam}
%%%%%%%%%%%%%%%%%%%%%%%%%%%%%%%%
%\noprintanswers % pour enlever les réponses
%\printanswers

\unframedsolutions
\SolutionEmphasis{\itshape\small}
\renewcommand{\solutiontitle}{\noindent\textbf{A: }}
%%%%%%%%%%%%%%%%%%%%%%%%%%%%%%%%

\usepackage[T2A]{fontenc}
\usepackage[utf8]{inputenc}
\usepackage[english, russian]{babel}


\usepackage[margin=0.73in]{geometry}
%\usepackage[top=1in, bottom=1in, left=1in, right=1in]{geometry}

%\usepackage{fullpage}


\usepackage{hyperref}
\usepackage{appendix}
\usepackage{enumerate}


\usepackage{times,graphicx,epsfig,amsmath,latexsym,amssymb,verbatim}%,revsymb}
\usepackage{algorithmicx, enumitem, algpseudocode, algorithm, caption}


%%%%%%%%%%%%%%%%%%%%%
% Handling comments and versions %%%
%%%%%%%%%%%%%%%%%%%%%
\newcommand{\extra}[1]{}

\renewcommand{\comment}[1]{\texttt{[#1]}}


%%%%%%%%%%%%%%%%%%%%%%%%%%%
%% THEOREMS
%%%%%%%%%%%%%%%%%%%%%%%%%%%

\usepackage{amsmath,amssymb,amsfonts}
\usepackage{amsthm}

% Landau 
\newcommand{\bigO}{\mathcal{O}}
\newcommand*{\OLandau}{\bigO}
\newcommand*{\WLandau}{\Omega}
\newcommand*{\xOLandau}{\widetilde{\OLandau}}
\newcommand*{\xWLandau}{\widetilde{\WLandau}}
\newcommand*{\TLandau}{\Theta}
\newcommand*{\xTLandau}{\widetilde{\TLandau}}
\newcommand{\smallo}{o} %technically, an omicron
\newcommand{\softO}{\widetilde{\bigO}}
\newcommand{\wLandau}{\omega}
\newcommand{\negl}{\mathrm{negl}} 


\newtheorem{theorem}{Теорема}
\newtheorem{corollary}[theorem]{Следствие}
\newtheorem{lemma}[theorem]{Лемма}
\newtheorem{observation}[theorem]{Observation}
\newtheorem{proposition}[theorem]{Предложение}

\theoremstyle{definition}
\newtheorem{definition}[theorem]{Определение}


\newcommand{\nc}{\newcommand}
\nc{\eps}{\varepsilon}
\nc{\RR}{{{\mathbb R}}}
\nc{\CC}{{{\mathbb C}}}
\nc{\FF}{{{\mathbb F}}}
\nc{\NN}{{{\mathbb N}}}
\nc{\ZZ}{{{\mathbb Z}}}
\nc{\PP}{{{\mathbb P}}}
\nc{\QQ}{{{\mathbb Q}}}
\nc{\UU}{{{\mathbb U}}}
\nc{\OO}{{{\mathbb O}}}
\nc{\EE}{{{\mathbb E}}}

\newcommand{\val}{\operatorname{val}}

\newcommand{\wt}{\ensuremath{\mathit{wt}}}
\newcommand{\Id}{\ensuremath{I}}
\newcommand{\transpose}{\mkern0.7mu^{\mathsf{ t}}}
\newcommand*{\ScProd}[2]{\ensuremath{\langle#1\mathbin{,}#2\rangle}} %Scalar Product
%\newcommand*{\eps}{\ensuremath{\varepsilon}}
\newcommand*{\Sphere}[1]{\ensuremath{\mathsf{S}^{#1}}}

\pretolerance=1000

%%%%%%%%%%%%%%%%%%%%%%%%%%%%%%%%
%%%%%%%%%%%%%%%%%%%%%%%%%%%%%%%%
%% DOCUMENT STARTS
%%%%%%%%%%%%%%%%%%%%%%%%%%%%%%%%
%%%%%%%%%%%%%%%%%%%%%%%%%%%%%%%%
\usepackage{tikz}
\usetikzlibrary{automata}
\DeclareMathOperator{\Vol}{Vol}

\begin{document}
	{\noindent
		\textsc{БФУ им. И. Канта -- Теория кодирования и сжатия информации}
		\hfill {Е. Киршанова // 2021\\}
	\hrule
	\begin{center}
		{\Large\textbf{
				\textsc{Практика № 5} \\[5pt] {8.10.21}
		} } 
	\end{center}
	\hrule \vspace{5mm}
	
	\thispagestyle{empty}
	
	\vspace{0.2cm}
	

	

\section{Лемма из лекции}
	Докажите Лемму~1 из лекции: для любого линейного бинарного кода $C$ справедливо
	\[
		\sum_{c \in C} (-1)^{\alpha c} = \begin{cases}
		|C|, & \alpha \in C^\perp \\
		0, & \text{иначе}.
		\end{cases}
	\]
	
%\section{Самодуальный код}
%	Докажите, что если $C = C^\perp$ (т.е.\ $C$ -- самодуальный), то $C$ содержит только слова чётного веса

\section{Обобщенный код Хэмминга}
Напомним, что обобщённый код Хэмминга  $\textsc{Ham}_r$ с параметрами $[2^r - 1, 2^r - r -1, 3]_2$ задаётся проверочной матрицей $r \times 2^r - 1$, столбцами которой являются все ненулевые строки длины $r$. Пусть $W_i$ -- количество кодовых слов веса $i$ в $\textsc{Ham}_r$.
\begin{questions}
	\question Пусть $c \in \textsc{Ham}_r$-- кодовое слово веса  $t$. Для каждого из следующих $i$, найдите число слов из $\{0,1\}^n$ веса $i$, которые будут декодированы к $c$:
	\begin{enumerate}
		\item $i = t-1$
		\item $i = t+1$
		\item $i = t$.
	\end{enumerate}
	\question Докажите, что
	\[
	(i+1) W_{i+1} + W_i + (n-i+1)W_{i-1} = \binom{n}{i},
	\]
	где $W_1 = 0, W_0 = 1$.
	\question Вычислите $W_3$.
\end{questions}
 	
\section{Минимальное расстояние совершенного кода}
	Докажите, что минимальное расстояние совершенного кода обязательно нечётно. \textit{Напоминание:} $q$-арный код $C$  длины $n$ называется совершенным, если $|C| = \frac{q^n}{ \sum_{i = 0}^{\lfloor (d-1)/2 \rfloor} \binom{n}{i} (q-1)^i} $.
	

\end{document}