% !TeX program = xelatex
% !BIB TS-program = biber

\documentclass{beamer}

\usepackage[utf8]{inputenc}
\usepackage[russian]{babel}

%---tikz----
\usepackage{tikz}
\usetikzlibrary{arrows, chains, matrix, positioning, scopes, patterns, shapes}
\usepackage{pgfplots, subfigure}
\usepackage{extarrows}

\usepackage[backend=biber,firstinits=true,hyperref=true,style=numeric-comp]{biblatex}

\usepackage{../beamerthemeec2020}
{\footnotesize\bibliography{../biblio}}

\title{Эллиптические кривые}
\subtitle{Лекция 4. Алгоритм вычисления $E[n]$}
\author{Семён Новосёлов}
\institute{БФУ им. И. Канта}
\date{2021}

\begin{document}

\frame{\titlepage}

\begin{frame}{Поле определения $E[n]$}
$E$ -- эллиптическая кривая над полем $K = \mathbb{F}_q$, $\operatorname{char}{K} \ne 2,3$.
%\begin{itemize}
    %\item 
    \structure{Точки $n$-кручения:} $E[n] = \left\{ {P \in E\left( \bar{K} \right) : nP = \mathcal{O}} \right\}$.

%    \structure{В случае $p \nmid n$:}
    В случае $p \nmid n$:
    \[
    E\left[ n \right] \simeq \mathbb{Z}/n\mathbb{Z} \times \mathbb{Z}/n\mathbb{Z}
    \]
    \[\Downarrow\]
    \[
    E\left[n\right] = \left\{ {\mathcal{O},\;\left( {{x_1},\;{y_1}} \right), ... \left( {{x_m},\;{y_m}} \right)} \right\},
    \]
    где $m=n^2-1$.
    \[\Downarrow\]
    Поле, в котором лежит $E\left[ n \right]$ (расширение
    $K$), можно записать как
    \[
    K_{E,n} := K\left( {{x_1}, {y_1}, \ldots, x_m, y_m} \right)
    \]
    \[
    \left[ K_{E,n} : K \right] = d < \infty 
    \]
%\end{itemize}
\end{frame}

\begin{frame}{Мотивация}
    \structure{Зачем нужно находить $E[n]$?}
    \begin{enumerate}
        \item нахождение точек из~$E[n]$ "--- часть полиномиальных алгоритмов вычисления $\#E(\mathbb{F}_q)$.
        \item $d = [K_{E,n} : K]$ -- степень вложения, $K_{E,n}$ -- поле определения спаривания Вейля $e_n: E[n] \times E[n] \mapsto \mu_r$.
        \begin{center}
            \structure{$\Downarrow$}
        \end{center}
        %\begin{itemize}
            %\item
            сложность \structure{DLOG} в $E(\mathbb{F}_q)$ \structure{$\rightleftarrows$} сложность \structure{DLOG} в $\mathbb{F}_{q^{d}}$
        %\end{itemize}
    \end{enumerate}
\end{frame}

\begin{frame}{Многочлены деления}
    \structure{Как вычислить $E[n]$?}
    
    Рассмотрим метод на основе факторизации многочленов деления (из лекции \textnumero~$3$):
    
    \begin{itemize}
        \item $\psi_m \in \mathbb{Z}\left[x,y,A,B\right]$
        \item $\varphi_m = x \cdot \psi_m^2 - \psi_{m + 1} \psi_{m - 1}$
        
        \item $\omega_m = \dfrac{1}{{4y}}\left( {{\psi_{m + 2}}\psi_{m - 1}^2 - {\psi_{m - 2}}\psi_{m + 1}^2} \right)$
    \end{itemize}
    Сложение точки $P$ с самой собой $n$ раз:
    \begin{equation*}
        nP = \left( {\frac{{{\varphi _n}(x)}}{{\psi _n^2(x)}},\;\frac{{{\omega _n}( x )}}{{\psi _n^3(x,y)}}} \right).
    \end{equation*}
\end{frame}

\begin{frame}{Нахождение $E[n]$}
\begin{lemma}
Многочлены ${\varphi _n}$ и $\psi _n^2 \in K\left[ x \right]$ -- взаимно просты, если $\Delta (E) \ne 0$. Т.е. для $E$ -- эллиптической кривой, ${\phi _n}, \psi _n^2$ -- взаимно просты.
\end{lemma}
\structure{$\triangleleft$} Доказательство: [Lang, II 2.3].\structure{$\triangleright$}

\begin{corollary}
Пусть $P = (x,y) \in E(\bar{K})$. Тогда $n P = \mathcal{O} \Leftrightarrow \psi_n^2(x) = 0$.
\end{corollary}
\end{frame}

\begin{frame}{Нахождение $E[n]$}
    \begin{center}
       $\psi_n^2(x) = {n^2}{x^{{n^2} - 1}} +  \ldots$ \hfill {\small(Washington, \S3.2)}
    \end{center}

Факторизуем $\psi_n$ в $\mathbb{F}_q[x]$.
\[\psi_n = f_1 \cdot \ldots \cdot f_r,\]
где $f_r$ -- неприводимые над $\mathbb{F}_q$.
\begin{block}{Замечание}
\begin{itemize}
    \item все $f_i$ различны
    \item в $E[n]$ всего ${n^2} - 1$ точек $\neq \mathcal{O}$
    \item $\forall P_i \in E[n]$ имеем $-P_i \in E[n]$
    \item $\deg {\psi _n}( x ) = \frac{{{n^2} - 1}}{2} \Rightarrow {\psi _n}( x )$ имеет $\frac{{{n^2} - 1}}{2}$ корней в $\overline{\mathbb{F}}_q$ и каждый корень кратности 1 (иначе мы имели бы меньше чем ${n^2} - 1$ точек $ \ne \mathcal{O}$ в $E[n]$). 
\end{itemize}
\end{block}
\end{frame}

\begin{frame}{Определение степени вложения $d$}
Определим $d = [K_{E,n} : K] \ne 0$ из разложения $\psi_n$ над $\mathbb{F}_q$: 
\[\psi_n = f_1 \cdot \ldots \cdot f_r\]
\begin{block}{Теорема}
Пусть $n$ -- простое $> 2$, $K = \mathbb{F}_q$, $n \ne \operatorname{char}(K)$,
$d_i = \deg {f_i}$,
$\ell = \mathrm{lcm}(d_1, \ldots, d_r)$.
Пусть $K'_{E,n} = K( {{x_1}, \ldots, {x_{{n^2} - 1}}})$, где ${x_i}$ -- $x$-координаты точек $n$-кручения. Тогда 
    \[[ {K'_{E,n}:\:K} ] = \ell.
    \]
    Кроме того, $[ {{K_{E,n}}:\:K'_{E,n}} ] = 1$, либо $2$. То есть $d = \ell$ либо $2\ell$. 
\end{block}

\structure{$\triangleleft$}
%\begin{itemize}
    %\item
    $\exists x_i$ т.ч. $y_i = \sqrt{x_i^3 + a x_i + b} \not\in K'_{E,n} = \mathbb{F}_{q^\ell}$ \structure{$\implies$} $d = 2 \ell$.
    %\item
    В противном случае: $d = \ell$.
%\end{itemize}
\structure{$\triangleright$}
\end{frame}

\begin{frame}{Обобщенный символ Лежандра}
\begin{definition}
    $K = \mathbb{F}_q$, $x \in K$. Квадратичный характер $\left( \frac{ \cdot }{K} \right)$ -- это 
    $$
    \left( \frac{x}{K} \right ) =
    \begin{cases}
        1,& \exists y \in K:\:{y^2} = x \\
        -1,& \nexists y \in K:\:{y^2} = x \\
        0, & x = 0.
    \end{cases}
    $$
\end{definition}
\begin{center}
\structure{$\Downarrow$}
\end{center}
чтобы определить $d = \ell$ или $d = 2\ell$, необходимо вычислить 
\[
\left(  {\frac{x_i^3 + a{x_i} + b}{\mathbb{F}_{q^\ell}}} \right ),
\]
$\forall x_i$ -- корней $\psi_n$. 
\end{frame}

\begin{frame}
\begin{lemma}
Если $K = \mathbb{F}_q$ и $d = [K_{E,n} : K]$, то $q^d \equiv 1 \bmod{n}$. В частности, $\operatorname{ord}(q,n) | d$.
\end{lemma}

\vspace*{1em}
\begin{block}{Замечание}
Так как $DLOG$ на $E(\mathbb{F}_q)$ сводится в $\mathbb{F}_{q^d}$ для проверки безопасности достаточно проверить, что $q^d \not\equiv 1 \bmod{n}$ для $d \leq 1000$ (требование ГОСТ и др.).
\end{block}
\end{frame}

\begin{frame}%{title}
\begin{block}{Лемма (van Tuyl)}
    Пусть $f_i$~---~неприводимый многочлен в разложении ${\psi _n}$, т.ч. $2 d_i \nmid \ell$, $d_i = \deg f_i$. Положим
    \[
    {d^*} = \mathrm{lcm}( {\operatorname{ord} ( {q,n} ),\;{d_i}} ),
    \]
    \[
    c = \left(\frac{x_i^3 + a x + b}{\mathbb{F}_{q^{d_i}}} \right),{\text{ где }} f_i( x_i ) = 0.
    \]
    Тогда
    $$
    d = 
    \begin{cases}
        \ell, &\text{ если } c=1 \text{ и } d^*| \ell  \\
        2\ell & \text{ иначе. }
    \end{cases}
    $$
\end{block}
\begin{itemize}
    \item Лемма позволяет рассмотреть лишь один $f_i$ (и его корень $x_i$) для определения $d$. 
\end{itemize}
\end{frame}

\begin{frame}{Алгоритм вычисления $d = [K_{E,n}:K]$}
    \structure{Вход:} $ E/\mathbb{F}_q: y^2 = x^3 + a x + b$, $n \geqslant 3$ -- нечётное.
    
    \structure{Выход:} $d$ т.ч. $E[n] \subseteq E(\mathbb{F}_{q^d})$.
    
    \begin{enumerate}
        \item Построить $\psi_n \in \mathbb{F}_q[x]$
        \item Факторизовать $\psi_n = {f_1} \cdot \ldots \cdot {f_r}$
        \item $\ell := \operatorname{lcm}(\deg{f_1}, \ldots, \deg{f_r})$
        \item Выбрать $f_i$ т.ч. $2 \cdot \deg {f_i} \nmid \ell$
        \item Вычислить $c = \left(\dfrac{x_i^3 + a {x_i} + b}{\mathbb{F}_{q^{d_i}}}\right)$, где $x_i$ -- корень $f_i$.
        
        \item if $c = - 1$:
        
        \quad return $d = 2\ell$
        
        \item ${d^*} = \operatorname{lcm}( \operatorname{ord} ({q,n}),{d_i})$
        
        if ${d^*} = \ell$ or $\ell = n \cdot {d^*}$:
        
        \quad return $d = \ell$
        
        return $d = 2\ell$ 
    \end{enumerate}
\end{frame}

\begin{frame}{}
    \begin{itemize}
        \item Алгоритм может быть адаптирован для вычисления самой группы точек $n$-кручения $E[ n]$, если для ${x_i}$ -- корня ${f_i}$, вычислять соответствующие ${y_i}$
        \item для $n = 2, E[n]$ вычисляется разложением многочлена ${x^3} + a x + b$ (см. лекцию~\#~3)
        \item для $n = 1, E[n] = \{ \mathcal{O} \}$. 
    \end{itemize}
\end{frame}

\begin{frame}{Оценка сложности}
    \begin{itemize}
        \item Шаг $1$. $\deg \psi_n = \frac{n^2 - 1}{2}$. Грубо: $\operatorname{poly}(n)$. \\
        \item Шаг $2$.
        %Berlekamp-Zassenhaus: $O( (\deg \psi_n)^3 + (\deg \psi_n)^2 \cdot \operatorname{lg}(n^2) \cdot \operatorname{lg}q  )$.
        %
        Факторизация многочлена [MCA, Th. 14.14]:
        $\widetilde{O}((\deg{\psi_n})^2 \log^2{q})$.
        
        \item %Шаг $5$. Наивно (baby step / giant step): $O(\sqrt{q^{d_i}})$. Можно свести к вычислению символа Лежандра над $\mathbb{F}_q$.
        Шаг~$5$. Обобщённый символ Лежандра (CohenFrey+'05, Alg. 11.69): $poly(n)$ (грубо).
        \item Шаг $7$. Сводится к факторизации $n-1$. Время: $L_n(1/3)$.
    \end{itemize}
    \structure{Итого:}
    %$\operatorname{poly}(n) + O(\operatorname{poly}(n) + \sqrt{q^{d_i}})$
    $\operatorname{poly}(n)\log{q}$ операций в $\mathbb{F}_q$.
\end{frame}

%\begin{frame}{Дополнение. Конструктивные приложения билининейных спариваний}
%    content...
%\end{frame}

\begin{frame}{Литература}
    %\nocite{Menezes1993}
    %\nocite{Blake1999}
    \nocite{Washington2008}
    \nocite{Lang1978}
    \nocite{vanTuyl1997}
    \nocite{GathenJurgen2013}
    \nocite{CohenFrey+2005}
    \printbibliography
    
\begin{center}
    \begin{tcolorbox}[enhanced,hbox,colback=block-green-color-bg,colframe=subsection-color!120,title=Контакты,center title]
        \begin{varwidth}{\textwidth}
            \begin{center}
                \href{mailto:snovoselov@kantiana.ru}{snovoselov@kantiana.ru}
            \end{center}
        \end{varwidth}
    \end{tcolorbox}	
\end{center}

\structure{Страница курса:}\\
{\footnotesize
    \href{https://crypto-kantiana.com/semyon.novoselov/teaching/elliptic_curves_2021}{crypto-kantiana.com/semyon.novoselov/teaching/elliptic\_curves\_2021}
}
\end{frame}

\end{document}