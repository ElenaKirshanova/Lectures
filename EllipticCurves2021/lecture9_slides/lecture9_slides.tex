% !TeX program = xelatex
% !BIB TS-program = biber

\documentclass{beamer}

\usepackage[utf8]{inputenc}
\usepackage[russian]{babel}

%---tikz----
\usepackage{tikz}
\usetikzlibrary{arrows, chains, matrix, positioning, scopes, patterns, shapes}
\usepackage{pgfplots, subfigure}
\usepackage{extarrows}

\usepackage[backend=biber,firstinits=true,hyperref=true,style=numeric-comp]{biblatex}

\usepackage{../beamerthemeec2020}
{\footnotesize\bibliography{../biblio}}

\title{Эллиптические кривые}
\subtitle{Лекция 9. Выбор кривой для криптографии}
\author{Семён Новосёлов}
\institute{БФУ им. И. Канта}
\date{2021}

\begin{document}

\frame{\titlepage}

%\begin{frame}{Мотивация}
%\begin{itemize}
%    \item
%\end{itemize}
%\end{frame}

\begin{frame}{Как выбрать кривую, подходящую для криптографии?}
\structure{Требования:}
\begin{enumerate}
    \item Безопасность: Для заданного параметра безопасности $\lambda$ сложность наилучшей известной атаки должна быть $\approx 2^\lambda$. На данный момент $\lambda \approx 128$.
    \item Эффективность: групповой закон должен вычисляться быстро.
\end{enumerate}
%Рассмотреть все известные атаки на выявить условия применимости и отсечь все слабые кривые.
%    \begin{enumerate}
%        \item DLOG
%        \item Криптография на спариваниях
%        \item Криптография на изогениях 
%    \end{enumerate}
\end{frame}

\begin{frame}{I. Безопасность}
\[
E/\mathbb{F}_q: y^2 + a_1 x y + a_3 y = x^3 + a_2 x^2 + a_4 x + a_6
\]
\begin{itemize}
    \item $N = \#E(\mathbb{F}_q)$, вычисляем с помощью SEA за $O(\log^4{q})$
    \item $N = O(q)$ (граница Хассе-Вейля)
    \item \structure{DLOG:}
    $
    Q = [\ell] P, \quad (P, Q) \mapsto \ell
    $
    \item $G = \left<P\right>$ для $P \in E(\mathbb{F}_q)$
    \item для эффективности: $\#G = \operatorname{ord}{P} \approx \#E(\mathbb{F}_q)$
\end{itemize}
Рассмотрим обзорно атаки и соответствующие им ограничения на~$(N,q,\ell)$.
\end{frame}

\begin{frame}{Атака BSGS или $\rho-$методом Полларда}
    %\begin{itemize}
        %\item 
        Адаптируем алгоритм BSGS нахождения порядка точки из лекции по подсчёту точек.
    %\end{itemize} 
    
    \structure{Сложность:} $\widetilde{O}(\sqrt{\#G}) = \widetilde{O}(\sqrt{q})$ по времени и по памяти.
    \begin{itemize}
        \item $\rho$-метод Полларда позволяет снизить сложность по памяти до $O(\operatorname{polylog} q)$.
    \end{itemize}
    \structure{Вывод:} для уровня безопасности $\lambda = 128$ требуется кривая с подгруппой $G$ порядка $\approx 2^{256}$. Т.е. над полем $\mathbb{F}_q$ размера $q \approx 2^{256}$.
\end{frame}

\begin{frame}{Атака Полига-Хеллмана}
    \structure{Идея:} решить задачу DLOG в подгруппах $G$ с помощью $\rho$-метода Полларда и восстановить искомый DLOG в $G$ по КТО.
    \begin{center}
        $\#G = p_1^{e_1} \cdot \ldots \cdot p_m^{e_m}$ \structure{$\implies$} $G \simeq \mathbb{Z} / p_1^{e_1} \mathbb{Z} \oplus \ldots \oplus \mathbb{Z} / p_m^{e_m} \mathbb{Z}$
    \end{center}
    
    %\begin{itemize}
        %\item 
        т.е. $G \simeq G_1 \oplus ... \oplus G_m$, где $\#G_1 = p_1^{e_1}, \ldots, \#G = p_m^{e_m}$.
    %\end{itemize}
    
    \structure{Сложность:} $\widetilde{O}(\sum e_i (\log{\#G} + \sqrt{p_i}))$
    
    \begin{center}
        \structure{$\Downarrow$}
    \end{center}
    
    \structure{Вывод:} для безопасности $\#G = c r$, где $r$ -- большое простое число, $c$ -- малое число.
\end{frame}

\begin{frame}
    Комбинируя условия двух атак получаем, что группа точек кривой должна как минимум:
\begin{itemize}
    \item содержать подгруппу \structure{простого} порядка размера $256$-бит для уровня безопасности $128$-бит.
    \item соответственно, размер поля $q \approx 2^{256}$.
\end{itemize} 
\end{frame}

\begin{frame}{Атака спуском Вейля}
    \begin{itemize}
        \item В случае $q = p^n$ можно определить ограничение Вейля:
        \[
        W/\mathbb{F}_p := W_{\mathbb{F}_{p^n} / \mathbb{F}_p}(\mathbb{F}_p) = E(\mathbb{F}_{p^n}).
        \]
        \item Это абелево многообразие размерности $n$, т.е. проективное многообразие, обладающее структурой группы.
        \item Поэтому $DLOG$ на $E/\mathbb{F}_{p^n}$ можно свести к $W/\mathbb{F}_p$.
        %\item Если $W \subseteq Jac_D$ для некоторой кривой $D/\mathbb{F}_p$ рода $g \geq n$, то изменение сложности $DLOG$:
        %\begin{itemize}
        %    \item для $g = O()$, алгоритм [Enge–Gaudry'2002]
        %    \item $g = 2, n = 2$, $D$ -- гиперэллиптическая:
        %    \begin{center}
        %    $\widetilde{O}(p)$ \structure{$\implies$}
        %    \end{center}
        %\end{itemize}
    \end{itemize}
\end{frame}

\begin{frame}
Если $W \subseteq Jac_D$ для некоторой кривой $D/\mathbb{F}_p$ рода $g \geq n$, то получаем изменение сложности $DLOG$:
\begin{itemize}
    \item $g \geq \log_g{p}$, $D$ -- гиперэллиптическая,% алгоритм [Enge–Gaudry'2002] для DLOG в $Jac_D(\mathbb{F}_p)$:
    \begin{center}
      $\widetilde{O}(p^{n/2})$ (Pollard) \structure{$\implies$} $L_{p^g}(1/2, 2.732)$ (Enge–Gaudry)
    \end{center}
    получаем переход к субэкспоненциальной сложности при $g \approx n$.
    \item $g < \log_g{p}$, $D$ -- гиперэллиптическая,
    \begin{center}
        $\widetilde{O}(p^{n/2})$ (Pollard) \structure{$\implies$} $\widetilde{O}(p^{2-2/g})$ (GTTD).
    \end{center}
    Например, при $n = 8, g = 4$ получаем переход от $\widetilde{O}(p^{4})$ к $\widetilde{O}(p^{1.5})$.
\end{itemize}
\end{frame}

\begin{frame}
\begin{itemize}
    \item Кривая $D$ не всегда существует для кривой $E(\mathbb{F}_{q^n})$ и заданного рода $g \geq n$.
    \item В общем случае найти кривую $D$ не просто.
    \item Условия при которых $\exists$ $D$ не до конца ясны.
\end{itemize}
\structure{Консервативный выбор размера поля} для криптографии с учётом существования атаки спуском Вейля: $q = p$.
\end{frame}

\begin{frame}{Атака с помощью билинейных спариваний}
Пусть $r = \#G$, $G \subseteq E(\mathbb{F}_q)$ и $\mu_r = \{x \in \overline{\mathbb{F}}_q | x^r = 1\}$.

Атака использует следующее отображение на $E[r]$.
    \begin{block}{Теорема (спаривание Вейля)}
    $\exists$ отображение
    $
    e_n: E[r] \times E[r] \rightarrow \mu_r
    $
    со свойствами:
    \begin{enumerate}
        \item $e_r(T,T) = 1$
        \item $e_r(T,S) = e_r(S,T)^{-1}$
        \item $e_r(S_1 + S_2, T) = e_r(S_1, T) e_r(S_2, T)$ \hfill \textit{(билинейность)} \\
        $e_r(S, T_1 + T_2) = e_r(S, T_1) e_r(S, T_2)$
        \item $e_r(S,T) = 1, \forall T \implies S = \mathcal{O}$ \hfill \textit{(невырожденность)}\\
        $e_n(S,T) = 1, \forall S \implies T = \mathcal{O}$
    \end{enumerate}
\end{block}

Другие билинейные отображения: спаривание Тейта, эта-спаривание.
\end{frame}

\begin{frame}
    \structure{Степень вложения:} минимальное целое $k$ т.ч. $E[r] \subseteq E(\mathbb{F}_{q^k})$.
    
    \begin{itemize}
        \item Атака на DLOG: $|\left<P\right>| = r$, $Q = \ell P$.
        \begin{enumerate}
            \item Выбрать случайную точку $R$.
            \item $\alpha = e_r(P, R)$
            \item $\beta  = e_r(Q, R)$ \hfill 
            \begin{scriptsize}
                $(\beta = e_r(\ell P, R) = e_r(P, R)^\ell = \alpha^\ell)$
            \end{scriptsize}
            \item $\ell = DLOG(\alpha, \beta)$ в $\mathbb{F}_{q^k}$
        \end{enumerate}    
        \item Конструктивное использование: ZCash, IBE, SIKE.
    \end{itemize}
\end{frame}

\begin{frame}
\begin{itemize}
    \item Сложность решения DLOG в $\mathbb{F}_{q^k}$ используя
NFS (и его модификации): $L_{q^k}(1/3, c)$.
    \item Для уровня безопасности $\lambda = 128$ требуется поле размера $3072$-бит [ECRYPT'18].
    
    \begin{center}
        \structure{$\Downarrow$}
    \end{center}

    Для стойкости к атаке с помощью билинейных спариваний необходимо: $k \geq 24$ ($3072/128$).
    \item Т.к. $\mu_r \subseteq \mathbb{F}_{q^k} \iff q^k \equiv 1 \pmod{r}$. Достаточно проверить, что:
    \[
    r \nmid q^k - 1, 
    \]
    для $k = 1, \ldots, 24$.
\end{itemize}
\end{frame}

\begin{frame}{Аномальные кривые}
Кривые с $\#E(\mathbb{F}_p) = p$ называются \structure{аномальными}.
\begin{itemize}
    \item Если $\#G = p$ для кривой $E/\mathbb{F}_p$, то $\exists$ гомоморфизм $E[p] \rightarrow \Omega^0_E(\mathbb{F}_p)$
\end{itemize}
Здесь $\Omega^0_E(\mathbb{F}_p)$ -- $\mathbb{F}_p$-векторное пространство голоморфных дифференциалов, где $DLOG$ решается время $O(polylog(p))$
\begin{itemize}
    \item Подробнее: [Galbraith'12, \S26.4.1].
    \item Условия легко проверяются.
\end{itemize}
\end{frame}

%\begin{frame}{Атаки методом исчисления индексов}
%    \begin{itemize}
%        \item Semaev
%        \item Joux \& Vitse
%    \end{itemize}
%\end{frame}
%
%\begin{frame}{Атаки на параметры с малым весом Хемминга}
%    \begin{itemize}
%        \item 
%    \end{itemize}
%\end{frame}

\begin{frame}{Атаки на кривые с автоморфизмами}
Существуют модификации методов $BSGS$ или $\rho$-метода Полларда, использующие автоморфизмы.

\begin{itemize}
    \item \structure{Идея:} при поиске $DLOG$ перебирать вместо точек $P$ классы эквивалентности $(P, \psi(P), \psi^2(P), \ldots, \psi^{\alpha-1}(P))$ для $\alpha = \operatorname{ord}{\psi}$.
    \item \structure{Сложность:} для модифицированного $\rho$-метода Полларда -- $O(\sqrt{\frac{\pi}{2\alpha}}\sqrt{\#G})$ [Galbraith'12, Th.~14.4.3]
\end{itemize}
\end{frame}

\begin{frame}
    \begin{itemize}
    \item Может быть обобщено на эндоморфизмы, в случае если их можно эффективно вычислить.
    \end{itemize}

    \structure{Пример кривой:}
    \[E/\mathbb{F}_p: y^2 = x^3 + a_6,\]
    \begin{itemize}
        \item Автоморфизм: $(x,y) \mapsto (\zeta_3 x, y)$ для $p \equiv 1 \pmod{3}$, $\alpha = 3$.
        \item Эффективная арифметика, т.к. $a_4 = 0$.
        \item Однако нужно учитывать ускорение $DLOG$.
    \end{itemize}
\end{frame}

\begin{frame}{Условия безопасности для $\lambda = 128$ относительно основных атак.}
    \[E/\mathbb{F}_q: y^2 = x^3 + a_4 x + a_6\]
    \begin{enumerate}
        \item $r = \#\left<P\right> \subseteq E(\mathbb{F}_q)$ -- простое число, $\#E(\mathbb{F}_q) / r$ -- малое число (стойкость к методу Полига-Хеллмана)
        \item $r \approx 2^{256}$ (стойкость к $\rho$-методу Полларда) 
        \item $q = p$ (стойкость к спуску Вейля)
        \item $r \nmid q^k - 1$ для $k \leq 24$ (стойкость к атакам на спариваниях)
        \item $r \neq p$ (кривая не аномальная)
    \end{enumerate}
\end{frame}

\begin{frame}{Дополнительно}
    \begin{itemize}
        \item Параметры кривой должны сопровождаться детальным описанием откуда они взялись.
        \begin{itemize}
            \item сиды всех псевдослучайных функций
            \item выбор псевдослучайных функций / хеш-функций (если $a_4 = hash(seed), a_6 = hash(seed)$)
        \end{itemize}
        \item Условия только для DLOG, не гарантируется безопасное использование $E$ в протоколах
    \end{itemize}
\end{frame}

%\begin{frame}{Требования стандарта (ГОСТ)}
%\begin{itemize}
%    \item 
%\end{itemize}
%\end{frame}

\begin{frame}{II. Эффективность}
Есть $3$ основных формы кривой $E$.
\begin{enumerate}
    \item Краткая форма Вейерштрасса:
    \[y^2 = x^3 + a x + b\]
    \item Кривые Монтгомери:
    \[
    B y^2 = x^3 + A x^2 + x
    \]
    \item Кривые Эдвардса:
    \[
    x^2 + y^2 = 1 + d x^2 y^2
    \]
\end{enumerate}
\end{frame}

\begin{frame}{Сравнение операций}
\begin{table}[]
    \begin{tabular}{l|lc}
     Кривая/Операция   & $P + Q$ & $2 P$  \\
        \hline
     Кривая Вейерштрасса (проект. коорд.)  & $12 M + 2 S$ & $5M + 2 S$ \\
     Кривая Вейерштрасса (коорд. Якоби)  & $11 M + 5 S$ & $1M + 8 S$ \\
     Кривая Эдвардса  & $10 M + 1 S$  & $3M + 4 S$ \\
     Кривая Монтгомери & $6M + 2 S$\footnote{ для $2P + Q$} & $4M$
    \end{tabular}
\end{table}

\end{frame}

%\begin{frame}{II. Криптография на спариваниях}
%    \begin{itemize}
%        \item 
%    \end{itemize}
%\end{frame}

%\begin{frame}{III. Криптография на изогениях}
%    \begin{itemize}
%        \item 
%    \end{itemize}
%\end{frame}

\begin{frame}{Литература}
    %\nocite{Menezes1993}
    %\nocite{Lenstra1987}
    %\nocite{Blake1999}
    \nocite{CohenFrey+2005}
    %\nocite{Washington2008}
    %\nocite{GoldwasserKilian1999}
    %\nocite{CohenLenstra1984}
    \nocite{Galbraith2012}
    \nocite{SafeCurves}
    \printbibliography
    
    \begin{center}
        \begin{tcolorbox}[enhanced,hbox,colback=block-green-color-bg,colframe=subsection-color!120,title=Контакты,center title]
            \begin{varwidth}{\textwidth}
                \begin{center}
                    \href{mailto:snovoselov@kantiana.ru}{snovoselov@kantiana.ru}
                \end{center}
            \end{varwidth}
        \end{tcolorbox}	
    \end{center}\end{frame}

\end{document}