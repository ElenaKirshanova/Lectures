% !TeX spellcheck = ru_RU-Russian
\documentclass[12pt]{article}
\usepackage{amsmath,amssymb,amsthm}
\usepackage{algorithm}
\usepackage[noend]{algpseudocode} 
\usepackage{enumitem}

%---enable russian----

\usepackage[utf8]{inputenc}
\usepackage[russian]{babel}


%---tikz----
\usepackage{tikz}
\usetikzlibrary{arrows, chains, matrix, positioning, scopes, patterns, shapes}
\usepackage{pgfplots, subfigure}


% MACROS 
% PROBABILITY SYMBOLS
\newcommand*\PROB\Pr 
\DeclareMathOperator*{\EXPECT}{\mathbb{E}}


% GROUPS/DISTRIBUTIONS/SETS/LISTS
\newcommand{\N}{{{\mathbb N}}}
\newcommand{\Z}{{{\mathbb Z}}}
\newcommand{\Q}{{{\mathbb Q}}}
\newcommand{\R}{{{\mathbb R}}}
\newcommand{\F}{{{\mathbb F}}}
\newcommand{\PP}{{{\mathbb P}}}
\newcommand*{\IZ}{\ensuremath{\mathbb{Z}}}
\newcommand*{\IN}{\ensuremath{\mathbb{N}}}
\newcommand*{\IR}{{{\mathbb R}}}
\newcommand{\Zp}{\ints_p} % Integers modulo p
\newcommand{\Zq}{\ints_q} % Integers modulo q
\newcommand{\Zn}{\ints_N} % Integers modulo N
\newcommand{\Zr}{\ensuremath{\mathbb{Z}/r\mathbb{Z}}} % Integers modulo N
\newcommand*{\dDR}{\mathrm{d}} %de-Rham-Differential (the d in dx, dy, dz and so on)
\newcommand{\transpose}{\mkern0.1mu^{\mathsf{t}}}
\newcommand*{\union}{\mathbin{\cup}}

% Landau 
\newcommand{\bigO}{\mathcal{O}}
\newcommand*{\OLandau}{\bigO}
\newcommand*{\WLandau}{\Omega}
\newcommand*{\xOLandau}{\widetilde{\OLandau}}
\newcommand*{\xWLandau}{\widetilde{\WLandau}}
\newcommand*{\TLandau}{\Theta}
\newcommand*{\xTLandau}{\widetilde{\TLandau}}
\newcommand{\smallo}{o} %technically, an omicron
\newcommand{\softO}{\widetilde{\bigO}}
\newcommand{\wLandau}{\omega}
\newcommand{\negl}{\mathrm{negl}} 

% Misc
\newcommand{\eps}{\varepsilon}
\newcommand{\inprod}[1]{\left\langle #1 \right\rangle}

 
\newcommand{\handout}[5]{
  \noindent
  \begin{center}
  \framebox{
    \vbox{
      \hbox to 5.78in { {\bf  } \hfill #2 }
      \vspace{4mm}
      \hbox to 5.78in { {\Large \hfill #5  \hfill} }
      \vspace{2mm}
      \hbox to 5.78in { {\em #3 \hfill #4} }
    }
  }
  \end{center}
  \vspace*{4mm}
}

\newcommand{\lecture}[4]{\handout{#1}{#2}{#3}{}{Лабораторная работа #1}}

\newtheorem{theorem}{Теорема}
\newtheorem{corollary}[theorem]{Следствие}
\newtheorem{lemma}[theorem]{Лемма}
\newtheorem{observation}[theorem]{Observation}
\newtheorem{proposition}[theorem]{Предложение}

\theoremstyle{definition}
\newtheorem{definition}[theorem]{Определение}

\newtheorem{claim}[theorem]{Утверждение}
\newtheorem{fact}[theorem]{Факт}
\newtheorem{assumption}[theorem]{Предположение}

\theoremstyle{definition}
\newtheorem{examples}[theorem]{Примеры}

\theoremstyle{definition}
\newtheorem{example}[theorem]{Пример}

% 1-inch margins
\topmargin 0pt
\advance \topmargin by -\headheight
\advance \topmargin by -\headsep
\textheight 8.9in
\oddsidemargin 0pt
\evensidemargin \oddsidemargin
\marginparwidth 0.5in
\textwidth 6.5in

\parindent 0in
\parskip 1.5ex

\begin{document}
    
    \lecture{\textnumero 1}{Осень 2021}{Преподаватель: Новоселов Семен}{}
    
    Разработать программу в системе компьютерной алгебры Sage, реализующую следующие функции:
    
    \begin{enumerate}
        \item \texttt{jInvariant}($a_1$, $a_2$, $a_3$, $a_4$, $a_6$), где $a_1$, $a_2$, $a_3$, $a_4$, $a_6$ -- коэффициенты кривой, заданной уравнением Вейерштрасса. Если кривая является эллиптической, функция возвращает $j$-инвариант кривой, иначе сообщение о том, что кривая сингулярна.
        \item \texttt{randIsomorphic}($a_1 = 0$, $a_2 = 0$, $a_3 = 0$, $a_4 = 0$, $a_6 = 0$, $a = 0$, $b = 0$), где $a_1$, $a_2$, $a_3$, $a_4$, $a_6$, $a$, $b$ -- коэффициенты эллиптической кривой $E_1$ в общем случае, или в случае $char(K)\neq 2, 3$. Функция возвращает коэффициенты кривой $E_2$, изоморфной $E_1$ над $\mathbb{Q}$ путём случайного выбора параметров $(u, r, s, t)$. Если коэффициенты $a_1$, $a_2$, $a_3$, $a_4$, $a_6$ задают сингулярную кривую, функция терминирует с соответствующим сообщением.
        \item \texttt{isIsomorphic}($a_1$, $a_2$, $a_3$, $a_4$, $a_6$, $\_a_1$, $\_a_2$, $\_a_3$, $\_a_4$, $\_a_6$, $p$), где $a_1$, $a_2$, $a_3$, $a_4$, $a_6$ -- коэффициенты эллиптической кривой $E_1$, $\_a_1$, $\_a_2$, $\_a_3$, $\_a_4$, $\_a_6$ -- коэффициенты эллиптической кривой $E_2$, $p$ -- простое число (означает кривые заданы над $\mathbb{F}_p$) или $0$ (кривые заданы над $\mathbb{Q}$). Функция определяет, являются ли кривые изоморфными над $\mathbb{F}_p$ (или $\mathbb{Q}$), и возвращает одно из значений $\in$ $\{isomorphic, non-isomorphic\}$. Если коэффициенты $a_1$, $a_2$, $a_3$, $a_4$, $a_6$ или $\_a_1$, $\_a_2$, $\_a_3$, $\_a_4$, $\_a_6$ задают сингулярную кривую, функция терминирует с соответствующим сообщением.
        \item \texttt{findExtension}($a_1$, $a_2$, $a_3$, $a_4$, $a_6$, $\_a_1$, $\_a_2$, $\_a_3$, $\_a_4$, $\_a_6$, $p$), коэффициенты эллиптической кривой $E_1$,
        $\_a_1$, $\_a_2$, $\_a_3$, $\_a_4$, $\_a_6$ -- коэффициенты эллиптической кривой $E_2$, заданные над $\mathbb{F}_p$ ($p$ интерпретировать аналогично предыдущей функции).Функция определяет, над каким полем кривые $E_1 \simeq E_2$ и возвращает степень расширения этого поля над $\mathbb{F}_p$. Если коэффициенты $a_1$, $a_2$, $a_3$, $a_4$, $a_6$ или $\_a_1$, $\_a_2$, $\_a_3$, $\_a_4$, $\_a_6$ задают сингулярную кривую, функция терминирует с соответствующим сообщением.
    \end{enumerate}

    Требования к сдаче
    \begin{itemize}
        \item Исходный код должен содержать комментарии к каждой из функций с описанием входных и выходных параметров
        \item Лабораторную следует выполнять модификацией файла с тестами, заменяя строку "\texttt{\# your code here.}" на код, реализующий функцию.
        \item Функции должны работать на всех примерах, что проверяется запуском команды:
        \\\texttt{sage -t file\_with\_tests.sage}
        \item Студент должен понимать, что он написал, зачем, а также ответить на теоретические вопросы.
    \end{itemize}    
    
\end{document}